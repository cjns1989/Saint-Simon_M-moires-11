\PassOptionsToPackage{unicode=true}{hyperref} % options for packages loaded elsewhere
\PassOptionsToPackage{hyphens}{url}
%
\documentclass[oneside,10pt,french,]{extbook} % cjns1989 - 27112019 - added the oneside option: so that the text jumps left & right when reading on a tablet/ereader
\usepackage{lmodern}
\usepackage{amssymb,amsmath}
\usepackage{ifxetex,ifluatex}
\usepackage{fixltx2e} % provides \textsubscript
\ifnum 0\ifxetex 1\fi\ifluatex 1\fi=0 % if pdftex
  \usepackage[T1]{fontenc}
  \usepackage[utf8]{inputenc}
  \usepackage{textcomp} % provides euro and other symbols
\else % if luatex or xelatex
  \usepackage{unicode-math}
  \defaultfontfeatures{Ligatures=TeX,Scale=MatchLowercase}
%   \setmainfont[]{EBGaramond-Regular}
    \setmainfont[Numbers={OldStyle,Proportional}]{EBGaramond-Regular}      % cjns1989 - 20191129 - old style numbers 
\fi
% use upquote if available, for straight quotes in verbatim environments
\IfFileExists{upquote.sty}{\usepackage{upquote}}{}
% use microtype if available
\IfFileExists{microtype.sty}{%
\usepackage[]{microtype}
\UseMicrotypeSet[protrusion]{basicmath} % disable protrusion for tt fonts
}{}
\usepackage{hyperref}
\hypersetup{
            pdftitle={SAINT-SIMON},
            pdfauthor={Mémoires XI},
            pdfborder={0 0 0},
            breaklinks=true}
\urlstyle{same}  % don't use monospace font for urls
\usepackage[papersize={4.80 in, 6.40  in},left=.5 in,right=.5 in]{geometry}
\setlength{\emergencystretch}{3em}  % prevent overfull lines
\providecommand{\tightlist}{%
  \setlength{\itemsep}{0pt}\setlength{\parskip}{0pt}}
\setcounter{secnumdepth}{0}

% set default figure placement to htbp
\makeatletter
\def\fps@figure{htbp}
\makeatother

\usepackage{ragged2e}
\usepackage{epigraph}
\renewcommand{\textflush}{flushepinormal}

\usepackage{indentfirst}
\usepackage{relsize}

\usepackage{fancyhdr}
\pagestyle{fancy}
\fancyhf{}
\fancyhead[R]{\thepage}
\renewcommand{\headrulewidth}{0pt}
\usepackage{quoting}
\usepackage{ragged2e}

\newlength\mylen
\settowidth\mylen{...................}

\usepackage{stackengine}
\usepackage{graphicx}
\def\asterism{\par\vspace{1em}{\centering\scalebox{.9}{%
  \stackon[-0.6pt]{\bfseries*~*}{\bfseries*}}\par}\vspace{.8em}\par}

\usepackage{titlesec}
\titleformat{\chapter}[display]
  {\normalfont\bfseries\filcenter}{}{0pt}{\Large}
\titleformat{\section}[display]
  {\normalfont\bfseries\filcenter}{}{0pt}{\Large}
\titleformat{\subsection}[display]
  {\normalfont\bfseries\filcenter}{}{0pt}{\Large}

\setcounter{secnumdepth}{1}
\ifnum 0\ifxetex 1\fi\ifluatex 1\fi=0 % if pdftex
  \usepackage[shorthands=off,main=french]{babel}
\else
  % load polyglossia as late as possible as it *could* call bidi if RTL lang (e.g. Hebrew or Arabic)
%   \usepackage{polyglossia}
%   \setmainlanguage[]{french}
%   \usepackage[french]{babel} % cjns1989 - 1.43 version of polyglossia on this system does not allow disabling the autospacing feature
\fi

\title{SAINT-SIMON}
\author{Mémoires XI}
\date{}

\begin{document}
\maketitle

\hypertarget{chapitre-premier.}{%
\chapter{CHAPITRE PREMIER.}\label{chapitre-premier.}}

1713

~

{\textsc{Constitution \emph{Unigenitus} fabriquée et subitement publiée
à Rome.}} {\textsc{- Soulèvement général difficilement arrêté.}}
{\textsc{- Soulèvement général contre la constitution à son arrivée en
France.}} {\textsc{- Singulières conversations entre le P. Tellier et
moi sur la forme de faire recevoir la constitution, et sur elle-même.}}
{\textsc{- Retour par Petit-Bourg de Fontainebleau à Versailles.}}
{\textsc{- Étrange tête-à-tête sur la constitution entre le P. Tellier
et moi, qui me jette en un \emph{sproposito} énorme.}}

~

Aubenton et Fabroni étaient cependant venus à bout de leur ténébreux
ouvrage, sans qu'aucun tiers eût su ce qui se faisait par eux, sinon en
gros qu'on travaillait à une constitution pour l'affaire de France. La
pièce fut mise avec le même secret dans l'état de perfection que le P.
Tellier l'avait commandée. Tout y brillait, excepté la vérité. L'art et
l'audace y étaient sur le trône, et toutes les vues qu'on s'y était
proposées s'y trouvèrent plus que parfaitement remplies. L'art s'y était
épuisé, l'audace y surpassait celle de tous les siècles, puisqu'elle
alla jusqu'à condamner en propres termes des textes exprès de saint
Paul, que tous les siècles depuis Jésus-Christ avaient respectés comme
les oracles du Saint-Esprit même, sans en excepter aucun hérétique, qui
se sont au moins contentés de détourner les passages de l'Écriture à des
sens étrangers et forcés, mais qui n'ont jamais osé aller jusqu'à les
rejeter ni à les condamner. C'est ce que cette constitution eut
au-dessus d'eux\,; et ce qu'elle y eut de commun fut le mépris et la
condamnation expresse de saint Augustin et des autres Pères, dont la
doctrine a toujours été adoptée par les papes, par les conciles
généraux, par toute l'Église comme la sienne propre.

L'inconvénient était un peu fort, mais tout à fait indispensable pour le
but auquel on tendait. Les deux auteurs de la pièce le sentirent. Ils
n'espérèrent pas de la faire passer aux cardinaux, qu'une nouveauté si
étonnante révolterait, ni en particulier au cardinal de La Trémoille sur
les maximes ultramontaines absolument nécessaires pour gagner Rome par
un intérêt si cher. Aubenton avait fourni l'adresse\,; ce fut à Fabroni
à se charger de l'impudence. Ils enfermèrent des imprimeurs, tirèrent ce
qu'ils voulurent d'exemplaires, gardèrent les planches et les imprimeurs
tant que le secret leur fut important, puis ils allèrent trouver le
pape, auquel ils en firent une rapide lecture.

Elle ne put l'être assez pour que Clément ne fût pas frappé de la
condamnation des textes formels de saint Paul, de saint Augustin, des
autres Pères. Il se récria. Fabroni insista pour achever la lecture
qu'Aubenton en faisait modestement. Le pape voulut garder la pièce pour
la relire à son aise, et y faire ses corrections. Fabroni le traita
comme autrefois\,; il étourdit le pape et le malmena. Clément crut au
moins s'en tirer de biais, en représentant à Fabroni le danger d'exposer
à l'examen des cardinaux une censure expresse des termes formels de
saint Paul, dont il n'y avait point d'exemple dans l'Église, et même de
saint Augustin, dans une matière où elle avait adopté sa doctrine pour
sienne. Mais cela n'arrêta point Fabroni, qui lui répondit qu'il serait
plaisant de donner son ouvrage à des réviseurs\,; et qu'il ne se
laisserait point mettre sur la sellette, ni le pape, sous le nom duquel
l'ouvrage était fait, et qui le prononçait y parlant et y décidant
lui-même. Clément dit qu'il était engagé de parole, au cardinal de La
Trémoille en particulier, de ne rien donner là-dessus que de concert
avec lui\,; et qu'il avait solennellement promis au sacré collège que la
pièce ne verrait pas le jour qu'ils ne l'eussent examinée par petites
congrégations les uns avec les autres, et que conformément à l'avis du
plus grand nombre d'entre eux. Fabroni s'emporta de colère, traita le
pape de faible et qui se rendait un petit garçon, lui soutint la
constitution belle et bonne, toute telle qu'il la fallait, et que, s'il
avait fait la sottise de donner cette parole, il ne fallait pas la
combler en la tenant, laisse le pape éperdu, sort, et de ce pas l'envoie
afficher par tous les lieux publics, où on a coutume d'afficher et de
publier les bulles et les constitutions nouvellement faites à Rome.

Ce coup fit un grand bruit parmi les cardinaux, qui se virent joués et
moqués par un manquement de parole si complet, et si peu attendu. Ils
s'assemblèrent par troupes les uns chez les autres, et leurs plaintes
les plus fortes y furent promptement résolues. Les chefs d'ordre, et les
plus considérables d'entre les autres, allèrent par huit, par dix, par
six, trouver le pape, à qui ils témoignèrent l'étonnement d'un
manquement de parole aussi éclatant, et d'une parole si solennellement
sortie de sa bouche\,; et leur scandale de voir émaner une constitution
doctrinale et de jugement en première instance dans Rome, sans avoir été
consultés comme l'exigent leur droit, leur pourpre, leur qualité
d'assesseurs et de conseillers nécessaires, sur des matières de cette
importance et de cette qualité. Le pape confus ne sut que leur répondre.
Il protesta que la publication s'était faite à son insu\,; et les paya
de compliments, d'excuses et de larmes qu'il avait fort à commandement.

Cela n'apaisa point le bruit. Les cardinaux prétendirent revenir à
l'examen, et à soutenir leur dignité violée. Casoni, Davia, quelques
autres de la première considération pour leur savoir ou pour les
affaires qu'ils avaient maniées, trouvèrent la substance de la chose
plus intolérable encore que le procédé. Ils allèrent représenter au pape
que sa constitution renversait la doctrine de l'Église reçue de tous les
siècles, celle de saint Augustin et d'autres Pères adoptée pour telle
par les conciles généraux et par tous les papes jusqu'à lui\,; que
jamais les hérétiques mêmes n'avaient osé attenter à condamner
expressément des textes formels de l'Écriture\,; et qu'il était le
premier qui depuis Jésus-Christ eut ébranlé les fondements les plus
incontestables de la religion, en condamnant des propositions mot pour
mot de saint Paul. Que fût devenue la constitution en France, et les
projets si avancés du P. Tellier, si elle eût avorté dans Rome presque
avant que de naître\,? Aussi fut-ce le chef-d'oeuvre de l'art, de
l'argent, des souplesses des jésuites et des leurs, de parer un coup si
funeste. Le cardinal Albani et les créatures du pape les plus attachées
à lui s'employèrent par degrés pour des tempéraments qu'en effet ils ne
voulaient pas admettre, mais en leurrer pour émousser le premier feu\,;
et, pour ne nous pas trop arrêter à Rome, le grand intérêt des cardinaux
de ne pas se désunir du pape, celui de son infaillibilité qui rejaillit
si utilement sur eux, celui des maximes ultramontaines les plus fortes
et les plus habilement insérées dans la constitution, apaisèrent enfin
les ignorants et les politiques, qui eux-mêmes devinrent un frein à ceux
qui dans le sacré collège, dans la prélature et dans les emplois
réguliers, saisis par leurs lumières et guidés par leur conscience,
voulurent s'opposer à la constitution, et demeurèrent enfin réduits à la
détester presque en silence.

Le même jour qu'elle fut affichée dans Rome, elle fut envoyée au P.
Tellier, par un courrier secret qui prévint de peu de jours celui qui
l'apporta au nonce, qui la reçut à Fontainebleau, le lundi 2 octobre, et
la présenta au roi le lendemain matin dans son cabinet, en audience
particulière. Il fit au roi un beau discours en italien, auquel le roi,
qui l'entendait, et que le P. Tellier avait eu le temps de préparer,
répondit en français le plus favorablement du monde. On remarqua qu'il y
avait une grande promenade ordonnée autour du canal pour l'après-dînée,
et qu'il n'y en eut point, parce que le roi travailla sur cette affaire,
seul avec Voysin, jusqu'à six heures du soir. Le P. Tellier, pour sonder
les esprits, avait lâché quelques exemplaires de la constitution avant
que le nonce la portât au roi. Il avait mandé le premier président et le
parquet, qui, dès le 1er octobre, alarmés des maximes ultramontaines
dont la constitution était remplie, vinrent présenter un mémoire au roi.

Elle eut en France le même sort qu'elle avait essuyé à Rome\,: le cri
fut universel. Le cardinal de Rohan déclara qu'elle ne pouvait être
reçue, et Bissy même protesta contre elle\,; les uns indignés de sa
naissance des plus épaisses ténèbres, les autres de la proposition
touchant l'excommunication qui rendait le pape maître obliquement de
toutes les couronnes\,; les uns choqués de la condamnation de la
doctrine et des passages de saint Augustin et des autres Pères\,; tous
effrayés de celle des paroles mêmes de saint Paul. Il n'y eut pas deux
avis dans les premiers huit jours. Le cardinal de La Trémoille à qui le
pape avait en particulier manqué de parole, comme il en avait manqué à
tout le sacré collège, et sur lequel ses plaintes avaient eu aussi peu
d'effet, envoya un courrier exprès pour se justifier d'avoir laissé
publier une constitution si directement contraire aux maximes du royaume
qu'elle attaquait de front, et souleva tous les ministres, excepté le
duc de Beauvilliers. La cour, la ville et les provinces, à mesure que la
constitution y fut connue, se soulevèrent également.

Le P. Tellier tint ferme, fronça le sourcil sur Bissy, comme sur un
homme dans sa dépendance, qui ne tenait pas encore son chapeau, et à qui
en disant un mot, et ici et à Rome, il pouvait le faire manquer\,; il
parla ferme à Rohan, et lui fit entendre le péril qu'il courait à ne pas
tenir les promesses qui lui avaient valu la charge de grand aumônier\,;
et il n'oublia rien pour se rendre maître de tout ce qu'il put
d'évêques, et pour intimider ceux qui étaient déjà siens, {[}de façon{]}
qu'aucun ne lui put échapper.

Il fallait recevoir la constitution, et la manière de le faire était
embarrassante par la contradiction qu'elle rencontrait dès son premier
abord. Le Tellier, qui me cultivait toujours, m'avait parlé souvent de
cette affaire avant et depuis qu'elle fut portée à Rome\,; et moi, qui
évitais ces conversations, mais qui ne pouvais lui fermer ma porte,
surtout à Fontainebleau où il était toujours à demeure, je lui répondais
si franchement, et si fort selon la vérité et ma pensée que
M\textsuperscript{me} de Saint-Simon m'en reprenait souvent, et me
disait que je me ferais chasser, et peut-être mettre à la Bastille.

La constitution venue, le P. Tellier me demanda un rendez-vous pour
raisonner avec moi. Je crus que c'était pour me la montrer, car presque
personne encore ne l'avait vue, et le nonce ne l'avait pas encore portée
au roi. Quand nous fûmes tête à tête je lui demandai à la voir. Il me
dit qu'il n'en avait qu'un exemplaire sur lequel on travaillait, mais
qu'il me la donnerait au premier jour, et qu'il pouvait m'assurer
qu'elle était bien et bonne, et telle que j'en serais content\,; que ce
qui l'avait engagé à me demander cette conversation, c'était pour me
consulter sur la manière de la faire recevoir. Je me mis à rire de ce
qu'il voulait me demander ce qu'il savait bien mieux que moi, et
peut-être ce que déjà il avait résolu. Il se répandit en discours,
partie de compliments, partie de la difficulté de la chose sur un
premier effarouchement qui commençait à bourdonner. Il me pressa
tellement que je lui dis qu'il me paraissait qu'il avait sa leçon toute
tracée dans la manière dont le roi avait fait recevoir la condamnation
de M. de Cambrai, qui était parfaitement juridique, sans embarras, et
selon toutes les formes les plus ecclésiastiques.

Je n'eus pas lâché la parole que d'un air de confiance et d'ingénuité,
dont je ne reviens pas encore, il me dit en propres termes qu'il ne se
jouerait pas à cela, et que cette forme était trop dangereuse\,; qu'il
se garderait bien de livrer la constitution aux assemblées provinciales
de chaque métropolitain, au génie de chaque évêque du royaume, et à des
gens qui ne seraient pas dans Paris, sous ses yeux. Je sentis
incontinent la violence qu'il voulait exercer qui m'anima à disputer
contre, et à lui représenter l'irrégularité d'une réception faite par
des évêques qui au hasard se trouveraient à Paris. «\,Au hasard\,!
reprit le confesseur, je ne veux point me fier au hasard\,; je prétends
mander des provinces les évêques qui me conviendront, empêcher de venir
ceux que je croirai difficiles à conduire\,; et comme je ne puis pas
empêcher ceux qui sont à Paris d'être de l'assemblée qu'il y faut faire
pour recevoir, et qu'il peut y en avoir de dyscoles, j'y fourrerai les
évêques \emph{in partibus}, et ceux même qui sont nommés et qui n'ont
pas encore leurs bulles, pour être par eux plus fort en voix, et les
opposer à quiconque voudra raisonner.\,» Je frémis à ce langage, et je
lui répondis que cela s'appelait jardiner et choisir. «\,Vraiment,
répliqua-t-il avec feu, c'est bien aussi ce que je veux faire, et ne
m'abandonner pas aux députations. --- Mais, lui dis-je, quel pouvoir
auront des évêques fortuitement à Paris, ou qui y seront mandés,
d'accepter pour leurs comprovinciaux\,; destitués de procuration
d'eux\,? --- J'en conviens, me répondit le confesseur, mais de deux
inconvénients il faux éviter le pire\,; or le pire est de se livrer au
hasard, et de ne pas se bien assurer. Pourvu qu'ils acceptent dans
l'assemblée, je ne m'embarrasse pas du reste\,; et avec ce chausse-pied,
nous verrons qui osera résister au pape et au roi. Les défauts se
suppléeront par l'autorité, et la bulle sera reçue comme que ce soit\,:
voilà ce qu'il faut.\,»

Nous disputâmes et discourûmes encore quelque temps sur ces évêques
\emph{in partibus}, et ces autres nommés et encore sans bulles, moins de
ma part pour le persuader que pour le faire parler, et j'admirais en
moi-même également ce fond de supercherie, d'adresse, de violence, de
renversement de toute règle, et cette incroyable facilité de me le
montrer à découvert. C'est une franchise que je n'ai jamais pu
comprendre l'un homme si faux, si artificieux, si profond, encore moins
à quoi il la pouvait croire utile. Je le quittai épouvanté de lui, et
des suites que je prévoyais.

Nous prîmes un autre rendez-vous pour parler de la bulle même, après
qu'il m'en aurait donné un exemplaire. Nous nous revîmes très peu de
jours avant le départ de Fontainebleau. Je le trouvai radieux. Il avait
rangé Bissy et le cardinal de Rohan à ses volontés, et reçu apparemment
de bonnes nouvelles de ses batteries de Paris. Je ne cherchais pas à
gagner à la raison et à la vérité un homme que je voyais faire si peu de
cas de l'une et de l'autre, et engagé si avant à les opprimer, mais je
n'osois rompre avec un homme si dangereux qui me ménageait jusqu'à une
folle confiance. Je lui dis donc qu'encore que j'eusse fort ouï parler
sur la doctrine de la constitution, que je fusse choqué comme tout le
monde de cette foule de propositions condamnées, et avec une généralité
d'injures atroces et sans nombre, qui, en tombant sur toutes, ne
tombaient pourtant en particulier sur aucune, encore que je fusse
effrayé de censures directes sur des textes formels de saint Paul, et
peu édifié d'une constitution de doctrine qui s'enveloppait dans
l'obscurité, au lieu de porter dans l'esprit une clarté, une netteté,
une précision instructive, j'étais trop ignorant pour me jeter avec lui
dans des disputes théologiques\,; mais que pour ce qui regardait les
prétentions romaines, et en particulier la proposition touchant
l'excommunication, j'avais la présomption de me croire bastant pour lui
dire que ces endroits de la constitution étaient insoutenables, et ne se
pouvaient jamais recevoir. Il me dit que nous reviendrions là-dessus\,;
et tout de suite il enfila assez longuement ce qui lui plut sur la
doctrine, sur quoi je le contredis peu, parce que j'en sentais la plus
qu'inutilité. Cette matière consomma presque tout le temps de notre
conférence.

Revenu à l'excommunication, il se mit à battre la campagne, convint que
ses réponses n'étaient pas bien solides\,; mais ajouta qu'il me
demandait une audience chez moi à Versailles, le vendredi après le
premier vendredi que le roi y serait arrivé, parce que lui n'irait pas
sortant de Fontainebleau\,; et qu'il se promettait dans cette
conversation me convaincre que la censure dont je me plaignais
n'attaquait en rien les droits du roi ni de sa couronne.

Il me conta, toujours avec cette naïveté dont à peine je pus croire mes
oreilles, le nombre d'évêques qu'il avait mandés des provinces, à quoi
sans doute il s'était pris avant de m'en avoir parlé pour la première
fois, et pour les avoir à temps, et d'autres mesures générales, avec un
épanouissement singulier. Nous nous séparâmes de la sorte pour nous
revoir chez moi au jour dont nous venions de convenir.

Le mercredi 11 octobre, le roi tint conseil d'État à l'ordinaire et dîna
ensuite, puis alla coucher à Petit-Bourg chez d'Antin, et le lendemain à
Versailles.

L'intelligence de ce qui suit et de ce qui m'arriva demande celle de mon
logement à Versailles. Il donnait d'un côté et de plain-pied dans la
galerie de l'aile neuve qui est de plain-pied à la tribune de la
chapelle, appuyé de l'autre côté à un degré, et tenait la moitié du
large corridor qui est vis-à-vis du grand escalier qui communique la
galerie basse avec la haute\,: un demi-double d'abord sur ce corridor,
qui en tirait le jour pour des commodités et des sorties\,; une
antichambre à deux croisées qui distribuait à droite et à gauche, où de
chaque côté il y avait une chambre à deux croisées\,; et un cabinet
après à une croisée\,; et toutes ces cinq pièces à cheminée ainsi que la
première antichambre obscure. Tout ce demi-double obscur était coupé
d'entre-sols, sous lesquels chaque cabinet avait un arrière-cabinet. Cet
arrière-cabinet, moins haut que le cabinet, n'avait de jour que par le
cabinet même. Tout était boisé\,; et ces arrière-cabinets avaient une
porte et des fenêtres qui, étant fermées, ne paraissaient point du tout
et laissaient croire qu'il n'y avait rien derrière. J'avais dans mon
arrière-cabinet un bureau, des sièges, des livres et tout ce qu'il me
fallait\,; les gens fort familiers qui connaissoient cela l'appelaient
ma boutique, et en effet cela n'y ressemblait pas mal.

Le P. Tellier ne manqua pas au rendez-vous qu'il m'avait demandé. Je lui
dis qu'il avait mal pris son temps, parce que M. le duc et
M\textsuperscript{me} la duchesse de Berry avaient demandé une collation
à M\textsuperscript{me} de Saint-Simon, qu'ils allaient arriver, qu'ils
étaient tout propres à se promener dans tout l'appartement, et que je ne
pouvais être le maître de ma chambre ni de mon cabinet. Le P. Tellier
parut fort peiné du contre-temps\,; et il insista si fort à trouver
quelque réduit inaccessible à la compagnie, pour ne pas remettre notre
conférence à son retour à la huitaine, que, pressé par lui à l'excès, je
lui dis que je ne savais qu'un seul expédient, qui était qu'il renvoyât
son frère vatblé\footnote{Le mot \emph{vatblé} était consacré pour
  désigner le frère qui accompagnait un religieux.} pour que ce qui
allait arriver ne le trouvât pas dans l'antichambre\,; que lui et moi
nous enfermassions dans ma boutique, que je lui montrai\,; que nous y
eussions des bougies, pour ne point dépendre du jour du cabinet, et
qu'alors nous serions en sûreté contre les promenades, quittes pour nous
taire, si nous entendions venir dans mon cabinet, jusqu'à ce qu'on en
fût sorti. Il trouva l'expédient admirable, renvoya son compagnon\,; et
nous nous enfermâmes vis-à-vis l'un de l'autre, mon bureau entre-deux,
avec deux bougies allumées dessus.

Là il se mit à me paraphraser les excellences de la constitution
\emph{Unigenitus}, dont il avait apporté un exemplaire qu'il mit sur la
table. Je l'interrompis pour venir à la proposition de
l'excommunication. Nous la discutâmes avec beaucoup de politesse, mais
avec fort peu d'accord. Tout le monde sait que la proposition censurée
est\,: \emph{qu'une excommunication injuste ne doit point empêcher de
faire son devoir}\,; par conséquent qu'il résulte de sa censure\,:
\emph{que excommunication injuste doit empêcher de faire son devoir}.
L'énormité de cette dernière frappe encore plus fortement que ne fait la
simple vérité de la proposition censurée. C'en est une ombre qui la fait
mieux ressortir. Les suites et les conséquences affreuses de la censure
sautent aux yeux.

Je ne prétends pas rapporter notre dispute. Elle fut vive et longue.
Pour l'abréger je lui fis remarquer que dans la situation présente des
choses, où, quand on raisonne on doit tout prévoir, surtout les cas les
plus naturels, conséquemment les plus possibles, le roi pouvait mourir
et le Dauphin aussi, qui tous les deux se trouvaient aux deux extrémités
opposées de l'âge\,; que, si ce double malheur arrivait, la couronne par
droit de naissance appartiendrait au roi d'Espagne et à sa branche\,;
que par le droit que les renonciations venaient d'établir, elle
appartiendrait à M. le duc de Berry et à sa branche, et à son défaut à
M. le duc d'Orléans et à la sienne\,; que si les deux frères se la
voulaient disputer, ils auraient chacun des forces, des alliés et en
France des partisans\,; qu'alors le pape aurait beau jeu, si sa
constitution était crue et reçue sans restriction, de donner la couronne
à celui des deux contendants qu'il lui plairait, en excommuniant
l'autre, puisque, moyennant sa censure reçue et crue, quelque juste que
pût être le droit de l'excommunié, quelque devoir qu'il y eût à soutenir
son parti, il faudrait l'abandonner et passer de l'autre côté, puisqu'il
serait établi, et qu'on serait persuadé qu'une excommunication injuste
doit empêcher de faire son devoir\,; et dès là, d'une façon ou d'une
autre, voilà le pape maître de toutes les couronnes de sa communion, de
les ôter à qui les doit porter, à qui les porte même et de les donner à
quiconque il lui plaira, comme tant de papes depuis Grégoire VII ont osé
le prétendre, et tant qu'ils se sont crus en force de l'attenter.

L'argument était également simple, présent, naturel et pressant\,; il
s'offrait de soi-même. Aussi le confesseur en fut-il étourdi\,; le rouge
lui monta, il battit la campagne\,; moi de le presser. Il reprit ses
esprits peu à peu\,; et, avec un sourire de satisfaction de la solution
péremptoire qu'il m'allait donner\,: «\,Vous n'y êtes, me dit-il\,;
tenez, d'un seul mot je vais faire tomber tout votre raisonnement\,;
écoutez-moi\,: Si, dans le cas que vous proposez, et qui malheureusement
n'est que trop susceptible d'arriver, le pape s'avisait de prendre parti
pour l'un des deux contendants, et d'excommunier l'autre et ceux qui
l'assisteraient, alors cette excommunication ne serait pas dans le cas
de la censure que le pape fait dans sa bulle, elle ne serait pas injuste
seulement, mais elle serait fausse. Voyez bien, monsieur, cette
différence, et sentez-la\,; car le pape ne peut avoir aucune raison
d'excommunier aucun des deux partis, ni des deux contendants. Or, cela
étant comme cela est vrai, son excommunication serait fausse. Jamais il
n'a été décidé qu'une excommunication fausse puisse ni doive empêcher de
faire son devoir\,; par conséquent cette excommunication porterait faux,
et ne porterait aucun avantage à l'un ni aucun préjudice à l'autre, qui
agirait tout comme s'il n'y avait point d'excommunication. --- Voilà,
mon père, qui est admirable, lui répondis-je\,; la distinction est
subtile et habile, j'en conviens, et j'avoue encore que je ne m'y
attendais pas\,; mais quelques petites objections encore, je vous
supplie. Les ultramontains conviendraient-ils de la nullité de
l'excommunication\,? N'est-elle pas nulle dès qu'elle est injuste\,? car
qui peut enjoindre de commettre l'injustice, et l'enjoindre sous peine
d'excommunication\,? Si le pape a le pouvoir d'excommunier injustement,
et de faire obéir à son excommunication, qui est-ce qui a limité un
pouvoir aussi illimité, et pourquoi son excommunication nulle ne
serait-elle pas respectée et obéie autant que son excommunication
injuste\,? Enfin, quand, par la réception des évêques, des parlements de
tout le royaume, et qu'en conséquence par la chaire, les confessions et
les instructions, il sera bien établi et inculqué à toutes sortes de
personnes que l'excommunication injuste doit empêcher de faire son
devoir, qu'ensuite le cas proposé arrivera en France, et qu'en
conséquence le pape excommuniera l'un des contendants et ceux qui
soutiendront son parti, pensez-vous qu'alors il fût facile de faire
comprendre votre subtile distinction entre l'excommunication injuste et
l'excommunication fausse aux peuples, aux soldats, aux officiers, aux
bourgeois, aux seigneurs, aux femmes, au gros du monde, de leur en
prouver la différence, d'appliquer cette différence à l'excommunication
fulminée, de les en bien convaincre, et tout cela dans le moment qu'il
serait question d'agir et de prendre les armes\,? Voilà, mon père, de
grands inconvénients\,; et je n'en vois aucun à ne pas recevoir la
censure dont il s'agit entre nous dans la bulle, que celui de ne pas
laisser prendre au pape ce nouveau titre qu'il se donne à lui-même de
pouvoir déposer les rois, dispenser leurs sujets du serment de fidélité,
et disposer de leur couronne, contre les paroles formelles de
Jésus-Christ et de toute l'Écriture.\,»

Cette courte exposition transporta le jésuite, parce qu'elle mettait le
doigt sur la lettre malgré ses cavillations et ses adresses. Il évita
toujours de me rien dire de personnel, mais il rageait\,; et plus il se
contenait à mon égard, moins il le put sur la matière\,; et, comme pour
se dédommager de sa modération à mon égard, plus il s'emporta et se
lâcha sur la manière de forcer tout le royaume à recevoir la bulle sans
en modifier la moindre chose.

Dans cette fougue, où, n'étant plus maître de soi, il s'échappa à bien
des choses dont je suis certain qu'il aurait après racheté très
chèrement le silence, il me dit tant de choses sur le fond et sur la
violence pour faire recevoir, si énormes, si atroces, si effroyables, et
avec une passion si extrême, que j'en tombai en véritable syncope. Je le
voyais bec à bec entre deux bougies, n'y ayant du tout que la largeur de
la table entre-deux (j'ai décrit ailleurs son horrible physionomie)\,;
éperdu tout à coup par l'ouïe et par la vue, je fus saisi, tandis qu'il
parlait, de ce que c'était qu'un jésuite, qui, par son néant personnel
et avoué, ne pouvait rien espérer pour sa famille, ni par son état et
par ses vœux, pour soi-même, pas même une pomme ni un coup de vin plus
que tous les autres, qui par son âge touchait au moment de rendre compte
à Dieu, et qui, de propos délibéré et amené avec grand artifice, allait
mettre l'État et la religion dans la plus terrible combustion, et ouvrir
la persécution la plus affreuse pour des questions qui ne lui faisaient
rien, et qui ne touchaient que l'honneur de leur école de Molina.

Ses profondeurs, les violences qu'il me montra, tout cela ensemble me
jeta en une telle extase, que tout à coup je me pris à lui dire en
l'interrompant\,: «\, Mon père, quel âge avez-vous\,?» Son extrême
surprise, car je le regardais de tous mes yeux qui la virent se peindre
sur son visage, rappela mes sens, et sa réponse acheva de me faire
revenir à moi-même. «\,Hé\,! pourquoi, me dit-il en souriant, me
demandez-vous cela\,? L'effort que je me fis pour sortir d'un
\emph{sproposito} si unique, et dont je sentis toute l'effrayante
valeur, me fournit une issue\,: «\,C'est, lui dis-je, que je ne vous
avais jamais tant regardé de suite qu'en ce vis-à-vis et entre ces deux
bougies, et que vous avez le visage si bon et si sain avec tout votre
travail que j'en suis surpris.\,» Il goba la repartie, ou en fit si bien
le semblant qu'il n'y a jamais paru ni lors ni depuis, et qu'il ne cessa
point de me parler très souvent et presque en tous ses voyages de
Versailles comme il faisait auparavant, et avec la même ouverture,
quoique je ne recherchasse rien moins. Il me répliqua qu'il avait
soixante-quatorze ans, qu'en effet il se portait très bien, qu'il était
accoutumé de toute sa vie à une vie dure et de travail\,; et de là
reprit où je l'avais interrompu.

Nous le fumes peu après, et réduits au silence, et à n'oser même remuer,
par la compagnie que nous entendîmes entrer dans mon cabinet.
Heureusement elle ne s'y arrêta guère, et M\textsuperscript{me} de
Saint-Simon, qui n'ignorait pas mon tête-à-tête, contribua à nous
délivrer.

Plus de deux heures se passèrent de la sorte\,: lui, à payer de
subtilités puériles pour le fond, d'autorité et d'impudence pour
l'acceptation et pour la forme d'accepter\,; moi, à ne plus remuer que
des superficies, dans la parfaite conviction où il venait de me mettre
que les partis les plus désespérés et les plus enragés étaient pris et
bien arrêtés. Nous nous séparâmes sans nous être persuadés\,: lui, me
disant sur ce force gentillesses sur mon esprit, que je n'y étais pas,
que je n'entendais pas la matière, que je ne m'arrêtais qu'à du spécieux
futile, qu'il en était surpris, et qu'il me priait d'y faire bien mes
réflexions\,; moi, de répondre rondement qu'elles étaient toutes faites,
et que ma capacité ne pouvait aller plus loin. Malgré cette franchise il
parut lors et depuis fort content de moi, quoiqu'il n'en pût jamais
tirer autre chose\,; et je n'avais garde aussi de ne me pas montrer fort
content de lui.

Je le fis sortir par la petite porte de derrière mon cabinet, en sorte
que personne ne l'aperçut\,; et dès que je l'eus refermée je me jetai
dans une chaise comme un homme hors d'haleine, et j'y demeurai longtemps
seul dans mon cabinet, à réfléchir sur le prodige de mon extase, et sur
les horreurs qui me l'avaient causée.

Les suites en commencèrent incontinent après par l'assemblée des évêques
à Paris\,; et c'est ce qui appartient à l'histoire particulière de la
constitution, à laquelle je les laisserai pour n'y revenir que lorsque
j'aurai à y parler nécessairement de ce qui en aura passé par mes mains,
ou, d'une manière également curieuse, sous mes yeux ou par mes oreilles.

\hypertarget{chapitre-ii.}{%
\chapter{CHAPITRE II.}\label{chapitre-ii.}}

1713

~

{\textsc{M. de Savoie prend le titre de roi de Sicile.}} {\textsc{- Il
imite le roi sur ses bâtards.}} {\textsc{- Prie, nommé ambassadeur à
Turin, épouse la fille de Plénoeuf, qui devient fatale à la France.}}
{\textsc{- Gouvernement d'Alsace et de Brisach au maréchal d'Uxelles.}}
{\textsc{- Trois cent mille livres à Torcy\,; quatre cent mille livres à
Pontchartrain\,; quatre cent mille livres au duc de La Rochefoucauld.}}
{\textsc{- Lamoignon, greffier, Chauvelin grand trésorier de l'ordre\,;
Voysin et Desmarets en ont le râpé.}} {\textsc{- Chauvelin\,; quel\,; et
son beau-père.}} {\textsc{- Dalon\,; quel.}} {\textsc{- Chassé de sa
place de premier président du parlement de Bordeaux.}} {\textsc{- Prise
de Fribourg par Villars, qui envoie Contade à la cour.}} {\textsc{- Duc
de Fronsac apporte la prise de Brisach\,; le roi lui donne douze mille
livres et un logement à Marly.}} {\textsc{- Kirn rendu à Besons, qui
sépare son armée et revient à Paris.}} {\textsc{- Conférences à Rastadt
entre Villars et le prince Eugène, qui y traitent et y concluent la paix
entre la France, l'empereur et l'empire.}} {\textsc{- Réforme de
troupes.}} {\textsc{- Mort du prince de Toscane.}} {\textsc{- Mort
d'Harleville.}} {\textsc{- Mort du chevalier de Grignan ou comte
d'Adhémar.}} {\textsc{- Mort de Gassion\,; quel il était, et sa
famille.}} {\textsc{- Mort de la princesse de Courtenai, sa famille, que
le roi montre sentir être de son sang.}} {\textsc{- Saintrailles\,;
quel\,; sa mort.}} {\textsc{- Mort et caractère de Phélypeaux.}}
{\textsc{- Mort du duc de Medina-Sidonia.}} {\textsc{- Ronquillo
destitué de la place de gouverneur du conseil de Castille\,; on lui
donne une pension de dix mille écus.}} {\textsc{- Retour du duc
d'Aumont.}} {\textsc{- Le roi de Sicile passe avec la reine en Sicile,
et laisse le prince de Piémont régent avec un conseil.}} {\textsc{-
Peterborough et Jennings saluent le roi.}} {\textsc{- Électeur de
Bavière à Paris\,; voit le roi.}}

~

M. de Savoie, en vertu de la paix d'Utrecht, prit le 22 septembre le
titre de roi de Sicile, et trancha tout aussitôt non seulement du grand
roi, mais il imita leurs tours d'autorité les plus nouveaux. Il avait un
fils et une fille de M\textsuperscript{me} de Vérue\,; il les avait
légitimés\,; ils étaient demeurés jusqu'alors dans cet état simple\,; il
voulut que toute sa cour leur donnât de l'Altesse. Le fils fut tué sans
alliance, la fille était fort aimée de son père\,; il voulut imiter le
roi\,; il la maria au prince de Carignan, fils unique du fameux muet, et
l'héritier présomptif de ses États après ses deux fils. Il fit appeler
l'aîné duc de Savoie, l'autre prince de Piémont. Le roi nomma le marquis
de Prie ambassadeur à Turin, et lui donna quatre mille livres
d'augmentation de pension, mille écus par mois, et dix mille pour son
équipage. Il épousa avant son départ la fille de Plénœuf qui s'était
enrichi aux dépens des vivres et des hôpitaux des armées, et qui était
devenu depuis, pour se mettre à couvert, commis de Voysin.
M\textsuperscript{me} de Prie\footnote{M\textsuperscript{me} de Prie
  était fille d'un riche financier nommé Berthelot de Plénœuf ou
  Pleinœuf. Le marquis d'Argenson confirme, dans ses Mémoires
  (p.~201-202), ce que dit Saint-Simon de M\textsuperscript{me} de
  Prie\,: «\,Je ne crois pas, dit-il, qu'il ait jamais existé créature
  plus céleste. Une figure charmante et plus de grâces encore que de
  beauté\,; un esprit vif et délié, du génie, de l'ambition, de
  l'étourderie, et pourtant une grande présence d'esprit, etc.\,» Le
  Journal inédit du marquis d'Argenson donne, sur M\textsuperscript{me}
  de Prie, des détails qu'il aurait été difficile d'insérer dans les
  Mémoires.} était extraordinairement jolie et bien faite, avec beaucoup
d'esprit et une lecture surprenante. Elle fut à Turin avec son mari\,; à
son retour, elle devint maîtresse publique de M. le Duc, et la Médée de
la France pendant le ministère de ce prince.

Le roi donna le gouvernement d'Alsace et celui de Brisach, vacants par
la mort du duc Mazarin, au maréchal d'Huxelles, qui fut un présent de
près de cent mille livres de rente\,; cent mille écus à Torcy sur les
postes, et quatre cent mille livres à Pontchartrain, pour lui aider à
acheter les terres que la maréchale de Clérembault lui vendit pour après
sa mort\,; et autres quatre cent mille livres à M. de La Rochefoucauld,
qui, sous prétexte de pleurer pour avoir de quoi payer ses dettes voulut
gorger ses valets.

La Vrillière vendit sa charge de greffier de l'ordre à Lamoignon,
président à mortier, avec permission de conserver le cordon bleu\,;
Voysin eut le râpé\footnote{Le sens de ce mot a été expliqué t. II,
  p.~296, note.} de cette charge. Chamillart vendit aussi la sienne de
grand trésorier de l'ordre en conservant le cordon\,; Desmarets en eut
le râpé, et Chauvelin la charge. Il était fort jeune, et seulement
avocat général. Ce fut une chute nouvelle pour ces charges, qui mortifia
fort les ministres bien que décorés de les avoir eues, et les premiers
magistrats. Celui-ci, qui était frère aîné de celui qui longtemps après
fut garde des sceaux, en savait encore plus que lui\,; il avait su
gagner la confiance du roi qui s'en servait pour beaucoup de manèges des
jésuites\,; il avait des audiences longues et fréquentes par les
derrières\,; à peine encore cela s'apercevait-il, et il aurait été à
tout pour peu que le roi et lui eussent vécu davantage. Il était gendre
de Gruchy, qui avait été longtemps intendant de mon père, qui ne l'a
jamais oublié, qui l'a bien et fidèlement servi, qui s'était enrichi
dans les partis sous Pontchartrain, contrôleur général, et qui a vécu
près de cent ans dans une santé parfaite de corps et d'esprit.

Dalon, qui avait succédé à son père, un des meilleurs et des plus
honnêtes magistrats du royaume, et ami de mon père à la place de premier
président de Pau, et qui était homme de beaucoup d'esprit et de
capacité, avait passé à celle de premier président de Bordeaux. Il y fit
tant de folies et de friponneries insignes qu'il eut ordre d'en donner
la démission. Cette punition parut un prodige dans l'impunité que la
magistrature avait acquise avec tant d'autres usurpations de ce règne.
Dalon se cacha de honte les premières années après sa chute. Il reprit
après courage, et demanda longtemps avec impudence une autre place
pareille, ou une de conseiller d'État. Il ne se lassa point de frapper à
toutes les portes. On ne se lassa point non plus de le laisser aboyer.
Enfin, après bien des années, il s'en alla s'enterrer chez lui, où il a
vécu fort abandonné et encore plus méprisé jusqu'à sa mort, arrivée il
n'y a pas bien longtemps.

Le maréchal de Villars fit attaquer, le 14 octobre, la contrescarpe de
Fribourg, à cinq heures du soir. Vivans était lieutenant général de
jour, et s'y distingua fort. L'action fut longue et fort disputée. Il y
eut vingt-cinq capitaines de grenadiers tués, et douze cents hommes plus
tués que blessés\,; on s'établit enfin sur la contrescarpe et sur la
lunette\footnote{Petite fortification de forme triangulaire pratiquée
  dans l'intérieure des demi-lunes.}. Le maréchal de Villars demeura
dans la tranchée jusqu'à onze heures du soir, que le logement fut tout à
fait fini. La demi-lune fut attaquée le dernier octobre. On y trouva peu
de résistance, tout ce qui s'y trouva fut tué ou pris. On se préparait à
donner le lendemain l'assaut au corps de la place, lorsqu'on aperçut sur
le rempart deux drapeaux blancs. Le baron d'Arche, qui commandait dans
la place, avait abandonné la ville, et s'était retiré au château et dans
les forts avec tout ce qu'il avait pu y mettre de troupes. Il avait
laissé dans la ville plus de deux mille blessés où malades, huit cents
soldats sains, pour qui il n'avait pu trouver place dans le château et
dans les forts, et, toutes les femmes, les enfants, et force valets de
la garnison. Villars fit entrer le régiment des gardes dans la ville, ne
permit point à ces bouches inutiles de sortir, quelques cris qu'ils
fissent, fit demander un million aux bourgeois pour se racheter du
pillage, accorda cinq jours de trêve au gouverneur pour envoyer au
prince Eugène lui demander ses ordres, et dépêcha Contade au roi, qui
arriva à Marly le lundi matin 6 novembre. Villars donna encore jusqu'au
15 au baron d'Arche, sans tirer de part ni d'autre, mais le maréchal
faisant travailler à ses batteries, et le gouverneur envoyant la
nourriture à ce qu'il avait laissé dans la ville. Le mardi 21 novembre,
le duc de Fronsac arriva à Marly portant au roi la nouvelle de la
capitulation du château et des forts de Fribourg. Il y avait sept mille
hommes fort entassés, qui sortirent le 17 avec tous les honneurs de la
guerre, qui finit par cet exploit. Asfeld, longtemps depuis maréchal de
France, fut laissé à Fribourg pour y commander, et dans le Brisgau, sous
les ordres des du Bourg, commandant en Alsace. Villars revint à
Strasbourg\,; et le duc de Fronsac eut douze mille livres pour sa
course, et un logement à Marly pour le reste du voyage, et plus
retourner, parce que l'armée s'allait séparer.

Besons, en séparant la sienne, fit sommer Kirn qui se rendit\,; et lui
s'en revint à Paris et saluer le roi.

Il y avait eu des propositions secrètes, pendant les derniers temps du
siège, de la part du prince Eugène au maréchal de Villars, qui disparut
même une fois du siège fort peu accompagné pendant une journée. Contade,
en apportant la nouvelle de la contrescarpe, avait été chargé d'autres
choses sur ces propositions, et de rapporter les ordres du roi. Il y eut
encore depuis force courriers que n'exigeait pas la situation du siège
presque fini. En effet le maréchal de Villars partit le 27 novembre de
Strasbourg, accompagné du prince de Rohan, de Châtillon, Broglio et
Contade, pour arriver, le même jour et en même temps que le prince
Eugène, au château de Rastadt, bâti magnifiquement par le feu prince
Louis de Bade, et que sa veuve prêta pour y tenir entre ces deux
généraux les conférences de la paix entre la France, l'empereur et
l'empire. Ils conservèrent tous deux la plus entière égalité en tout, et
la plus parfaite politesse. Ils eurent chacun une garde de cent hommes.
Les conférences entre eux deux seuls commencèrent incontinent après. Le
prince de Rohan n'y demeura que deux ou trois jours, et s'en revint à
Paris. On trouvera dans les Pièces tout ce qui regarde ces conférences,
le traité qui en résulta et que les deux généraux y signèrent, et ce qui
se passa depuis en conséquence à Bade où le traité définitif fut
signé\,; ce qui me dispensera de m'étendre ici sur ces
matières\footnote{À défaut des pièces auxquelles renvoie Saint-Simon, on
  peut consulter les Mémoires de Torcy, qui a dirigé toutes ces
  négociations comme secrétaire d'État chargé des affaires étrangères.}.

Pendant ces conférences, le roi réforma soixante bataillons et dix-huit
hommes par compagnie du régiment des gardes et cent six escadrons, dont
vingt-sept de dragons. Outre que la paix paraissait sûre avec
l'Allemagne, le roi, en paix avec le reste de l'Europe, n'avait plus
besoin de tant de troupes, quand la guerre eût continué contre
l'empereur et l'empire.

L'année se termina par plusieurs morts. Le grand-duc perdit son fils
aîné, le 30 octobre, à cinquante ans, qui était un prince de grande
espérance, mais dont la santé était perdue il y avait longtemps. Il
avait épousé, en 1688, la soeur de M\textsuperscript{me} la dauphine de
Bavière, et des électeurs de Cologne et de Bavière, dont il n'eut jamais
d'enfants. M\textsuperscript{me} la grande-duchesse, sa mère, qui était
revenue en France depuis longues années, sentit moins cette perte que
toute la Toscane, et que le grand-duc, à qui il ne restait plus
d'héritier {[}que{]} son second fils, séparé de sa femme depuis
plusieurs années, dont il n'avait point d'enfants, et qui s'en était
retournée vivre chez elle en Allemagne. Elle n'avait point eu d'enfants.
Elle et sa sœur, la veuve du célèbre prince Louis de Bade, étaient les
dernières de cette ancienne et grande maison de Saxe-Lauenbourg. Le
deuil du roi fut en noir et de trois semaines.

Harleville mourut assez vieux. Son nom était Brouilly, comme la duchesse
d'Aumont et la marquise de Châtillon, ses issues de germaines, du père
duquel\footnote{Nous avons reproduit exactement le manuscrit de
  Saint-Simon, mais il faudrait lire probablement \emph{du père
  desquelles}.} il avait acheté le gouvernement de Pignerol. Il avait
bien servi, et il était fort honnête homme et considéré. Le roi avait
continué à lui en payer trente-cinq mille livres de rente
d'appointements, dont huit mille demeurèrent sur la tête de sa femme.

Le comte d'Adhémar mourut, à Marseille, sans enfants de
M\textsuperscript{lle} d'Oraison, que sa famille lui avait fait épouser
pour en avoir. Il avait été fort connu sous le nom de chevalier de
Grignan. Il avait été des premiers menins de Monseigneur, homme de
beaucoup d'esprit, de sens, de courage et de lecture, fort dans le grand
monde, et recherché de la meilleure compagnie. La goutte, qui l'affligea
à l'excès et de fort bonne heure, le fit retirer en Provence. Il était
frère du comte de Grignan, chevalier de l'ordre, lieutenant général et
commandant dans cette province. M\textsuperscript{me} de Sévigné en
parle beaucoup dans ses lettres.

Gassion, fort ancien lieutenant général, très distingué, gouverneur
d'Acqs et de Mézières, mourut, à Paris, d'une longue maladie à
soixante-treize ans. Il avait été longtemps lieutenant des gardes du
corps, et en avait quitté le corps pour servir plus librement de
lieutenant général, dans l'espérance de devenir maréchal de France. On
en avait fait plus d'un qui ne le valaient pas, mais on n'en avait
jamais tiré des gardes du corps, et c'est ce qui le pressa d'en sortir.
Le roi en fut secrètement piqué par jalousie pour ses compagnies des
gardes, le traita extérieurement honnêtement, l'employa, mais ce fut
tout. C'était un petit Gascon vif, ambitieux, ardent, qui se sentait
encore plus qu'il ne valait, et qui peu à peu en mourut de chagrin. Il
était propre neveu du célèbre maréchal de Gassion, et cela lui avait
tourné la tête. Gassion, son neveu, a été plus heureux que lui et à
meilleur marché. Le grand-père du maréchal, qui est le premier de ces
Gassion qu'on connaisse distinctement, fut procureur général au conseil
de Navarre, que Jeanne d'Albret, reine de Navarre, avait fait élever. Il
se jeta dans Navarreins assiégé par les Espagnols\,; le gouverneur y fut
tué, il y commanda en sa place, contraignit les Espagnols de se retirer
à Orthez jusqu'où il les poursuivit, les y assiégea et les força de se
rendre. Cette action lui valut la présidence du conseil souverain de
Navarre, et {[}il{]} fut depuis chef du conseil secret du roi de
Navarre. Le fils de celui-là fut procureur général, puis président du
conseil souverain de Navarre, et mourut, avec un brevet de conseiller
d'État, en 1598. Il fut père du maréchal de Gassion, d'un évêque
d'Oléron, et de leur aîné qui fut président à mortier après avoir été
procureur général au parlement de Navarre Il fut aussi intendant de la
généralité de Pau\,; eut, en 1636, de ces brevets de conseiller d'État
comme avait eu son père\,; et obtint, en 1660, l'érection de sa terre de
Camou en marquisat sous le nom de Gassion. Celui-ci est le père de
Gassion des gardes du corps qui a donné lieu à cette petite digression,
et de plusieurs enfants dont l'aîné fut président à mortier au parlement
de Pau, et eut, en 1664, un de ces brevets de conseiller d'État. Entre
plusieurs enfants, il a eu le marquis de Gassion, gendre d'Armenonville,
garde des sceaux, qui est devenu lieutenant général distingué,
gouverneur d'Acqs et de {[}Mézières{]}, chevalier du Saint-Esprit à la
Pentecôte 1743.

Le prince de Courtenai perdit sa femme, qui par son bien le faisait
subsister, et qui lui laissa un fils, et une fille qui épousa le marquis
de Bauffremont, chevalier de la Toison d'or, et depuis lieutenant
général. Le fils avait épousé la sœur de M. de Vertus des bâtards de
Bretagne, veuve de don Gonzalez Carvalho Palatin, grand maître des
bâtiments du roi de Portugal, d'où elle était revenue. Il avait peu
servi, et avait eu un frère aîné tué dans les mousquetaires au siège de
Mons, où son père était à la suite de la cour. Le roi l'alla voir sur
cette perte, ce qui parut très extraordinaire, et un honneur qu'il
voulut faire, lorsqu'il ne le faisait plus à personne depuis bien des
années, qui montra qu'il ne le pouvait ignorer être bien réellement
prince de son sang, mais que les rois ses prédécesseurs ni lui n'avaient
jamais voulu reconnaître. Ce prince de Courtenai était fils d'une
Harlay, n'eut point d'enfants d'une Lamet, sa première femme, et eut
ceux-ci de la seconde, qui était veuve de Le Brun, président au grand
conseil, et fille de Duplessis-Besançon, gouverneur d'Auxonne et
lieutenant-général. J'aurai lieu de parler encore de ce prince de
Courtenai et du fils qui lui resta, et qui a été le dernier de cette
branche infortunée de la maison royale.

Saintrailles mourut, qui était vieux et à M. le Duc dont j'ai eu
occasion de parler lors de la mort de M. le Duc, gendre du roi. C'était
un homme d'honneur et de valeur, le meilleur joueur de trictrac de son
temps, et qui possédait aussi tous les autres {[}jeux{]} sans en faire
métier. Il avait l'air important\,; le propos moral et sententieux,
avare et avait accoutumé à des manières impertinentes tous les princes
du sang et leurs amis particuliers qui étaient devenus les siens. Il
n'était ni Poton ni Saintrailles, mais un très petit gentilhomme et
point marié. Il n'avait qu'une nièce, fort jolie et sage, fille
d'honneur de M\textsuperscript{me} la Duchesse. Lorsqu'elle n'en eut
plus, elle demeura auprès de M\textsuperscript{me} la Princesse. Le
marquis de Lanques, de la maison de Choiseul, en devint si amoureux
qu'il la voulut épouser. Il était capitaine dans Bourbon, fut blessé
pendant la campagne, revint mourant à Paris, se fit porter à
Saint-Sulpice, où il l'épousa\,; et mourut deux jours après.
Saintrailles lui donna tout son bien, avec lequel elle épousa M.
d'Illiers.

On apprit par les lettres de la Martinique que Phélypeaux y était mort.
C'était un homme très extraordinaire, avec infiniment d'esprit, de
lecture, d'éloquence et de grâce naturelle\,; fort bien fait, point
marié, qui n'avait rien, avare quand il pouvait, mais honorable et
ambitieux, qui n'ignorait pas qui il était, mais qui s'échafaudait sur
son mérite et sur le ministère\,; poli, fort l'air du monde et
d'excellente compagnie, mais particulier, avec beaucoup d'humeur, et un
goût exquis en bonne chère, en meubles et en tout. Il était lieutenant
général, fort paresseux et plus propre aux emplois du cabinet qu'à la
guerre. Il avait été auprès de l'électeur de Cologne, puis ambassadeur à
Turin, et fort mal traité à la rupture, dont il donna une relation à son
retour, également exacte, piquante et bien écrite, à l'occasion de quoi
j'ai eu lieu de parler de lui. Il fut conseiller d'État d'épée à son
retour\,; mais, après cet écrit où M. de Savoie était cruellement
traité, et les propos que Phélypeaux ne ménagea pas davantage,
M\textsuperscript{me} la duchesse de Bourgogne lui devint un fâcheux
inconvénient, et M. de Savoie même après la paix. Il n'avait rien\,; et
il n'avait qu'un frère, évêque de Lodève, qui n'avait pas moins d'esprit
et plus de mœurs que lui, chez lequel il alla vivre en Languedoc. Ils
étaient cousins germains de Châteauneuf, secrétaire d'État, père de La
Vrillière, qui avec le chancelier et son fils trouva moyen de l'envoyer
à la Martinique général des îles, qui est un emploi indépendant, de plus
de quarante mille livres de rente, sans le tour du bâton qu'il savait
faire valoir.

La mort du duc de Medina-Sidonia termina l'année. Elle arriva subitement
à Madrid, comme il était prêt à monter dans le carrosse du roi
d'Espagne, dont il était grand écuyer, et chevalier du Saint-Esprit.
C'était un des plus grands seigneurs d'Espagne et des plus accomplis,
fort vieux et fort attaché au roi d'Espagne. J'ai eu occasion d'en
parler sur le testament de Charles II, et l'avènement de Philippe V à la
couronne. Il laissa un fils qui a eu aussi postérité. Il était l'aîné de
cette grande et ancienne maison de Guzman, et le plus ancien duc
d'Espagne. Mais c'est la grandesse qui y fait tout\,; et quoique la
sienne soit des premières, j'ai déjà remarqué que l'ancienneté ne s'y
observe point parmi les grands. J'aurai lieu d'en parler encore à
l'occasion de mon ambassade extraordinaire en Espagne. J'y parlerai
aussi de la grande charge de président du conseil de Castille, et à son
défaut de la place de gouverneur de ce conseil. Ronquillo l'avait, qui
en cette qualité ne donnait pas chez lui la main à M. de Vendôme, malgré
l'étrange Altesse et le genre que M\textsuperscript{me} des Ursins lui
avait fait donner pour en prendre le semblable. Il fut remercié avec une
pension de dix mille écus.

Le duc d'Aumont arriva de son ambassade d'Angleterre, et eut une longue
audience du roi, dans son cabinet. On remarqua qu'il affecta toutes les
manières anglaises jusqu'à nouer sa croix à son cordon bleu, comme les
chevaliers de la Jarretière portent leurs médailles attachées à leur
cordon. Son arrivée ne reçut pas de grands applaudissements. L'argent
qu'il en sut rapporter sut aussi l'en consoler.

Le nouveau roi de Sicile ne tarda pas à aller reconnaître cette île par
lui-même, et ce qu'il en pourrait tirer. Il y mena la reine sa femme,
fit un conseil pour gouverner à Turin en son absence, et offrit à
M\textsuperscript{me} sa mère la qualité de régente. Au peu de part
qu'il lui avait donné toute sa vie aux affaires, depuis qu'il en eut
pris l'administration de ses mains, elle sentit bien le vide d'un titre
offert par la seule bienséance, et s'excusa de l'accepter. Sur son
refus\,; il le donna au prince de Piémont, son fils, jeune prince de la
plus grande espérance, et partit sur les vaisseaux de l'amiral Jennings,
qui le portèrent à Palerme. Il y fut couronné\,; et les Siciliens
n'oublièrent rien par leurs empressements, leurs hommages, leurs fêtes,
pour se mettre bien avec un prince aussi jaloux et aussi clairvoyant. Il
donna cinquante mille livres, avec son portrait enrichi de diamants, à
Jennings pour son passage, et la reine de Sicile une fort belle bague.
Jennings vint après mouiller aux côtes de Provence, et reçut force
honneurs à Toulon. Il vint ensuite à Paris. Le comte de Peterborough,
qui avait tant de fois couru l'Europe, et servi l'archiduc en Espagne
avec tant de fureur, était aussi venu se promener à Paris. C'était un
homme qui, dans un âge fort avancé, et chevalier de la Jarretière, ne
pouvait durer en place. Torcy le présenta au roi à Versailles le lundi 4
décembre, et tout de suite Peterborough présenta Jennings au roi. Ces
amiraux d'escadre ne sont, sous ces grands noms, que ce que sont parmi
nous des chefs d'escadre. Celui-là disait qu'il avait gagné cinq cent
mille écus depuis qu'il servait. Il s'en faut tout que les nôtres
gagnent autant. Il s'en alla incontinent en Angleterre.

L'électeur de Bavière arriva le lundi 18 décembre de Compiègne à Paris,
et vint descendre chez Monasterol son envoyé en cette ville. Il alla le
mercredi 20 à Versailles. Il vit le roi l'après-dînée par les derrières
à l'ordinaire, il fut seul avec lui une demi-heure dans son cabinet, et
retourna après à Paris chez Monasterol, où il vit peu de monde, fort
triste de n'espérer plus le titre de roi de Sardaigne.

\hypertarget{chapitre-iii.}{%
\chapter{CHAPITRE III.}\label{chapitre-iii.}}

1714

~

{\textsc{L'Évangile présenté à baiser au roi par un cardinal, de
préférence à l'aumônier de jour, en absence du grand et du premier
aumônier.}} {\textsc{- Duc d'Uzeda peu compté à Vienne, et son fils
emprisonné au château de Milan.}} {\textsc{- Duc de Nevers dépouillé par
le roi de la nomination à l'évêché de Bethléem.}} {\textsc{- Duc de
Richelieu se brouille avec sa femme et la quitte.}} {\textsc{- Cavoye
prend soin de lui.}} {\textsc{- Force bals à la cour et à Paris.}}
{\textsc{- Bals, jeux, comédies et nuits blanches à Sceaux.}} {\textsc{-
M\textsuperscript{me} la duchesse de Berry, grosse, mange au grand
couvert en robe de chambre.}} {\textsc{- Abbé Servien à Vincennes.}}
{\textsc{- Mort, fortune, famille et caractère du duc de La
Rochefoucauld.}} {\textsc{- Bachelier\,; sa fortune\,; son mérite.}}
{\textsc{- Surprise étrange du duc de Chevreuse et de moi chez le duc de
La Rochefoucauld.}} {\textsc{- Hardie générosité du duc de La
Rochefoucauld.}} {\textsc{- Vieux levain de Liancourt. --Ses deux
fils.}} {\textsc{- Comte de Toulouse grand veneur.}} {\textsc{- Douze
mille livres de pension au nouveau duc de La Rochefoucauld.}} {\textsc{-
Le chancelier voit un homme se tuer.}} {\textsc{- Commencement de la
persécution en faveur de la constitution Unigenitus.}} {\textsc{-
Mariage du prince de Pons et de M\textsuperscript{lle} de Roquelaure.}}
{\textsc{- Gouvernement de Dunkerque à Grancey en épousant la fille de
Médavy, son frère.}} {\textsc{- Vingt-cinq mille livres de rente fort
bizarres au premier président.}} {\textsc{- Mort de Bragelogne.}}
{\textsc{- Ambassadeurs de Hollande saluent le roi.}} {\textsc{- Grande
maladie de la reine d'Angleterre à Saint-Germain.}} {\textsc{- Mort du
duc de Melford à Saint-Germain.}} {\textsc{- Mort de Mahoni.}}
{\textsc{- M. le duc de Berry entre au conseil des finances.}}

~

Le premier jour de cette année 1714, il ne se trouva ni grand ni premier
aumônier à la grand'messe de l'ordre, célébrée par l'abbé d'Estrées. Il
y eut difficulté à qui présenterait au roi l'Évangile à baiser, entre
l'aumônier de jour en quartier et le cardinal de Polignac qui n'avait
point l'ordre, mais qui se trouva au prie-Dieu, et en faveur duquel le
roi décida. Il ne donna aucunes étrennes cette année. Elles ne
regardaient que M\textsuperscript{me} la duchesse de Berry dont il
n'était guère content, et Madame à qui il venait d'augmenter très
considérablement ses pensions. Pour M. le duc de Berry il ne s'en
embarrassa pas\,; il n'y avait guère qu'un an qu'il lui avait augmenté
ses pensions de quatre cent mille livres. Peu de jours après il le fit
entrer au conseil de finances, où il fut quelques conseils sans opiner,
comme il avait été quelques-uns de même en ceux de dépêches lorsqu'il
avait commencé à y entrer. C'était le chemin d'être bientôt admis en
celui d'État. Le roi avait usé des mêmes gradations envers Monseigneur
et Mgr le duc de Bourgogne.

Le duc d'Uzeda, peu considéré de l'empereur, depuis qu'il avait quitté
si vilainement le parti de Philippe V pour s'attacher au sien, eut tout
au commencement de cette année le déplaisir de voir mettre son fils
prisonnier au château de Milan.

Il y a un fantôme d'évêché sous le titre de Bethléem dans le duché de
Nevers, sans territoire, dont la résidence est à Clamecy, qui ne vaut
que cinq cents écus de rente, que les ducs de Nevers avaient toujours
nommé. M. de Nevers l'avait donné au P. Sanlèque, religieux de
Sainte-Geneviève, qui excellait à régenter l'éloquence et les humanités
en leur collège de Nanterre, et qui était aussi bon poète latin, aux
mœurs duquel il n'y avait rien à reprendre. Mais les jésuites, jaloux de
tous collèges et qui n'aimaient pas les chanoines réguliers, ne
s'accommodèrent pas que cette figure d'évêché leur échappât, dont ils
pouvaient défroquer quelque moine, et s'en attacher beaucoup par cet
appât. Le P. Tellier, tirant sur le temps et sur le peu de considération
du collateur, fit entendre au roi qu'il ne convenait pas qu'un
particulier fit sans lui un évêque dans son royaume, acheva ce que les
jésuites avaient commencé avant lui, car il y avait douze ans que
Sanlèque était nommé sans avoir pu obtenir des bulles. Il les fit
accorder au P. Le Bel, récollet, nommé dès lors par le roi qui n'y
pensait plus. Le Bel fut sacré, et Sanlèque n'eut aucune récompense.
Depuis cela cette idée d'évêché est demeurée à la nomination du roi.

Le duc de Richelieu, remarié depuis assez longtemps pour la troisième
fois, et logé chez sa femme au faubourg Saint-Germain, se brouilla avec
elle, et voulut retourner à l'hôtel de Richelieu, à la place Royale,
qu'il avait loué à l'archevêque de Reims qui, faute de savoir où se
mettre, voulait soutenir son bail. Cavoye et sa femme, amis de tout
temps de M. de Richelieu et qui ne venaient presque jamais à Paris,
prêtèrent leur maison à l'archevêque jusqu'à ce qu'il en eût trouvé une
à louer, et se mirent à prendre soin de M. de Richelieu qui avait
quatre-vingt-six ans, et qui en sa vie n'avait su prendre soin de
lui-même. Ce leur fut un mérite auprès de M\textsuperscript{me} de
Maintenon, et par conséquent auprès du roi.

Cet hiver fut fertile en bals à la cour. Il y en eut plusieurs parés et
masqués chez M. le duc de Berry, chez M\textsuperscript{me} la duchesse
de Berry, chez M. le Duc et ailleurs. Il y en eut aussi à Paris, et à
Sceaux où M\textsuperscript{me} du Maine donna force fêtes et nuits
blanches, et joua beaucoup de comédies, où tout le monde allait de Paris
et de la cour, et dont M. du Maine faisait les honneurs.
M\textsuperscript{me} la duchesse de Berry était grosse et n'allait
guère aux bals hors de chez elle. Le roi lui permit, à cause de sa
grossesse, de souper avec lui en robe de chambre, comme en même cas il
l'avait permis aux deux Dauphines seulement.

L'abbé Servien, dont j'ai parlé ailleurs, étant à l'Opéra, ne put tenir
aux louanges du roi du prologue. Il lâcha tout à coup au parterre un mot
sanglant, mais fort juste et fort plaisant, en parodie, qui le saisit,
et qui fut trouvé tel, répété et applaudi. Deux jours après il fut
arrêté et conduit à Vincennes, avec défense de parler à personne, et
sans aucun domestique pour le servir. On mit pour la forme le scellé sur
ses papiers. Il n'était pas homme à en avoir de plus importants que pour
allumer du feu. Il est vrai que, à plus de soixante-cinq ans qu'il avait
alors, il était étrangement débauché.

Le duc de La Rochefoucauld mourut le jeudi 11 janvier\,; à
soixante-dix-neuf ans, aveugle, à Versailles, dans sa belle maison du
Chenil, où il s'était retiré depuis quelques années. Quoique j'aie eu
lieu de parler diverses fois de lui, il a été personnage si singulier et
si distingué toute sa vie qu'il est à propos de s'y arrêter un peu. Il
était fils aîné du second fils de La Rochefoucauld et de la fille unique
d'André de Vivonne, seigneur de La Chateigneraie, grand fauconnier de
France, capitaine des gardes de la reine Marie de Médicis, et de
Marie-Anne de Loménie. Cet André de Vivonne était petit-fils du frère
aîné de François de Vivonne, seigneur d'Ardelay, favori d'Henri II, qui
fut tué en sa présence en combat public et singulier par Guy Chabot,
fils du seigneur de Jarnac, d'où est venu le proverbe du coup de Jarnac,
10 juillet 1547. Marie-Anne de Loménie était fille du sieur de La
Ville-aux-Clercs, secrétaire d'État. M. de La Rochefoucauld porta le
vain titre de prince de Marsillac, sans rang ni distinction quelconque
pendant la vie de son père auquel il fut toujours très attaché, quoique
parfaitement dissemblable. Il le suivit dans le parti de M. le Prince,
et ne rentra qu'avec lui dans l'obéissance. Il épousa en 1659, en
novembre, Jeanne-Charlotte, fille et unique héritière d'Henri-Roger du
Plessis, comte de La Rocheguyon, premier gentilhomme de la chambre du
roi, en survivance de son père, qui fut depuis duc et pair de Liancourt,
et d'Anne-Élisabeth de Lannoy, remariée un an auparavant au duc
d'Elbœuf, père de celui d'aujourd'hui, dont elle fut la première femme,
et dont elle eut M. d'Elbœuf, dit le Trembleur, et M\textsuperscript{me}
de Vaudemont. M. et M\textsuperscript{me} de Marsillac étaient issus de
germains. Le premier duc de La Rochefoucauld, grand-père de M. de
Marsillac, avait épousé Gabrielle du Plessis, fille de M. de Liancourt,
premier écuyer, en faveur duquel cette charge fut soustraite à celle de
grand écuyer, et de la célèbre Antoinette de Pons, marquise de
Guercheville, père et mère du duc de Liancourt, ce qui faisait que le
grand-père et la grand'mère des mariés étaient frère et sœur. L'union
était parfaite entre les deux familles, et ils logeaient tous ensemble à
Paris, rue de Seine, dans ce bel hôtel de Liancourt qui est devenu
l'hôtel de La Rochefoucauld. Il y aurait bien des choses curieuses à
dire de ces deux Liancourt père et fils et de leurs femmes, mais qui
sont trop éloignées de notre temps. M. de Marsillac n'eut que deux fils
de sa femme\,; il la perdit le 14 août 1674. La duchesse de Liancourt sa
grand'mère était morte le 14 juin précédent, à soixante-treize ans, et
le duc de Liancourt le 1er août de la même année, à soixante-quinze ans.
Grand Dieu, quel bonheur de ne survivre que six semaines\,!

Jamais peut-être l'aveuglement qu'on reproche à la fortune ne parut dans
un plus grand jour que dans ce prince de Marsillac, qui rassemblait en
lui toutes les causes de disgrâces, et qui, sans secours d'aucune part,
brilla tout à coup de la plus surprenante faveur, et qui a été
pleinement constante toute sa vie, c'est-à-dire près de cinquante ans,
sans la plus légère interruption. Il était fils d'un père à qui le roi
n'a jamais pu pardonner, le seul peut-être de tous les seigneurs du
parti de M. le Prince, et M. de La Rochefoucauld le sentait si bien
qu'il ne se présentait presque jamais devant le roi. M. et
M\textsuperscript{me} de Liancourt étaient noircis d'un autre crime\,;
le mari ne faisait point sa charge de premier gentilhomme de la chambre
longtemps avant de ne l'avoir plus\,; la femme avait refusé d'être dame
d'honneur de la reine. Ils passaient presque toute leur vie à Liancourt,
dans les exercices de piété les plus édifiants et les plus continuels,
ne paraissaient plus à la cour\,; et comme ils y avaient vécu dans la
plus excellente et la plus brillante compagnie, ils avaient la meilleure
à Liancourt, mais la moins à la mode. Ce lieu était le réduit de tout ce
qui tenait à Port-Royal, et la retraite des persécutés de ce genre.
D'autres proches, M. de Marsillac n'en avait point\,; et ceux-là
n'étaient pas pour le produire ni l'étayer.

La figure, qui prévient souvent, et le roi presque toujours, n'était pas
un don qu'il eût en partage, j'ai ouï dire aux gens de la cour de son
temps que la sienne était tout à fait désagréable. Un homme entre deux
tailles, maigre avec de gros os, un air niais quoique rude, des manières
embarrassées, une chevelure de filasse, et rien qui sortît de là.

Fait de la sorte, et seul de sa bande, il arriva dans la plus brillante
et la plus galante cour, où le comte de Guiche, Vardes, le comte du
Lude, M. de Lauzun et tant d'autres se disputaient la faveur du roi et
le haut du pavé chez la comtesse de Soissons, de chez qui le roi
{[}ne{]} bougeait alors. Ce centre de la cour d'où tout émanait était
encore un lieu où Marsillac, fils de M. de La Rochefoucauld, devait être
de contrebande pour la nièce du cardinal Mazarin\,; aussi fut-il fort
mal reçu d'abord, et n'y fut accueilli de personne. Mais bientôt toute
la troupe choisie, qui s'en moquait, fut bien étonnée de voir le roi le
mettre de ses parties, sans autre chose de sa part que de se présenter
devant le roi, et sans que le roi lui eût montré auparavant aucune
bienveillance. Cela dura ainsi quelque temps, et commença à exciter
l'envie, lorsque la faveur se déclara et ne fit plus que croître.

M. de Lauzun fut arrêté en décembre 1671, à Saint-Germain, dans sa
chambre, un soir qu'il revenait de Paris rapporter des pierreries à
M\textsuperscript{me} de Montespan qui l'en avait chargé. Il était
capitaine des gardes, et fut arrêté par le marquis de Rochefort, depuis
maréchal de France, qui l'était aussi, car un capitaine des gardes ne
peut être arrêté que par un autre capitaine des gardes, et dès le
lendemain {[}il fut{]} mis en route de Pignerol. Il était gouverneur de
Berry, Marsillac en fut pourvu tout aussitôt, et M. de Luxembourg de sa
charge.

Guitry, favori pour qui le roi avait fait la charge de grand maître de
la garde-robe, fut tué au passage du Rhin en 1672. M. de Marsillac, qui
y avait été fort blessé à l'épaule, eut sa charge\,; et à la mort de
Soyecourt en 1679, qui était grand veneur, le roi écrivit à M. de
Marsillac, qui était venu voir son père, ce billet qu'on a rendu si
célèbre, par lequel il lui manda «\,qu'il se réjouissait avec lui, comme
son ami, de la charge de grand veneur qu'il lui donnait comme son
maître.\,»

Avec toute cette faveur, le père, de concert avec lui, eut beau
s'opiniâtrer à ne lui point céder son duché, jamais M. de Marsillac ne
put avoir le rang de prince, ni aucune autre distinction\,; et ses
instances furent aussi vaines depuis la mort de son père, qu'il perdit
au commencement de 1680. Sur la fin de sa vie la faveur et les efforts
de son fils lui avaient attiré quelques paroles du roi\,; on en voit des
traces dans les lettres de M\textsuperscript{me} de Sévigné, mais
toujours rares et peu naturelles.

M. de Marsillac, que je nommerai désormais duc de La Rochefoucauld,
était le seul confident des amours du roi, et le seul qui, le manteau
sur le nez comme lui, le suivait à distance lorsqu'il allait à ses
premiers rendez-vous. Il fut ainsi dans l'intimité de
M\textsuperscript{me} de La Vallière, de M\textsuperscript{me} de
Montespan, de M\textsuperscript{me} de Fontange, de tous leurs
particuliers avec le roi, et de tout ce qui se passait dans le secret de
cet intérieur. Il demeura toute sa vie intimement avec
M\textsuperscript{me} de Montespan, même depuis son éloignement, avec
M\textsuperscript{me} de Thianges, avec ses filles. Il eût aimé d'Antin
sans sa faveur. Aussi ne put-il jamais souffrir M\textsuperscript{me} de
Maintenon, quoi qu'elle et le roi pussent faire. Jamais aussi elle n'osa
l'entamer. Il se tenait dans un respectueux silence, n'en approcha
jamais\,; force révérences s'il la rencontrait par quelque hasard\,; et
payait toujours de monosyllabes et de révérences redoublées tout ce
qu'en ces occasions elle lui disait d'obligeant.

M. de La Rochefoucauld avait beaucoup d'honneur, de valeur, de probité.
Il était noble, bon, libéral, magnifique\,; il était obligeant et touché
du malheur. Il savait et osait plus que personne rompre des glaces, et
souvent forcer le roi. Mais, à force de prodiguer ses services avec peu
de choix et de discernement, il fatigua et lassa enfin le roi, mais ce
ne fut que sur les derniers temps\,; d'ailleurs sans aucun esprit, sans
discernement, glorieux au dernier point, rude et rustre en toutes ses
manières, très volontiers brutal, désagréable en toutes ses façons,
embarrassé avec tout ce qui n'était point ses complaisants, mais comme
un homme qui ne sait pas recevoir une visite\,:, ni entrer ou sortir
d'une chambre\,; surtout désespéré si une femme lui parlait en le
rencontrant. Hors M. de Bouillon et les maréchaux de Duras et de Lorges,
il n'allait chez qui que ce fût, excepté un instant pour des compliments
indispensables de mort, de mariage, etc., et encore tout le moins qu'il
pouvait. Il vivait chez lui avec un tel empire qu'il n'y voyait personne
aussi qu'à ces mêmes occasions, il n'y avait que des gens désoeuvrés qui
n'étaient guère, et la plupart point, reçus ailleurs, qu'on appelait les
ennuyeux de M. de La Rochefoucauld, et ses valets, qui étaient ses
maîtres, qui s'y mêlaient de la conversation, et pour lesquels il
fallait avoir toutes sortes d'égards et de complaisances, si on avait
envie de fréquenter la maison.

Il avait plusieurs gentilshommes tant à lui que de la vénerie\,; mais,
en cela très homogène à son maître, ils étaient peu comptés, et ses
valets l'étaient pour tout, jusque-là que ses enfants étaient réduits à
leur faire la cour, et n'obtenaient rien de lui que par Bachelier, qui
de son laquais était par sa protection devenu premier valet de
garde-robe, et qui, contre l'ordinaire de ces gens-là, ne s'était jamais
méconnu avec personne, quoique M. de La Rochefoucauld n'eût rien oublié
pour le gâter. C'était un des meilleurs et des plus honnêtes hommes que
j'aie vus dans ces étages-là, et le plus digne de sa fortune\,; toujours
faisant du bien tant qu'il pouvait, jamais de mal, infiniment
respectueux avec tout le monde, nullement intéressé\,; qui vivait avec
les valets de M. de La Rochefoucauld comme avec ses camarades, avec ses
enfants comme avec ses maîtres, toujours occupé de leur plaire et de
leur être utile, honteux du besoin qu'ils avaient de lui, faisant sans
eux mille choses pour eux, et, avec l'ascendant sans mesure qu'il avait
naturellement, et sans aucun soin de sa part sur M. de La Rochefoucauld,
toujours attentif à ne s'en servir que pour le bien, la paix, l'union,
l'avantage de sa famille, et pour l'honneur et la gloire de son maître,
sans jamais montrer au dehors tout ce qu'il pouvait sur lui.

Du reste M. de La Rochefoucauld ne regarda jamais sa belle-fille que
comme la fille de l'homme du monde qu'il haïssait le plus, ni son fils
que comme le gendre de Louvois. Il en avait si bien pris l'habitude que
la mort de ce ministre n'y changea rien. M. de Liancourt n'était pas
mieux traité de lui. Sa disgrâce du roi lui tourna toute sa vie à crime
auprès de son père. Ses sœurs, il ne faisait cas que de l'aînée, qui en
effet avait beaucoup d'esprit et de mérite, mais ce cas n'allait à rien.
Des autres et de son frère l'abbé de Verteuil, il n'en faisait aucun, et
le leur montrait sans cesse aussi bien qu'à ses fils, l'abbé de
Marsillac et le chevalier, morts depuis longtemps\,; il ne les aimait
pas davantage, mais il les comptait plus, parce que le monde les
comptait, et qu'ils se faisaient compter. Ils ressemblaient assez en
esprit à leur père. Il n'y avait donc que l'abbé de La Rochefoucauld que
M. de La Rochefoucauld aimât. Quoique son oncle paternel, ils étaient de
même âge, et il en avait tiré secours en jeunesse en ses besoins. En
tout temps, il fut panier percé, incapable de tout soin domestique et de
toute affaire, et toute sa vie livré à des valets qui, en vrais valets,
en abusèrent sans cesse, et s'enrichirent tous à ses dépens, et
quelques-uns de son crédit.

Je n'oublierai jamais ce qui nous arriva à la mort du fils unique du
prince de Vaudemont, par la mort duquel tous les biens de la première
femme du duc d'Elbœuf, père de celui-ci, revinrent aux enfants de M. de
La Rochefoucauld, fils de sa fille du premier lit. On était à Marly, et
le roi avait couru le cerf. M. de Chevreuse, que je trouvai au débotté
du roi, me proposa d'aller avec lui chez M. de La Rochefoucauld sur ce
compliment à lui faire, et nous nous amusâmes dans le salon pour le
laisser retourner et être quelque temps chez lui. En y entrant quelle
fut notre surprise, j'ajouterai notre honte, de trouver M. de La
Rochefoucauld seul dans sa chambre jouant aux échecs avec un de ses
laquais en livrée assis vis-à-vis de lui\,! La parole en manqua à M. de
Chevreuse et à moi qui le suivais. M. de La Rochefoucauld s'en aperçut
et demeura confondu lui-même. Il ne lui en fallait pas tant pour
recevoir la visite de M. de Chevreuse, qu'il ne voyait jamais qu'aux
occasions. Il balbutia, il s'empêtra, il essaya des excuses de ce que
nous voyions, il dit que ce laquais jouait très bien, et qu'aux échecs
on jouait avec tout le monde. M. de Chevreuse n'était pas venu pour le
contredire, moi encore moins. On glissa, on s'assit, on se releva
bientôt pour ne pas troubler la partie, et nous nous en allâmes au plus
tôt. Dès que nous fûmes dehors, nous nous dîmes, M. de Chevreuse et moi,
ce que nous pensions d'une rencontre si rare, mais nous ne voulûmes
point la publier.

M. de La Rochefoucauld ne fut donc regretté que de ses valets, qui le
déshonorèrent par l'empire qu'ils exercèrent dans tous les temps sur
lui, et par cette ridicule et sèche retraite du Chenil où ils le
tenaient écarté de sa famille et des honnêtes gens, mais à portée
d'aller importuner le roi pour eux. Ses \emph{ennuyeux} le regrettèrent
aussi, mais beaucoup moins depuis sa retraite. Jamais la cour ne l'avait
aimé, parce qu'il n'avait jamais vécu avec elle. Son goût et son
assiduité prodigieuse à toutes les heures de son service et des
promenades du roi l'en avait toujours entièrement séquestré, et cette
assiduité introduisit celle de tous les grands officiers, qui se
piquèrent à qui mieux mieux de l'imiter.

Le roi, qui ne s'en pouvait passer, mais à qui sur les fins il était
devenu à charge, qui se trouvait soulagé de sa retraite, mais qui était
fort importuné de sorties fréquentes qu'il en faisait sur lui pour ses
valets, et en dernier lieu pour sa famille, se trouva fort soulagé de sa
mort. Tels ont été ses sentiments à la mort de presque tous ceux qu'il a
aimés et comblés de faveurs et de grâces.

On a toujours cru que le peu d'esprit de M. de La Rochefoucauld avait
fait sa fortune. Le roi commençait lors à sentir la supériorité d'esprit
de la plupart de cet élixir de cour qui vivait sans cesse avec lui chez
la comtesse de Soissons. Le rogue, le dur, le désagréable de M. de La
Rochefoucauld n'était pas pour le roi\,; son court lui plut et le mit à
l'aise. Avec ce défaut il avait celui d'envier tout jusqu'à un prieuré
de cinq cents livres, et avec tant de charges et de grâces de toutes les
sortes pour lui et pour les siens, avec ses dettes payées trois ou
quatre fois par le roi, avec des présents d'argent gros et fréquents, il
trouvait tout le monde bien traité, hors lui.

Il ne s'était point consolé que le mariage de la fille de Louvois avec
son fils, que le roi avait exigé de lui pour raccommoder ces deux hommes
fort ennemis et qu'il voyait sans cesse, ne lui eût pu faire obtenir le
rang de prince étranger, à quoi son père et lui, comme on l'a vu
ailleurs (t. X, p. 290), tendirent toute leur vie, et que tout se fût
borné à cet égard au duché Ve de La Rocheguyon pour son fils, comme M.
de Luynes avait eu celui de Chevreuse pour le sien en épousant la fille
de Colbert.

Cette envie générale était bien plus forte à l'égard de ceux de sa sorte
qui paraissaient en faveur. M. de Chevreuse, M. de Beauvilliers, M. le
Grand surtout étaient ses bêtes. Il haïssait les ministres, et eux le
craignaient et le ménageaient. Quoiqu'il n'eût presque point de commerce
avec la maison de Condé et de Conti, il s'était conservé une tradition
d'estime et d'amitié qui se marquait en toute occasion, et qui était
fort entretenue par ses enfants, trop intimes du prince de Conti, comme
on l'a vu, et qui le sont demeurés jusqu'à sa mort.

Pour achever ce qui regarde un favori si singulier, il faut à son
honneur se souvenir du trait, rapporté t. II. p.~98, qu'il fit à
Portland, que, jusqu'à M. le Prince, tout ce qu'il y avait de plus
considérable s'empressait à festoyer et à courtiser.

J'ai été témoin d'un autre bien plus fort pour un courtisan tel qu'il
l'était. Ce fut pendant un voyage de Marly, dans les jardins où le roi
s'amusait à une fontaine qu'il faisait faire. Je ne me souviens plus sur
quoi le roi se mit en propos, lui qui fut toujours si réservé. Mais ce
jour-là il parla de Montgaillard, évêque de Saint-Pons, avec chaleur,
qui était alors en disgrâce profonde, et dans laquelle il est mort, à
l'occasion des affaires de Port-Royal et de celles de la
régale\footnote{Droit qu'avaient les rois de France de jouir des fruits
  et revenus des évêchés et archevêchés pendant la vacance des sièges,
  et de conférer les bénéfices qui en dépendaient.}. M. de La
Rochefoucauld laissa dire le roi, mais, dès qu'il eut cessé de parler,
il se mit sur les louanges de l'évêque. Le silence peu approbatif du roi
l'échauffa. Il poussa sa pointe, et il raconta que, visitant son
diocèse, il enfila un chemin qui alla toujours en étrécissant, et qui
aboutit à la fin à un précipice. Nul moyen d'en sortir qu'en retournant,
et aucun espace pour tourner ni pour pouvoir mettre pied à terre. Le
saint évêque, car ce fut son terme que je remarquai bien, leva les yeux
au ciel, rendit toute la bride, et s'abandonna à la Providence. Aussitôt
sa mule se dressa sur ses pieds de derrière, et, ainsi dressée, se
tourna doucement, lui toujours dessus, et ne remit les pieds de devant à
terre que lorsqu'elle se trouva la tête où elle avait la queue. Tout
aussitôt elle se remit à marcher par où elle était venue jusqu'à ce
qu'elle eut trouvé à rentrer dans le bon chemin. Tout ce qui était
autour du roi imita son silence, qui excita encore le duc à commenter ce
qu'il venait de raconter. Cette générosité me charma, et surprit tous
ceux qui en furent témoins.

Il avait toujours conservé de cet ancien levain de Liancourt un penchant
pour tout ce qu'il y avait vu et entendu, et du commerce et de la
liaison avec plusieurs de ceux qui avaient survécu à M. et à
M\textsuperscript{me} de Liancourt, jusque-là que quelques-uns de ces
saints persécutés passèrent de longues années dans Liancourt, de son
temps, et y sont morts. Il avait un tel respect pour M. et
M\textsuperscript{me} de Liancourt, qui fit ce beau lieu pour amuser M.
de Liancourt dans cette retraite, qu'il ne voulut jamais souffrir qu'on
y changeât rien de ce qu'ils y avaient fait, quoique bien des choses
eussent vieilli et eussent été bien mieux autrement\,; et c'était un
plaisir que de l'entendre parler d'eux avec l'affection et la vénération
qu'il conserva toujours pour eux.

Ses deux fils, malvoulus du roi, prirent différentes routes\,; aussi,
nonobstant leur intime et inaltérable union, chose également rare et
respectable entre deux frères, rien en tout de plus différent l'un de
l'autre\,: l'aîné, rogue, avare à l'excès, sans esprit que silence,
ricanerie, malignité qui lui avait fait donner le nom de Monseigneur le
Diable, force gloire et bassesse tout à la fois, et un long usage du
monde en supplément d'esprit, fit la charge de grand maître de la
garde-robe servilement, sans nul agrément, en valet assidu et enragé de
l'être. Son nom sonore à trois syllabes, car il prit celui de son père
qui, après avoir retenti dans les partis, s'était fait craindre dans les
cabinets, lui donna un reste de considération qui ne passa guère un
certain étage, et qui ne trouva en soi nul appui. Sans table, sans
équipage, mais de grands biens, une cour de caillettes de Paris les
soirs chez sa femme, avec un souper et des tables de jeu, et grande
bassesse avec la robe qui leur fit gagner force procès. Son frère, doux,
liant, poli, orné de beaucoup de simplicité, de lecture et d'esprit,
plein d'honneur, de courage, de sentiment, de bonne gloire, était, à
force de disgrâces, devenu solitaire et sauvage, et fut, ce qui est fort
rare, également estimé, honoré et peu compté.

Pour achever cette matière, le nouveau duc de La Rochefoucauld, qui
avait la goutte, se fit porter, peu de jours après la mort de son père,
dans le cabinet du roi, qui lui dit merveilles sur son père, et pas un
mot des cinquante mille livres que le roi lui donnait tous les ans de sa
cassette pour augmentation à sa charge de grand veneur, et que
l'équipage fût plus magnifique. Ce silence, soutenu pendant près de deux
mois, parmi les divers comptes que M. de La Rochefoucauld cherchait à
rendre au roi des chasses et de l'équipage, et la situation personnelle
en laquelle il se sentait auprès de lui, le persuadèrent qu'il n'avait
point de continuation à espérer, et par conséquent de se défaire d'une
charge fatigante, qu'il trouvait trop pesante sans ce supplément, et qui
ne le privait de rien avec l'autre qu'il conservait. Il en fit donner
envie par M\textsuperscript{me} la Duchesse à M. le comte de Toulouse,
qui l'acheta cinq cent mille livres comptant, dont il y en avait deux
cent trente mille livres en brevet de retenue pour les créanciers. Comme
survivancier, M. de La Rochefoucauld avait neuf mille livres de pension,
qui s'éteignait par le titre de la charge. Le roi, en faveur du marche,
lui donna douze mille livres de pension personnelle, et M. le comte de
Toulouse joignit sa meute à celle du roi, et augmenta fort l'équipage.

Le lendemain de la mort de M. de La Rochefoucauld, le chancelier essuya
une scène bien tragique. Un vice-bailli d'Alençon venait de perdre un
procès apparemment fort intéressant pour son honneur ou pour son bien.
Il vint à Pontchartrain, où était le chancelier, et l'attendit dans sa
cour, qui allait monter en carrosse. Là il lui demanda la révision de
son procès et un rapporteur. Le chancelier, avec douceur et bonté, lui
représenta que les voies de cassation étaient ouvertes de droit quand il
y avait lieu, mais que de révision on n'en connaissoit point l'usage, et
se mit à monter dans son carrosse. Pendant qu'il y montait, ce
malheureux dit qu'il y avait un moyen plus court pour sortir d'embarras,
et se donna en même temps deux coups de poignard. Aux cris des
domestiques le chancelier descendit de carrosse, le fit porter dans une
chambre, et envoya chercher un chirurgien qu'il avait, et un confesseur.
Cet homme se confessa assez tranquillement, et mourut une heure après.

Nous voici parvenus à l'époque des premiers coups d'État en faveur de la
constitution, et de la persécution qui a fait tant de milliers de
confesseurs et quelques martyrs, dépeuplé les écoles et les places,
introduit l'ignorance, le fanatisme et le dérèglement, couronné les
vices, mis toutes les communautés dans la dernière confusion, le
désordre partout, établi la plus arbitraire et la plus barbare
inquisition\,; et toutes ces horreurs n'ont fait que redoubler sans
cesse depuis trente ans. Je me contente de ce mot, et je n'en noircirai
pas ces Mémoires. Outre ce qu'on en voit tous les jours, bien des plumes
s'en sont occupées et s'en occuperont. Ce n'est pas là l'apostolat de
Jésus-Christ, mais c'est celui des révérends pères et de leurs ambitieux
clients.

Roquelaure arriva de Languedoc, où on l'avait envoyé commander après son
aventure des lignes, et d'où il n'était pas sorti depuis huit ans. Sa
femme, qui lui avait valu cet emploi, avait fait le mariage de sa
seconde fille avec le prince de Pons, fils aîné du feu comte de Marsan,
à qui, en haine de l'aînée, ils donnèrent tout ce qu'ils purent et qui
alla à un million, dont la moitié après eux et sans renoncer. Roquelaure
était très mal dans ses affaires, et son père aussi quand il se maria
sans quoi que ce soit en dot que son brevet de duc. De ce rien
M\textsuperscript{me} de Roquelaure trouva moyen, à force de procès, de
crédit, d'affairés et d'industrie, de parvenir à faire une des plus
riches maisons du royaume. La noce se fit à Paris chez Roquelaure avec
fort peu d'apparat.

Médavy, n'ayant qu'une fille, la voulut marier à son frère, et obtint
pour cela de faire passer sur sa tête son gouvernement de Dunkerque en
s'en réservant les appointements. C'est ainsi qu'on escobardait les
survivances depuis que le roi n'en voulait plus donner que des charges
de secrétaire d'État.

Le roi fit en ce même temps une grâce au premier président, sans
exemple, et qui ne se pouvait imaginer à demander que par un panier
percé de la dernière impudence, et aussi fortement appuyé qu'il l'était.
Il avait un brevet de retenue de cinq cent mille livres. Il osa proposer
que le roi lui en payât les intérêts, et il l'obtint tout de suite.
C'était une vraie pension de vingt-cinq mille livres qu'il eût été moins
énorme de lui donner à cru. M. du Maine avait ses raisons de le prendre
par son faible quoique déjà tout à lui, et le roi et
M\textsuperscript{me} de Maintenon les leurs de lui en donner tous les
moyens. Le scandale ne laissa pas d'être grand.

Bragelogne, qui avait été capitaine au régiment des gardes et major
général de l'armée d'Allemagne, mais qui ne servait plus par mauvaise
santé, tomba mort chez Le Rebours, à Paris, le jour de la Chandeleur,
jouant à l'hombre.

Buys et Goslinga, ambassadeurs d'Hollande, arrivèrent à Paris\,: le
premier pour y demeurer comme ambassadeur ordinaire, l'autre pour s'en
retourner au bout de quelques mois de la commission d'ambassadeurs
extraordinaires. Ils saluèrent le roi, quelques jours après, dans son
cabinet en particulier. Buys, qui portait la parole, fit un beau
discours. On a pu voir dans les Pièces quel était son caractère, son
animosité contre la France, et tout ce qu'il fit pour empêcher la paix.
Son ambassade le changea entièrement, et le séjour qu'il fit en France
le rendit tout français. Cette singularité m'a paru mériter d'être
remarquée.

La reine d'Angleterre tomba fort malade à Saint-Germain, et reçut tous
les sacrements. Les médecins la condamnaient, et elle en était
contente\,; la vie n'avait rien qui pût l'attacher depuis bien des
années, et elle faisait le plus saint usage de ses malheurs. Le roi lui
rendit de grands soins pendant cette maladie, et M\textsuperscript{me}
de Maintenon aussi.

Le duc de Melford mourut à Saint-Germain. Il avait la Jarretière, avait
été secrétaire d'État d'Écosse, et était frère du duc de Perth, aussi
chevalier de la Jarretière. Il avait essuyé des soupçons et des exils.
On a vu que le feu roi Jacques avait cru en mourant qu'ils avaient été
mal fondés, et qu'en réparation il l'avait fait duc. Tout le monde à
Saint-Germain et à Versailles n'en fut pas aussi persuadé que ce prince.

Mahoni, Irlandais, lieutenant général, qui avait beaucoup d'esprit,
d'honneur et de talents, et qui s'était fort distingué à la guerre,
surtout à la journée de Crémone, dont il apporta la nouvelle au roi,
mourut en Espagne, où il s'était attaché et où il avait acquis des
biens. Il avait épousé la sœur de la duchesse de Berwick, veuve et mère
des comtes de Clare\,; et le duc de Berwick vivait avec lui avec
beaucoup d'estime et d'amitié. Il laissa des enfants qui sont aussi
devenus officiers généraux avec distinction.

Le 3 février M. le duc de Berry entra, pour la première fois, au conseil
des finances. Le roi voulut qu'il assistât à plusieurs avant que d'y
opiner, comme il avait fait lorsqu'il fut admis en celui de dépêches, et
il se pressait pour le faire entrer au conseil d'État\footnote{Saint-Simon
  appelle ici \emph{conseil d'État} ce qu'il appelle ailleurs
  \emph{conseil d'en haut\,;} c'était le conseil qui s'occupait des
  affaires politiques et où Louis XIV n'admettait qu'un petit nombre de
  personnes.}.

\hypertarget{chapitre-iv.}{%
\chapter{CHAPITRE IV.}\label{chapitre-iv.}}

1714

~

{\textsc{Helvétius en Espagne pour la reine à l'extrémité.}} {\textsc{-
Orry et son fils.}} {\textsc{- La reine d'Espagne, pour ses derniers
sacrements, congédie son confesseur jésuite et prend un dominicain.}}
{\textsc{- Sa mort.}} {\textsc{- Retraite du roi d'Espagne chez le duc
de Medina-Celi.}} {\textsc{- Deuil de la reine d'Espagne.}} {\textsc{-
Conférences de Rastadt barbouillées.}} {\textsc{- Contade à la cour.}}
{\textsc{- {[}Conférences{]} renouées.}} {\textsc{- Malhabileté de
Villars.}} {\textsc{- La paix signée à Rastadt.}} {\textsc{- Contade en
apporte la nouvelle.}} {\textsc{- Mort, caractère, maison, famille du
duc de Foix.}} {\textsc{- Mort de M\textsuperscript{me} de Miossens\,;
son caractère.}} {\textsc{- Bâtards d'Albret expliqués.}} {\textsc{-
Maréchal d'Albret\,; sa fortune.}} {\textsc{- Mort et dépouille de
Montpéroux.}} {\textsc{- Mort du Charmel.}} {\textsc{- Dureté du roi.}}
{\textsc{- Mort et caractère de la maréchale de La Ferté et de sa soeur
la comtesse d'Olonne.}} {\textsc{- Le roi donne au prince Charles douze
mille livres de rentes en fonds\,; voit en particulier l'électeur de
Bavière\,; donne les grandes entrées au maréchal de Villars, et à son
fils la survivance de son gouvernement de Provence.}} {\textsc{-
Villars, du Luc et Saint-Contest, ambassadeurs plénipotentiaires à
Bade.}} {\textsc{- Époque de la première prétention des conseillers
d'État de ne céder qu'aux gens titrés.}} {\textsc{- Six mille livres de
pension à Saint-Contest.}} {\textsc{- Villars, chevalier de le Toison
d'or, fait donner trois mille livres de pension au comte de Choiseul,
son beau-frère.}} {\textsc{- Abbé de Gamaches auditeur de rote\,; son
caractère.}} {\textsc{- Maréchal de Chamilly fait donner à son neveu son
commandement de la Rochelle, etc.}}

~

La reine d'Espagne, depuis longtemps violemment attaquée d'écrouelles
autour du visage et de la gorge, se trouvait à l'extrémité. Ne tirant
aucun secours des médecins, elle voulut avoir Helvétius, et pria le roi
par un courrier exprès de le lui envoyer. Helvétius, fort incommodé, et
sachant d'ailleurs l'état de la princesse, n'y voulait point aller, mais
le roi le lui commanda absolument. Il partit aussitôt dans une chaise de
poste, suivi d'une autre en cas que la sienne vînt à rompre, et dans
cette autre était le fils d'Orry. Il eût fallu être bon prophète alors
pour dire que nous le verrions contrôleur général ici, très absolu, très
longtemps, et ministre d'État, dont la France se serait aussi utilement
passée que l'Espagne de son père, qui eut en ce même temps un bel
appartement dans le palais, et dont la faveur et l'administration
mécontentait de plus en plus les Espagnols.

Helvétius arriva à Madrid le 11 février. Dès qu'il eut vu la reine, il
dit qu'il n'y avait qu'un miracle qui pût la sauver. Elle avait un
confesseur jésuite. Elle fit comme M\textsuperscript{me} la Dauphine sa
sœur\,: lorsqu'il fut question des derniers sacrements et de penser tout
de bon à la mort, elle le remercia et prit un dominicain. Le roi
d'Espagne ne cessa que le 9 de coucher dans le lit de la reine. Elle
mourut le mercredi 14 avec beaucoup de courage, de connaissance et de
piété.

Le roi sortit aussitôt après du palais, et alla se mettre à l'autre bout
de la ville de Madrid, dans une des plus belles maisons, où logeait le
duc de Medina-Celi, assez près du Buen-Retiro, où les princes d'Espagne
furent conduits bientôt après. Ce choix au lieu du Retiro parut
bizarre\,; il n'est pas encore temps d'en parler.

La désolation fut générale en Espagne, où cette reine était
universellement adorée. Point de famille dans tous les états où elle ne
fût pleurée, et personne en Espagne qui s'en soit consolé depuis.
J'aurai lieu d'en parler à l'occasion de mon ambassade. Le roi d'Espagne
en fut extrêmement touché, mais un peu à la royale. On l'obligea à
chasser et à aller tirer pour prendre l'air. Il se trouva en une de ces
promenades lors du transport du corps de la reine à l'Escurial, et à
portée du convoi. Il le regarda, le suivit des yeux, et continua sa
chasse. Ces princes sont-ils faits comme les autres humains

Le roi regretta fort la reine d'Espagne. Il en prit le deuil en violet
pour six semaines. M. le duc de Berry drapa. M\textsuperscript{me} de
Saint-Simon ne voulait point draper. Elle disait avec raison que,
n'étant point séparée comme les duchesses de Ventadour et de Brancas
l'étaient de leurs maris, les équipages étaient à moi qui ne drapais
point. Cela fut contesté quelques jours, mais M. {[}le duc{]} et
M\textsuperscript{me} la duchesse de Berry le prirent à l'honneur, et en
prièrent M\textsuperscript{me} de Saint-Simon si instamment, qu'il
fallut céder à la complaisance, tellement que nous fûmes mi-partis dans
notre maison, avec des carrosses et une livrée moitié noir et moitié
ordinaire.

Les conférences continuaient à Rastadt. Villars s'y embarbouilla si mal
à propos qu'il fallut le désavouer, c'est-à-dire lui ordonner de courir
après ce qu'il avait lâché, et, comme que ce fût, de raccommoder la
sottise qu'il avait faite. Le chancelier, que j'en vis en grand dépit,
me le conta sur-le-champ, et trouvait Villars un bien malhabile homme
dans toutes ses conférences, et longtemps depuis que je fus en commerce
intime avec Torcy, il ne m'en parla pas mieux, non seulement sur
Rastadt, mais sur toutes les négociations dont Villars s'est mêlé, et
c'est ce qui est bien visible par les Pièces ici jointes. Ce retour de
Villars à ce qu'il avait lâché, et que je n'explique point non plus que
cette négociation de paix avec l'empereur et l'empire, parce qu'elle se
trouve dans les Pièces\footnote{Tous les passages où Saint-Simon parle
  des Pièces annexées à ses Mémoires ont été supprimés dans les
  précédentes éditions.}, ce retour, dis-je, surprit fort le prince
Eugène qui avait compté sur ce que Villars avait lâché. Cela forma entre
eux une contestation toujours polie, mais au fond si forte que le prince
Eugène fit semblant de rompre, pour forcer la main au maréchal, qui à la
fin ne put éviter de convenir d'envoyer au roi, et de se séparer en
attendant ses ordres. Il se retira à Strasbourg le même jour que le
prince Eugène à Stuttgard, et que Contade fut dépêché au roi. Torcy,
chez qui il descendit, le mena au roi chez M\textsuperscript{me} de
Maintenon, où Contade demeura plus d'une heure. C'était le samedi 10
février. Contade repartit le jeudi suivant, 15. À son retour les deux
généraux se rassemblèrent à Rastadt, et y continuèrent leurs
conférences. Elles finirent le mardi matin 6 mars, par la signature de
la paix. Les deux généraux convinrent de se rassembler à Bade en Suisse,
promptement après l'échange des ratifications, pour y ajuster plusieurs
détails, et quelques intérêts de prince de l'empire, qui n'avaient pas
paru assez importants pour arrêter la paix. Contade en apporta la
nouvelle.

Le duc de Foix mourut à Paris à soixante-treize ans, sans enfants, sans
charge, sans gouvernement. Il était chevalier de l'ordre et le dernier
de sa maison. Avec lui son duché-pairie fut éteint. C'était un fort
petit homme, de fort petite mine, qui, avec de la noblesse dans ses
manières, de l'honneur dans sa conduite, de la valeur dans le peu qu'il
avait servi, et un esprit médiocre, n'avait jamais été de rien, ni
figuré nulle part\,; mais il s'était fait aimer partout par l'agrément
et la douceur de sa société. Il ne s'était jamais soucié que de s'amuser
et de se divertir. Il avait trouvé la duchesse de Foix de même humeur,
et on disait d'eux avec raison qu'ils n'avaient jamais eu que dix-huit
ans, et étaient demeurés à cet âge, mais toujours dans la meilleure
compagnie, et peu à la cour où il était peu considéré\,; il finit la
plus heureuse maison du monde, mais en qui le bonheur ne se fixa pas.

Elle était de Bresse, du nom de Greilly, et par corruption Grailly. Le
hasard d'une alliance redoublée de la maison des comtes de Foix lui
porta, contre toute apparence, le comté de Foix et tous les États de
cette puissante maison. Un autre hasard aussi peu apparent la rendit
héritière du royaume de Navarre. Un troisième hasard aussi bizarre lui
enleva le tout presque aussitôt pour le faire passer dans la maison
d'Albret, et de là bientôt après dans la maison de Bourbon par la mère
d'Henri IV. Celle d'Anne, duchesse héritière de Bretagne et deux fois
reine de France, était Greilly-Foix\,; et le fameux Gaston de Foix, duc
de Nemours, qui gagna la bataille de Ravenne où il fût tué, et sa soeur
germaine, seconde femme du roi d'Aragon Ferdinand le Catholique, étaient
aussi Greilly-Foix, et enfants d'une soeur de notre roi Louis XII. Si
c'en était le lieu j'en pourrais rapporter d'autres grandeurs. M. de
Foix avait aussi les siennes dans sa branche, quoiqu'il ne vînt pas de
celles-là. Cependant, avec toute la faveur constante de la marquise de
Senecey et de la comtesse de Fleix sa fille, mère du duc de Foix, il ne
fut pas mention de rang de prince pour une maison si distinguée, dans un
temps où la reine mère était régente, où elle pouvait tout, où elle se
piquait de reconnaissance, d'amitié et de toute sorte de considération
pour M\textsuperscript{me} de Senecey qui avait été chassée pour elle
étant sa dame d'honneur, qu'elle rappela et remit dans sa charge dès
qu'elle fut la maîtresse, et en donna la survivance à sa fille\,; dans
un temps où les Bouillon y parvinrent à force de félonies et
d'épouvanter le cardinal Mazarin\,; dans un temps où les menées et la
faveur de la duchesse de Chevreuse et de M\textsuperscript{me}s de
Montbazon et de Guéméné en eurent quelques prémices et s'en frayèrent le
chemin pour les Rohan\,; qu'auraient fait ces gentilshommes princisés
s'ils avaient eu comme les Greilly des États étendus, et des royaumes
dans leur maison, et surtout les Bouillon, des alliances pareilles\,?

M\textsuperscript{me} de Senecey n'avait d'enfants que la comtesse de
Fleix, veuve comme elle, et celle-ci que deux garçons. Ces dames
cependant n'eurent qu'un tabouret de grâce avec la pointe de celui des
Rohan. Le bruit qu'en fit la noblesse, plus sage et plus instruite de
ses intérêts dans la minorité de Louis XIV qu'elle ne se l'est montrée
en celle de Louis XV, les fit ôter\footnote{Voy., sur les discordes
  relatives à ces tabourets, t. II. p.~153, 154, note.}. Les troubles
passés, ils furent rendus, c'est-à-dire à la seule princesse de Guéméné
pour les Rohan, qui seule l'avait attrape, je dis attrapé comme on l'a
vu (t. II, p.~153, 154), et aux deux dames d'honneur mère et fille,
lesquelles enfin furent comprises dans cette étrange fournée de ducs et
pairs de la fin de 1663\footnote{Voy., à la fin du Ier volume, p.~449,
  la note relative à cette fournée de ducs et pairs.}.

Randan fut érigé en leur faveur à toutes deux, et en celle du fils aîné
de la comtesse de Fleix\,; et le cadet, qui est celui dont il s'agit
ici, fût appelé dans les lettres. L'aîné parut à peine dans le monde et
mourut très promptement, sans enfants de la fille unique du duc de
Chaulnes frère aîné de l'ambassadeur, et de la fille aînée du premier
maréchal de Villeroy, qui se remaria si étrangement à ce M. d'Hauterive
dont on a parlé, et qui fut toujours connue depuis sous le nom de
M\textsuperscript{me} d'Hauterive de Chaulnes. M. de Foix, de la mort
duquel on vient de parler, devint ainsi duc et pair de fort bonne
heure\,; il ne prétendit jamais à princerie, mais il était bon à
entendre et à voir sur ces rangs étrangers, quoique d'ailleurs simple et
modeste. Il fut généralement et beaucoup regretté, et mérita de l'être.

M\textsuperscript{me} de Miossens mourut en même temps à
soixante-dix-huit ans, dans un beau logement complet des basses cours de
Luxembourg que le roi lui avait donné, et que M\textsuperscript{me} de
Caylus eut après elle. M\textsuperscript{me} de Miossens était aussi
bonne femme que sa sœur cadette, M\textsuperscript{me} d'Heudicourt,
était méchante. Elle avait fort peu de bien et paraissait très rarement
à la cour. C'était une femme très maigre, d'une taille qui effrayait par
sa hauteur extraordinaire, avec des yeux vifs, un visage allumé, de
longues dents blanches qui paraissaient fort\,; elle ressemblait à une
sorcière. Elle vivait très retirée et dans la piété. Elle n'avait point
eu d'enfants de son mari, tué en duel en 1672 par Saint-Léger-Corbon\,;
et ce mari était frère cadet du maréchal d'Albret, dont le frère aîné
fut premier mari de la duchesse de Richelieu, dame d'honneur de la
reine, puis par confiance de la dauphine de Bavière à son mariage.
L'occasion est trop naturelle d'expliquer une fois pour toutes ces
bâtards d'Albret pour la manquer, d'autant que la fortune si étrangement
prodigieuse dont M\textsuperscript{me} de Maintenon trouva la source
chez le maréchal d'Albret, et celles que les connaissances qu'elle fit
dans cette maison ont faites, doivent exciter la curiosité sur le
maréchal d'Albret.

Gilles d'Albret était cinquième fils de Charles II sire d'Albret, comte
de Dreux, vicomte de Tartas, fils aîné du connétable d'Albret Charles
Ier tué à la bataille d'Azincourt, 25 octobre 1415, gagnée par les
Anglais, si funeste à la France. Les frères de Gilles d'Albret
étaient\,: Jean d'Albret vicomte de Tartas, grand-père de Jean sire
d'Albret, qui devint roi de Navarre, comte de Foix, etc., par son
mariage avec Catherine de Greilly, dite de Foix, héritière de tous ces
États, et dont la petite-fille Jeanne d'Albret fut héritière, et les
porta dans la maison de Bourbon en épousant Antoine de Bourbon duc de
Vendôme, dont elle eut notre roi Henri IV. Les autres frères de Gilles
furent le cardinal d'Albret, le seigneur d'Orval dont la branche finit à
son fils qui n'eut que des filles, et le seigneur de Sainte-Bazeille qui
ne laissa point d'enfants, et eut la tête coupée à Poitiers, 7 avril
1473, pour avoir trahi Pierre de Bourbon sire de Beaujeu, et l'avoir
livré au comte d'Armagnac. Mais, si de bons auteurs mettent notre Gilles
pour le dernier fils de Charles II d'Albret avec le titre de seigneur de
Castelmoron, d'autres aussi bons lui contestent cette naissance, et le
font bâtard de Jean d'Albret grand-père de celui-ci qui par son mariage
fut roi de Navarre, comte de Foix, etc.

Quoi qu'il en soit, ce Gilles d'Albret, bâtard ou légitime, ne fut point
marié\,; et de Jeannette Le Sellier eut un bâtard nommé Étienne, qui est
la souche des Miossens, dont il s'agit ici. Cet Étienne fut sénéchal de
Foix, premier chambellan de Jean d'Albret, roi de Navarre et comte de
Foix, par son mariage avec Catherine susdite, et obtint quelques terres
de ce prince. Il fut aussi le premier des ambassadeurs de cette reine
Catherine pour son traité de confédération avec Louis XII en 1512\,; et
il eut de ce prince, en 1527, des lettres de légitimation, où il est
traité de cousin, et son père nommé fils puîné de Charles II d'Albret.
Étienne porta le nom de seigneur de Miossens depuis son mariage avec
Françoise, fille et héritière de Pierre, baron de Miossens, qu'il épousa
en 1510, dont il eut un fils unique, qui fut Jean dit d'Albret, baron de
Miossens et de Coaraze. Il fut lieutenant général d'Henri d'Albret, roi
de Navarre comte de Foix, etc., en ses pays et États\,; il épousa
Suzanne, dite de Bourbon, fille du seigneur de Busset, bâtard de Liège,
laquelle fut gouvernante de notre roi Henri IV. Ils eurent un fils et
une fille qui épousa un Cochefilet. Le fils fut Henri dit d'Albret,
baron de Miossens, etc., qui fut, en 1595, chevalier du Saint-Esprit,
gouverneur et sénéchal de Navarre et Béarn. Il épousa Antoinette de
Pons, sœur d'autre Antoinette de Pons, qui fut la célèbre marquise de
Guiercheville, dame d'honneur de la reine Maria de Médicis, femme de M
de Liancourt et mère du duc de Liancourt. De ce mariage une fille qui
épousa, en 1609, Jean de Grossolles, baron de Flamarens, et deux fils,
dont le cadet fut d'église et peu connu.

L'aîné, Henri dit d'Albret, baron de Miossens et, par sa mère, comte de
Marennes, épousa Anne de Pardaillan soeur du père de Montespan, mari de
la trop célèbre M\textsuperscript{me} de Montespan.

De ce mariage trois fils, en qui finit cette bâtardise, et six filles
dont l'aîné épousa, en 1637, Renée Gruel, seigneur de La Frette, comte
de Jonsac en Saintonge\footnote{Jonsac est en Saintonge, tandis que
  Lonsac, que portent les précédentes éditions, est en Angoumois.},
frère du père de MM. de La Frette, si connus par leur célèbre duel\,;
deux autres mariées et trois abbesses.

Les trois fils furent Fr.~Alexandre dit d'Albret, comte de Marennes,
mort, en 1648, premier mari d'Anne Poussart, depuis remariée au duc de
Richelieu, et dame d'honneur de la reine, etc. Il mourut de bonne heure,
ne figura point, et laissa un fils qui porta hardiment le nom de marquis
d'Albret et les armes pleines sans nulle brisure, moins encore de
marques de bâtardise, comme avaient fait ses pères depuis l'extinction
de la maison d'Albret. M\textsuperscript{me} de Richelieu, sa mère, le
maria fort jeune à la fille unique du maréchal d'Albret, son beau-frère
et oncle paternel de son fils. Elle était franche héritière,
c'est-à-dire riche, laide et maussade. Le marquis d'Albret, jeune,
galant, bien fait, étourdi, et qui se croyait du sang des rois de
Navarre, n'en fit pas grand cas, et se fit tuer malheureusement pour une
galanterie, à la première fleur de son âge. Sa veuve demeura sans
enfants avec sa belle-mère, qui la fit faire dame du palais de la reine,
aux premières que le roi lui donna. Le comte de Marsan, jeune, avide et
gueux, qui avait accoutumé de vivre d'industrie, et qui avait ruiné la
maréchale d'Aumont, fit si bien sa cour à la marquise d'Albret, qui
n'avait pas accoutumé d'être courtisée, qu'elle l'épousa en lui donnant
tout son bien par le contrat de mariage, sans que la duchesse de
Richelieu en sût rien que lorsqu'il fallut s'épouser. Elle en fut la
dupe. M. de Marsan la laissa dans un coin de sa maison, avec le dernier
mépris et dans la dernière indigence, tandis qu'il se réjouissait de son
bien. Elle mourut dans ce malheur sans enfants.

Le maréchal d'Albret fut le second des trois frères\,; il porta le nom
de Miossens. C'était un homme d'esprit, de main, de tête et plus encore
d'intrigue et d'industrie, qui se dévoua au cardinal Mazarin, mais qui
sut s'en faire compter, et monter rapidement à la tête des gens d'armes
de la garde, que le comte de Coligny commandait, mais qui paraissait
peu. Lorsque le cardinal eut tout arrangé pour arrêter M. le Prince, M.
le prince de Conti et M. de Longueville dans l'appartement de la
reine-mère, l'après-midi du 18 janvier 1650, au Palais-Royal à Paris, il
confia leur conduite du Palais-Royal à Vincennes à Miossens, et à un
détachement qu'il choisit des gens d'armes de la garde. Le carrosse où
étaient les illustres prisonniers rompit hors de Paris. Il fallut le
raccommoder, et ce fut là où M. le Prince s'écria\,: «\,Ah\,! Miossens,
si tu voulais\,!» en offrant monts et merveilles. Mais Miossens en
savait trop pour prendre le change. Il avait fait son marché, et à force
d'exagérer la délicatesse et le danger de cette conduite, il avait tiré
parole d'un bâton de maréchal de France. Moins d'une année après, il
succéda à Coligny. Le cardinal crut l'amuser en lui donnant la compagnie
des gens d'armes, et se délivrer de la sommation fréquente qu'il lui
faisait de sa parole. Miossens prit toujours la charge, mais, au bout de
fort peu de temps, il se remit aux trousses du cardinal, et avec la
force qu'il tirait de plus de cette compagnie dont il était alors
capitaine, il lui fit si grande peur qu'il en arracha le bâton, à la
promotion qu'on fit le 15 février 1653. Ainsi il ne l'attendit pas
longtemps. Il avait lors trente-neuf ans, et avait très peu servi,
jamais nulle part en chef, et depuis ne vit plus de guerre\,; mais il
sut se donner et se continuer toute sa vie une grande considération, et
obtenir le gouvernement de Guyenne.

Il avait épousé en 1645 la fille cadette de Guénégaud, trésorier de
l'épargne, sœur du secrétaire d'État, dont il fut veuf d'assez bonne
heure, et n'en eut qu'une fille dont on vient d'expliquer la vie.
L'hôtel d'Albret fut toujours à Paris le rendez-vous de la meilleure et
de la plus illustre compagnie, et devint le berceau de la fortune de
M\textsuperscript{me} de Maintenon, et par elle des amis qu'elle y avait
faits. M\textsuperscript{me} d'Heudicourt s'en sentit des premières. Sa
sœur aînée, M\textsuperscript{me} de Miossens, n'en ramassa que peu de
miettes. Son mari fut le troisième frère et le dernier, dont on a déjà
vu la fin. Le maréchal d'Albret alla mourir à Bordeaux le 3 septembre
1676 à soixante-huit ans et fut fort regretté. M\textsuperscript{me} de
Miossens et M\textsuperscript{me} d'Heudicourt étaient Pons, ainsi que
la grande-mère du maréchal d'Albret, qui avec raison se faisait grand
honneur de cette alliance. M\textsuperscript{lle}s de Pons, par là ses
parentes, ne bougeaient de chez lui. Elles n'avaient pas de chausses\,;
il les aidait, et trouvait la cadette fort à son gré par sa beauté et
par son esprit, et la maria pour rien à Heudicourt qu'il en embâta pour
l'honneur de l'alliance, et il décrassa ce Sublet par la charge de grand
louvetier, que Saint-Hérem lui vendit lorsqu'il eut le gouvernement et
la capitainerie de Fontainebleau. L'agrément que le maréchal d'Albret en
obtint à Heudicourt fut en faveur de ce mariage.

Montpéroux, lieutenant-général et mestre de camp général de la
cavalerie, mourut assez jeune. Il dormait partout depuis longtemps, et
debout et en mangeant. C'était un brave homme, assez officier, sans
aucun esprit. Il ne laissa point d'enfants. La Vallière, commissaire
général, monta à sa charge, et vendit la sienne au comte de Châtillon,
gendre de Voysin.

On a vu en son temps l'exil du Charmel et ses causes, dont son
opiniâtreté à ne vouloir point voir le roi, et le dépit du roi contre
les gens retirés qui ne le voyaient point, fut, comme je l'ai raconté
alors, la cause foncière de sa disgrâce. Cette pique du roi à son égard
ne se passa point, et dégénéra en une dureté étrange, pour en parler
sobrement. Le Charmel, attaqué de la pierre, fit demander la permission
de venir se faire tailler à Paris. La permission fut impitoyablement
refusée. Le mal pressait\,; il fallut faire l'opération au Charmel. Elle
fut si rude et peut-être si mal faite, qu'il en mourut trois jours
après, dans les plus grands sentiments de piété et de pénitence. Il est
bien rare de la pousser aussi loin et de la soutenir aussi longtemps
avec la même ferveur et la même exactitude qu'il fit la sienne, parmi
une infinité de bonnes œuvres et toutes celles qu'il put pratiquer. Il
n'avait presque point d'étude, et il n'avait d'esprit que ce que lui en
avait donné l'usage du grand monde. La piété avait suppléé à tout. Je
n'en dirai pas davantage, en ayant assez parlé ailleurs. Il avait
soixante-huit ans, et il avait passé autant d'années dans la retraite
qu'il en avait vécu dans le grand monde. Il avait toujours été persuadé
que cela lui arriverait, et il me l'avait dit plusieurs fois. M. de
Beauvau-Craon, mari de la dame d'honneur de M\textsuperscript{me} la
duchesse de Lorraine, à qui M. de Lorraine a fait et procuré une si
incroyable fortune, est fils de la sœur du Charmel.

La maréchale de La Ferté mourut à Paris en ce même temps, à plus de
quatre-vingts ans. Elle était mère du feu duc de La Ferté et du P. de La
Ferté jésuite, et soeur de la comtesse d'Olonne qui était son aînée et
fort riche sans enfants, et elle fort pauvre. M\textsuperscript{me}
d'Olonne était veuve d'un cadet de la maison de La Trémoille qui tint
toute sa vie chez lui tripot de jeu et de débauche. Les deux sœurs
étaient d'Angennes, d'une branche cadette éteinte en elles. Leur beauté
et le débordement de leur vie fit grand bruit. Aucune femme, même des
plus décriées pour la galanterie, n'osait les voir ni paraître nulle
part avec elles. On en était là alors. La mode a bien changé depuis.
Quand elles furent vieilles et que personne n'en voulut plus, elles
tâchèrent de devenir dévotes. Elles logeaient ensemble, et un mercredi
des Cendres elles s'en allèrent au sermon. Ce sermon, qui fut sur le
jeûne et sur la nécessité de faire pénitence, les effraya. «\,Ma soeur,
se dirent-elles au retour, mais c'est tout de bon, il n'y a point de
raillerie, il faut faire pénitence, ou nous sommes perdues. Mais, ma
soeur, que ferons-nous\,?» Après y avoir bien pensé\,: «\,Ma sœur, dit
M\textsuperscript{me} d'Olonne, voici ce qu'il faut faire, faisons
jeûner nos gens.\,» Elle était fort avare\,; et avec tout son esprit,
car elle en avait beaucoup, elle crut avoir très bien rencontré. À la
fin pourtant elle se mit tout de bon dans la piété et la pénitence, et
mourut trois mois après sa soeur la maréchale de La Ferté. Quelque
impétueux que fut le maréchal son mari, il fut sa dupe toute sa vie ou
le voulut bien paraître. On n'oubliera jamais que ce fut d'elle que se
fit la planche de légitimer un bâtard sans nommer la mère, comme je l'ai
raconté ailleurs, pour, sur cet exemple, légitimer ceux du roi sans
nommer M\textsuperscript{me} de Montespan.

Le roi donna douze mille livres de rente en fonds d'un droit de péage en
Normandie au prince Charles, fils et survivancier de M. le Grand\,; et
il vit une demi-heure seul dans son cabinet l'électeur de Bavière, qui y
était monté par les derrières. Il demeurait en une maison de
Saint-Cloud, où il était venu de Compiègne.

Le maréchal de Villars arrivant de Rastadt le salua le 15 mars dans son
cabinet à Versailles, au retour de courre le cerf à Marly. Le roi
l'embrassa, le loua fort, lui donna pour son fils la survivance de son
gouvernement de Provence, et à lui les entrées des premiers
gentilshommes de la chambre, dont il prit possession le soir même au
coucher. Ces grâces si singulièrement grandes surprirent fort la cour,
et, envie à part, ne l'édifièrent pas.

En même temps le roi le nomma son premier ambassadeur plénipotentiaire
pour aller à Bade, le comte du Luc pour le second, qui se trouvait tout
porté, étant ambassadeur en Suisse\,; et pour troisième La Houssaye,
conseiller d'État et intendant d'Alsace, qui se trouvait aussi tout
porté à Strasbourg. La surprise fut extrême du refus de La Houssaye qui
ne pouvait, disait-il, céder au comte du Luc, qui n'était pas conseiller
d'État\,; et le scandale plus grand encore de ce que le roi ne fit qu'en
rire et s'en moquer tout haut, et nomma Saint-Contest, maître des
requêtes, intendant à Metz, qui en eut six mille livres de pension.
Outre que le comte du Luc était par sa naissance un seigneur, et qu'il
était actuellement ambassadeur, on n'avait jamais ouï parler encore
qu'en magistrat eût osé prétendre aucune compétence avec un homme de
qualité, ou passant pour tel. C'est donc ici l'époque où cela fut
imaginé pour la première fois, et passé toute de suite. On cria\,; les
gens de robe eux-mêmes en furent honteux, mais il n'en fut autre chose.
Ainsi la robe ose tout, usurpe tout et domine tout. Les premiers
magistrats prétendent ne plus céder qu'aux ducs et aux officiers de la
couronne. C'est encore une grande modestie dont il leur faut être très
obligé.

Peu de jours après, le maréchal de Villars qui voulait tout atteindre,
et qui, sans avoir jamais servi l'Espagne, en avait obtenu la Toison,
reçut le collier de cet ordre à Versailles, dans l'appartement de M. le
duc de Berry, des mains de ce prince, en présence de tous ceux qui
avaient cet ordre en France, et qui s'y trouvèrent en collier. Le
maréchal fit presque en même temps donner mille écus de pension au comte
de Choiseul son beau-frère.

L'abbé de Gamaches fut nommé auditeur de rote\footnote{Voy., sur le
  tribunal de la rote, t. II, p.~383, note.} en la place du cardinal de
Polignac. C'était un garçon d'esprit, de savoir, encore plus d'ambition,
et qui compta bien se faire cardinal. Mais pour le devenir quand on est
François, il faut d'autres degrés que celui de la rote, et force
ressorts dont cet abbé se flattait bien aussi de ne pas manquer. Il y
fit bien tout ce qu'il put, mais il mourut en la peine, après avoir
frisé la corde plus d'une fois d'être rappelé et disgracié.

Le maréchal de Chamilly qui, à soixante-dix-huit ans, était sans
enfants, et à qui le commandement de la Rochelle et des pays voisins ne
pouvait plus être bon à rien, obtint du roi de le faire passer au comte
de Chamilly, ancien lieutenant général et fils de son frère, qui avait
été ambassadeur en Danemark.

\hypertarget{chapitre-v.}{%
\chapter{CHAPITRE V.}\label{chapitre-v.}}

1714

~

{\textsc{Le roi tête à tête avec le chancelier, qui lui rapporte le
procès d'entre M. de La Rochefoucauld et moi, m'adjuge toute
préséance.}} {\textsc{- Mort de Saint-Chamant.}} {\textsc{- Tessé
demandé par l'Espagne pour le siège de Barcelone.}} {\textsc{- Berwick
choisi et Ducasse pour y mener une escadre.}} {\textsc{- Souveraineté
manquée de la princesse des Ursins.}} {\textsc{- Palais qu'elle se
prépare près d'Amboise, et ce qu'il devient.}} {\textsc{- Décadence de
la princesse des Ursins dans l'esprit du roi et de M\textsuperscript{me}
de Maintenon.}} {\textsc{- Princesse des Ursins gouvernante des
infants.}} {\textsc{- Ses mesures pour se glisser en la place de la feue
reine.}} {\textsc{- Générosité de Robinet, jésuite, confesseur du roi
d'Espagne.}} {\textsc{- Princesse des Ursins se hâte de faire le mariage
du roi d'Espagne avec la princesse de Parme\,; ses raisons.}} {\textsc{-
Situation du marquis de Brancas en Espagne.}} {\textsc{- Raisons qui le
déterminent à demander d'aller passer quinze jours à Versailles\,; il
l'obtient.}} {\textsc{- Alarme de la princesse des Ursins.}} {\textsc{-
Elle dépêche brusquement le cardinal del Giudice en France.}} {\textsc{-
Brancas court après et le devance.}} {\textsc{- Quel était Giudice.}}
{\textsc{- Brancas à Marly.}} {\textsc{- Giudice après lui avec son
neveu Cellamare.}} {\textsc{- Caractère del Giudice.}} {\textsc{- Mort
et caractère de la chancelière de Pontchartrain.}} {\textsc{- Mort de la
reine douairière de Danemark.}} {\textsc{- Mort et caractère de l'évêque
de Senlis.}} {\textsc{- Chamillart obtient un logement à Versailles.}}
{\textsc{- Mort et caractère de M\textsuperscript{me} Voysin.}}
{\textsc{- Caractère de M\textsuperscript{me} Desmarets.}} {\textsc{-
Mort de Zurbeck.}} {\textsc{- Mort du président Le Bailleul, dont le
fils obtient la charge.}} {\textsc{- Leur caractère.}}

~

J'ai eu trop souvent occasion de parler ici de la question de préséance
qui était entre M. de La Rochefoucauld et moi, et des diverses choses
qui s'y sont passées, principalement lors de ma réception au parlement,
et à l'occasion de l'édit de 1711. Il suffira donc de se rappeler ici
que M. de La Rochefoucauld ayant obtenu à force de cris que la question
serait revue et jugée de nouveau, comme si elle ne se le trouvait pas
dans cet édit de 1711, et enregistré, le roi s'en était réservé à lui
seul le jugement, sans qui que ce soit avec lui que le chancelier seul
pour rapporter l'affaire, à qui les parties sans autre formalité
donneraient leurs mémoires signés d'eux-mêmes, et en recevraient la
communication par lui. On a vu aussi ce qui s'était passé entre eux en
conséquence. L'adresse de l'un était de piquer le roi de jalousie sur
son autorité à l'égard du parlement\,; et celle de l'autre de bien
expliquer que ce qui regardait le parlement dans l'enregistrement des
lettres, et dans la réception des impétrants, était une forme
nécessaire, mais émanée du roi même, et qui par conséquent n'intéressait
en rien son autorité.

Je fis seul mes mémoires. Je les rendis les plus courts qu'il me fut
possible. Je tâchai de n'y rien omettre de ce qui servait à une
instruction parfaite, et de guérir le roi sur les soupçons qu'on
essayait de lui jeter, et qui m'avaient, comme on l'a vu, mis une fois
au moment de perdre ma cause.

Enfin tous les mémoires étant remis de part et d'autre au chancelier, et
n'y ayant plus rien de part et d'autre à répondre ni à ajouter, le
chancelier prit l'ordre du roi pour le jugement.

Le dimanche de la Passion, 18 mars, le roi tint conseil d'État après sa
messe, dîna au petit couvert, entendit le sermon, remonta chez lui, où
il trouva le chancelier, comme il le lui avait ordonné, pour lui
rapporter l'affaire. Elle dura bien deux heures.

Je m'étais présenté devant le roi au retour du sermon, sans lui rien
dire. Le hasard fit que, passant au bas du grand escalier pour monter
par le petit qui donnait dans la première antichambre, je vis le
chancelier qui descendait. Je m'arrêtai pour l'attendre et lui demander
à quoi j'en étais. Il eut la malice de faire avec moi le chancelier pour
la première fois de sa vie. Il me dit avec une gravité austère\,:
«\,Monsieur, je ne puis parler.\,» Je fus assez simple pour en demeurer
interdit. Je le laissai passer, et quelques instants après je le suivis.
J'entrai dans son cabinet comme il changeait de robe. «\,Eh bien\,!
monsieur, lui dis-je, au moins sommes-nous jugés\,?» La malignité le
possédait encore. De ce même ton, du bas du degré\,: «\,Oh\,! pour cela,
oui, monsieur, me répondit-il, pour jugés, vous l'êtes, et vous l'êtes
entièrement sur tout\,;» et fixant des yeux tristes et sévères sur moi,
«\,et jugés sans retour.\,» L'air, le ton, les paroles si différentes
pour moi de ce qu'il avait accoutumé, me glacèrent. Je savais qu'il
était pour moi\,; il eut l'art de me persuader qu'il avait été tondu,
que le roi avait prononcé contre moi malgré lui, et que c'était le
chagrin d'être tondu qui le rendait tel que je le trouvais. Je me tus
dans la plus mortelle angoisse tandis que les valets de chambre
achevaient de sortir. Dès que la porte fut fermée\,: «\,De grâce,
monsieur, lui dis-je, suis-je mort\,? apprenez-moi mon sort.\,» Il se
prit à rire, m'embrassa, et me dit que j'avais gagné en plein, en tout
et partout.

Il est difficile d'ôter en un instant à quelqu'un une meule plus
pesante. Je l'embrassai encore, et le baisai comme on baise une
maîtresse, en lui reprochant sa méchanceté qui m'avait pensé faire
mourir. Il m'avoua qu'il avait voulu se divertir un moment, et se payer
par là de toute la peine que je lui avais donnée. On peut juger que je
lui pardonnai. À mon tour j'avouerai que je sentis une grande joie et un
grand soulagement.

J'allai aussitôt tirer M\textsuperscript{me} de Saint-Simon de peine, et
de là attendre le roi à la sortie de son cabinet comme il allait passer
chez M\textsuperscript{me} de Maintenon. Dès qu'on m'y vit, chacun
comprit que j'avais gagné, mais on était curieux si j'avais emporté la
cour avec le parlement, dont on n'avait pas douté, et M. de La
Rochefoucauld si peu lui même, qu'il n'est rien qu'il n'eût tenté pour
m'engager jusque dans les fins de nous accommoder de la sorte, ce que
j'avais toujours constamment refusé. J'essuyai donc presque autant de
questions que de compliments, mais je fus froid et modeste, et je me
contentai de répondre court que j'étais content, et, quand on l'est
autant que je l'étais, cela est aisé à faire.

Comme le roi sortit, je lui fis ma révérence et mon remerciement.
«\,Monsieur, me dit le roi, vous avez tout gagné, et je suis bien aise
de vous avoir fait plaisir en faisant justice.\,» Comme je ne m'étais ni
expliqué ni ouvert à pas une des questions qu'on m'avait faites, les
oreilles avaient été très attentives à la réponse du roi qui courut
aussitôt de bouche en bouche, et nouveaux compliments. Je ne cachai plus
que j'avais pleinement gagné, mais j'eus grand soin de continuer à être
modeste, et de me dérober au monde qui se réjouissait avec moi,
peut-être avec chagrin, sûrement, au moins pour la plupart, sans y
prendre la moindre part que celle de la curiosité de m'examiner.

M. de La Rochefoucauld fut outré et tout ce qui tenait à lui. Quoiqu'il
ne pût ignorer sa situation personnelle avec le roi, la faveur de son
père l'avait accoutumé à ne douter de rien de ce qui était affaire. Il
n'avait rien oublié sur celle-ci, jusqu'aux artifices les plus propres à
entraîner le roi par l'intérêt d'une autorité qui était son idole, et il
s'en était tout promis, au moins qu'à la cour la préséance lui
demeurerait. Il alla donc chez le chancelier fort peu après que j'en fus
sorti, qui me conta le lendemain qu'il en avait essuyé d'étranges
lamentations.

Deux jours après j'eus mon arrêt. Plus j'étais content, plus je voulus
outrer les procédés honnêtes. J'allai à Paris, et je pris mon temps
d'aller à l'hôtel de La Rochefoucauld, que je m'étais assuré de n'y
trouver personne. Je leur fis dire que j'y étais allé pour le prier de
ne pas trouver mauvais que je leur fisse signifier l'arrêt.
M\textsuperscript{me} de La Rochefoucauld surtout était enragée\,; ils
auraient voulu au moins pouvoir crier sur les procédés L'arrêt fut
signifié, puis enregistré au parlement et la contestation finie. Le
commerce très fréquent et très libre l'était devenu beaucoup moins entre
les deux beaux-frères et moi depuis la mort de la duchesse de Villeroy.
La reprise de cette dispute le rendit encore plus froid et plus rare, et
cette fin l'éteignit tout à fait\,; on en demeura aux simples
bienséances des rares occasions. J'avais mon compte, je m'en consolai.
On verra dans la suite que cette aigreur secrète les conduisit fort mal.

Saint-Chamant mourut à la campagne où il s'était retiré depuis
longtemps. Il avait été lieutenant des gardes du corps. Il commanda le
détachement de la maison du roi qui conduisit la reine d'Espagne, fille
de Monsieur, à la frontière. La reine allongea ce voyage tant qu'elle
put. Saint-Chamant était fort bien fait\,; il avait de l'esprit, encore
plus d'audace\,; la reine peu d'expérience, de ménagement, de
contrainte. Tout cela fit un grand bruit à la cour et retentit fort en
Espagne, qui y fit grand tort à la reine, et qui perdit Saint-Chamant
ici.

M. de Berwick fut nommé pour aller faire au roi d'Espagne les
compliments de condoléance\,; il s'agissait du siège de Barcelone, et de
soumettre les Catalans qui tenaient bon malgré la paix, et qui sous main
étaient secourus. M\textsuperscript{me} des Ursins s'était trop bien
trouvée du flexible et courtisan Tessé pour vouloir un autre général, et
le faisait demander par le roi d'Espagne. Tessé, qui n'avait plus rien à
gagner en ce pays-là, ne se souciait point d'être chargé d'une si forte
expédition. Le roi et M\textsuperscript{me} de Maintenon, par des
raisons qu'il sera bientôt temps de développer, préférèrent le duc de
Berwick à tout autre, qui, outre sa capacité, sa bonne volonté et son
expérience d'Espagne, était depuis longtemps fort mal avec Orry pour
l'avoir traité souvent comme il le méritait, et par conséquent fort peu
au gré de M\textsuperscript{me} des Ursins, qui le trouvait droit,
ferme, libre, barre de fer, toutes qualités qu'elle n'aimait pas à
rencontrer, surtout dans un général d'armée. Le roi donna quinze
bataillons au duc de Berwick\,; et Ducasse fut chargé du commandement de
l'escadre, qui porta tous les besoins du siège, que sa maladie et,
après, les vents contraires retardèrent assez.

Il faut maintenant voir dans les Pièces ce qui se passa sur la
souveraineté que la princesse des Ursins voulut obtenir par le traité de
paix, qui en fut si longtemps et si scandaleusement arrêté par le roi
d'Espagne. Elle y avait tellement compté, et de l'échanger après avec le
roi pour la Touraine et le pays d'Amboise, et y venir jouir de cette
nouvelle grandeur, qu'elle avait chargé son fidèle Aubigny de lui
acheter un terrain près d'Amboise, situé à souhait, d'y bâtir un vaste
palais, avec des basses cours et des communs pour une cour, de le
meubler avec magnificence, de n'y épargner ni dorures ni peintures, de
l'accompagner des plus beaux jardins et de ne s'y soucier d'aucun fief
ni d'aucune seigneurie, parce que la souveraine du pays n'en avait pas
besoin. Aubigny méprisé à Utrecht où il était allé négocier cette
souveraineté, et où il n'avait jamais pu passer les antichambres, relevé
par Bournonville, comme on l'a vu, était revenu à Paris et en Touraine,
et travaillait à force à ce magnifique bâtiment. Il fut mené si vite
qu'il se trouva presque achevé lorsque la corde cassa sur la
souveraineté\,; et, pour n'avoir plus à revenir à cette folie,
d'Aubigny, voyant que cela ne pouvait plus servir à ce que sa maîtresse
s'était proposé, retrancha tout ce qui pouvait encore l'être, acheta
comme il put quelques fiefs, pour qu'un si beau lieu ne fût pas
absolument dans l'état d'une guinguette, et M\textsuperscript{me} des
Ursins, honteuse après de ce pot au lait de la bonne femme, laissa le
tout à d'Aubigny, pas assez seigneur pour remplir le lieu, mais
suffisamment riche pour y bien recevoir le voisinage et les passants. Il
y a passé le reste de sa vie, aimé et considéré dans le pays, avec assez
d'esprit pour avoir laissé en Espagne ses grands airs et ses plus hautes
espérances. Ce lieu s'appelle {[}Chanteloup{]}, et a passé à
M\textsuperscript{me} d'Armentières, fille d'Aubigny. C'est un des beaux
et des plus singuliers lieux de France, et le plus superbement meublé.

Cette souveraineté, dont M\textsuperscript{me} de Maintenon se trouvait
si peu à portée, la choqua. Cette extrême différence offensa son
orgueil, en lui faisant sentir la distance des rangs et des naissances,
qui étaient la base d'un si grand essor. Elle sentit avec jalousie que
le crédit sans mesure qui portait M\textsuperscript{me} des Ursins si
haut n'était que l'effet de la protection qu'elle lui avait donnée. Elle
ne put souffrir qu'elle en abusât au point de s'élever si fort au-dessus
d'elle, et que cette souveraineté elle l'établît et en jouît sous ses
yeux. Le roi sentit aussi tout l'excès de ce dessein, mais il fut aussi
piqué d'en voir la paix retardée, de se trouver obligé à prendre des
ménagements, et à la fin forcé de ne plus rien ménager, de fâcher le roi
d'Espagne, de menacer, de parler en père et en maître, et de faire
conclure la paix sans cette souveraineté, malgré son petit-fils qui n'en
voulait point démordre, et qui ne céda qu'à l'impuissance de tenir
contre tant d'ennemis, abandonné de la France, et pour un si bizarre et
si mince sujet. On peut juger aussi quelle fut la rage de
M\textsuperscript{me} des Ursins, après avoir poussé sa pointe jusqu'à
une opiniâtreté si démesurée, s'être donnée en spectacle à toute
l'Europe, et ne remporter que le mépris et la honte d'une si folle
entreprise Telle fut la pierre d'achoppement entre les deux modératrices
suprêmes de la France et de l'Espagne. Telle fut aussi la raison de la
préférence de Berwick sur Tessé. Depuis cet essor de souveraineté, le
concert ne fut plus le même entre M\textsuperscript{me} de Maintenon et
M\textsuperscript{me} des Ursins. Mais cette dernière était parvenue à
un point en Espagne, qu'elle crut pouvoir plus qu'aisément s'en passer.

On a vu avec quel art elle avait sans cesse isolé le roi d'Espagne,
jusqu'à quel point elle l'avait enfermé avec la reine, et rendu
inaccessible, non seulement à sa cour, mais à ses grands officiers, a
ses ministres, jusqu'aux valets les plus nécessaires, en sorte qu'il
n'était servi que par trois ou quatre, qui étaient François et tout à
elle. Le prétexte de la douleur de la mort de la reine continua cette
solitude\,; et la retraite au palais de Medina-Celi fut préférée à celle
du Buen-Retiro, pour être plus resserrée dans un lieu infiniment moins
étendu que ce palais royal, où la cour pouvait abonder, et où il aurait
été plus embarrassant de ne laisser approcher le roi de personne. Elle
prit elle-même la place de la reine\,; et pour avoir une sorte de
prétexte d'être au près du roi dans la même solitude, elle se fit nommer
gouvernante de ses enfants. Mais, pour y être toujours, et qu'on ne pût
savoir quand ils étaient l'un chez l'autre, elle fit faire un corridor
de bois depuis le cabinet du roi jusque dans l'appartement de ses
enfants dans lequel elle logeait, pour pouvoir passer de l'un à l'autre
sans cesse sans être aperçus, et sans traverser un long espace de pièces
qui étaient entre-deux, et qui étaient remplies de courtisans. Ainsi on
ne savait jamais si le roi était seul ou avec M\textsuperscript{me} des
Ursins, ni elle de même, lequel des deux était chez l'autre, ni quand,
ni combien ils étaient ensemble. Cet appentis couvert et vitré fut
ordonné avec tant de hâte, qu'avec toute la dévotion du roi, les fêtes
et les dimanches ne furent point exceptés de ce travail. Il déplaisait
extrêmement à toute la cour, qui en sentait l'usage, et jusqu'à ceux qui
le dirigeaient. Le contrôleur des bâtiments, qui avait ordre d'y faire
travailler fêtes et dimanches, demanda un jour dans une de ces pièces où
la cour était, et que M\textsuperscript{me} des Ursins était si pressée
d éviter, il demanda, dis-je, au P. Robinet, confesseur du roi, et le
seul excellent qu'il ait eu, s'il ferait travailler le lendemain
dimanche et le surlendemain fête de la Vierge. Robinet répondit que le
roi ne lui en avait point parlé\,; et à une seconde instance fit même
réponse. À la troisième il ajouta qu'il attendrait que le roi lui en
parlât. Enfin excédé d'une quatrième, la patience lui échappa, et il
répondit que, si c'était pour détruire l'ouvrage commencé, il croyait
qu'on y pourrait travailler le propre jour de Pâques, mais que pour
continuer ce corridor, il ne pensait pas que cela se pût un dimanche ni
une fête. Toute la cour applaudit\,; mais M\textsuperscript{me} des
Ursins, à qui ce propos ne tarda pas à être rapporté, en fut très
irritée.

On soupçonna qu'elle pensait à plus qu'à devenir l'unique compagnie du
roi. Il avait plusieurs princes. On sema des discours qui parurent
équivoques, et qui effrayèrent\,: il se débita que le roi n'avait plus
besoin de postérité avec toute celle dont il avait plu à Dieu de le
bénir, mais seulement d'une femme, et qui pût les gouverner. Non
contente de passer toutes les journées avec le roi, et comme la feue
reine de ne le laisser travailler avec ses ministres qu'en sa présence,
la princesse des Ursins comprit qu'il fallait rendre cette conduite
durable en s'assurant du roi dans tous les moments. Il était accoutumé à
prendre l'air, et il en était d'autant plus affamé qu'il était demeuré
fort enfermé dans les derniers temps de la reine, et dans les premiers
qui avaient suivi sa mort. M\textsuperscript{me} des Ursins choisit
quatre ou cinq hommes pour accompagner le roi privativement à tous
autres, même à ses officiers grands ou autres les plus nécessaires.
Chalais, Masseran, Robecque et deux ou trois autres sur la servitude de
qui elle pouvait compter, furent nommés pour suivre le roi toutes les
fois qu'il sortait. On les appela \emph{recreadores} du roi, ceux qui
étaient chargés de l'amuser. Avec tant de mesures, d'obsession, de
discours préparatoires, jetés avec soin, on ne douta pas qu'elle n'eût
le projet de l'épouser, et l'opinion ainsi que la crainte en devint
générale\,; le roi son grand-père en fut vivement alarmé, et
M\textsuperscript{me} de Maintenon, qui n'avait jamais pu parvenir à
être déclarée après en avoir frisé le moment de bien près par deux fois,
en fut poussée à bout de jalousie. Cependant, si M\textsuperscript{me}
des Ursins s'en flatta, ce ne fut pas pour longtemps.

Le roi d'Espagne toujours curieux de nouvelles de France en demandait
souvent à son confesseur, le seul homme à qui il pût parler qui ne fût
pas à M\textsuperscript{me} des Ursins. L'habile et le hardi Robinet,
aussi inquiet que personne des progrès du dessein dont personne ne
doutait dans les deux cours de France et d'Espagne, se laissa pousser de
questions dans une embrasure de fenêtre où le roi l'avait attiré, et fit
le réservé et le mystérieux pour exciter la curiosité davantage\,: quand
il la vit au point où il la voulait, il dit au roi que puisqu'il le
forçait il lui avouerait que ses nouvelles de France étaient conformes à
toutes celles de Madrid, où on ne doutait plus qu'il ne fît à la
princesse des Ursins l'honneur de l'épouser. Le roi rougit et répondit
brusquement\,: «\,Oh\,! pour cela, non,\,» et le quitta.

Soit que la princesse des Ursins fût informée de cette vive repartie, ou
qu'elle désespérât déjà du succès, elle tourna court, et jugeant que cet
état d'interstice au palais de Medina-Celi ne pouvait durer toujours,
résolut de s'assurer du roi par une reine qui lui dût un si grand
mariage, et qui n'ayant aucun soutien se jetât entre ses bras par
reconnaissance et par nécessité. Dans cette vue elle s'ouvrit à Albéroni
qui, depuis la mort du duc de Vendôme, était demeuré à Madrid chargé des
affaires de Parme, et lui proposa le mariage de la princesse, fille de
la duchesse de Parme, et du feu duc, frère du régnant, qui avait épousé
la veuve de son frère.

Albéroni eut peine à croire ses oreilles\,; une alliance si
disproportionnée lui parut d'autant plus incroyable, qu'il n'espéra pas
que la cour de France y pût consentir, et qu'il crut encore moins qu'on
osât la conclure sans elle. En effet, une personne issue de double
bâtardise, d'un pape par père, d'une fille naturelle de Charles-Quint
par mère, fille d'un petit duc de Parme, et d'une mère tout autrichienne
sœur de l'impératrice douairière, de la reine d'Espagne douairière, dont
on était si mécontent, et qu'on avait fait passer de l'exil de Tolède à
la relégation de Bayonne, de la reine de Portugal, qui avait déterminé
le roi son mari à recevoir l'archiduc à Lisbonne, et à porter là guerre
en Espagne, n'était pas un parti auquel il fût vraisemblable de songer
pour en faire une reine d'Espagne.

Rien de tout cela néanmoins n'arrêta la princesse des Ursins\,; son
intérêt pressant fut sa considération la plus forte\,; elle disposait de
la volonté du roi d'Espagne, elle sentait tout le changement du roi et
de M\textsuperscript{me} de Maintenon pour elle, elle n'en espérait plus
de retour\,: elle crut même devoir s'appuyer contre l'autorité qui
l'avait si puissamment établie, et qui aurait pu la détruire, et ne
s'occupa plus qu'à brusquer un mariage dont elle se promettait tout, et
de faire de la nouvelle reine le même usage qu'elle avait fait de celle
qu'elle venait de perdre. Le roi d'Espagne était dévot, il avait besoin
d'une femme, la princesse des Ursins était d'un âge où ses agréments
n'étaient plus que de l'art\,: en un mot, elle mit Albéroni en besogne,
et on peut croire qu'elle ne fut pas difficile dès l'instant qu'on put
les persuader à Parme qu'elle était sérieuse, et qu'on ne se moquait pas
d'eux. Orry, toujours un avec M\textsuperscript{me} des Ursins et le
tout-puissant par elle, fut le seul confident de cette importante
affaire.

Le marquis de Brancas était lors ambassadeur de France à Madrid, comme
on l'a vu en son temps. Il s'était flatté de la grandesse au sortir de
Girone, il avait été tout près de l'obtenir. Il crut toujours que
M\textsuperscript{me} des Ursins l'avait fait changer en Toison, et il
ne lui avait pas pardonné cet échange. Il était tout à
M\textsuperscript{me} de Maintenon. On a vu ailleurs par quelles rares
conjonctures il en avait obtenu la protection, que son adroite mère et
lui avaient bien su cultiver et conserver. Par cela même il était fort
suspect à la princesse des Ursins, qui d'ailleurs se doutait bien de la
dent qu'il lui gardait de sa grandesse manquée\,: elle ne lui laissait
aucun accès, et avait les yeux fort ouverts sur toute sa conduite.
Brancas voyait et n'ignorait rien de tout ce qui se passait. Le
confesseur s'expliquait à ce client de sa compagnie de ses inquiétudes
sur la conduite de la princesse des Ursins, et les principaux d'une cour
universellement mécontente allaient décharger leur cœur avec lui, dans
la pensée qu'il n'y avait que la France qui pût mettre ordre à la
situation de l'Espagne. Brancas en sentit toute l'importance, mais
instruit par l'aventure de l'abbé d'Estrées, craignant même pour ses
courriers, il prit le parti de mander au roi qu'il avait pressamment à
lui rendre compte d'affaires les plus importantes, qui ne se pouvaient
confier au papier, et qui exigeaient qu'il lui permit d'aller passer
quinze jours à Versailles. La réponse fut la permission qu'il demandait,
mais avec ordre de s'arrêter où il rencontrerait le duc de Berwick sur
la route, qui allait faire le siège de Barcelone, pour conférer avec
lui.

M\textsuperscript{me} des Ursins, qui trouvait toujours moyen d'être
instruite de tout, la fut non seulement du voyage de Brancas, mais
encore de l'ordre qu'il avait reçu de conférer avec Berwick\,; elle en
fut alarmée\,: elle fit presser par le roi d'Espagne le départ du
maréchal comme si tout eût été prêt pour le siège de Barcelone, pour
éviter que Brancas le rencontrât en chemin. Elle fit disposer seize
relais de mules sur le chemin de Bayonne, et fit tout à coup partir pour
France, le jeudi saint, le cardinal del Giudice, grand inquisiteur et
ministre d'État, qui eut pour elle cette basse complaisance. C'était
coup double\,: le cardinal était à ses ordres, mais un cardinal-ministre
et grand inquisiteur l'embarrassait, elle s'en délivrait au moins pour
un temps de la sorte, en attendant mieux, et par le poids de sa pourpre
et de ses établissements en Espagne, elle en donnait à la commission
dont elle le chargeait, et prévenait Brancas, ce qui en notre cour
n'était pas un point médiocre. Brancas qui en sentait toute l'importance
le suivit dès le vendredi saint, et fit si bien qu'il l'atteignit à
Bayonne la nuit qu'il y était couché\,: Il chargea, en passant tout
droit, le commandant, qui était Dudoncourt, d'amuser et de retarder le
cardinal tout le lendemain tant qu'il pourrait, gagna pays et arriva à
Bordeaux avec vingt-huit chevaux de poste qu'il emmena de partout avec
lui pour les ôter au cardinal. Il arriva de la sorte deux jours plus tôt
que lui à Paris, d'où il alla aussitôt à Marly, où le roi était, lui
rendre compte des affaires qui l'avaient amené si roide\,; il en eut une
longue audience avec Torcy en tiers, et un logement pour le reste du
voyage.

Le cardinal del Giudice se reposa quatre ou cinq jours à Paris, puis
vint de Paris chez Torcy à Marly qui le mena dans le cabinet du roi à
l'issue de son lever. Il lui présenta le prince de Cellamare, fils du
duc de Giovenazzo son frère, grand d'Espagne et conseiller d'État assez
considéré à Madrid\,; Cellamare sortit aussitôt du cabinet, et le
cardinal y demeura seul avec le roi et Torcy une bonne heure. Torcy lui
donna à dîner\,; au sortir de table, ils retournèrent à Paris. Le
cardinal, à ce que longtemps depuis Torcy m'a compté, fut un peu
embarrassé de sa personne\,; il n'était chargé d'aucune affaire\,; toute
sa mission n'allait qu'à louer M\textsuperscript{me} des Ursins et se
plaindre du marquis de Brancas. Ces louanges de M\textsuperscript{me}
des Ursins n'étaient que vagues\,; elle ne comptait pas assez sur le
cardinal pour lui avouer la situation où elle se trouvait en notre cour,
et pour le charger de rien à cet égard, de sorte que la matière fut
bientôt épuisée. Sur le marquis de Brancas il n'y avait nul fait à
alléguer\,; son crime était de voir trop clair, et de n'être pas dévoué
à la princesse

Le cardinal était un homme d'esprit, de cour, d'affaires et d'intrigue,
qui sentait pour un homme de son état et de son poids le vide de sa
commission, et qui en était peiné. Il parut d'une conversation aimable,
d'une société aisée, écartant les embarras du rang et du personnage, et
il fut fort goûté et recueilli par la bonne compagnie. Il se rendit
assidu auprès du roi sans l'importuner d'audiences qu'il n'avait pas
matière à remplir, et à tout son manège il donna lieu de soupçonner
qu'il se doutait de la décadence de la princesse des Ursins dans notre
cour, et qu'il cherchait à s'en attirer l'estime et la confiance pour, à
l'appui du roi, devenir premier ministre en Espagne\,; mais nous verrons
bientôt que la marotte ultramontaine de sa charge, de son chapeau,
rompirent toutes ses mesures. Tout le succès de son voyage se borna à
empêcher Brancas de retourner en Espagne, et quoique bien sans concert,
Brancas fut de moitié avec lui\,: il n'avait rien à espérer de cette
cour dans la situation où il était avec M\textsuperscript{me} des
Ursins, et il n'était pas homme à perdre sciemment son temps. Il a fallu
conduire jusqu'ici cette affaire de suite\,; il faut maintenant un peu
retourner sur nos pas.

Il y avait longtemps que la chancelière était menacée d'une hydropisie
de poitrine après un asthme de presque toute sa vie. Elle était fille de
Maupeou, président d'une des chambres des enquêtes et peu riche, mais
bon parti pour Pontchartrain qui l'était encore moins quand elle
l'épousa. On ne peut guère être plus laide, mais avec cela une grosse
femme, de bonne taille et de bonne mine, qui avait l'air imposant, et
quelque chose aussi de fin. Jamais femme de ministre ni autre n'eut sa
pareille pour savoir tenir une maison, y joindre plus d'ordre à toute
l'aisance et la magnificence, en éviter tous les inconvénients avec le
plus d'attention, d'art et de prévoyance, sans qu'il y parût, et y avoir
plus de dignité avec plus de politesse, et de cette politesse avisée et
attentive qui sait la distinguer et la mesurer, en mettant tout le monde
à l'aise. Elle avait beaucoup d'esprit sans jamais le vouloir montrer,
et beaucoup d'agrément, de tour et d'adresse dans l'esprit, et de la
souplesse, sans rien qui approchât du faux, et quand il le fallait, une
légèreté qui surprenait\,; mais bien plus de sens encore, de justesse à
connaître les gens, de sagacité dans ses choix et dans sa conduite, que
peu d'hommes même ont atteint comme elle de son temps. Il est surprenant
qu'une femme de la robe qui n'avait vu de monde qu'en Bretagne, fût en
si peu de temps au fait aux manières, à l'esprit, au langage de la
cour\,; elle devint un des meilleurs conseils qu'on pût trouver pour s'y
bien gouverner. Aussi y fut-elle dans tous les temps d'un grand secours
à son mari, qui tant qu'il la crut n'y fit jamais de fautes, et ne se
trompa en ce genre que lorsqu'il s'écarta de ses avis. Avec tout cela
elle avait trop longtemps trempé dans la bourgeoisie pour qu'il ne lui
en restât pas quelque petite odeur. Elle avait naturellement une
galanterie dans l'esprit raffinée, charmante, et une libéralité si
noble, si simple, si coulant de source, si fort accompagnée de grâces
qu'il était impossible de s'en défendre. Personne ne s'entendait si
parfaitement à donner des fêtes. Elle en avait tout le goût et toute
l'invention, et avec somptuosité et au dehors et au dedans, mais elle
n'en donnait qu'avec raisons et bien à propos, et tout cela avec un air
simple, tranquille et sans jamais sortir de son âge, de sa place, de son
état, de sa modestie. La plus secourable parente, l'amie la plus solide,
la plus effective, la plus utile, la meilleure en tous points et la plus
sûre. Délicieuse à la campagne et en liberté\,; dangereuse à table pour
la prolonger, pour se connaître en bonne chère sans presque y tâter, et
pour faire crever ses convives\,; quelquefois fort plaisante sans jamais
rien de déplacé\,; toujours gaie quoique quelquefois elle ne fût pas
exempte d'humeur. La vertu et la piété la plus éclairée et la plus
solide, qu'elle avait eue toute sa vie, crût toujours avec la fortune.
Ce qu'elle donnait de pensions avec discernement, ce qu'elle mariait de
pauvres filles, ce qu'elle en faisait de religieuses, mais seulement
quand elle s'était bien assurée de leur vocation, ce qu'elle en dérobait
aux occasions, ce qu'elle mettait de gens avec choix et discernement en
état de subsister, ne se peut nombrer.

Sa charité mérite ce petit détail\,: sortant un dimanche de la
grand'messe de la paroisse de Versailles avec M\textsuperscript{me} de
Saint-Simon, elle s'amusa en chemin. M\textsuperscript{me} de
Saint-Simon, qui était pressée, parce qu'elle devait aller dîner chez
Monseigneur à Meudon avec M\textsuperscript{me} la duchesse de
Bourgogne, la hâtait, et lui demanda avec surprise ce que c'était qu'une
petite fille du bas peuple avec qui elle s'était arrêtée. «\,Ne
l'avez-vous pas trouvée fort jolie\,? lui dit la chancelière\,: elle m'a
frappée en passant. Je lui ai demandé qui étaient ses parents. Cela
meurt de faim, cela a quatorze ou quinze ans. Jolie comme elle est, elle
trouvera aisément pratique. La misère fait tout faire. Je l'ai un peu
langueyée\,; demain matin elle viendra chez moi\,; et tout de suite je
la paquetterai en lieu où elle sera en sûreté, et apprendra à gagner sa
vie.\,»

Voilà de quoi cette femme-là était sans cesse occupée sans qu'elle le
parût jamais\,: car elle ne l'aurait pas dit à une autre qu'à
M\textsuperscript{me} de Saint-Simon, qu'elle regardait comme une autre
elle-même. Outre tout ce qui vient d'être dit, ses aumônes réglées
étaient abondantes\,; les extraordinaires les surpassaient. Elle avait
toute une communauté à Versailles, de trente à quarante jeunes filles
pauvres qu'elle élevait à la piété et à l'ouvrage, qu'elle nourrissait
et entretenait de tout, et qu'elle pourvoyait quand elles étaient en
âge. Elle avait fondé avec le chancelier et bâti un hôpital à
Pontchartrain, où tout le spirituel et le temporel abondait, où ils
allaient souvent servir les pauvres, et qui leur coûta plus de deux cent
mille livres, et de l'entretien duquel ils n'étaient pas quittes à huit
ni à dix mille livres par an. De tant de bonnes œuvres il n'en
paraissait que cet hôpital et sa communauté de Versailles, qui ne se
pouvaient cacher et dont encore on ne voyait que l'écorce. Tout le reste
était enseveli dans le plus profond secret. Elle donnait ordre à tout
les matins, et aux choses domestiques, et il n'était plus mention de
rien après, et tout dans une règle admirable.

Mais l'année 1709 la trahit. La disette et la cherté fit une espèce de
famine. Elle redoubla ses aumônes, et, comme tout mourait de faim dans
les campagnes, elle établit des fours à Pontchartrain, des marmites et
des gens pour distribuer des pains et des potages à tous venants, et de
la viande cuite à la plupart tant que le soleil était sur l'horizon.
L'affluence fut énorme. Personne ne s'en allait sans emporter du pain de
quoi nourrir deux ou trois personnes plusieurs jours, et du potage pour
une journée. Ce concours a eu bien des journées de trois mille
personnes, et avec tant d'ordre que nul ne se pressait, ne passait son
tour d'arrivée, et avec tant de paix qu'on n'eût pas dit qu'il y eût
plus de cinquante personnes. Plus la donnée avait été nombreuse, plus la
chancelière était aise, et cela dura six à sept mois de la sorte.

Le chancelier, ravi de faire aussi ces bonnes œuvres, l'en laissait
entièrement maîtresse. Leur union, leur amitié, leur estime était
infinie et réciproque. Ils ne se séparaient de lieu que par une rare
nécessité, et ils couchaient partout dans la même chambre. Ils avaient
mêmes amis, mêmes parents, même société. En tout ils ne furent qu'un.
Ils le furent bien aussi dans les regrets de leur première belle-fille,
dont jamais ils ne purent se consoler. Telle fut la chancelière de
Pontchartrain, que Dieu épura de plus en plus par de longues et pénibles
infirmités, qui finirent par une hydropisie de poitrine, qu'elle porta
avec une patience, un courage et une piété qui fut l'exemple de la cour
et du monde. Elle s'en sépara entièrement au milieu de Versailles
plusieurs mois avant sa mort, pour ne voir plus que sa plus étroite
famille, M\textsuperscript{me} de Saint-Simon et des gens de bien,
uniquement occupée jour et nuit de son salut. Elle y mourut le jeudi 12
avril, à {[}\ldots\ldots.{]}, à {[}\ldots\ldots.{]} ans, universellement
regrettée de toute la cour, qui l'aimait et la respectait, et pleurée
des pauvres presque avec désespoir. Le chancelier alla cacher le sien
dans son petit appartement de l'institution de l'Oratoire. Jamais
M\textsuperscript{me} de Saint-Simon et moi n'eûmes de meilleure amie.
Nous en fûmes amèrement touchés. Son fils fut le seul de toute la
famille qui essuya cette perte avec tranquillité, et même des
domestiques.

La reine douairière de Danemark mourut en ce même temps. Elle était
Hesse, et petite-fille de la fameuse landgrave, dont le courage, l'âme
haute et guerrière et l'attachement à la France ont tant fait parler
d'elle. Elle était cousine germaine de Madame.

L'évêque de Senlis mourut aussi. Il était frère de Chamillart, le
meilleur et le plus imbécile des hommes, dont le visage et le maintien
ne le témaignaient guère moins que le discours. Sans quoi que ce soit de
l'orgueil ni de l'impertinence si ordinaire aux enfants, aux frères, aux
proches des ministres, c'était une fatuité de bonté et de confiance qui
le persuadait de l'amitié de tout le monde, qui le rendait libre et
caressant. Il était ravissant sur M. le Prince qui lui faisait mille
bassesses qu'il prenait toutes pour soi, et avec grand soin de bien
faire entendre que la place de son frère n'y avait aucune part, que M.
le Prince était le meilleur homme du monde, le plus agréable voisin, et
qu'il ne comprenait pas qu'on pût le trouver autrement\,; mais quand la
place du frère fut perdue, les bonnes grâces et les prévenances de M le
Prince s'évanouirent avec elle. Il n'allait plus le voir, il ne
l'attirait plus à Chantilly. Il l'en bannit bientôt par ses manières.
Plus de présents de gibier, plus de liberté à ses gens de chasser même
chez leur maître. Le pauvre homme ne put digérer ce changement qui lui
fut peut-être plus sensible que la chute de son frère, parce qu'il lui
montrait sa sottise. Pendant la faveur, ses nièces et tout ce qui le
voyait en familiarité se moquait de lui grossièrement, et il le
comprenait si peu, qu'il en riait le premier. Son frère même s'en
divertissait quelquefois. Avec tout cela tout le monde l'aimait tant il
était bon homme. Il ne savait rien, mais des moeurs excellentes,
peut-être avait-il conservé son innocence baptismale. C'était un homme à
mettre bien richement à Mende ou à Auch, et à l'y confiner pour qu'on ne
le vît jamais. Son frère fit la sottise de le faire passer de Dol à
Senlis, de le mettre à la cour, de l'y attacher à la mort de M. de Meaux
par la charge de premier aumônier de M\textsuperscript{me} la Dauphine,
où il fut la risée de toutes ses dames\,; enfin de le mettre de
l'Académie française en sa place, qui avait eu la misère de l'élire.
Cela combla toute mesure parce qu'il se crut bel esprit. Chamillart
écrivit au roi pour lui demander le logement qu'il avait conservé, et
l'obtint aussitôt. Ce qui montra que le goût du roi n'était pas
affaibli, malgré M\textsuperscript{me} de Maintenon et toutes les
machines qui le dépostèrent.

M\textsuperscript{me} Voysin mourut à Paris d'une assez longue
maladie\,: pourrait-on croire, si on ne le savait, que ce fut de
chagrin, unie comme elle était avec son mari, et dans l'état radieux où
il était, et qu'il ne devait qu'à elle\,? On a vu (t. VII, p.~254)
quelle était cette femme, et à quel point elle fut utile à Voysin, qui
sans elle n'avait rien qui pût lui faire faire fortune qu'il ne mérita
jamais, beaucoup moins une aussi démesurée qui l'a enfin porté à la tète
de la guerre et de la robe. M\textsuperscript{me} de Maintenon était
changeante\,: elle n'avait mis le mari en place que pour avoir sa femme
à la cour. Outre qu'elle les comptait tous et avec raison à elle sans
réserve, ce qu'elle brassa depuis par lui pour M. du Maine ne pouvait
entrer dans ses vues, alors que la petite vérole et le poison n'avaient
pas détruit la maison royale, et que les princes du sang d'âge étaient
encore pleins de vie. M\textsuperscript{me} Voysin eut dans les premiers
temps de son arrivée à la cour toute la faveur de M\textsuperscript{me}
de Maintenon et toute sa confiance. Elle ne s'aperçut pas assez tôt
qu'il ne fallait pas rassasier d'elle. L'indigestion vint peu à peu.
Toute la faveur, toute la confiance passa de la femme au mari. Elle le
trouva homme à tout faire, et que pour lui plaire aucune considération
ne l'arrêterait. Cela soutint quelque temps sa femme, mais le goût était
passé. Tout ce qui lui avait tant plu en elle, commença à lui être à
charge ou à lui paraître ridicule. Son assiduité, ses empressements, ses
flatteries l'importunèrent\,; ses douceurs et ses complaisances la
dégoûtèrent. Son vêtement et sa coiffure imitée de la sienne lui
semblèrent ridicules. M\textsuperscript{me} Voysin commençait à sentir
sa décadence, lorsque sa jalousie de M\textsuperscript{me} Desmarets
acheva de la perdre.

Vauxbourg, conseiller d'État, d'une vertu, d'une probité, d'une piété
rare dans tous ses emplois, où il s'était montré assez capable, était
frère aîné de Desmarets, et il avait épousé la sœur de Voysin. Cette
alliance des deux ministres réussit assez bien entre-deux, mais ne put
concilier leurs femmes. M\textsuperscript{me} Desmarets, grande, bien
faite, toujours bien mise sans affectation, avoir un air simple, naturel
et, avec de l'esprit, beaucoup de monde, rien du tout de bourgeois, un
air et des manières nobles, un dehors de franchise qui n'était pas sans
art, mais cet art n'était pas sans duplicité. Ses soins et ses respects
pour M\textsuperscript{me} de Maintenon étaient sans bassesse. Elle se
ménagea toujours si bien à l'approcher, que, bien loin de lui devenir à
charge, elle eut l'adresse de s'en faire toujours désirer. Tout cela
était bien loin de l'air doucereux, composé, préparé et de l'extrême
bourgeoisie de M\textsuperscript{me} Voysin\,: aussi en fut-elle coulée
à fond. Elle ne put soutenir une disgrâce personnelle ni une rivale
d'autant plus odieuse qu'elle n'y trempait en rien, et ne lui donnait
aucun sujet de plainte. La cour s'aperçut du changement, le mari le
sentit. Il en fut outré sans toutefois oser en rien montrer. La douleur
extrême prit sur la santé de M\textsuperscript{me} Voysin jusqu'alors
ferme et brillante. La maladie se déclara, elle s'en alla à Paris, elle
y mourut enfin de désespoir le vendredi 20 avril, à cinquante-un ans,
peu regrettée. Ce fut une délivrance pour M\textsuperscript{me} de
Maintenon. Le mari, tout dévoué à la fortune, s'en consola aisément\,;
peut-être même se trouva-t-il soulagé de n'avoir plus quelqu'un de si
nécessairement intime pris en aversion par M\textsuperscript{me} de
Maintenon, auprès de laquelle il n'avait plus besoin de personne.

Peu de jours après mourut Zurbeck, ancien lieutenant général, colonel du
régiment des gardes suisses et des neuf autres régiments suisses au
service de France. Ce fut une grande dépouille à distribuer pour M. du
Maine.

Le Bailleul, président à mortier, mourut en même temps. Il était fils de
l'ami de mon père, et petit-fils du surintendant des finances. Lui et le
maréchal d'Huxelles, et Saint-Germain-Beaupré étaient enfants du frère
et des deux sœurs. C'était un homme d'honneur et de vertu, d'ailleurs
fort peu de chose. Il ne laissa qu'un fils qui, excepté l'honneur et la
vertu, lui ressembla au reste. Il était dès lors fort décrié, mais les
efforts du maréchal d'Huxelles, qui fit valoir son nom dans le
parlement, et les services de ses pères, lui obtinrent enfin la charge
avec grand'peine. Il ne prit pas celle de l'exercer, se ruina avec honte
et scandale, et la vendit enfin à Chauvelin, depuis garde des sceaux,
dont la fortune et la disgrâce ont tant fait parler. Ce dernier Bailleul
est mort sans s'être marié, dans la dernière obscurité.

\hypertarget{chapitre-vi.}{%
\chapter{CHAPITRE VI.}\label{chapitre-vi.}}

1714

~

{\textsc{Mariage du fils du marquis du Châtelet avec la fille du duc de
Richelieu\,; {[}il obtient{]} la survivance de Vincennes.}} {\textsc{-
Publication et réjouissances de la paix.}} {\textsc{- Contade
grand'croix surnuméraire de Saint-Louis.}} {\textsc{- Marly.}}
{\textsc{- Giudice bien traité du roi.}} {\textsc{- Ducasse malade.}}
{\textsc{- Chalais mandé de l'armée à Madrid.}} {\textsc{- Ronquillo et
d'autres exilés.}} {\textsc{- Bergheyck se retire tout à fait des
affaires\,; son éloge.}} {\textsc{- Réforme de troupes.}} {\textsc{-
Électeur de Bavière à la chasse à Marly.}} {\textsc{- M. le duc de Berry
malade et empoisonné.}} {\textsc{- Mort de M. le duc de Berry\,; son
caractère.}} {\textsc{- Quel avec sa famille.}} {\textsc{- M. {[}le
duc{]} et M\textsuperscript{me} la duchesse de Berry\,; comment
ensemble.}} {\textsc{- Ordres du roi.}} {\textsc{- Le corps de M. le duc
de Berry très promptement porté à Paris aux Tuileries.}} {\textsc{-
Deuil drapé de six mois.}} {\textsc{- Le roi ne veut point de
révérences, de manteaux, de mantes, de harangues ni de compliments.}}
{\textsc{- État du roi.}} {\textsc{- Sa visite à M\textsuperscript{me}
la duchesse de Berry.}} {\textsc{- M. {[}le duc{]} et
M\textsuperscript{me} la duchesse d'Orléans fort touchés.}} {\textsc{-
Raisons particulières à M. le duc d'Orléans.}} {\textsc{-
M\textsuperscript{me} de Maintenon et duc du Maine.}} {\textsc{-
Duchesse du Maine.}} {\textsc{- Évêques usurpent pour la première fois,
en gardant, fauteuils et carreaux.}} {\textsc{- Eau bénite.}} {\textsc{-
Comte de Charolais et duc de Fronsac conduisent le coeur au
Val-de-Grâce.}} {\textsc{- M. le Duc et le duc de La Trémoille
conduisent le corps à Saint-Denis.}} {\textsc{- Fils et petits-fils de
France tendent seuls chez le roi.}} {\textsc{- Précautions chez
M\textsuperscript{me} la duchesse de Berry, qui font quelques aventures
risibles.}}

~

Un événement singulier, et qui fit honneur à la cour, reposera pour
quelques moments de ces tristesses. Parmi toutes les dames du palais
dont il y avait force dévotes, une seule n'était occupée que de Dieu,
son mari un très galant homme, et les deux personnes du monde, lui par
peu d'entregent, elle par n'être occupée que de son salut, les moins
propres à tirer le moindre parti d'aucune chose, et fort pauvres.
C'était la marquise du Châtelet, fille du feu maréchal de Bellefonds. Un
reste de considération pour la mémoire de son père et d'avoir été fille
d'honneur de M\textsuperscript{me} la dauphine de Bavière, avec une
grande réputation de sagesse et de vertu, la tirèrent de Vincennes où
elle vivait avec sa mère, pour la faire dame du palais lorsqu'elle y
pensait le moins. Elle aimait tellement sa retraite qu'elle évita le
voyage du Pont-Beauvoisin, et tant qu'elle put, Marly dans la suite,
pour s'en aller à Vincennes\,; et à Versailles tant qu'elle pouvait
aussi à la chapelle ou dans sa chambre. Du reste gaie, paisible, assidue
à ses fonctions, ne se mêlant de rien, mais à force de vertu, de
douceur, de piété sincère, aimée, considérée, respectée de tout le
monde, de M\textsuperscript{me} la duchesse de Bourgogne même, et de la
jeunesse de la cour dont la vie ressemblait le moins à la sienne.

Ni elle ni son mari, ancien lieutenant général et de qualité distinguée,
et fort estimé, ne savaient que faire de leur fils qui avait un régiment
et peu ou point de quoi y vivre\,; avec cela brave et honnête garçon,
mais aussi demeuré que le père, et faute de savoir qu'en faire, ils n'y
songeaient point du tout. Un beau jour qu'ils étaient tous à Vincennes
et la cour à Versailles, Cavoye, qui prenait soin du vieux duc de
Richelieu, le trouva fort en peine de sa fille qui venait chez lui d'un
couvent de province. Il lui conseilla de s'en défaire promptement à un
mari. Il chercha, il imagina Clefmont, fils de M. et de
M\textsuperscript{me} du Châtelet, avec la survivance de Vincennes. Sur
tout le bien qu'il lui dit d'eux tous, le bon homme y entra si bien que
dans la même conversation Cavoye régla tout ce qu'il pouvait donner, et
l'affaire tout de suite résolue. Pour savoir des nouvelles de ce
qu'aurait le prétendu, ils envoyèrent à l'heure même chercher
M\textsuperscript{me} de Saint-Géran, qui avait passé ses premières
années chez le maréchal de Bellefonds, et qui était leur amie intime.
Elle vint et leur dit ce qu'elle en savait. Malgré le peu de bien, M. de
Richelieu la chargea de parler au père et à la mère. Au sortir d'avec
eux M\textsuperscript{me} de Saint-Géran en parla à
M\textsuperscript{me} de Nogaret son amie, et qui l'était aussi de
M\textsuperscript{me} du Châtelet, et avait été sa compagne fille
d'honneur et dame du palais chez les deux Dauphines.
M\textsuperscript{me} de Nogaret qui avait un excellent esprit trouva
que rien ne pouvait être plus avantageux à M. de Clefmont, et tandis
qu'elles envoyèrent chercher M\textsuperscript{me} du Châtelet à
Vincennes, M\textsuperscript{me} de Saint-Géran retourna, de l'avis de
M\textsuperscript{me} de Nogaret, presser l'affaire, tellement que le
même soir, car cela ne fut pas plus long, M. de Richelieu fut parler à
M\textsuperscript{me} de Maintenon un moment avant que le roi y entrât.
Elle se piquait d'amitié pour lui, et sa porte lui était toujours
ouverte. Elle le renvoya écrire au roi et se chargea du reste. Il lui
envoya sa lettre dès qu'elle fut faite\,; elle la présenta au roi qui
accorda la survivance en faveur du mariage, et sur-le-champ
M\textsuperscript{me} de Maintenon le manda à M. de Richelieu, de
manière que du dîner au souper l'affaire fut imaginée, réglée et
consommée, sans que M. ni M\textsuperscript{me} du Châtelet en eussent
la première notion.

Le lendemain ils arrivèrent à Versailles. M\textsuperscript{me}s de
Saint-Géran et de Nogaret les furent trouver aussitôt et leur apprirent
que leur fils était marié, et marié avec cinq cent mille livres, à la
vérité un peu légères, et peu présentes, à la fille d'un duc et pair
bien élevée, et qui sortait tout à l'heure d'un couvent, et avec la
survivance de Vincennes. Jamais surprise ne fut pareille à la leur. À la
surprise succéda la joie. Ils ne pouvaient comprendre que la chose fût
vraie Le mariage se fit aussitôt après. On a vu que la considération
seule de M\textsuperscript{me} du Châtelet avait valu à son mari, et
sans qu'elle s'en mêlât ni lui non plus, le gouvernement de Vincennes à
la mort de son neveu. Ainsi la vertu fut doublement récompensée
uniquement par des traits de Providence, et il est bien remarquable que
de toutes les dames du palais, ce fut la seule qui en tira parti, et
toujours sans s'en donner aucun soin, et même sans le savoir.

La paix avec l'empereur et l'empire fut publiée, le \emph{Te Deum}
chanté, des feux de joie le soir. Le roi qui était à Marly où le
\emph{Te Deum} ne put être chanté à sa messe, l'alla entendre sur les
cinq heures du soir à la paroisse. Le duc de Tresmes donna une grande
collation à l'hôtel de ville, et à minuit un grand repas chez lui à
beaucoup de dames et d'étrangers, et à des gens de la cour.

En même temps {[}le roi{]} donna à Contade une grand'croix de l'ordre de
Saint-Louis surnuméraire, n'y en ayant point de vacante, en attendant un
gouvernement.

Ce Marly-ci fut encore bien funeste. Il est à propos de le reprendre des
le commencement, car c'est le même où arriva le marquis de Brancas, et
où le cardinal del Giudice vit le roi, et pendant lequel se sont passées
les choses qui ont été racontées depuis.

Quelque temps auparavant, M\textsuperscript{me} de Saint-Simon s'en
était allée de Versailles à Paris incommodée\,; elle y eut la rougeole.
Sur la fin de cette rougeole, le roi alla à Marly le mercredi 11
avril\,; peu de jours après, M\textsuperscript{me} de Lauzun et moi
reçûmes chacun un billet de Bloin, qui nous mandait que le roi nous
avait donné à chacun un logement à Marly, que la rougeole n'était pas
comme la petite vérole, et que nous pouvions aller à Marly dès le
lendemain. Permettre en ce genre c'était ordonner, et cet ordre était
une distinction et une grâce, qui, sous prétexte de peur, fit jalousie à
bien des gens. M\textsuperscript{me} de Saint-Simon alla s'établir chez
M\textsuperscript{me} de Lauzun, à Passy dès qu'elle fut en état de le
faire, pour prendre l'air, en changer, et revenir à Versailles le même
jour que le roi y retournerait, car le voyage de Marly était annoncé
pour être long. M\textsuperscript{me} la duchesse de Berry, qui était
grosse, se trouvait incommodée, et avait été bien aise de demeurer à
Versailles comme il lui arrivait quelquefois pendant les Marly\,; et
comme il s'en fallait tout qu'elle fût l'amusement du roi et de
M\textsuperscript{me} de Maintenon, comme avait été
M\textsuperscript{me} la Dauphine, le roi s'en trouvait soulagé
quoiqu'il n'aimât pas ces séparations.

Le roi permit au cardinal del Giudice de lui venir faire sa cour à Marly
sans le demander, toutes les fois qu'il voudrait. Il le distingua fort,
et prit plaisir à lui montrer ses jardins, et tout cela finit enfin par
lui donner un logement à Marly. On y apprit la maladie de Ducasse\,; que
Chalais qui était avec les troupes qui allaient faire le siège de
Barcelone, avait été mandé à Madrid pour une commission secrète\,; que
Ronquillo avait été exilé avec quelques autres qui déplaisaient à la
princesse des Ursins. Le roi apprit aussi avec chagrin que Bergheyck
avait obtenu de se retirer de toutes les affaires, et d'aller achever sa
vie tranquillement dans une de ses terres en Flandre. C'était un homme
infiniment modeste, affable, doux, équitable et parfaitement
désintéressé\,; avec beaucoup d'esprit, mais sage et réglé, et qui
possédait à fond toutes les parties du ministère dont il était charge,
qui étaient les finances et le commerce des Pays-Bas espagnols où il fut
toujours adoré. C'était l'homme du monde le plus véritable, le plus
hardi à dire la vérité, qui aimait et cherchait le plus le bien pour le
bien, et qui était le plus attaché aux intérêts du roi d'Espagne. Poussé
enfin à bout de tous les obstacles qu'il trouvait à tout à la cour de
Madrid, où on ne s'accommodait pas d'un ministre si intègre, si éclairé,
si libre, et désespérant de rien faire de bon, qui était son ambition
unique, quoiqu'il eût des enfants, il prit le parti de tout quitter, au
grand soulagement d'Orry et de M\textsuperscript{me} des Ursins. Nous le
verrons passer à la cour revenant de Madrid et allant se confiner dans
une petite terre de Flandre, où il vécut retiré encore fort longtemps,
aimé, respecté et considéré de tout le monde. Le roi l'aimait, le
croyait et l'estimait beaucoup.

Le roi réforma cinq hommes par compagnie d'infanterie qui demeurèrent à
quarante-cinq, et de cavalerie qui restèrent à trente. L'électeur vint
courre le cerf a Marly le jeudi 26 avril, et ne vit le roi qu'à la
chasse\,; il soupa chez d'Antin et joua dans le salon après avec M. le
duc de Berry à un grand lansquenet, puis retourna à Saint-Cloud.

Le lundi 30 avril, le roi prit médecine, et travailla l'après-dînée avec
Pontchartrain\,; sur les six heures du soir il entra chez M. le duc de
Berry qui avait eu la fièvre toute la nuit. Il s'était levé sans en rien
dire, avait été à la médecine du roi, et comptait aller courre le
cerf\,; mais, en sortant de chez le roi sur les neuf heures du matin, il
lui prit un grand frisson qui l'obligea de se remettre au lit. La fièvre
fut violente ensuite. Il fut saigné, le roi dans sa chambre, et le sang
fut trouvé très mauvais\,; au coucher du roi, les médecins lui dirent
que la maladie était de nature à leur faire désirer que c'en fût une de
venin. Il avait beaucoup vomi, et ce qu'il avait vomi était noir. Fagon
disait avec assurance que c'était du sang\,; les autres médecins se
rejetaient sur du chocolat, dont il avait pris le dimanche. Dès ce
jour-là je sus qu'en croire. Boulduc, apothicaire du roi, qui était
extrêmement attaché à M\textsuperscript{me} de Saint-Simon et à moi, et
dont j'ai eu quelquefois occasion de parler, me glissa à l'oreille qu'il
n'en reviendrait pas, et qu'avec quelques petits changements, c'était au
fond la même chose qu'à M. {[}le Dauphin{]} et M\textsuperscript{me} la
Dauphine. Il me le confirma le lendemain, ne varia ni pendant la courte
maladie, ni depuis\,; et il me dit le troisième jour que nul des
médecins qui voyaient ce prince n'en doutait, et ne s'en était caché à
lui qui me parlait. Ces médecins en demeurèrent persuadés dans la suite,
et s'en expliquèrent même assez familièrement.

Le mardi 1er mai, saignée du pied à sept heures du matin, après une très
mauvaise nuit\,; deux fois de l'émétique qui fit un grand effet, puis de
la manne, mais deux redoublements. Le roi y alla au sortir de sa messe,
tint conseil de finances, ne voulut point aller tirer comme il l'avait
résolu, et se promena dans ses jardins. Les médecins, contre leur
coutume, ne le rassurèrent jamais. La nuit fut cruelle. Le mercredi 2
mai le roi alla après sa messe chez M. le duc de Berry qui avait été
encore saigné du pied. Le roi tint le conseil d'État à l'ordinaire, dîna
chez M\textsuperscript{me} de Maintenon, et alla après faire la revue de
ses gardes du corps. Coettenfao, chevalier d'honneur de
M\textsuperscript{me} la duchesse de Berry, était venu le matin prier le
roi de sa part que Chirac, médecin fameux de M. le duc d'Orléans, vît M.
le duc de Berry. Le roi le refusa sur ce que tous les médecins étaient
d'accord entre eux, et que Chirac, qui serait peut-être d'avis
différent, ne ferait que les embarrasser. L'après-dînée,
M\textsuperscript{me}s de Pompadour et de La Vieuville vinrent de sa
part prier le roi de trouver bon qu'elle vînt, avec force propos de son
inquiétude, et qu'elle viendrait plutôt à pied. Il y fallait venir en
carrosse si elle en avait eu tant d'envie, et avant de descendre le
faire demander au roi. La vérité est qu'elle n'avait pas plus d'envie de
venir que M. le duc de Berry de désir de la voir, qui ne proféra jamais
son nom, ni n'en parla indirectement même. Le roi répondit des raisons à
ces dames\,; sur ce qu'elles insistèrent, il leur dit qu'il ne lui
fermerait pas la porte, mais qu'en l'état où elle était cela serait fort
imprudent. Il dit ensuite à Madame et à M. le duc d'Orléans d'aller à
Versailles pour l'empêcher de venir. Au retour de la revue, le roi entra
chez M. le duc de Berry. Il avait encore été saigné du bras, il avait eu
tout le jour de grands vomissements où il y avait beaucoup de sang, et
il avait pris pour l'arrêter de l'eau de Rabel jusqu'à trois fois. Ce
vomissement fit différer la communion\,; le P. de La Rue était auprès de
lui dès le mardi matin, qui le trouva fort patient et fort résigné.

Le jeudi 3, après une nuit encore plus mauvaise, les médecins dirent
qu'ils ne doutaient pas qu'il n'y eût une veine rompue dans son estomac.
Il commençait dès la veille, mercredi, à se débiter que cet accident
était arrivé par un effort qu'il avait fait à la chasse le jeudi
précédent que l'électeur de Bavière y était venu, en retenant son cheval
qui avait fait une grande glissade, et on ajouta que le corps avait
porté sur le pommeau de la selle, et que depuis il avait craché et rendu
du sang tous les jours. Les vomissements cessèrent à neuf heures du
matin, mais sans aucun mieux. Le roi, qui devait courre le cerf,
contremanda la chasse. À six heures du soir, M. le duc de Berry
étouffait tellement qu'il ne put plus demeurer au lit\,; sur les huit
heures, il se trouva si soulagé qu'il dit à Madame qu'il espérait n'en
pas mourir\,; mais bientôt après le mal augmenta si fort, que le P. de
La Rue lui dit qu'il était temps de ne plus penser qu'à Dieu, et à
recevoir le viatique. Le pauvre prince parut lui-même le désirer. Un peu
après dix heures du soir, le roi alla à la chapelle où on gardait une
hostie consacrée dès les premiers jours de la maladie\,; M. le duc de
Berry la reçut et l'extrême onction, en présence du roi, avec beaucoup
de dévotion et de respect. Le roi demeura près d'une heure dans sa
chambre, vint souper seul dans la sienne, ne vit point les princesses
après souper, et se toucha. M. le duc d'Orléans alla à deux heures après
minuit à Versailles, sur ce que M\textsuperscript{me} la duchesse de
Berry voulait encore venir à Marly. Un peu avant de mourir, M. le duc de
Berry dit au P. de La Rue, qui au moins le conta ainsi, l'accident de la
glissade dont on vient de parler, mais, à ce qui fut ajouté, la tête
commençait à s'embarrasser\,; après qu'il eut perdu la parole, il prit
le crucifix que le P. de La Rue tenait, il le baisa et le mit sur son
cœur. Il expira le vendredi 4 mai, à quatre heures du matin, en sa
vingt-huitième année, étant né à Versailles le dernier août 1686.

M. le duc de Berry était de la hauteur ordinaire de la plupart des
hommes, assez gros, et de partout, d'un beau blond, un visage frais,
assez beau, et qui marquait une brillante santé. Il était fait pour la
société et pour les plaisirs, qu'il aimait tous\,; le meilleur homme, le
plus doux, le plus compatissant, le plus accessible, sans gloire et sans
vanité, mais non sans dignité, ni sans se sentir. Il avait un esprit
médiocre, sans aucunes vues et sans imagination, mais un très bon sens,
et le sens droit, capable d'écouter, d'entendre, et de prendre toujours
le bon parti entre plusieurs spécieux. Il aimait la vérité, la justice,
la raison\,; tout ce qui était contraire à la religion le peinait à
l'excès, sans avoir une piété marquée\,; il n'était pas sans fermeté, et
haïssait la contrainte. C'est ce qui fit craindre qu'il ne fit pas aussi
souple qu'on le désirait d'un troisième fils de France, qui ne pouvait
entendre dans sa première jeunesse qu'il y eût aucune différence entre
son aîné et lui, et dont les querelles d'enfant avaient souvent fait
peur.

C'était le plus beau et le plus accueillant des trois frères, par
conséquent le plus aimé, le plus caressé, le plus attaqué du monde\,; et
comme son naturel était ouvert, libre, gai, on ne parlait dans sa
jeunesse que de ses reparties à Madame et à M. de La Rochefoucauld qui
l'attaquaient tous les jours. Il se moquait des précepteurs et des
maîtres, souvent des punitions\,; il ne sut jamais guère que lire et
écrire, et n'apprit jamais rien depuis qu'il fut délivré de la nécessité
d'apprendre. Ces choses avaient engagé à appesantir l'éducation\,; mais
cela lui émoussa l'esprit, lui abattit le courage, et le rendit d'une
timidité si outrée qu'il en devint inepte à la plupart des choses,
jusqu'aux bien séances de son état, jusqu'à ne savoir que dire aux gens
avec qui il n'était pas accoutumé, et n'oser ni répondre ni faire une
honnêteté dans la crainte de mal dire, enfin jusqu'à s'être persuadé
qu'il n'était qu'un sot et une bête propre à rien. Il le sentait, et il
en était outré. On peut se souvenir là-dessus de son aventure du
parlement, et de M\textsuperscript{me} de Montauban.
M\textsuperscript{me} de Saint-Simon, pour qui il avait une ouverture
entière, ne pouvait le rassurer là-dessus, et il est vrai que cette
excessive défiance de lui-même lui nuisait infiniment. Il s'en prenait à
son éducation, dont il disait fort bien la raison, mais elle ne lui
avait pas laissé de tendresse pour ceux qui y avaient eu part.

Il était le fils favori de Monseigneur par goût, par le naturel du sien
pour la liberté et pour le plaisir, par la préférence du monde, et par
cette cabale expliquée ailleurs, qui était si intéressée et si appliquée
à éloigner et à écraser Mgr le duc de Bourgogne. Comme ce prince, depuis
leur sortie de première jeunesse, n'avait jamais fait sentir son
aînesse, et avait toujours vécu avec M. le duc de Berry dans la plus
intime amitié et familiarité, et avait eu pour lui toutes les
prévenances de toute espèce, aussi M. le duc de Berry, qui était tout
bon et tout rond, ne se prévalut jamais à son égard de la prédirection.
M\textsuperscript{me} la duchesse de Bourgogne ne l'aimait pas moins, et
n'était pas moins occupée de lui faire tous les petits plaisirs qu'elle
pouvait que s'il avait été son propre frère, et les retours de sa part
étaient la tendresse même et le respect les plus sincères et les plus
marqués pour l'un et pour l`autre. Il fut pénétré de douleur à la mort
de l'un et à celle de l'autre, surtout à celle de Mgr le duc de
Bourgogne lors Dauphin, et de la douleur la plus vraie, car jamais homme
n'a su moins feindre que celui-là. Pour le roi, il le craignait à un tel
point qu'il n'en osait presque approcher, et si interdit dès que le roi
le regardait d'un œil sérieux, ou lui parlait d'autre chose que de jeu
ou de chasse, qu'à peine l'entendait-il, et que les pensées lui
tarissaient. On peut juger qu'une telle frayeur ne va guère de compagnie
avec une grande amitié.

Il avait commencé avec M\textsuperscript{me} la duchesse de Berry comme
font presque tous ceux qu'on marie fort jeunes et tout neufs. Il en
était devenu extrêmement amoureux, ce qui, joint à sa douceur et à sa
complaisance naturelle, fit aussi l'effet ordinaire, qui fut de la gâter
parfaitement. Il ne fut pas longtemps sans s'en apercevoir\,; mais
l'amour fut plus fort que lui. Il trouva une femme haute altière,
emportée, incapable de retour, qui le méprisait, et qui le lui laissait
sentir, parce qu'elle avait infiniment plus d'esprit que lui, et qu'elle
était de plus suprêmement fausse et parfaitement déterminée. Elle se
piquait même de l'un et de l'autre, et de se moquer de la religion, de
railler avec dédain M. le duc de Berry parce qu'il en avait, et toutes
ces choses lui devinrent insupportables. Tout ce qu'elle fit pour le
brouiller avec M. {[}le duc{]} et M\textsuperscript{me} la duchesse de
Bourgogne, et à quoi elle ne put parvenir pour les deux frères, acheva
de l'outrer. Ses galanteries furent si promptes, si rapides, si peu
mesurées, qu'il ne put se les cacher. Ses particuliers journaliers et
sans fin avec M. le duc d'Orléans, et où tout languissait pour le moins
quand il y était en tiers, le mettaient hors des gonds. Il y eut entre
eux des scènes violentes et redoublées. La dernière qui se passa à
Rambouillet, par un fâcheux contre-temps, attira un coup de pied dans le
cul à M\textsuperscript{me} la duchesse de Berry, et la menace de
l'enfermer dans un couvent pour le reste de sa vie\,; et il en était,
quand il tomba malade, à tourner son chapeau autour du roi comme un
enfant, pour lui déclarer toutes ses peines, et lui demander de le
délivrer de M\textsuperscript{me} la duchesse de Berry. Ces choses en
gros suffisent, les détails seraient et misérables et affreux\,; un seul
suffira pour tous.

Elle voulut à toute force se faire enlever au milieu de la cour par La
Haye, écuyer de M. le duc de Berry, qu'elle avait fait son chambellan.
Les lettres les plus passionnées et les plus folles de ce projet ont été
surprises, et d'un tel projet, le roi, son père, et son mari pleins de
vie, on peut juger de la tête qui l'avait enfanté et qui ne cessait d'en
presser l'exécution. On en verra dans la suite encore d'autres. Elle
sentit donc moins sa chute à la mort de M. le duc de Berry que sa
délivrance. Elle était grosse, elle espérait un garçon, et elle compta
bien de jouir en plein de sa liberté, délivrée de ce qui lui avait
attiré tant de choses fâcheuses du roi et de M\textsuperscript{me} de
Maintenon, qui ne prendraient plus la même part dans sa conduite.

M. le duc de Berry était fort aimé et fut généralement regretté. Le
vendredi matin, qu'il mourut, M\textsuperscript{me} de Maintenon, les
princes, les princesses se trouvèrent au réveil du roi dans le petit
salon, devant sa chambre. Tout s'y passa à peu près comme on l'a vu à la
mort de Mgr le duc de Bourgogne, lors Dauphin. Le roi, dans son lit,
donna ses ordres à Dreux, grand maître des cérémonies, se leva, entendit
la messe à la chapelle plus tôt qu'à l'ordinaire, et passa tout le reste
de la matinée chez M\textsuperscript{me} de Maintenon. Dès qu'il eut
dîné, il alla se promener en calèche dans la forêt de Marly,
c'est-à-dire entre trois et quatre heures. Dès qu'il fut sorti, le corps
de M. le duc de Berry fut mis dans son carrosse, environné de ses pages
et de ses gardes, suivi d'un autre de ses carrosses rempli de ses
officiers principaux\,: MM\hspace{0pt}. de Béthune, depuis duc de Sully,
premier gentilhomme de la chambre en année\,; le chevalier de Roye,
capitaine des gardes en quartier\,; Sainte-Maure, premier écuyer\,;
Montendre, capitaine des Suisses de sa garde\,; Pons, maître de sa
garde-robe en année\,; et Champignelle, premier maître d'hôtel. On avait
préparé à la hâte un appartement funèbre à Paris, aux Tuileries, où il
fut déposé. Ainsi il ne demeura pas douze heures à Marly après sa mort.
Le roi régla le même jour que la maison subsisterait jusqu'aux couches
de M\textsuperscript{me} la duchesse de Berry, pour continuer si c'était
d'un prince.

Le lendemain, samedi, le roi ordonna à son lever que le deuil
commencerait le mardi suivant, que les princes du sang, ducs, officiers
de la couronne, princes étrangers et grands officiers, draperaient,
quoiqu'il ne portât point le deuil\,; qu'il durerait six mois\,; et
déclara qu'il ne voulait point de révérences, ni voir personne en
manteau ni en mante, ce qui fut cause qu'il n'y en eut pas même chez
M\textsuperscript{me} la duchesse de Berry. Il chargea Breteuil,
introducteur des ambassadeurs, d'avertir les ministres étrangers qu'il
recevrait leurs compliments en allant et en revenant de la messe, mais
qu'il ne donnerait d'audience pour cela à pas un d'eux\,; et il dit au
premier président, qui était venu recevoir ses ordres, qu'il ne voulait
de compliment d'aucune compagnie. Il manda la perte qu'il venait de
faire à la reine d'Angleterre, à Saint-Germain, par le duc de Tresmes,
et à M\textsuperscript{me} la duchesse de Berry qu'il irait la voir le
lendemain. II vécut ce jour-là à l'ordinaire, et alla faire une dernière
revue de ses gardes du corps, qu'il renvoya dans leurs quartiers. Il
avait l'âme fort noircie\,; mais il était d'ailleurs peu touché, et il
ne cherchait pas à s'affliger. Les bienséances en souffrirent.

Le dimanche après-dîner, le roi fut à Versailles voir
M\textsuperscript{me} la duchesse de Berry. M\textsuperscript{me} de
Saint-Simon y était revenue, qui en reçut beaucoup d'honnêtetés, et
force caresses de M\textsuperscript{me} la duchesse de Berry. M. {[}le
duc{]} et M\textsuperscript{me} la duchesse d'Orléans étaient auprès
d'elle. Le roi lui fit fort bien\,; mais il n'y demeura qu'un quart
d'heure, et s'en retourna à Marly se promener dans ses jardins.

M. {[}le duc{]} et M\textsuperscript{me} la duchesse d'Orléans sentirent
toute la grandeur de la perte. C'était un lien qui les attachait au roi
de fort près. Sa rupture était irréparable. L'idée de régence ne consola
point M. le duc d'Orléans. Il ne pouvait se dissimuler sa supériorité
d'esprit sur un gendre avec qui d'ailleurs ses intérêts étaient communs,
et qu'il conduirait nécessairement. D'ailleurs cette régence ne
paraissait pas encore prochaine. Il fut véritablement affligé par
intérêt et par amitié.

La nature du mal qui avait emporté ce gendre ne tarda pas à devenir
publique, et le contre-coup en fut pareil à celui des précédentes
pertes. Plus elles augmentaient, plus M. le duc d'Orléans demeurait
seul, plus l'intérêt s'augmentait de l'affubler de ce qu'il y avait de
plus odieux, de le rendre tel au roi et au monde, et on y était enhardi
par l'expérience des précédents essais. M\textsuperscript{me} de
Maintenon et un intérieur de valets affidés y prétaient toute leur
assistance, et on n'oubliait pas à s'aider au dehors des ressorts qui
avaient donné tant de succès à M. de Vendôme dans tous les temps,
surtout contre M. le duc de Bourgogne. Ces ressorts, M. du Maine en
disposait\,; il les avait trop maniés dans ce temps-là pour se trouver
rouillé à les remettre en pratique, et s'en était trop utilement servi à
la mort des deux Dauphins et de la Dauphine. Le roi ne montra rien au
dehors\,; mais ces bons ouvriers n'y perdirent rien, comme on le verra
en plus d'un endroit, et qu'ils surent toujours croître et s'élever sur
un si bon fondement. M. le duc d'Orléans n'était pas encore revenu avec
le roi, ni avec le monde des premiers bruits excités contre lui. Ceux
qui les avaient tramés avaient su ne les pas laisser s'évanouir. Ces
derniers les réchauffèrent, et formèrent un étrange groupe, sous lequel
il n'y eut qu'à baisser la tête et ployer les épaules.

Un intérêt domestique affligeait encore M. {[}le duc{]} et
M\textsuperscript{me} la duchesse d'Orléans. Ils avaient éprouvé ce dont
leur fille avait été capable ayant un fils de France pour époux. Ils
comprirent donc aisément quel essor elle était capable de prendre veuve,
et ils avaient raison d'en trembler. M. le duc d'Orléans, attaqué et
miné de la sorte, était l'unique prince légitime qui eût âge d'homme.

Jamais aussi ne vit-on M. du Maine si solaire et si désinvolte qu'alors.
On voyait qu'il se cachait encore plus qu'à l'ordinaire\,; mais, dans le
peu qu'on l'apercevait quelquefois, on sentait qu'il se tenait à quatre,
et toutefois qu'il ne touchait pas à terre. Jamais les Guise si
accueillants qu'il se le montra malgré lui en partie, et en partie il
voulait l'être, parce qu'il voulait tout gagner. Tout cela, et tout à la
fois, se sentait comme au nez. À peine osait-on s'en couler un demi-mot
à l'oreille entre les plus clairvoyants et les plus sûrs l'un de
l'autre. M\textsuperscript{me} du Maine gardait moins de mesures. Elle
triomphait à Sceaux\,; elle y nageait dans les plaisirs et les fêtes, et
M. du Maine, qui, assis vers la porte, en faisait les honneurs plus
souvent qu'il n'eut voulu, en paraissait embarrassé et honteux.

Les obsèques de M. le duc de Berry furent un peu cavalières. Cela fut
pitoyable aux Tuileries. Les évêques prirent des fauteuils et des
carreaux pour garder. Dreux les laissa faire. Ce rut la première fois
que cette usurpation eut lieu. Les princes du sang, les ambassadeurs,
les ducs allèrent en manteaux à l'eau bénite, et les compagnies\,; tout
cela reçu par les principaux officiers en forme de maison et conduits.
Le comte de Charolais et le duc de Fronsac conduisirent le jeudi 10 mai
le cœur au Val-de-Grâce. M. le duc d'Orléans devait mener le corps à
Saint-Denis, il pria le roi de l'en dispenser\,; M. le duc en fut chargé
à sa place avec le duc de La Trémoille. Ce fut le mercredi 16 mai. La
décence fut fort observée chez M\textsuperscript{me} la duchesse de
Berry, à quoi M\textsuperscript{me} de Saint-Simon eut grande attention.
Les fils et petits-fils de France tendent leurs appartements chez le
roi, ce que ne peuvent faire les princes du sang. M\textsuperscript{me}
la Duchesse même, malgré les distinctions de la bâtardise, n'eut rien de
veuve dans le sien.

Celui de M\textsuperscript{me} la duchesse de Berry fut entièrement
fermé et sans jours, c'est-à-dire la chambre où elle était\,; le reste
n'était que tendu. Cette précaution fut prise pour qu'on ne la vit pas
dans son lit\,; et la première fois que le roi y vint, on ne donna de
jour qu'au moment qu'il entra pour qu'il vît à se conduire. Personne que
lui n'eût ce privilège, ce, qui causa force scènes ridicules et des
rires assez indécents qu'on avait peine à retenir. Les personnes
habitantes de la chambre étaient accoutumées, à y voir un peu, mais
celles qui venaient du grand jour n'y voyaient rien, trébuchaient et,
avaient besoin de secours. Le P. du Trévoux et le P. Tellier après lui
firent leur compliment à la muraille, d'autres au pied du lit\,; cela
devint un amusement secret. Les dames et le domestique étaient affligés,
mais il arrive des accidents ridicules qui surprennent le rire, et puis
on en est honteux. Cet aveuglement factice ne dura que le moins qu'on
put.

\hypertarget{chapitre-vii.}{%
\chapter{CHAPITRE VII.}\label{chapitre-vii.}}

1714

~

{\textsc{Le roi voit en particulier le cardinal del Giudice, tous deux
avec surprise\,; et peu après l'électeur de Bavière.}} {\textsc{- Mort
de La Taste\,: sa femme.}} {\textsc{- Mort du duc de Guastalla.}}
{\textsc{- Cardinal de Bouillon à Rome.}} {\textsc{- Mort, naissance et
caractère de la maréchale d'Estrées douairière.}} {\textsc{- Congrès de
Bade.}} {\textsc{- Camps de paix.}} {\textsc{- Nesle quitte le
service\,; en est puni.}} {\textsc{- Succession de M. le duc de Berry.}}
{\textsc{- Deux cent mille livres d'augmentation de pension à
M\textsuperscript{me} la duchesse de Berry.}} {\textsc{- Canal de
Mardick.}} {\textsc{- Trente mille livres d'augmentation de pension à
Ragotzy, et quarante mille livres de pension à distribuer dans son
parti.}} {\textsc{- Survivances des gouvernements du duc de Beauvilliers
à son gendre et à son frère.}} {\textsc{- Mort et caractère de la
duchesse de Lorges.}} {\textsc{- Des Forts conseiller d'État.}}
{\textsc{- Mort et caractère de Saint-Georges, archevêque de Lyon.}}
{\textsc{- Mort de Matignon, évêque de Lisieux.}} {\textsc{- Petite
sédition à Lyon\,; le maréchal de Villeroy y va.}} {\textsc{- Chalais à
Paris\,; Giudice à Marly.}} {\textsc{- Le roi, à qui il échappe un mot
inintelligible sur la princesse des Ursins, résout entièrement sa
perte.}} {\textsc{- L'Espagne signe la paix sans plus parler de
souveraineté pour la princesse des Ursins.}} {\textsc{- Soixante-huit
bataillons français avec Berwick pour le siège de Barcelone.}}
{\textsc{- Giudice, puis Chalais, voient le roi en particulier.}}
{\textsc{- Ducasse, malade, revient\,; remplacé par Bellefontaine.}}
{\textsc{- Mort de Menager\,; son caractère.}} {\textsc{- Duchesse de
Berry blessée d'une fille.}} {\textsc{- M\textsuperscript{me} de
Saint-Simon, par méprise du roi, la conduit à Saint-Denis, et le coeur
au Val-de-Grâce.}} {\textsc{- Mort de la première électrice d'Hanovre.}}
{\textsc{- Mort, naissance, famille et caractère de la duchesse de
Bouillon.}} {\textsc{- Mariage de La Mothe avec M\textsuperscript{lle}
de La Roche-Courbon\,; et d'une fille du marquis de Châtillon avec
Bacqueville.}} {\textsc{- Mariage de Creuilly avec une Spinola.}}
{\textsc{- Giudice établi à Marly.}} {\textsc{- Berwick part pour faire
le siège de Barcelone.}} {\textsc{- Chalais donne part particulière au
roi du mariage du roi d'Espagne avec la princesse de Parme.}} {\textsc{-
Giudice voit aussitôt après le roi en particulier.}} {\textsc{- Retraite
de Bergheyck\,; il arrive d'Espagne, vient à Marly.}}

~

Le roi vécut à son ordinaire à Marly dès aussitôt après la perte de son
petit-fils, mais les musiques chez M\textsuperscript{me} de Maintenon ne
recommencèrent que quelques jours après le retour à Versailles. Il fit
entrer le cardinal del Giudice un matin dans son cabinet, qui ne s'y
attendait point, peu de jours après la mort de M. le duc de Berry. Il le
croyait chargé de quelque affaire qu'il ne voulait pas être sue des
ministres, et le roi était seul, mais le cardinal ne lui dit rien de
nouveau, et montra ainsi le vide de sa commission.

L'électeur de Bavière vint peu de jours après de Saint-Cloud, sur les
six heures du soir, à Marly. Il entra d'abord dans le cabinet du roi. Il
y demeura tête à tête un quart d'heure, et s'en retourna tout de suite à
Saint-Cloud. Il revint le lendemain courre le cerf, et ne vit le roi
qu'à la chasse.

Le gros La Taste mourut subitement à Versailles\,: c'était une manière
de gros brutal que le roi traitait bien, et que tout le monde
connaissoit, parce qu'il avait passé presque toute sa vie aide-major des
gardes du corps. Il se retira, demeura à Versailles, ne connaissant
point d'autre pays, et se maria par inclination. Il était pourtant fort
vieux, et il avait plus de quatre-vingts ans quand il mourut. Le roi
laissa deux mille livres de pension à cette femme, qui était jolie et
qui avait des protecteurs. Chamlay prit soin d'elle, et il prit soin de
lui quand il fut vieux et apoplectique. Elle n'y perdit pas.

En même temps mourut le duc de Guastalla, qui aurait dû succéder au duc
de Mantoue si l'empereur, qui s'était emparé de ses États pendant la
guerre, n'eût mieux aimé les garder à la paix. La grandeur d'âme, la
fidélité et la valeur personnelle de Louis XIII au célèbre Pas de Suse,
son opiniâtreté et sa capacité pour le forcer, avait sauvé autrefois la
maison de Gonzague des griffes de la maison d'Autriche\,; mais ce héros
n'était plus.

Le cardinal de Bouillon était enfin arrivé des Pays-Bas à Rome. Il
semblait que ce fût malgré lui, tant il avait prolongé son voyage. Tous
les François et les attachés à la couronne eurent défense de le voir, et
de tout commerce. Les cardinaux Gualterio et de La Trémoille eurent
permission de l'aller voir une seule fois comme doyen du sacré collège,
et reçurent d'ailleurs la même défense que tous les autres François. Le
cardinal de Bouillon fit à Rome une figure triste, et y parut fort
délaissé et fort peu considéré.

La maréchale d'Estrées, douairière, mourut à Paris. Elle avait eu à
Marly, ce voyage-ci, dont elle ne manquait guère aucun, un logement tout
neuf qui la tua. Elle s'y trouva fort mal, se fit porter à Paris, et y
mourut bientôt après. Elle était fille d'un riche financier nommé Morin,
qu'on n'appelait que Morin le juif. C'était une grande et assez grosse
femme, de bonne mine, quoique avec des yeux un peu en dedans, qui avait
une physionomie haute, audacieuse, résolue, et pleine d'esprit\,; aussi
n'a-t-on guère vu de femme qui en eût tant, qui sût tant de choses, ni
qui fût de plus excellente compagnie. Elle était brusque et pourtant
avec politesse, et savait très bien rendre ce qu'elle devait, et se le
faire rendre aussi. Elle avait passé sa vie à la cour, et dans le
meilleur du plus grand monde, jouant gros jeu nettement et avec
jugement. On la craignait fort, et on ne laissait pas de la rechercher.
Elle passait pour méchante. Elle ne l'était que par dire franchement et
très librement son avis de tout, souvent très plaisamment, toujours avec
beaucoup d'esprit et de force, et de n'être pas d'humeur à rien
souffrir. Dangereuse alors à se lâcher en peu de mots d'une manière
solide et cruelle, et à parler en face aux gens, à les faire rentrer
sous terre. D'ailleurs n'aimant ni les querelles ni à médire pour
médire, mais à se faire considérer et compter, et elle l'était beaucoup,
et vivait très bien dans sa famille.

Elle était avare à l'excès, et en riait la première\,; avec cela
brocanteuse, se connaissoit aux choses et aux prix, avait le goût
excellent et ne se refusait rien. Quand il lui prenait fantaisie de
donner un repas, rien de plus choisi, de plus exquis ni de plus
magnifique. Elle était bonne amie, de très bon conseil, fidèle et sûre,
et sans être de ses amis on ne risquait jamais à parler devant elle.

M\textsuperscript{lle} de Tourbes\footnote{On appelait aussi cette fille
  de la maréchale d'Estrées M\textsuperscript{lle} de Tourpes.} qui
n'avait pas moins d'esprit qu'elle, et de la même sorte, mais plus
impérieux et plus aigre, se laissa un jour tomber à Marly, au milieu du
salon, chargée de pierreries, en dansant au bal devant le roi. Sa mère
qui, comme les vieilles, était assise au second rang, escalada le
premier, courut à sa fille, et sans s'informer si elle était blessée,
car elle était encore par terre, ne pensa qu'aux pierreries. On en rit
beaucoup, elle aussi.

Elle lui laissa plus de huit cent mille livres\,; presque autant au
maréchal d'Estrées son fils\,; à M\textsuperscript{me} de Courtenvaux et
à l'abbé d'Estrées ses autres enfants six cent mille livres chacun, sans
compter un amas prodigieux de meubles, de bijoux, de porcelaines\,; de
la vaisselle en quantité et des pierreries. Elle avait soixante-dix-sept
ou soixante-dix-huit ans, avait l'esprit et la santé comme à quarante,
et sans ce logement neuf aurait encore vécu très longtemps. Quoiqu'elle
aimât peu de gens, elle fut regrettée, mais avec tout son esprit elle
n'aurait jamais pu durer hors de la cour et du grand monde. Elle vivait
bien avec sa belle-fille et avec les Noailles, et ne laissait pas d'être
excellente sur eux et avec eux.

Le mercredi 16 mai, jour du convoi de M. le duc de Berry, le roi quitta
ce funeste Marly et retourna à Versailles. En même temps, le prince
Eugène manda au maréchal de Villars que le comte de Goëz et le baron
Seylern, plénipotentiaires de l'empereur avec lui à Bade, s'y
acheminaient, et qu'ils avaient les pouvoirs de l'empire pour ce qui le
concernait. On fit partir aussitôt Saint-Contest, et Villars, qui ne
tarda pas à le suivre, se mesura sur l'arrivée du prince Eugène à Bade.
En même temps, on fit deux camps de paix pour exercer les troupes, qui
n'en avaient pas grand besoin, mais ce ne fut que de la cavalerie pour
consommer les fourrages dont on avait trop de magasins.

Le marquis de Nesle, qui avait la compagnie écossaise de la gendarmerie,
se sentant peu propre au service, la vendit à son cousin germain le
comte de Mailly, qui n'y fit pas plus de fortune. Nesle l'avait achetée
deux cent dix mille livres. Le roi, qui n'aimait pas qu'on quittât le
service de si bonne heure, la taxa à cent cinquante mille livres.

Le roi dit le soir après souper dans son cabinet, à Madame, qu'il
voulait être tuteur de M\textsuperscript{me} la duchesse de Berry et de
l'enfant dont elle était grosse. Il avait, le même jour, envoyé Voysin
et Pontchartrain faire l'inventaire des pierreries de M. le duc de
Berry. Celles que M\textsuperscript{me} la duchesse de Berry avaient
apportées lui furent rendues, celles que M. le duc de Berry avait à lui
avant son mariage furent réservées à l'enfant qui naîtrait, les acquises
depuis partages entre la mère et l'enfant. En même temps, le roi donna à
M\textsuperscript{me} la duchesse de Berry deux cent mille livres
d'augmentation de pension.

La perte de Dunkerque, dont les Anglais avaient exigé la ruine des
fortifications et du port, fit imaginer un canal à Mardick, pour y faire
peu à peu un port en supplément. Le Blanc, intendant de cette province,
le proposa à Pelletier, chargé de l'intendance des fortifications et du
génie. Cela fut fort goûté, et on se mit à y travailler avec chaleur.
Les Anglais s'en sont fort scandalisés dans tous les temps\,; on leur a
répondu qu'on ne faisait rien en cela contre les conventions de la paix,
et cet ouvrage, quoique quelquefois interrompu par leurs cris et leurs
menaces, a assez bien réussi, en sorte qu'on n'a cessé depuis de
l'augmenter.

Ragotzi avait du roi six cent mille livres au denier
vingt-cinq\footnote{Quatre pour cent.} sur l'hôtel de ville, mais dont
les deux cinquièmes étaient retranchés, et vingt-quatre mille écus de
pension. Il eut en ce temps-ci dix mille écus d'augmentation de pension,
et de plus une autre de quarante mille livres à distribuer à son gré
entre les principaux de son parti dont les biens de Hongrie étaient
confisqués. M. de Beauvilliers, encore malgré tout ce que je lui pus
dire, fit donner au duc de Mortemart la survivance de son gouvernement
du Havre de Grâce, qui est indépendant et vaut trente-trois mille livres
de rente, et au duc de Saint-Aignan celle de Loches, qui ne vaut rien,
mais qui est au milieu des terres qu'il lui a données en le mariant. La
justice y eut plus de part que l'inclination. Il prétendait qu'il devait
ce dédommagement à son gendre des avantages qu'il a faits à son frère.

La duchesse de Lorges, troisième fille de Chamillart, mourut à Paris, en
couche de son second fils, le dernier mai, jour de la Fête-Dieu, dans sa
vingt-huitième année. C'était une grande créature, très bien faite, d'un
visage agréable, avec de l'esprit et un naturel si simple, si vrai, si
surnageant à tout, qu'il en était ravissant\,; la meilleure femme du
monde et la plus folle de tout plaisir, surtout du gros jeu. Elle
n'avait quoi que ce soit des sottises de gloire et d'importance des
enfants des ministres\,; mais tout le reste, elle le possédait en plein.
Gâtée dès sa première jeunesse par une cour prostituée à la faveur de
son père, avec une mère incapable d'aucune éducation, elle ne crut
jamais que la France ni le roi pût se passer de son père. Elle ne connut
aucun devoir, pas même de bienséance. La chute de son père ne put lui en
apprendre aucun, ni émousser la passion du jeu et des plaisirs. Elle
l'avouait tout le plus ingénument du monde, et ajoutait après qu'elle ne
pouvait se contraindre. Jamais personne si peu soigneuse d'elle-même, si
dégingandée\,: coiffure de travers, habits qui traînaient d'un côté, et
tout le reste de même, et tout cela avec une grâce qui réparait tout. Sa
santé, elle n'en faisait nul compte\,; et pour sa dépense, elle ne
croyait pas que terre pût jamais lui manquer. Elle était délicate, et sa
poitrine s'altérait. On le lui disait\,: elle le sentait, mais de se
retenir sur rien, elle en était incapable. Elle acheva de se pousser à
bout de jeu, de courses, de veilles en sa dernière grossesse. Toutes les
nuits elle revenait couchée en travers dans son carrosse. On lui
demandait en cet état quel plaisir elle prenait. Elle répondait d'une
voix qui de faiblesse avait peine à se faire entendre qu'elle avait bien
du plaisir. Aussi finit-elle bientôt. Elle avait été fort bien avec
M\textsuperscript{me} la Dauphine et dans la plupart de ses confidences.
J'étais fort bien avec elle\,; mais je lui disais toujours que pour rien
je n'eusse voulu être son mari. Elle était très douce, et pour qui
n'avait que faire à elle, fort aimable. Son père et sa mère en furent
fort affligés.

Orcey, frère de feue M\textsuperscript{me} de Montchevreuil, qui avait
été prévôt des marchands, mourut en même temps. Il était conseiller
d'État. Sa place fut donnée à des Forts, qui a depuis été deux fois
contrôleur général, et qui était lors encore fort jeune, fils de
Pelletier de Sousy et intendant des finances.

Saint-Georges, archevêque de Lyon, y mourut, prélat pieux, décent,
réglé, savant, imposant, résidant et de grande mine avec sa haute taille
et ses cheveux blancs. Il y avait longtemps que cette grande église,
dont il avait été chanoine ou comte, comme ils les nomment, et
archevêque de Tours, n'avait vu d'évêque\,; et depuis lui elle n'en a
pas vu, j'entends des évêques qui prissent la peine de l'être. Bientôt
après mourut l'évêque de Lisieux, frère du comte et du maréchal de
Matignon.

Il y eut un petit désordre à Lyon pour une imposition que la ville avait
nouvellement mise sur la viande. Les bouchers excitèrent le peuple, dont
quantité prit les armes et fit une assez grande sédition, tellement que
Méliand, intendant, fut obligé d'ôter l'imposition, et apaisa tout par
là. Cette imposition n'avait pas été trop approuvée\,: ainsi l'intendant
le fut. Le maréchal de Villeroy, qui sur tous les hommes du monde aimait
à se faire de fête, se trouvait lors à Villeroy avec un peu de goutte.
Il écrivit au roi pour lui permettre d'aller à Lyon. Il l'obtint et
partit. On envoya ordre à quelques troupes du camp de la Saône d'y
marcher, et le maréchal de Villeroy trouva en arrivant qu'il n'y avait
plus rien à faire\,; mais il ne laissa pas d'y demeurer. Au moins
était-il mieux là qu'à la tête d'une armée.

Chalais, qu'on a vu mandé de l'armée destinée à Barcelone, s'était peu
arrêté à Madrid. Il était arrivé à Paris, dépêché par la princesse des
Ursins, et elle l'avait chargé de lettres pour le cardinal del Giudice.
La corde venait de casser par le roi sur sa souveraineté, et la paix
{[}était{]} enfin conclue avec l'Espagne, sans en faire mention,
laquelle était demeurée seule en arrière accrochée sur ce point. Dans
ces entrefaites, le roi alla, le mardi 29 mai, à Marly, et y donna un
logement au cardinal del Giudice.

J'étais du voyage à mon ordinaire, quoique M\textsuperscript{me} de
Saint-Simon fût restée à Versailles auprès de M\textsuperscript{me} la
duchesse de Berry. Le roi n'avait pas ouï parler encore par le roi
d'Espagne qu'il pensât se remarier, beaucoup moins à une fille de
Parme\,; mais il en était informé d'ailleurs. Ce procédé, enté sur la
souveraineté prétendue par la princesse des Ursins et sur toute sa
conduite avec le roi d'Espagne depuis la mort de la reine, mit le sceau
à la résolution de la perdre sans retour.

Il échappa au roi, toujours si maître de soi et de ses paroles, un mot
et un sourire sur M\textsuperscript{me} des Ursins tellement
énigmatique, quoique frappant, que Torcy, à qui il le dit, n'y comprit
rien. Dans sa surprise il le conta à Castries, son ami intime, et
celui-ci à M\textsuperscript{me} la duchesse d'Orléans, qui le conta à
M. le duc d'Orléans et à moi. Nous nous cassâmes vainement la tête pour
y comprendre quelque chose. Toutefois un mot si peu intelligible sur une
personne comme M\textsuperscript{me} des Ursins, et qui jusqu'à ces
derniers temps avait été si parfaitement avec le roi et avec
M\textsuperscript{me} de Maintenon, ne me parut pas favorable. J'y étais
confirmé par ce qui venait de se passer sur sa souveraineté, mais à
mille lieues de la foudre que cet éclair annonçait, et qui ne nous le
développa que par sa chute. Mais il n'est pas temps encore d'en parler.

Le mariage de Parme était conclu, et le roi n'en ouït point encore
parler de quelque temps de la part de l'Espagne. Tout portait à croire
néanmoins que Chalais n'était venu que pour cette affaire, que les
dépêches qu'il avait apporiées au cardinal del Giudice la regardaient.
Peut-être s'en trouvèrent-ils embarrassés, et qu'ils différèrent. Je
n'en ai pas pénétré davantage là-dessus. Peut-être aussi cela ne
regardait-il encore que la souveraineté manquée, et l'ordre envoyé aux
plénipotentiaires d'Espagne de signer la paix, sans en plus parler. Quoi
qu'il-en soit, Chalais apporta lui-même les paquets dont il était chargé
au cardinal del Giudice à Marly. Il s'en retourna sans voir le roi ni
personne. C'était le samedi 2 juin.

Le lendemain dimanche 3, le roi, satisfait enfin de l'ordre du roi
d'Espagne envoyé à Utrecht, fit entrer le duc de Berwick dans son
cabinet, à qui il ordonna de se tenir prêt à partir pour le siège de
Barcelone avec soixante-huit bataillons français, à qui en même temps on
envoya ordre d'y marcher, et quatre lieutenants généraux, et quatre
maréchaux de camp français, outre ceux qui y sont déjà. Le duc de
Mortemart obtint d'y être le cinquième de ces maréchaux de camp. On
remarquera en passant que ce départ fut bien retardé, tandis que les
Espagnols en corps d'armée se morfondaient en Catalogne, sous le duc de
Popoli qui s'en retourna vilainement à Madrid dès que le siège commença.
Brancas, courant au plus fort avec le cardinal del Giudice, avait eu
ordre, comme on l'a vu, de s'arrêter en chemin, où il rencontrerait
Berwick, pour conférer avec lui. Le roi sans doute s était ravisé sur
l'opiniâtreté de l'Espagne à arrêter la paix sur la souveraineté de
M\textsuperscript{me} des Ursins. Il y avait longtemps que Brancas et le
cardinal étaient arrivés, sans qu'il fût mention du départ de Berwick ni
des troupes qui lui étaient destinées, et l'ordre n'en fut donné, comme
on le voit, qu'immédiatement après que le roi fut assuré que le roi son
petit-fils avait enfin envoyé les siens à Utrecht de signer sans plus
songer à la souveraineté.

Aussitôt après que le duc de Berwick fut sorti du cabinet du roi, il y
fit entrer le cardinal del Giudice, apparemment pour lui dire ce qu'il
venait de commander, et trois jours après, Chalais revint passer
quelques heures à Marly, où Torcy le mena pour quelques moments dans le
cabinet du roi.

Ducasse, retombé malade à la mer, demanda son congé. On le fit remplacer
par Bellefontaine, lieutenant général.

Menager, troisième plénipotentiaire à Gertruydemberg et à Utrecht, dont
on a suffisamment parlé alors pour le faire connaitre, mourut
d'apoplexie à Paris, fort riche, sans avoir été marié. Ce fut dommage
pour sa probité, sa modestie, sa capacité dans le commerce et son
intelligence dans les affaires. Il n'était point vieux.

M\textsuperscript{me} la duchesse de Berry se blessa dans sa chambre, le
samedi 16 juin, d'une fille qui ne vécut que douze heures. Le roi, qui
était à Rambouillet, nomma M\textsuperscript{me} de Saint-Simon, comme
duchesse, pour mener ce petit corps à Saint-Denis, et le cœur au retour
au Val-de-Grâce. Deux heures après il dit qu'il l'avait nommée parce
qu'elle lui était venue la première dans l'esprit comme étant à
Versailles, et M\textsuperscript{me} de Pompadour de même pour femme de
qualité, mais que s'il eût pensé que l'une était dame d'honneur, l'autre
gouvernante, laquelle par son emploi y devait toujours aller, il aurait
nommé une autre duchesse et une autre dame. Mais la chose était faite et
de Rambouillet, et M\textsuperscript{me} de Saint-Simon en eut la
corvée. L'évêque de Séez, premier aumônier de feu M. le duc de Berry,
était avec elle, et à droite au fond du carrosse, portant le cœur\,;
M\textsuperscript{me} de Pompadour et M\textsuperscript{me} de
Vaudreuil, gouvernante et sous-gouvernante au devant\,; le curé à la
portière\,; et à l'autre portière le petit corps\,; des gardes, des
pages, des carrosses de suite. Il en eurent pour quatorze ou quinze
heures.

La princesse Sophie, palatine, veuve du premier électeur d'Hanovre, et
mère du premier Hanovre roi d'Angletterre, mourut à quatre-vingts ans.
Elle était fille de la sœur du roi Charles Ier d'Angleterre, qui eut la
tête coupée, et fille de l'électeur palatin, à qui il en prit si mal de
s'être voulu faire roi de Bohême. Ce fut par elle que le droit à la
couronne d'Angleterre vint à la maison d'Hanovre, non qu'indépendamment
de la ligne royale des Stuarts il n'y eût plusieurs héritiers plus
proches, mais tous catholiques, et elle était la plus proche d'entre les
protestants. C'était une princesse de grand mérite, qui avait
quatre-vingts ans. Elle avait élevé Madame, qui était fille de son
frère, laquelle avait conservé un extrême attachement pour elle, et qui
toute sa vie lui écrivit, deux fois la semaine, des vingt à vingt-cinq
pages par ordinaire. C'était à elle à qui elle écrivait ces lettres si
étranges que le roi vit, et qui la pensèrent perdre à la mort de
Monsieur, comme on l'a vu alors. Elle fut affligée au dernier point de
la perte de cette tante.

M. de Bouillon avait eu une assez grande maladie à Versailles, dont on
crut même qu'il ne reviendrait pas. Lorsqu'il se trouva en état de
changer d'air, il alla le prendre à Clichy. M\textsuperscript{me} de
Bouillon l'y alla voir de bonne heure le mercredi 20 juin. En entrant
dans sa chambre elle se trouva si mal et si subitement, qu'elle tomba à
ses pieds et y mourut à l'instant même. Elle avait eu deux ou trois
attaques d'apoplexie si légères qu'elles furent traitées d'indigestion,
et qu'elle ne prit aucune sorte de précaution. Elle avait soixante-huit
ans, et on voyait encore en elle de la beauté et mille agréments. Cet
épouvantable spectacle fut regardé de tout le monde comme une amende
honorable à son mari de sa conduite, dont elle ne s'était jamais
contrainte un moment, au point qu'elle ne voyait que très peu de femmes
qui n'avaient rien à perdre, mais la meilleure et la plus florissante
compagnie en hommes, dont sa maison, d'où elle ne sortait guère, était
le rendez-vous, avec grand jeu et grande chère. Mais sur la fin elle
était devenue avare, et avait éclairci sa compagnie par son humeur, sa
mauvaise chère, et se faire donner à souper partout où elle pouvait.

Elle avait été mariée en 1662, et elle était la dernière des nièces du
cardinal Mazarin, mort 9 mars 1661, au château de Vincennes, où il
s'était fait porter. Elle était née à Rome en 1646, de Michel-Laurent
Mancini, mort en 1657, et d'une soeur du cardinal Mazarin, mariée en
1634, et morte en 1656. Ces Mancini ne sont connus depuis 1380 que par
des contrats d'acquisitions et de vente du prix de quarante ou cinquante
florins, et des dots de quarante et cinquante ducats jusque très tard.
Jamais aucun emploi de nulle sorte, jamais ni fiefs ni terre, jamais une
alliance qui se puisse nommer, ni active ni passive. On trouve vers 1530
une Jacqueline Mancini, mariée à Jean-Paul Orsini\,; mais ce Jean-Paul
est entièrement ignoré par Imhof\footnote{Voy. t. III, p. 265, où il est
  question des \emph{Recherches} d'Imhof sur la noblesse espagnole.},
qui est exact et instruit des maisons d'Italie, et ne se trouve nulle
part. On ne voit même personne de la maison Ursine qui ait porté le nom
de Jean-Paul. Ajoutez à cette obscurité les alliances actives et
passives contemporaines des Mancini, celle de cet inconnu n'imposera
pas.

Une seule acquisition d'un château ruiné et quelque terre autour, aux
portes de Rome, appelé Leprignana, de Jacques Conti pour cinq mille
florins, revendue longtemps après quarante mille écus à un Justiniani,
fait toute leur illustration. On voit aussi que, vers les temps de cette
vente, leurs dots passaient mille ducats, et vers ces mêmes temps un
Laurent Mancini est dit avoir servi les Vénitiens avec distinction, mais
en quelle qualité\,? c'est ce qui n'est point exprimé. Enfin Paul
Mancini, grand-père de M\textsuperscript{me} de Bouillon, servit en 1597
à la guerre de Ferrare, on ne dit point encore en quelle qualité, épousa
en 1600 Vittoria Capoccia, fille de Vincent se qualifiant patrice
romain, et en eut quinze mille écus de dot. Voilà l'illustre de la race.
Il revint à Rome, s'adonna à l'étude, et l'académie des Humoristes prit
naissance dans sa maison. Enfin devenu veuf, il prit l'habit
ecclésiastique, il laissa trois fils et deux filles. L'une épousa en
1624 Jacques Velli, l'autre Sartorio Teofilo. Jusqu'ici les alliances ne
brillent pas\,; les trois fils furent Laurent, qui épousa la sœur du
cardinal Mazarin, longtemps avant sa fortune, et qui mourut en 1657,
veuf depuis un an. Le second, Fr.~Marie Mancini, eut par la nomination
du roi le chapeau de cardinal en 1660. Il était né en 1606 et mourut en
1672. Le troisième, Laurent-Grégoire, qui était de 1608, mourut jeune et
obscur\,: aucun des trois ne sortit d'Italie.

Michel-Laurent Mancini n'eut aucun emploi, point de terres connues, ne
brilla pas plus que ses pères, et comme eux, vécut en citadin obscur à
Rome, et fort inconnu. Ses enfants furent plus heureux. Le cardinal
Mazarin en fit comme des siens, et les fit venir en France. Il y avait
trois garçons et cinq filles\,; deux autres étaient mortes à Rome
enfants.

L'aîné des fils fut tué au combat de Saint-Antoine, en 1652, tout jeune.
Il promettait beaucoup et la fortune encore davantage. Le cardinal
Mazarin en fut très affligé. M. de Nevers était le second, dont il a été
parlé en son lieu. Le troisième, qui ne promettait pas moins pour son
âge que l'aîné, mourut à quatorze ans, en 1658. Il était au collège des
jésuites. La jalousie que quelques écoliers conçurent des distinctions
qu'il y avait les poussa à le berner dans une couverture. Il en tomba,
et se blessa tellement qu'il en mourut, dont le cardinal Mazarin fut
outré. Cet exemple, et celui du fils aîné du maréchal de Boufflers par
les jésuites mêmes, avec bien d'autres, montrent que ce collège des
jésuites n'est pas un lieu sûr pour ceux que la fortune élève dès leur
première jeunesse. Voici maintenant les filles\footnote{Voy. notes à la
  fin du volume p.~399.}\,:

Laure-Victoire, mariée, 4 février 1651, au duc de Mercœur, fils aîné du
duc de Vendôme, bâtard d'Henri IV, puis duc de Vendôme, morte à Paris, 4
février 1657, mère du dernier duc de Vendôme, dont il a été tant parlé
en ces Mémoires, et du grand prieur de France. Elle n'avait pas vingt et
un ans encore. Son mari fut cardinal en mars 1667, et mourut en août
1668.

Olympe, mariée, 20 février 1657, à Eugène-Maurice de Savoie, comte de
Soissons, colonel général des Suisses et Grisons, gouverneur de
Champagne et Brie, dont, entre autres enfants, elle eut le comte de
Soissons et le fameux prince Eugène. J'ai tant parlé d'elle en divers
endroits que je n'ai rien à y ajouter.

Marie, qui fut l'objet des premières amours du roi, qui la voulait
épouser. Cette raison la fit dépayser et marier à Rome, en avril 1661,
au connétable Colonne, qu'elle perdit en 1689. On aura lieu de parler
d'elle encore.

Hortense, qui, avec vingt-huit millions de dot, des dignités, des
gouvernements, etc., et l'obligation de prendre en seul le nom et les
armes de Mazarin, épousa le duc Mazarin, fils unique du maréchal de La
Meilleraye, desquels aussi on a suffisamment parlé.

Enfin Marie-Anne, mariée, 20 avril 1662, au duc de Bouillon qui avait
acheté en 1658 de la maison de Guise la charge de grand chambellan de
France.

Ajoutons à tant de grandeur que la sœur aînée du cardinal Mazarin avait
épousé en 1634 Jérôme Martinozzi, soi-disant gentilhomme romain, dont
elle n'eut que deux filles que le cardinal Mazarin maria aussi
passablement, l'aînée en 1655 à Alphonse d'Este, duc de Modène, et la
reine d'Angleterre, épouse de Jacques II, morts à Saint-Germain, était
leur fille\,: l'autre au prince de Conti, frère de M. le Prince le
héros, dont deux fils\,: l'aîné mort fort jeune, gendre naturel du
roi\,; l'autre si connu par sa réputation, qui fut un instant roi de
Pologne, et dont le prince de Conti d'aujourd'hui est petit-fils. Ainsi
M\textsuperscript{me} de Bouillon, avec quatre sœurs si grandement
établies, se trouvait comme elles cousine germaine de la princesse de
Conti et de la duchesse de Modène, mère de la reine d'Angleterre
réfugiée en France. Le cardinal Mazarin avait doté ses sept nièces, et
on peut imaginer comment, pour les placer si haut d'une naissance si
persévéramment basse, pauvre et obscure. Ajoutez-y les vingt-huit
millions de sa véritable héritière, les biens qu'il donna à M. de
Nevers, dont le duché est une province, les meubles, les maisons, les
bijoux, les pierreries, les statues et les tableaux, les gouvernements
et les charges, et on verra ce que c'est qu'un premier ministre pour un
roi, pour ses sujets, pour un royaume. Encore faut-il avouer que cet
effréné pillage en est le plus léger et le moins dangereux, peut-être
encore le moins honteux de tous les inconvénients, et sûrement, quelque
monstrueux qu'il soit, le moins nuisible.

Si les pères de ces nièces n'étaient rien, leurs mères, sœurs du
cardinal Mazarin, étaient, s'il se peut, encore moins. Jamais on n'a pu
remonter plus haut que le père de cette trop fameuse Éminence, ni savoir
où elle est née, ni quoi que ce soit de sa première jeunesse\,; tout ce
qui l'a suivie est si connu qu'on n'en parlera pas ici. On sait
seulement qu'ils étaient de Sicile\,; on les a crus des manants de la
vallée de Mazzare qui avaient pris le nom de Mazarin, comme on voit à
Paris des gens qui se font appeler Champagne et Bourguignon. La mère du
cardinal était Buffalini. On ignore toutes les antérieures puisqu'on ne
sait rien des Mazarin. Le père du cardinal vécut si obscur toute sa vie
à Rome, que lorsqu'il y mourut en novembre 1654 à soixante-dix-huit ans,
cela n'y fit pas le moindre bruit. Les nouvelles publiques de Rome
eurent la malice d'y insérer ces mots\,: «\,Les lettres de Paris nous
apprennent que le seigneur Pietro Mazarini, père du cardinal de ce nom,
est mort en cette ville de Rome, le,\,» etc. Revenons maintenant à
M\textsuperscript{me} de Bouillon.

Avec des grandeurs en tel nombre, et si proches, M\textsuperscript{me}
de Bouillon trouva en se mariant M. de Turenne dans le comble de son
lustre et du crédit auprès du roi jusqu'à anéantir publiquement à son
égard celui des plus puissants ministres, et la comtesse de Soissons, la
reine de la cour, le centre de la belle galanterie qui dominait le
monde, de chez qui le roi ne bougeait, et qui tenait le sort de tous
entre ses mains. Ce radieux état dura longtemps, celui de M. de Turenne
jusqu'à sa mort en 1675. Elle vit de plus le frère de son mari cardinal
à vingt-six ans, en 1669, et grand aumônier en 1671, dans la plus grande
faveur\,; et son autre beau-frère recueillir la charge de la cavalerie,
et le gouvernement de M. de Turenne\,: aussi poussa-t-elle l'orgueil
jusqu'à l'audace, et un orgueil qui s'étendait à tout\,; mais comme elle
avait beaucoup d'esprit et de tour, et d'agrément dans l'esprit, elle
sentait les proportions, et avait le jugement de ne les outre-passer
guère et de couvrir son jeu de beaucoup de politesse pour les personnes
qu'il ne fallait pas heurter, et d'un air de familiarité avec les
autres, qui voiloit comme par bonté celui d'autorité. En quelque lieu
qu'elle fût, elle y donnait le ton et y paraissait la maîtresse. Il
était dangereux de lui déplaire\,; elle se refusait peu de choses, et
encore n'était-ce que par rapport à elle-même, d'ailleurs très bonne
amie, et très sûre dans le commerce.

Son air libre était non seulement hardi, mais audacieux, et, avec la
conduite dont on a d'abord touché un mot, elle ne laissa pas d'être une
sorte de personnage dans Paris, et un tribunal avec lequel il fallait
compter\,; je dis dans Paris, où elle était une espèce de reine\,; car à
la cour, elle n'y couchait jamais, et n'y allait qu'aux occasions, ou
une ou deux fois au plus l'année.

Le roi personnellement ne l'avait jamais aimée\,; sa liberté
l'effarouchait\,; elle avait été souvent exilée, et quelquefois
longtemps. Malgré cela elle arrivait chez le roi la tète haute, et on
l'entendait de deux pièces\,; ce parler haut ne baissait point de ton,
et fort souvent même au souper du roi, où elle attaquait Monseigneur et
les autres princes ou princesses qui étaient à table, derrière qui elle
se trouvait, et les dames assises auprès d'elle.

Elle traitait ses enfants et souvent aussi ses amis et ses compagnies
avec empire\,; elle l'usurpait sur les frères et les neveux de son mari
et sur les siens, sur M. le prince de Conti et sur M. le Duc même, tout
féroce qu'il était, et qui à Paris ne bougeait de chez elle. Elle
traitait M. de Bouillon avec mépris, et tous étaient plus petits devant
elle que l'herbe. Elle n'allait chez personne qu'aux occasions, mais
elle y était exacte et chez quelques amis fort particuliers\,; et ces
visites, elle y conservait un air de grandeur et de supériorité sur tout
le monde, qu'elle savait néanmoins pousser ou mesurer et assaisonner de
beaucoup de politesse selon les personnes qu'elle connaissoit très bien,
et qu'elle savait distinguer.

Sa maison était ouverte dès le matin\,; jamais femme qui s'occupât moins
de sa toilette\,; peu de beaux et de singuliers visages comme le sien
qui eussent moins besoin de secours, et à qui tout allât si bien,
toutefois toujours de la parure et de belles pierreries. Elle savait,
parlait bien, disputait volontiers, et quelquefois allait à la botte. La
splendeur dont les douze ou quinze premières années de son mariage elle
s'était vue environnée l'avait gâtée\,; ce qui lui en resta après ne la
corrigea pas\,; l'esprit et la beauté la soutinrent, et le monde
s'accoutuma à en être dominé. Tant qu'elle put elle fit la princesse, et
hasarda sur cela quelquefois des choses dont elle eut du dégoût, mais
qui ne ralentirent point cette passion en elle. En tout ce fut une perte
pour ses amis, surtout pour sa famille\,; c'en fut même une pour Paris.
Elle n'était ni grande ni menue, mais tout le reste admirable et
singulier. C'était grande table soir et matin, grand jeu et de toutes
les sortes à la fois, et en hommes la plus grande, la plus illustre et
souvent la meilleure compagnie. Au demeurant, une créature très
audacieuse, très entreprenante, par conséquent toujours embarrassante et
dangereuse. Elle sortit plus d'une fois du royaume\,; elle se promena en
Italie et en Angleterre sous prétexte de ses soeurs, et vit aussi les
Pays-Bas\,; mais elle régna moins à Rome et à Londres qu'à Paris.

Le fils aîné du comte de La Mothe épousa M\textsuperscript{lle} de La
Roche-Courbon, riche, sage et bien faite\,; et le marquis de Châtillon,
qui n'avait rien à donner à ses filles, en maria une à Bacqueville, fils
d'un premier président de la chambre des comptes de Rouen, dont le père
était un gros laboureur qui s'était fort enrichi dans les fermes qu'il
avait tenues. Le mariage ne fut pas heureux.

Creuilly, second fils de feu M. de Seignelay, ministre et secrétaire
d'État, épousa en même temps une Spinola qui n'avait rien, soeur de
celle que le fils de M. de Nevers avait épousée. Cela ne fit pas non
plus un mariage fort heureux.

Le roi était revenu de Rambouillet droit à Marly, le mardi 19 juin, d'où
il fut voir M\textsuperscript{me} la duchesse de Berry à Versailles,
sans y coucher. Je fus à mon ordinaire de ce voyage\,; j'en avertis
parce qu'il fut étrangement curieux\,; le cardinal del Giudice en fut
aussi. Dès les premiers jours du voyage, le maréchal de Berwick y prit
congé du roi, et partit pour aller faire le siège de Barcelone.

Chalais y vint, sur un courrier d'Espagne, conférer, le mardi 26 juin,
après dîner, avec le cardinal del Giudice, puis avec Torcy\,; il ne vit
point le roi, mais il revint le lendemain matin à la fin du lever du
roi, qui le fit entrer dans son cabinet avec Torcy. Sa commission était
embarrassante\,: il s'agissait de donner part au roi du mariage du roi
d'Espagne fait et conclu, et c'était la première fois que le roi
d'Espagne lui en faisait parler. L'audience finie, Chalais prit congé
pour retourner en Espagne. M\textsuperscript{me} des Ursins, inquiète de
cette hardiesse, voulut savoir par un homme uniquement à elle comment
elle aurait été reçue, et ce qu'il y aurait remarqué. Peu de moments
après que Chalais fut sorti du cabinet, le cardinal del Giudice y fut
appelé. Ce fut sur la même matière\,; tout cela ne fut su que depuis. Le
roi passa le plus doucement et le plus légèrement du monde cet étrange
mariage et le mystère si long et si entier qui lui en avait été fait,
plus étrange, s'il se peut, que le mariage même. Il ne le pouvait
empêcher, et il était sûr dès lors de sa vengeance sur celle qui l'avait
fait et achevé de la sorte.

Bergheyck arriva de Madrid, ayant, comme on l'a dit, renoncé aux emplois
et aux affaires, et allant se retirer dans une de ses terres en Flandre.
Le roi le vit longtemps dans son cabinet, et, comme il en avait toujours
été parfaitement content, il lui permit de venir à Marly toutes les fois
qu'il le voudrait. Comme il se proposa d'user souvent de cette liberté,
il se logea à Versailles, vint souvent à Marly, où le roi le distingua
toujours, et le vit plusieurs fois dans son cabinet. Avec toutes ses
mesures, sa sagesse et sa modestie, les affaires d'Espagne, qu'il
connaissoit à fond, et celles de cette cour, qu'outre ses épreuves
particulières il avait vues à revers, il ne raccommoda pas la princesse
des Ursins dans l'esprit du roi. Tant qu'il demeura en ce pays-ci il fut
fort accueilli de la cour, et toujours avec le roi et ses ministres sur
un grand pied de privance et de distinction, sans jamais sortir des
bornes de sa discrétion et de sa modestie. Cellamare eut aussi la
liberté de venir sans demander, de temps en temps à Marly faire sa cour,
mais sans coucher\,; le cardinal del Giudice l'avait obtenu ainsi.

\hypertarget{chapitre-viii.}{%
\chapter{CHAPITRE VIII.}\label{chapitre-viii.}}

1714

~

{\textsc{Retraite du chancelier de Pontchartrain.}} {\textsc{- Voysin
chancelier, et conserve sa place de secrétaire d'État.}} {\textsc{- M.
du Maine.}} {\textsc{- Mot plaisant et salé de M. de Lauzun.}}
{\textsc{- Électeur de Bavière deux fois à Marly.}} {\textsc{- Roi
Stanislas aux Deux-Ponts.}} {\textsc{- Arrivée de la flotte des Indes au
Port-Louis.}} {\textsc{- Trois mille livres d'augmentation de pension à
M\textsuperscript{me} de Saint-Géran.}} {\textsc{- Le fils de Fagon
intendant des finances.}} {\textsc{- Mariage de Brassac avec la fille du
feu maréchal de Tourville.}} {\textsc{- Reine de Pologne veuve de Jean
Sobieski\,; causes de sa haine pour la France, de son séjour à Rome, de
sa retraite à Blois.}} {\textsc{- Égalité de rois du cardinal Mazarin.}}
{\textsc{- Reine de Pologne, médiocrement reçue, ne veut aucune
réception\,; va droit à Blois, sans pouvoir approcher de la cour ni de
Paris.}} {\textsc{- Service de M. le duc de Berry à Saint-Denis.}}
{\textsc{- Prince de Dombes y fait le troisième deuil.}} {\textsc{-
Tranchée ouverte devant Barcelone, 12 juillet.}} {\textsc{- Maisons
président à mortier\,; sa femme\,; leur famille, leur caractère, leur
conduite, leur situation, leurs vues.}} {\textsc{- Désir de Maisons de
lier avec moi\,; comment il y réussit.}} {\textsc{- Première entrevue de
Maisons avec moi fort singulière.}} {\textsc{- Notre commerce
s'établit.}} {\textsc{- Maisons me fait aller de Marly le trouver.}}
{\textsc{- Il m'apprend que les bâtards et leur postérité sont devenus
princes du sang en plein, et capables de succéder à la couronne.}}
{\textsc{- Scène singulière chez Maisons.}} {\textsc{- La nouvelle se
publie à Marly, effet qu'elle y produit.}} {\textsc{- Mon compliment aux
bâtards.}} {\textsc{- Comte de Toulouse.}} {\textsc{- Cause secrète de
la conservation de la place de secrétaire d'État au nouveau
chancelier.}}

~

Le chancelier fit alors un événement qui n'avait point encore eu de
semblable et qui surprit étrangement, on pourrait ajouter funestement.
Toute sa vie il avait formé le dessein de mettre un intervalle entre la
vie et la mort, souvent il me l'avait dit. Sa femme l'avait empêché bien
des fois de se retirer avant qu'il fût chancelier, elle le retint encore
depuis, et en mourant elle lui fit promettre que, s'il voulait enfin se
retirer, il demeurerait encore six semaines à y penser. Dès qu'il alla
après sa mort à l'institution des pères de l'Oratoire, dans un petit
appartement qu'il y avait, où il se retirait les bonnes fêtes, il songea
à exécuter son dessein, et il y prit secrètement toutes ses mesures.

Elles ne purent être si cachées qu'elles ne transpirassent dans sa
famille. La Vrillière, qui en fut alarmé, m'en avertit\,; nous
consultâmes le premier écuyer lui et moi\,; ils me pressèrent de lui
parler sur les inconvénients de cette retraite pour lui-même, et pour
son fils si détesté qu'il laisserait par là à découvert. J'eus beau
dire, je ne gagnai rien.

Il attendit son terme, et il parla au roi, dont la surprise fut extrême.
Il ne croyait pas qu'un chancelier pût se démettre, et il est vrai qu'il
n'y en avait point d'exemple. Quoique l'aversion que
M\textsuperscript{me} de Maintenon avait conçue pour lui, qui, depuis la
mort de sa femme qu'elle avait toujours aimée et considérée, n'eut plus
de contre-poids\,; que cette haine et l'opinion que le roi avait prise
de longue main du jansénisme du chancelier, l'eût fort changé à son
égard\,; l'habitude et l'ancien goût qu'il avait pour lui ne laissaient
pas de prévaloir, et de se faire sentir dans toute leur étendue quand il
fut question d'une véritable séparation. Le roi n'oublia rien pour le
retenir par ses raisons et par tout ce qu'il y put ajouter de tendre, et
qui marquait le plus son estime\,; il le trouva ferme et déterminé. Le
roi se rabattit à lui demander quinze jours pour y penser encore. Ce
terme finit avec le mois de juin\,; le chancelier retourna à la charge,
et obtint enfin, quoiqu'à grand'peine, la liberté après laquelle il
soupirait, et dont il a fait un si courageux et si saint usage.

La netteté de son esprit, l'agrément de ses manières, la justesse et la
précision de ses raisonnements toujours courts, lumineux, décisifs,
surtout son antipode de pédanterie, et cet alliage qu'il savait faire
avec tant de mesure et de légèreté du respect avec la liberté, du
sérieux avec la fine plaisanterie qui était en lui des traits vifs et
perçants, plaisait toujours infiniment au roi, qui d'ailleurs était
peiné que tout homme qui l'approchait le quittât.

Le bruit de l'événement qui se préparait ne bourdonna que quatre ou cinq
jours avant l'exécution, et d'une manière encore fort douteuse. Le
dimanche 1er juillet, le chancelier resta seul assez longtemps avec le
roi après que les autres ministres furent sortis du conseil d'État, et
ce fut là où, malgré les derniers efforts du roi, le chancelier arracha
son congé. Le roi, fort attendri, lui fit donner parole de le venir voir
de temps en temps par les derrières. En entrant, en sortant, ni pendant
le conseil, à ce que dirent après les autres ministres, il ne parut quoi
que ce soit sur le visage ni dans les manières du chancelier, et la
plupart de la cour était encore dans l'incertitude.

Le lendemain lundi, 2 juillet, comme le roi fut rentré chez lui après sa
messe\,; on vit arriver le chancelier en chaise, à la porte du petit
salon d'entre l'appartement du roi et celui de M\textsuperscript{me} de
Maintenon. Comme il n'y avait point de conseil, chacun courut du grand
salon. On le vit entrer chez le roi avec la cassette des sceaux, et on
ne douta plus alors de la retraite. Ce fut une louange et une
consternation générale. Je savais la chose par lui-même. Je le vis
entrer et sortir avec le cœur bien serré, lui avec l'air de l'avoir bien
au large. Le roi le combla d'amitiés et de marques d'estime, de
confiance et de regrets\,; et sans qu'il lui demandât rien, lui donna
une pension de trente-six mille livres, et la conservation du rang et
des honneurs de chancelier. En finissant l'audience, il demanda au roi
d'avoir soin de ses deux secrétaires, qui en effet étaient de très
honnêtes gens, et sur-le-champ le roi donna à chacun une pension de deux
mille livres.

Pendant qu'il était chez le roi, la nouvelle courut, et fit amasser tout
ce qui se trouva d'hommes dans Marly qui firent presque foule sur son
passage. Il sortit de chez le roi comme il y était entré, sans qu'il
parût en rien différent de son ordinaire\,; saluant à droite et à
gauche, mais sans parler à personne, ni personne à lui. Il se mit dans
sa chaise où il l'avait laissée, gagna son pavillon, où il monta tout de
suite dans son carrosse qui l'attendait, et s'en alla à Paris. Il y fut
plus d'un mois dans sa maison en butte à ce qu'il ne put refuser les
premiers jours, puis se resserra tant qu'il put. La maison que la mort
du Charmel avait laissée tout à fait vacante, et qu'il faisait
accommoder pour lui, n'était pas encore prête. Dès qu'il y put habiter,
il s'y retira. J'aurai lieu ailleurs de parler de sa solitude, et de la
vie qu'il y mena également sainte et contente.

Outre l'âge, la douleur, et la liberté que lui donnait la perte de la
chancelière pour cette résolution de tous les temps de mettre un
intervalle entre la vie et la mort, il se sentit hâté de l'exécuter par
les événements qu'il prévoyait devenir de jour en jour plus difficiles à
soutenir dans sa place. Il voyait les desseins du P. Tellier, les
progrès de l'affaire de la constitution, le renversement des libertés de
l'Église gallicane, de celles des écoles, la persécution qui
s'échauffait, et les plus saintes barrières qui n'arrêtaient plus. Il
prévit que la tyrannie des jésuites et de leurs supports, qui avaient
transformé leur cause en celle de l'autorité du roi en ce monde et de
son salut en l'autre, se porterait peu à peu à toutes les sortes de
violences. Il n'en voulait pas être le ministre par le sceau, ni même le
témoin muet. Parler et refuser le sceau, c'était se perdre sans rien
arrêter, et ce fut une de ses plus pressantes raisons de ne différer pas
de se mettre à l'écart. Une autre, qui ne le diligenta pas moins, fut le
vol rapide qu'il voyait prendre à la bâtardise, qui, délivrée des fils
de France et des princes du sang d'âge à la contenir, ne donnerait plus
de bornes à son audace et à ses conquêtes. C'était encore un article sur
lequel on ne pouvait se passer de son ministère, auquel il avait horreur
de le prêter où ses représentations l'auraient perdu sans en pouvoir
espérer aucun fruit. La prompte suite a fait sentir toute la sagacité de
ses vues. Il avait été contrôleur général dix ans, et peu après qu'il le
fut ministre d'État, puis secrétaire d'État à la mort de Seignelay en
1690, le 5 septembre 1699 chancelier et garde des sceaux\,; et lors de
sa retraite il avait soixante et onze ans, sans jamais la plus légère
infirmité, et la tête comme à quarante.

Fort peu après qu'il fut sorti du cabinet du roi, Pelletier de Sousy y
entra pour son travail ordinaire sur les fortifications. Cela dura
peu\,; et quand il eut fini, le roi, qui avait eu le temps de choisir un
chancelier depuis que celui qui quittait cette place lui en avait
demandé la permission avec tant de persévérance instante, envoya
chercher Voysin, lui remit la cassette des sceaux, et le déclara
chancelier. On ne douta pas qu'il ne remît sa charge de secrétaire
d'État du département de la guerre. Il n'y avait point d'exemple d'aucun
chancelier secrétaire d'État à la fois, mais celui-ci avait l'appétit
bon, et il fut l'un et l'autre.

De Mesmes, bien éveillé, bien averti, avait tourné vers cette première
charge de la robe une gueule béante. Le grand appui et l'unique qu'il
eût lui manqua. M. du Maine, plein de tout ce qui ne tarda pas à éclore,
avait plus besoin du premier président totalement et servilement à lui
que d'un chancelier\,; il ne pouvait jamais trouver de premier président
plus en sa main, ni plus parfaitement corrompu et vendu à la fortune,
par conséquent à la faveur et à la protection, que Mesmes\,; il était
donc de son intérêt principal de l'y conserver. Pour chancelier il avait
Voysin tout prêt, tout initié dans le conseil, dans l'habitude, dans la
privance du roi, et aussi corrompu que l'autre pour la fortune et la
faveur, mais nullement propre à manier rien que par voie d'autorité et
de violence, et qui d'ailleurs était dans la confiance intime de
M\textsuperscript{me} de Maintenon, et valet à tout faire et à tout
entreprendre\,; aussi elle et lui ne balancèrent-ils pas à préférer
Voysin, qu'ils gouvernèrent comme ils voulurent auprès du roi, tandis
que le premier président, vendu à M. du Maine, fut réservé pour le
servir à la cour et dans le parlement par tout l'art et les manéges
infâmes, dont il sera temps incontinent de parler à plus d'une reprise.
J'ai suffisamment expliqué ailleurs quels étaient ces deux chanceliers
et ce premier président pour n'avoir rien ici à y ajouter qu'un mot sur
l'écorce.

Voysin porta ses deux {[}charges{]} comme on vient de le dire, et le roi
eut l'enfantillage de s'amuser à le montrer. Au conseil, et tous les
matins même qu'il n'y en avait point, Voysin était vêtu en chancelier.
L'après-dînée, il était en manteau court de damas, et travaillait ainsi
avec le roi. Les soirs, comme c'était l'été, il quittait son manteau, et
paraissait à la promenade du roi en justaucorps de damas. Cela parut
extrêmement ridicule et parfaitement nouveau. M. de Lauzun, qui allait
volontiers faire des courses de Marly à Paris, se trouva en compagnie,
où on lui demanda des nouvelles de Marly. «\, Rien, répondit-il de ce
ton bas et ingénu qu'il prenait si souvent, il n'y a aucunes
nouvelles\,; le roi s'amuse à habiller sa poupée.\,» L'éclat de rire
prit aux assistants qui entendirent bien ce qu'il voulait dire, et lui
en sourit aussi malignement, et gagna la porte.

L'électeur de Bavière vint courre le cerf à Marly, et vit le roi avec
tout le monde à la chasse. Il joua après dans le salon jusqu'à minuit.
Le roi, au sortir de son souper, entra, contre sa coutume, dans le
salon, s'approcha de l'électeur, et le vit jouer quelques moments.
L'électeur alla faire \emph{media noche} chez d'Antin, avec
M\textsuperscript{me} la Duchesse et grande compagnie, puis retourna à
Saint-Cloud. Il y fit deux autres chasses de même, sans voir le roi en
particulier ni ailleurs qu'à la chasse.

On sut en même temps que le roi Stanislas, après avoir fort longtemps
erré et ne sachant où se retirer, était enfin arrivé aux Deux-Ponts avec
quatre officiers seulement du régiment du baron Spaar. Ce duché, qui a
un beau château logeable et meublé, appartenait au roi de Suède, qui
l'avait fait recevoir là en asile.

On apprit en même temps une nouvelle plus intéressante, l'arrivée au
Port-Louis de la flotte des Indes orientales, riche de dix millions en
marchandises.

Le roi donna mille écus d'augmentation de pension à
M\textsuperscript{me} de Saint-Géran\,; et choisit Fagon, maître des
requêtes, fils de son premier médecin, pour la charge d'intendant des
finances qu'avait du Buisson, qui l'avait très dignement remplie, mais
devenu trop vieux pour en pouvoir continuer les fonctions. Ce fut une
grande distinction pour Fagon à son âge, et qui n'avait point été
intendant de province. Il parut depuis homme de beaucoup d'esprit et de
capacité, et figura grandement dans les finances.

Brassac épousa la fille du feu maréchal de Tourville, qui fut quelque
temps après dame de M\textsuperscript{me} la duchesse de Berry. Personne
n'avait été plus singulièrement ni plus délicatement jolie, avec une
taille charmante qui y répondait. La petite vérole la changea à tel
point qu'il n'y eut personne qui la pût reconnaître. Je le rapporte par
l'extraordinaire de la chose portée à cet excès. La graisse survint
bientôt après, et en fit une tour, d'ailleurs une bonne, honnête et très
aimable femme.

Il y avait du temps que la reine de Pologne, veuve du célèbre Jean
Sobieski, était embarrassée de sa retraite, et qu'elle avait eu envie de
venir finir sa vie en France. La passion qu'elle avait eue autrefois de
venir montrer sa couronne dans sa patrie, sous prétexte des eaux de
Bourbon, l'en avait rendue la plus mortelle ennemie. Elle voulut savoir
sur quoi compter précisément. À l'égard du cérémonial, il se trouva que,
la Pologne étant couronne élective, la reine ne pouvait lui donner la
main. Il était même bien nouveau que le roi la donnât aux rois
héréditaires, et c'est du cardinal Mazarin que l'introduction de
l'égalité des rois est venue, et que ceux du Nord, qui ne faisaient pas
difficulté de donner la main aux ambassadeurs de nos rois, ont non
seulement abrogé cet usage, mais en sont venus à se parangonner à eux.
La reine de Pologne, qui n'avait d'autre objet de son voyage que
l'orgueil de se voir égalée à la reine, le rompit aussitôt et ne le
pardonna jamais.

On a prétendu que ses menées avaient eu grande part à former la fameuse
ligue d'Augsbourg contre la France\,; et il est certain qu'elle se
servit toute sa vie du pouvoir presque entier qu'elle s'était acquis sur
le roi son mari, pour l'éloigner de la France contre son goût, et
l'attacher à la maison d'Autriche, dont elle fut récompensée par le
grand mariage de son fils aîné avec une sœur de l'impératrice, et des
reines d'Espagne et de Portugal, de la duchesse de Modène et de
l'électeur palatin Neubourg.

Elle ne laissa pas parmi ses desservices de demander au roi de faire son
père duc et pair. Le peu de succès qu'eurent ses instances lui inspira
un nouveau dépit, qu'elle fit éclater dans toute son étendue, contre la
France et contre le prince de Conti, à la mort du roi son époux. À bout
d'espérance d'un duché pour son père, qui était veuf depuis longtemps et
chevalier du Saint-Esprit, elle le fit cardinal par la nomination de
Pologne.

Son humeur altière et son extrême avarice l'avaient fait détester en
Pologne\,; et l'aversion publique qu'elle témoigna sans mesure au prince
Jacques, son fils aîné, coûta la couronne à sa famille. Elle ne put donc
se résoudre à demeurer dans un pays où, après avoir été tout, elle se
trouvait haïe, méprisée, étrangère et sans appui par la division de ses
enfants, et prit le parti d'aller avec son père s'établir à Rome. Elle
avait compté y être traitée comme l'avait été la reine Christine de
Suède\,; mais celle-ci était reine héréditaire par elle-même, et avait
de plus touché la cour de Rome par sa conversion du luthéranisme. Il y
eut donc des différences, qui mortifièrent tellement la reine de Pologne
qu'elle ne put plus soutenir le séjour de Rome dès qu'elle y eut perdu
le cardinal d'Arquien, et que, ne sachant que devenir, elle voulut venir
en France. De la façon qu'elle s'était comportée il n'est pas surprenant
que la demande qu'elle en fit fût reçue froidement, et que la liberté
d'y venir se fît attendre. À la fin le roi consentit, mais à condition
qu'elle ne songerait pas à venir, ni même à s'approcher de la cour ni de
Paris, et lui donna le choix d'une ville sur la Loire, et même des
châteaux de Blois, d'Amboise et de Chambord.

Elle arriva, le 4 juillet, à Marseille, sur les galères du pape, et y
trouva pour la recevoir, de la part du roi, le marquis de Béthune, fils
de sa sœur, et père de la maréchale de Belle-Ile, qui n'était pas encore
mariée pour la première fois. Elle ne voulut point d'honneurs nulle
part, de peur apparemment qu'ils ne fussent pas tels qu'elle les aurait
souhaités, séjourna peu à Marseille, et s'en alla par le plus droit à
Blois, qu'elle avait choisi, et dont elle ne sortit plus. Elle avait
avec elle la fille aînée du prince Jacques son fils, qui épousa depuis,
à Rome, le roi Jacques d'Angleterre, que les Anglais appellent le
Prétendant. Elles vécurent à Blois dans la plus grande solitude et sans
nul éclat.

M. le Duc, M. le comte de Charolais son frère, et M. le prince de Conti
devaient faire le deuil du service de M. le duc de Berry à Saint-Denis.
Le comte de Charolais se trouva malade\,; M. le duc de Chartres avait
onze ans. Des princes aussi jeunes et plus jeunes ont fait le deuil en
pareilles cérémonies\,; et, sans remonter bien loin, les fils de
M\textsuperscript{me} la dauphine de Bavière à son enterrement, qui
étaient plus chers à la France\,; et M. de Chartres n'avait pas les
mêmes raisons de s'en dispenser que M. le duc d'Orléans\,; mais le temps
pressait, on en voulut profiter, et le roi ne voulut pas manquer
l'occasion d'y faire figurer le prince de Dombes en troisième. Cette
parité sembla fort étrange\,: ce n'était pourtant qu'un léger essai. Il
n'y eut à ce service que les compagnies à l'ordinaire, et les seuls
officiers de la maison de Berry. L'abbé Prévost fit l'oraison funèbre.
Ce fut le lundi 16 juillet.

Le maréchal de Berwick fit ouvrir, le 12 juillet au soir, la tranchée
devant Barcelone.

Maisons, président à mortier, et sa femme, soeur aînée de la maréchale
de Villars, furent deux espèces de personnages dont il est temps de
parler. Son grand-père, aussi président à mortier, fut surintendant des
finances, bâtit le superbe château de Maisons, était ami de mon père,
qui pour l'obliger, car rien ne lui coûta jamais pour ses amis, lui
vendit presque pour rien la capitainerie de Saint-Germain en Laye qu'il
avait, et qui était nécessaire au président par la position de Maisons
tout prés de Saint-Germain et au milieu de la capitainerie. C'est lui
qui, lorsqu'on lui ôta les finances, dit tout haut\,: «\,Ils ont tort\,;
car j'ai fait mes affaires, et j'allais faire les leurs. » Tant qu'il
vécut l'amitié subsista avec mon père. Son fils, père de celui dont il
s'agit, et président à mortier, voyait aussi mon père. C'est lui qui
présida si indignement au jugement de notre procès avec M. de
Luxembourg, comme je l'ai rapporté en son lieu. Sa conduite ne me donna
pas envie de cultiver l'ancienne amitié, et je n'en eus pas davantage à
l'égard de son fils, de qui aussi je n'entendis point parler jusque tout
au commencement de cette année, et tout au plus tôt tout à la fin de la
précédente. Cet exposé était nécessaire pour l'intelligence de ce qui va
suivre.

Maisons était un grand homme, de fort belle représentation, de beaucoup
d'esprit, de sens, de vues et d'ambition, mais de science dans son
métier fort superficielle, fort riche, la parole fort à la main, l'air
du grand monde, rien du petit-maître ni de la fatuité des gens de robe,
nulle impertinence du président à mortier. Je pense que l'exemple de M.
de Mesmes lui avait fort servi à éviter ces ridicules dont l'autre
s'était chamarré. Loin comme lui de faire le singe du grand seigneur, de
l'homme de la cour et du grand monde, il se contentait de vivre avec la
meilleure compagnie de la ville et de la cour, que sa femme et lui
avaient su attirer chez eux par les manières les plus polies, même
modestes, et sans jamais s'écarter de ce qu'ils devaient à chacun\,;
respect aux uns, civilité très marquée aux autres\,; avec un air de
liberté et de familiarité mesurée, qui, loin de choquer ni d'être
déplacée, leur attirait le gré de savoir mettre tout le monde à son
aise, sans jamais la moindre échappée qui fût de trop.

Sa femme, avec très peu ou point d'esprit, avait celui de savoir tenir
une maison avec grâce et magnificence, et de se laisser conduire par
lui. Elle n'avait donc rien de la présidente, ni des femmes de robe,
seulement quelque petit grain plus que lui du grand monde, mais avec la
même politesse et les mêmes ménagements. C'était une grande femme qui
avec moins d'embonpoint eût eu la taille belle, et une beauté romaine
que bien des gens préféraient à celle de sa soeur. Elle eut le bon sens
de bien vivre toujours avec elle, et de ravaler bien soigneusement la
jalousie du rang et de la concurrence de beauté\,; et Maisons, de son
côté, vivait en déférence très marquée, mais intimement, avec le
maréchal de Villars.

Il eut le bon esprit de sentir de fort bonne heure que le parlement
était la base sur laquelle il devait porter\,; que du crédit qu'il y
aurait dépendrait sa considération dans le monde\,; et que tout celui
dans lequel il se mêlait ne lui deviendrait utile qu'autant que sa
compagnie le compterait. Il fut donc assez avisé pour en faire son
principal, attirer chez lui les magistrats du parlement, courtiser, pour
ainsi dire, les plus estimés dans toutes les chambres, les persuader
qu'il se faisait honneur d'être l'un d'eux, faire conduire sa femme en
conséquence, être très assidu au palais et y gagner la basse robe en
général, et en particulier ce qui se distinguait le plus parmi les
avocats, les procureurs, les greffiers, par ses manières gracieuses,
ouvertes, affables, par des louanges et des prévenances qui l'en firent
adorer. De cette conduite il en résulta une réputation qui dans tout le
parlement n'eut pas deux voix, qui gagna la cour et le monde\,; qui
donna jalousie au premier président, et qui fit regarder Maisons comme
celui qui mènerait toujours le parlement à tout ce qu'il voudrait.

La situation de Maisons si près de Marly lui fournit des occasions,
qu'il sut bien ménager, d'y attirer des gens principaux de la cour. Il
devint du bon air d'y aller de Marly, et il se contenta longtemps d'y
voir la cour de ses terrasses. Il allait peu à Versailles, il rapprocha
mesurément ses voyages à une fois la semaine\,; et, à force de gens
principaux d'autour du roi qui pendant les longs Marlys allaient dîner à
Maisons, le roi s'accoutuma à lui parler de ce lieu presque toutes les
fois qu'il le voyait, et jamais il n'en fut gâté. Il avait si bien fait
que M. le Duc et M. le prince de Conti étaient en liaison avec lui, et
qu'il regarda leur mort comme une perte qu'il faisait. Il travaillait
aussi en dessous, et je ne sais par où il s'était mis fort en commerce
avec M. de Beauvilliers, mais un commerce qui ne paraissait point, et
dont je n'ai démêlé ni le comment ni la date.

Ces deux princes du sang morts, il se tourna vers M. le duc d'Orléans,
et il lui fut aisé de s'en approcher par Canillac, son ami intime, qui
l'était de tout temps de ce prince, mais qui ne le voyait qu'à Paris,
parce qu'il ne venait comme jamais à la cour. Il vanta donc tant le
mérite de Maisons, son crédit dans le parlement et dans le monde, les
avantages qui s'en pouvaient tirer et de son conseil, que M. le duc
d'Orléans, accoutumé à se laisser dominer à l'esprit de Canillac, crut
trouver un trésor dans la connaissance et l'attachement de Maisons.

Celui-ci, qui voulait circonvenir le prince, ne trouva pas Canillac
suffisant, leurs séparations de lieu étaient trop continuelles\,; il
jeta son coussinet sur moi. Je pense qu'il me craignait par ce que j'ai
raconté de son père. Il avait un fils unique à peu près de l'âge de mes
enfants\,; il y avait déjà longtemps qu'il avait fait toutes les avances
et qu'il les voyait souvent. Cela ne rendait rien au delà, et ce n'était
pas le compte du père\,; enfin il me fit parler par M. le duc d'Orléans.
Ce fut alors que j'appris cette liaison nouvelle, combien Maisons en
désirait avec moi, estime, louanges, amitié des pères que ce prince me
rapporta\,; je fus froid, je payai de compliments, j'alléguai que je
n'allais que très peu à Paris, et pour des moments, et je m'en crus
quitte. Peu de jours après, M. le duc d'Orleans rechargea, je ne fus pas
plus docile. Quatre ou cinq jours après, je fus fort surpris que M. le
duc de Beauvilliers m'en parlât, me dit les mêmes choses, m'apprit sa
liaison, me voulût persuader que celle que Maisons désirait que je
prisse avec lui pouvait être extrêmement utile à bien des choses, et
finalement, voyant que je n'y prenais point, employât l'autorité qu'il
avait sur moi, et me dit qu'il m'en priait, et qu'il le désirait puisque
je n'avais point de raison particulière ni personnelle pour m'en
défendre. Je vis bien clairement alors que Maisons, n'avançant pas à son
gré par M. le duc d'Orléans, était bien au fait de moi, et qu'il avait
bien compris que je ne résisterais pas au duc de Beauvilliers si
celui-ci entreprenait de former la liaison, et ne voulût pas être
éconduit\,; aussi ne le fut-il pas, mais après être demeuré sur la
défensive avec M. le duc d'Orléans, je ne voulus pas lui montrer que je
rendais les armes à un autre.

L'attente ne fut pas longue. Ce prince m'attaqua de nouveau, me maintint
que rien ne serait plus utile pour lui qu'une liaison de Maisons avec
moi, qui n'osait le voir que rarement et comme à la dérobée, et avec qui
il ne pouvait avoir le même loisir ni la même liberté de discuter bien
des choses qui pouvaient se présenter. J'avais d'autres fois répondu à
tout cela, mais comme j'avais résolu de me rendre à lui depuis que
l'autorité du duc de Beauvilliers m'avait vaincu, je consentis à ce que
le prince voulut.

Maisons en fut bientôt informé. Il ne voulut pas laisser refroidir la
résolution. M. le duc d'Orléans me pressa d'aller coucher une nuit à
Paris. En y arrivant j'y trouvai un billet de Maisons, qui m'avait déjà
fait dire merveilles par le prince et par le duc. Ce billet, pour les
raisons qu'il réservait à me dire, contenait un rendez-vous à onze
heures du soir, ce jour-là même, derrière les Invalides, dans la plaine,
avec un air fort mystérieux. J'y fus avec un vieux cocher de ma mère et
un laquais, pour dépayser mes gens. Il faisait un peu de lune. Maisons
en mince équipage m'attendait. Nous nous rencontrâmes bientôt. Il monta
dans mon carrosse. Je n'ai jamais compris le mystère de ce rendez-vous.
Il n'y fut question que d'avances, de compliments, de protestations, de
souvenirs des anciennes liaisons de nos pères, et de tout ce que peut
dire un homme d'esprit et du monde qui veut former une liaison
étroite\,; du reste de propos généreux, de louanges et d'attachement
pour M. le duc d'Orléans et pour M. de Beauvilliers, sur la situation
présente de la cour, en un mot toutes choses qui n'allaient à rien
d'important ni de particulier. Je répondis le plus civilement qu'il me
fut possible à l'abondance qu'il me prodigua. J'attendais ensuite
quelque chose qui méritât l'heure et le lieu\,; ma surprise fut grande
de n'y trouver que du vide, et seulement pour raison que cette première
entrevue devait être secrète, après laquelle il n'y aurait plus
d'inconvénient qu'il vînt quelquefois chez moi à Versailles, et serrer
les visites, après qu'on se serait accoutumé à l'y voir quelquefois, et
me priant de n'aller point chez lui à Paris de longtemps, où il se
trouvait toujours trop de monde. Ce tête-à-tête ne dura guère plus de
demi-heure. C'était beaucoup encore pour ce qu'il s'y passait. Nous nous
séparâmes en grande politesse, et dès la première fois qu'il alla à
Versailles, il vint chez moi sur la fin de la matinée.

Il ne fut pas longtemps sans y venir ainsi tous les dimanches. Nos
conversations peu à peu devinrent plus sérieuses. Je ne laissais pas
d'être en garde, mais je le promenais sur plusieurs sujets, et lui s'y
prêtait très volontiers.

Nous raisonnions et nous étions sur ce pied-là ensemble, lorsque,
rentrant chez moi à Marly sur la fin de la matinée du dimanche 29
juillet, je trouvai un laquais de Maisons avec un billet par lequel il
me conjurait, toutes affaires cessantes, de venir sur-le-champ chez lui
à Paris où il m'attendrait seul, et où je verrais qu'il s'agissait de
chose qui ne pouvait souffrir le moindre retardement, qui ne se pouvait
même désigner par écrit, et qui était de la plus extrême importance. Il
y avait longtemps que ce laquais était arrivé, et qu'il me faisait
chercher partout par mes gens. M\textsuperscript{me} de Saint-Simon
était à Versailles avec M\textsuperscript{me} la duchesse de Berry, qui
venait souper les soirs avec le roi sans coucher encore à Marly, et je
devais dîner chez M. et M\textsuperscript{me} de Lauzun. Y manquer
aurait mis la curiosité et la malignité de M. de Lauzun en besogne\,: je
n'osois donc pas disparaître. Je donnai ordre à ma voiture\,; dès que
j'eus dîné je m'éclipsai. Personne ne me vit monter en chaise\,;
j'arrivai fort diligemment chez moi à Paris, d'où j'allai sur-le-champ
chez Maisons avec l'empressement qu'il est aisé d'imaginer.

Je le trouvai seul avec le duc de Noailles. Du premier coup d'œil je vis
deux hommes éperdus, qui me dirent d'un air mourant, mais après une vive
quoique courte préface, que le roi déclarait ses deux bâtards, et à
l'infini leur postérité masculine, vrais princes du sang, en droit d'en
prendre la qualité, les rangs et honneurs entiers, et capables de
succéder à la couronne au défaut de tous les autres princes du sang. À
cette nouvelle, à laquelle je ne m'attendais pas, et dont le secret
jusqu'alors s'était conservé sans la plus légère transpiration, les bras
me tombèrent. Je baissai la tête et je demeurai dans un profond silence,
absorbé dans mes réflexions. Elles furent bientôt interrompues par des
cris auxquels je me réveillai. Ces deux hommes se mirent en pied à
courir la chambre, à taper des pieds, à pousser et à frapper les
meubles, à dire rage à qui mieux mieux, et à faire retentir la maison de
leur bruit. J'avoue que tant d'éclat me fut suspect de la part de deux
hommes, l'un si sage et si mesuré, et à qui ce rang ne faisait rien,
l'autre toujours si tranquille, si narquois, si maître de lui-même. Je
ne sus quelle subite furie succédait en eux à un si morne accablement,
et je ne fus pas sans soupçon que leur emportement ne fût factice pour
exciter le mien. Si ce fut leur dessein, il réussit tout au contraire.
Je demeurai dans ma chaise, et leur demandai froidement à qui ils en
voulaient. Ma tranquillité aigrit leur furie. Je n'ai de ma vie rien vu
de si surprenant.

Je leur demandai s'ils étaient devenus fous, et si au lieu de cette
tempête il n'était pas plus à propos de raisonner, et de voir s'il y
avait quelque chose à faire. Il s'écrièrent que c'était parce qu'il n'y
avait rien à faire à une chose non seulement résolue, mais exécutée,
mise en déclaration, et envoyée au parlement, qu'ils étaient outrés de
la sorte\,; que M. le duc d'Orléans, en l'état où il était avec le roi,
n'oserait souffler\,; les princes du sang en âge de trembler comme des
enfants qu'ils étaient\,; les ducs hors de tout moyen de s'opposer, et
le parlement réduit au silence et à l'esclavage\,; et là-dessus à qui
des deux crierait le plus fort et pesterait davantage, car rien de leur
part ne fut ménagé, ni choses, ni termes, ni personnes.

J'étais bien aussi en colère, mais il est vrai que ce sabbat me fit rire
et conserva ma froideur. Je convins avec eux que quant alors je n'y
voyais point de remède, et nulles mesures à prendre\,; mais qu'en
attendant ce qui pouvait arriver à l'avenir, je les aimais encore mieux
princes du sang capables de la couronne, qu'avec leur rang
intermédiaire. Et il est vrai que je le pensai ainsi dès que j'eus
repris mes esprits.

Enfin l'ouragan s'apaisa peu à peu. Nous raisonnâmes et ils m'apprirent
que le premier président et le procureur général, qui en effet étaient
venus ce jour-là de très bonne heure à Marly chez le chancelier, qui
avait vu le roi dans son cabinet, à l'issue de son lever, et qui étaient
revenus à Paris tout de suite, en avaient rapporté la déclaration tout
expédiée. Il fallait néanmoins que liaisons l'eût sue plus tôt
d'ailleurs, parce qu'à l'heure que le laquais qu'il m'envoya arriva à
Marly, ces messieurs n'en pouvaient pas être revenus à Paris quand il
partit. Nos discours n'allant à rien, je pris congé et regagnai Marly au
plus vite, afin que mon absence ne fît point parler.

Tout cela néanmoins me conduisit vers l'heure du souper du roi. J'allai
droit au salon, je le trouvai très morne. On se regardait, on n'osait
presque s'approcher, tout au plus quelque signe dérobé ou quelque mot en
se frôlant coulé à l'oreille. Je vis mettre le roi à table, il me sembla
plus morgué qu'à l'ordinaire, et regardant fort à droite et à gauche. Il
n'y avait qu'une heure que la nouvelle avait éclaté, on en était glacé
encore, et chacun fort sur ses gardes. À chose sans ressource il faut
prendre son parti, et il se prend plus aisément et plus honnêtement
quand la chose ne porte pas immédiatement comme le rang intermédiaire
dont les bâtards n'eurent jamais de moi ni compliment ni la moindre
apparence. J'avais donc pris ma résolution.

Dès que le roi fut à table, et qui m'avait fort fixement regardé en
passant, j'allai chez M. du Maine\,; bien que l'heure fût un peu indue,
les portes tombèrent devant moi, et je remarquai un homme surpris d'aise
de ma visite, et qui vint au-devant de moi presque sur les airs, tout
boiteux qu'il était. Je lui dis que pour cette fois je venais lui faire
mon compliment, et un compliment sincère\,; que nous n'avions rien à
prétendre sur les princes du sang\,; que ce que nous prétendions et ce
qui nous était dû, c'était qu'il n'y eût personne entre les princes du
sang et nous\,; que dès qu'il l'était et les siens, nous n'avions plus
rien à dire qu'à nous réjouir de n'avoir plus à essuyer ce rang
intermédiaire que je lui avouais qui m'était insupportable. La joie de
M. du Maine éclata à ce compliment. Tout ce qu'il m'en fit, tout ce
qu'il m'en dit ne peut se rendre, avec une politesse, un air même de
déférence que l'esprit inspire dans le transport du triomphe.

J'en dis autant le lendemain au comte de Toulouse et à
M\textsuperscript{me} la duchesse d'Orléans, cent fois plus bâtarde et
plus aise que ses frères, et qui les voyait déjà couronnés.
M\textsuperscript{me} la Duchesse fort princesse du sang, et point du
tout comme M\textsuperscript{me} sa soeur, parut fort sérieuse, et
n'ouvrit point sa porte. M. le duc d'Orléans fut fâché, mais fâché à sa
manière, et n'eut pas grand'peine à ne rien montrer. Ducs et princes
étrangers enragés, mais de rage mue. La cour éclata en murmures sourds
bien plus qu'on n'aurait cru. Paris se déchaîna et les provinces\,; le
parlement, chacun à part, ne se contraignit pas. M\textsuperscript{me}
de Maintenon, transportée de son ouvrage, en recevait les adorations de
ses familières. Elle et M. du Maine n'avaient pas oublié ce qui avait
pensé arriver du rang de ses enfants. Quoiqu'il n'y eût plus personne du
sang légitime à craindre, ils ne laissèrent pas d'être effarouchés, et
le roi fut gardé à vue, et persuadé par des récits apostés de la joie et
de l'approbation générale à ce qu'il venait de faire. M. du Maine n'eut
garde de se vanter de l'air triste, morne, confondu, qui accompagnait
tous les compliments, dont une cour esclave lui portait un hommage
forcé, et qui n'en cachait pas la violence. M\textsuperscript{me} du
Maine triompha à Sceaux de la douleur publique. Elle redoubla de fêtes
et de plaisirs, prit pour bons les compliments les plus secs et les plus
courts, et glissa sur le grand nombre de gens qui ne purent se résoudre
d'aller eux-mêmes à son adoration. Les bâtardeaux déifiés ne parurent
que quelques moments à Marly. M. du Maine crut nécessaire cet air de
modestie et de ménagement pour le public. Il n'eut pas tort.

Le comte de Toulouse profita de ce monstrueux événement sans y avoir eu
aucune part. Ce fut l'ouvrage de son frère, de sa fidèle et
toute-puissante protectrice, et de l'art qui fut lors aperçu d'avoir
fait conserver à Voysin, devenu chancelier, sa charge de secrétaire
d'État. Comme chancelier il n'aurait rien eu qui l'eût approché du roi,
plus de travail réglé avec lui, plus de prétextes de lui aller parler
quand il le jugeait à propos. Il n'aurait eu que les occasions de la fin
des conseils, quand les ministres en sortent\,; et comme il n'était
chargé de rien qui eût rapport au roi, il eût fallu l'attaquer sans
préface, sans prétexte, sans insinuation, et sans moyen de sonder le
terrain\,; quoique sur les bâtards, il aurait trouvé le roi en garde.
L'usurpation de ses audiences l'eut effarouché et rendu Voysin
désagréable, et comme le chancelier n'a point de travail avec le roi que
pour des affaires extraordinaires, rares, courtes, qui même pour
l'ordinaire ne sont pas secrètes, comme mon affaire avec M. de La
Rochefoucauld et autres pareilles quoique de différentes natures, ces
audiences, si elles avaient été répétées, auraient fait nouvelle, excité
une curiosité dangereuse au secret dont ce mystère d'iniquité avait tant
intérêt de se couvrir, et dont les artisans sentaient si bien
l'importance. Ce fut aussi ce qui fit conserver à Voysin cette place de
secrétaire d'État, qui lui donnait une occasion nécessaire de travailler
presque tous les jours seul avec le roi ou M\textsuperscript{me} de
Maintenon en tiers unique, et la faculté des prétextes d'y travailler
extraordinairement et tous les jours, et plus d'une fois par jour tant
que bon lui semblait, sans que cela parût extraordinaire au roi ni à sa
cour. Par là Voysin se trouvait à portée d'examiner les moments, les
humeurs, de sonder, d'avancer, de s'arrêter\,; par là nul temps perdu
qui ne se pût retrouver le lendemain, et quelquefois le jour même\,; par
là liberté de discuter et de pousser sa pointe quand il y trouvait lieu,
et de prolonger la conversation tant qu'il était nécessaire\,; sans quoi
ils n'en seraient jamais venus à bout.

Le roi, malgré tout ce qu'il sentait d'affection pour ses bâtards, avait
toujours des restes de ses anciens principes. Il n'avait pas oublié
l'adresse de la planche de la légitimation du chevalier de Longueville
sans nommer la mère, pour parvenir à donner un état à ses enfants,
lorsqu'il avait voulu les tirer de leur néant propre, et de l'obscurité
secrète dans laquelle ils avaient été élevés. De ce néant, ce qu'il fit
par degrés pour les conduire possiblement au trône est si prodigieux que
ce tout ensemble mérite d'être exposé ici sous un même coup d'œil tout à
la fois, et comparer les premiers degrés qui, par un effort inconnu
jusqu'alors de puissance, les égala peu à peu aux autres hommes, en les
égalant aux droits communs de tous\,; avec les derniers qui les
portèrent à la couronne. On ne parlera ici que des enfants de
M\textsuperscript{me} de Montespan.

\hypertarget{chapitre-ix.}{%
\chapter{CHAPITRE IX.}\label{chapitre-ix.}}

~

{\textsc{Degrés rapides qui, du plus profond non-être, portent à la
capacité de porter à la couronne, par droit de naissance, la postérité
sortie du double adultère du roi et de M\textsuperscript{me} de
Montespan.}} {\textsc{- Adresse de la réception de César, duc de
Vendôme, au parlement.}} {\textsc{- Traversement du parquet par les
princes du sang\,; son époque.}} {\textsc{- Réflexions.}} {\textsc{-
Position de l'esprit du roi sur ses bâtards paraît bien peu égale.}}

~

\begin{enumerate}
\def\labelenumi{\arabic{enumi}.}
\tightlist
\item
  Lettres de légitimation en faveur de Charles-Louis (le chevalier de
  Longueville), avec permission de porter le nom de bâtard d'Orléans, et
  déclaré capable de posséder toutes charges\,; vérifiées au parlement
  sans que le nom de la mère y fût exprimé\,; dont c'est le premier
  exemple, 7 septembre 1673.
\end{enumerate}

Telle fut la planche pour légitimer les enfants du roi, leur faire
porter le nom de Bourbon, leur pouvoir donner des charges, et sans
nommer M\textsuperscript{me} de Montespan.

\begin{enumerate}
\def\labelenumi{\arabic{enumi}.}
\setcounter{enumi}{1}
\item
  Lettres de légitimation en faveur de Louis-Auguste, né dernier mars
  1670 (le duc du Maine)\,; de Louis-César, né 1672 (le comte du
  Vexin)\,; de Louise-Françoise, née en 1673 (M\textsuperscript{lle} de
  Nantes, depuis M\textsuperscript{me} la Duchesse)\,; toutes de
  décembre 1673, vérifiées 20 des mêmes mois et an.
\item
  Noms de provinces imposés, qui ne se donnent qu'à des fils de France.
\item
  Avant le pouvoir, le duc du Maine pourvu en février 1674, c'est-à-dire
  avant l'âge de quatre ans, de la charge de colonel général des Suisses
  et Grisons.
\end{enumerate}

Lettres de légitimation en faveur de Louise-Marie-Antoinette
(M\textsuperscript{lle} de Tours), janvier 1676. Elle mourut 15
septembre 1681.

5 et 6. Lettres de décembre 1676, qui déclarent Louis-Auguste de Bourbon
capable de posséder toutes charges et qu'il serait nommé duc du Maine.
(Le comte de Toulouse n'a rien eu d'écrit pour porter ce nom.)

Ainsi cette déclaration donna la faculté que le fait avait précédé de
deux ans, tant pour les charges que pour l'appellation de duc du Maine,
et suppose en lui d'avance, comme on le va voir, le nom de Bourbon qu'il
n'avait pas.

\begin{enumerate}
\def\labelenumi{\arabic{enumi}.}
\setcounter{enumi}{6}
\tightlist
\item
  Le comte de Vexin, tout contrefait, nommé à l'abbaye de Saint-Germain
  des Prés et à celle de Saint-Denis\,; mort le 10 janvier 1683, à dix
  ans et demi, dans l'abbatial de Saint-Germain des Prés.
\end{enumerate}

8 et 9. Lettres patentes portant que le duc du Maine, le comte de Vexin,
M\textsuperscript{lle} de Nantes et M\textsuperscript{lle} de Tours,
porteront le surnom de Bourbon, et se succéderont les uns aux autres
tant pour les biens qu'ils ont reçus de notre libéralité, que pour ceux
qu'ils pourront acquérir d'ailleurs, comme aussi que leurs enfants se
succéderont selon l'ordre des successions légitimes. Données au mois de
janvier 1680, registrées en parlement le 11 janvier même année, et en la
chambre des comptes le lendemain.

Ainsi les voilà égalés aux autres hommes, élevés du néant à la condition
commune, enrichis de tous les droits des légitimes dans la société, en
même temps décorés du surnom de la maison régnante, et de noms de
provinces que les princes du sang même ne portent pas.

\begin{enumerate}
\def\labelenumi{\arabic{enumi}.}
\setcounter{enumi}{9}
\tightlist
\item
  Don fait (c'est-à-dire arraché pour tirer de Pignerol M. de Lauzun) au
  duc du Maine de la principauté de Dombes, etc., par Mademoiselle, 2
  février 1681.
\end{enumerate}

Lettres de légitimation en faveur de Françoise-Marie, née en mai 1677
(M\textsuperscript{lle} de Blois, depuis duchesse d'Orléans), et de
Louis-Alexandre, né le 6 juin 1678 (le comte de Toulouse), avec
permission de porter le nom de Bourbon\,; et la faculté tant à eux qu'à
Louis-Auguste, Louis-César, Louise-Françoise, de se succéder les uns aux
autres, etc. Ces lettres données en novembre 1681, registrées le 22 du
même mois et an.

\begin{enumerate}
\def\labelenumi{\arabic{enumi}.}
\setcounter{enumi}{10}
\item
  Le duc du Maine pourvu du gouvernement de Languedoc en juin 1682, à
  douze ans.
\item
  Le comte de Toulouse pourvu de l'office d'amiral de France en novembre
  1683, à cinq ans.
\end{enumerate}

Cet office, si nuisible par ses droits pécuniaires, et si embarrassant
par son autorité, avait été supprimé avec grande raison. Le roi l'avait
rétabli en faveur du comte de Vermandois, enfant qu'il avait eu de
M\textsuperscript{me} de La Vallière, à la mort duquel il le donna au
comte de Toulouse.

On remarquera que, le parlement et le monde une fois accoutumés aux
bâtards de double-adultère, le roi fit par une seule et même
déclaration, pour les deux derniers, ce qu'il n'avait osé présenter
qu'en plusieurs pour les premiers.

\begin{enumerate}
\def\labelenumi{\arabic{enumi}.}
\setcounter{enumi}{12}
\tightlist
\item
  Louise-Françoise de Bourbon, mariée, 24 juillet 1685, à Louis III, duc
  de Bourbon.
\end{enumerate}

Outre sa dot, ses pierreries et ses pensions, M. son mari eut les
survivances de l'office de grand maître de France et du gouvernement de
Bourgogne, une forte pension, et toutes les entrées, même celles d'après
le souper. M. son père, qui, comme lui, n'en avait aucunes, eut les
premières entrées, qui ne sont pas même celles des premiers
gentilshommes de la chambre. Avant que le roi eût, à l'occasion d'une
longue goutte, l'année de la mort du premier duc de Bretagne, supprimé
son coucher aux courtisans, on voyait M. le Prince, qu'il était lors,
sur un tabouret dans le coin de la porte du cabinet du roi, en dehors,
dans la pièce où tout le monde attendait le coucher, et dormant là
tandis que M. son fils était avec le roi, et ce qu'il appelait sa
famille. Quand la porte s'ouvrait pour le coucher, M. le Prince se
réveillait et voyait sortir M. son fils, M. le duc d'Orléans,
Monseigneur, et le roi ensuite, au coucher duquel il demeurait comme les
courtisans, et au petit coucher après avec les entrées, et qui était
fort court. Le reste de la famille sortait par les derrières.

\begin{enumerate}
\def\labelenumi{\arabic{enumi}.}
\setcounter{enumi}{13}
\tightlist
\item
  Le duc du Maine, à seize ans chevalier de l'ordre, à la Pentecôte
  1686.
\end{enumerate}

Je n'ose dire qu'à douze ans que je n'avais pas encore, j'étais fort en
peine et je m'informais souvent de l'état du duc de Luynes qui avait la
goutte\,; je mourais de peur qu'elle ne le quittât, parce qu'il aurait
été parrain de M. le prince de Conti avec-le duc de Chaulnes, et M. du
Maine eût échu à mon père. La goutte persévéra, et mon père présenta le
prince de Conti avec le duc de Chaulnes. L'ordre à un âge inouï, rare
aux fils de France, et en quatrième avec M. le duc de Chartres, à qui
cette considération le fit avancer alors, {[}avec{]} M. le duc de
Bourbon (car le grand prince de Condé ne mourut qu'à la fin de
l'automne), et {[}avec{]} M. le prince de Conti, parut une distinction
bien extraordinaire. Monseigneur et Monsieur furent les parrains de M.
le duc de Chartres, M. le Prince et M. le Duc de M. le duc de Bourbon\,;
feu M. le prince de Conti, gendre naturel du roi, était mort sans avoir
été chevalier de l'ordre, et celui-ci ne l'eût pas été sans le cri
général, que le roi craignit, de faire M. du Maine en laissant le prince
de Conti. Il était lors exilé à Chantilly, et ne coucha qu'une nuit à
Versailles pour la cérémonie. C'était la suite de son voyage en Hongrie.
Il ne fut rappelé qu'à l'instante prière de M. le Prince mourant, mais
jamais pardonné, comme on l'a pu voir ci-dessus en plus d'un endroit.

\begin{enumerate}
\def\labelenumi{\arabic{enumi}.}
\setcounter{enumi}{14}
\item
  Le duc du Maine pourvu de la charge des galères, en 1688, à la mort du
  duc de Mortemart.
\item
  Le comte de Toulouse gouverneur de Guyenne, en janvier 1689, à onze
  ans.
\item
  Le duc du Maine commande la cavalerie en Flandre en 1689.
\end{enumerate}

Jusqu'alors les princes du sang faisaient une ou deux campagnes à la
tête d'un de leurs régiments. M. du Maine, à dix-huit ans, et dès sa
première campagne, a la distinction que les princes du sang n'obtenaient
pas de si bonne heure, qui leur était nouvelle, et qui même en eux
blessait fort les trois généraux nés de la cavalerie par leurs charges.

\begin{enumerate}
\def\labelenumi{\arabic{enumi}.}
\setcounter{enumi}{17}
\tightlist
\item
  Marie-Françoise, mariée, 18 février 1692, à Philippe d'Orléans, duc de
  Chartres, petit-fils de France.
\end{enumerate}

Ce prodige fut le chef-d'œuvre du double adultère et de la sodomie, l'un
et l'autre publics et bien récompensés. La violence ouverte avec
laquelle ce mariage du propre neveu du roi, fils unique de son frère,
fut fait, eut toute la cour pour témoin, et ce qui s'y passa est
détaillé à l'entrée de ces Mémoires. Comparer ce mariage avec ceux de
toutes les bâtardes reconnues et légitimées de nos rois et de simple
adultère jusqu'à Henri IV inclusivement, la chute est à perte d'haleine.

\begin{enumerate}
\def\labelenumi{\arabic{enumi}.}
\setcounter{enumi}{18}
\tightlist
\item
  Le duc du Maine épouse, 19 mars 1692, une fille de M. le Prince\,;
  encore eut-il le choix des trois.
\end{enumerate}

Le roi donna des espèces de fêtes et se para lui-même aux mariages de
ses filles, à celui-ci, et y donna un festin royal, à la totale
différence du mariage du prince de Conti avec la fille aînée de M. le
Prince, à la célébration duquel il assista et n'y donna ni repas ni
fête.

Le duc du Maine lieutenant général, 3 avril 1692.

Il ne fut pas longtemps à acquérir un grade dont il ne fit pas un bon
usage, mais par lequel le roi comptait le mener rapidement loin. Ce sont
choses qui se sont vues ici en leur lieu.

\begin{enumerate}
\def\labelenumi{\arabic{enumi}.}
\setcounter{enumi}{19}
\tightlist
\item
  Le comte de Toulouse fait chevalier de l'ordre, et seul, 2 février
  1693, avant quinze ans.
\end{enumerate}

21 et 22. Déclaration du roi en faveur des duc du Maine et comte de
Toulouse, du 5 mai 1694, registrée le 8 du même mois et an, par laquelle
le roi veut qu'eux et leurs enfants qui naîtront en légitime mariage
aient le premier rang, immédiatement après les princes du sang, et
qu'ils précèdent en tous lieux, actes et cérémonies\ldots. même en la
cour de parlement de Paris et ailleurs, en tous actes de pairie quand
ils en auront, tous les princes des maisons qui ont des souverainetés
hors de notre royaume, et tous autres seigneurs de quelque qualité et
dignité qu'ils puissent être, nonobstant toutes lettres, si aucunes y
avait à ce contraires, et quand même les pairies desdits princes et
seigneurs se trouveraient plus anciennes que celles desdits enfants
naturels.

C'est ce qui s'appela le rang intermédiaire, et on va voir que les deux
bâtards n'étaient pas encore pairs alors. On a vu plus haut que leur
légitimation et ceci fut l'ouvrage de Harlay, procureur général au
premier {[}de ces actes{]}, premier président à l'autre, et qu'à tous
les deux il eut parole des sceaux, qu'il n'eut point, et dont il creva
enfin de rage.

\begin{enumerate}
\def\labelenumi{\arabic{enumi}.}
\setcounter{enumi}{22}
\tightlist
\item
  Lettres de continuation de la pairie d'Eu, en faveur du duc du Maine,
  données en mai 1694, registrées le 8 du même mois et an, pour lui, ses
  hoirs et ayants cause mâles et femelles, sous le titre ancien du comté
  et pairie d'Eu, pour en jouir aux rangs, droits et honneurs, etc.,
  ainsi que les anciens comtes d'Eu avaient fait depuis la première
  érection de 1458.
\end{enumerate}

Le 6 mai 1694 le premier président dit au parlement que le roi l'avait
mandé pour lui expliquer ses intentions au sujet des honneurs qu'il
voulait être rendus au duc du Maine et au comte de Toulouse, lorsqu'ils
iraient au parlement\,;

Que le roi lui dit qu'il voulait qu'il y eût toujours de la différence
entre les princes du sang et les duc du Maine et comte de Toulouse, et
d'eux aux ducs et pairs.

Tout ceci fut encore de l'invention du premier président. On verra enfin
que cette différence d'avec les princes du sang fut bien solennellement
et bien totalement bannie.

\begin{enumerate}
\def\labelenumi{\arabic{enumi}.}
\setcounter{enumi}{23}
\tightlist
\item
  Qu'il savait (le roi) que le duc de Vendôme avait été reçu très jeune
  et sans information, Henri IV l'ayant ainsi souhaité. Il croyait que
  son témoignage pouvait bien servir d'information, et que M. du Maine
  en pouvait être dispensé.
\end{enumerate}

Ce fut une hardiesse et une supercherie. M. de Sully se faisait recevoir
au parlement. On peut juger qu'un favori, surintendant des finances et
grand maître de l'artillerie, y alla bien accompagné. Le duc de Vendôme
y parût tout à coup sans que personne s'y attendît, et prit subitement
sa place. Le parlement se trouva si surpris et en même temps si étonné
qu'il n'osa dire mot, et la chose demeura faite. Pour l'âge, on a vu que
le duc de Luynes, sans aucune faveur ni distinction, fut reçu sans
difficulté, 24 novembre 1639, à dix-neuf ans, et par quel art et quelles
raisons Louis XIV a le premier conduit à la fixation de l'âge.

Qu'il savait aussi qu'il n'y avait que les enfants de France qui
traversassent le parquet de la grand'chambre\,; cependant les princes du
sang étant en possession de le faire, il ne fallait pas donner atteinte
à cette possession, puisque lorsque le duc du Maine prendrait place au
parlement il passerait par le barreau\,;

C'était pour apaiser et flatter les princes du sang, en confirmant pour
la première fois une usurpation qui ne l'avait jamais été et qui n'était
que tolérée. Le prince de Condé, qu'Henri IV fit venir de Saint-Jean
d'Angély pour l'élever à sa cour, se trouvait le plus prochain à
succéder à la couronne. Il traversa le parquet, et comme les honneurs ne
se perdent point, il le traversa toute sa vie, et prétendit que c'était
un droit du premier prince du sang. Traversant un jour le parquet, dans
la minorité de Louis XIII\footnote{Le manuscrit porte \emph{minorité de
  Louis XIII\,;} mais il faut lire évidemment \emph{Minorité de Louis
  XIV}, puisque le duc d'Enghien, dont il est ici question, n'était pas
  né à l'époque où cessa la minorité de Louis XIII.}, M. son fils, qui
le suivait et qui était fier de ses victoires, se mit aussi à le
traverser. M. le Prince se tourna pour l'en empêcher. «\,Allez, allez,
monsieur, votre train et laissez-moi faire, lui répondit le fameux duc
d'Enghien, nous verrons qui osera m'en empêcher.\,» Personne n'osa en
effet, et depuis cette époque tous les princes du sang l'ont toujours
traversé.

\begin{enumerate}
\def\labelenumi{\arabic{enumi}.}
\setcounter{enumi}{24}
\tightlist
\item
  Qu'il voulait que le premier président se découvrit en demandant
  l'avis à M. du Maine, et qu'il lui fît une inclination moindre que
  celle qu'il fait aux princes du sang, en le nommant par le nom de sa
  pairie\,;
\end{enumerate}

Il ne nomme point les princes du sang\,; et les pairs ecclésiastiques il
les nomme par leur nom de pairie, et jamais évêque, mais M. le duc de
Reims, M. le comte de Beauvais, etc.\,; pour le bonnet il en sera
bientôt mention\,: ainsi on n'en dit rien ici.

\begin{enumerate}
\def\labelenumi{\arabic{enumi}.}
\setcounter{enumi}{25}
\tightlist
\item
  Et enfin que les princes du sang à leur sortie de la cour étant
  précédés par deux huissiers jusqu'à la Sainte-Chapelle, le duc du
  Maine ne le serait que par un seul.
\end{enumerate}

Les pairs sortant ensemble, ou un seul s'il n'y en avait qu'un en
séance, ont aussi un huissier devant eux jusque par delà la grande
salle, et quelque chose de plus loin.

\begin{enumerate}
\def\labelenumi{\arabic{enumi}.}
\setcounter{enumi}{26}
\tightlist
\item
  Que l'enregistrement des lettres de la continuation de la comté d'Eu
  en pairie se ferait la grand'chambre et tournelle assemblées.
\end{enumerate}

Non toutes les chambres du parlement.

\begin{enumerate}
\def\labelenumi{\arabic{enumi}.}
\setcounter{enumi}{27}
\tightlist
\item
  Arrêt d'enregistrement et réception du 8 mai 1694, de M. le duc du
  Maine, en qualité de comte d'Eu et de pair de France au parlement
  {[}qui{]}, après le serment par lui fait, sans différence aucune des
  pairs à cet égard, a pris place au-dessous de M. le prince de Conti.
\end{enumerate}

Les princes du sang ne prêtent point de serment.

\begin{enumerate}
\def\labelenumi{\arabic{enumi}.}
\setcounter{enumi}{28}
\item
  Arrêt de réception du 8 juin 1694, de Louis-Joseph, duc de Vendôme, en
  la dignité de pair de France, pour avoir rang et séance, conformément
  aux lettres patentes du roi Henri IV, du 15 avril 1610 (qui depuis la
  mort d'Henri IV étaient demeurées ensevelies), en prêtant par lui le
  serment accoutumé, lequel fait a repris son épée, et a passé sur le
  banc au-dessus de M. l'archevêque-duc de Reims.
\item
  Le premier président avait dit auparavant au parlement, par ordre du
  roi, que l'intention de Sa Majesté était qu'on en usât à la réception
  de M. de Vendôme, et lorsqu'il viendrait en la cour, ainsi qu'on avait
  fait à M. du Maine.
\end{enumerate}

31 Lettres d'érection et de rétablissement de la terre et seigneurie
d'Aumale en titre et dignité de duché-pairie de France, en faveur du duc
du Maine et de ses enfants mâles et femelles, ses héritiers, successeurs
et ayants cause, pour en jouir et user aux mêmes titres, droits et
honneurs que les autres ducs et pairs, etc. Ces lettres données au mois
de juin 1695, registrées 1er juillet même année.

\begin{enumerate}
\def\labelenumi{\arabic{enumi}.}
\setcounter{enumi}{31}
\item
  Lettres de nouvelle érection de la terre et seigneurie de Penthièvre,
  en titre et dignité de duché et pairie de France, en faveur du comte
  de Toulouse, ses hoirs et successeurs et ayants cause, tant mâles que
  femelles, préférant l'aîné et plus capable d'iceux, etc. Ces lettres
  données au mois d'avril 1697, registrées en parlement le 15 décembre
  1698.
\item
  Le comte de Toulouse, gouverneur de Bretagne en mars 1698.
\end{enumerate}

On a vu la violence avec laquelle l'échange des gouvernements de
Bretagne et de Guyenne fut fait, que le duc de Chaulnes ne s'en cacha
pas, et qu'il en mourut tôt après de douleur. On a vu aussi à quel point
Monsieur en fut outré, et combien il éclata sur le manquement de parole
du roi à lui, pour le premier gouvernement de province vacant, qu'au
mariage de M. de Chartres, il s'était engagé de lui donner, et qu'il
éludait par là, et sur la puissance dont il revêtait ses bâtards.

\begin{enumerate}
\def\labelenumi{\arabic{enumi}.}
\setcounter{enumi}{33}
\item
  Le comte de Toulouse, lieutenant général en 1703, et commande la
  cavalerie sur la Meuse\,; va plusieurs fois à la mer.
\item
  Lettres de nouvelle érection des terres d'Arc et de Châteauvillain,
  unies et incorporées ensemble avec leurs dépendances, en duché pairie
  sous le nom de Châteauvillain, en faveur du comte de Toulouse, pour en
  jouir par lui, ses enfants tant mâles que femelles qui naîtront de lui
  en loyal mariage, etc., données en mai 1703, registrées au parlement
  29 août même année.
\end{enumerate}

Il avait d'abord, et avant Penthièvre, eu l'érection en sa faveur de la
terre de Damville en duché-pairie, et c'est sous ce nom qu'il fut reçu
au parlement. On ne la tire point ici en ligne, parce qu'il vendit
depuis cette terre à M\textsuperscript{me} de Parabère, ce qui a éteint
le duché-pairie. Elle est tombée depuis en d'autres mains.

\begin{enumerate}
\def\labelenumi{\arabic{enumi}.}
\setcounter{enumi}{35}
\item
  Le comte de Toulouse, chevalier de la Toison d'or en 1704, revenant de
  commander l'armée navale.
\item
  Dès qu'ils commencèrent à pointer à la cour, le roi leur fit usurper
  peu à peu toutes les manières, l'extérieur et les distinctions des
  princes du sang, sans autre chose marquée que le simple usage qui fut
  bientôt établi chez eux et partout, sans que le roi s'en expliquât que
  par le fait.
\end{enumerate}

C'est ce qui fit que la duchesse du Maine n'eut point en se mariant le
brevet ordinaire aux filles des princes du sang, qui n'épousent pas des
princes du sang, de conservation du rang et honneurs de princesse du
sang, et qu'elle fut obligée de le prendre lors du règlement de
préséance que le roi fit entre les femmes et les filles des princes du
sang.

\begin{enumerate}
\def\labelenumi{\arabic{enumi}.}
\setcounter{enumi}{37}
\item
  Brevet qui conserve à M\textsuperscript{me} la duchesse du Maine son
  rang de princesse du sang, du 13 mars 1710.
\item
  Règlement fait par le roi, le 17 mars 1710, en faveur du prince de
  Dombes, né 4 mars 1700, et du comte d'Eu, né 15 octobre 1701, enfants
  du duc du Maine légitimé de France, portant qu'ils auront, comme
  petits-fils de Sa Majesté, le même rang, les mêmes honneurs et les
  mêmes traitements dont a joui jusqu'à présent ledit duc du Maine.
\end{enumerate}

C'est-à-dire les rang, honneurs, traitement et l'extérieur en plein des
princes du sang sans différence. Cela se glisse ainsi parce que M. du
Maine et M. le comte de Toulouse s'en étaient mis d'abord en possession
par la volonté du roi tacite, sans ordre public, ni par écrit ni verbal.
Ce règlement fut seulement mis en note sur le registre du secrétaire
d'État de la maison du roi. On a vu en son lieu ce qui se passa de
curieux en cette occasion.

40 et 41. Démission de la charge de général des galères faite par le duc
du Maine, 1er septembre 1694, en faveur du duc de Vendôme.

Le duc du Maine pourvu le 10 septembre 1694 de l'office de grand maître
de l'artillerie, vacant par la mort du maréchal-duc d'Humières.

\begin{enumerate}
\def\labelenumi{\arabic{enumi}.}
\setcounter{enumi}{41}
\item
  Le prince de Dombes pourvu en survivance de la charge de colonel
  général des Suisses et Grisons.
\item
  Le comte d'Eu pourvu en survivance de l'office de grand maître de
  l'artillerie, tous deux 16 mai 1710.
\end{enumerate}

44 et 45. Le roi ôte à tous les régiments de cavalerie la compagnie de
carabiniers de chaque régiment, sans les dispenser d'en fournir les
cavaliers, en fait un corps à part divisé en cinq brigades, avec chacune
leur colonel et état-major, en donne le commandement général, détail et
toute nomination des cinq colonels et tous les autres officiers au duc
du Maine.

Outre ce corps, celui des Suisses et Grisons, et celui de l'artillerie,
le duc du Maine avait en particulier, et le comte de Toulouse aussi,
chacun un régiment d'infanterie et un de cavalerie.

46, 47, 48 et 49. L'article 2 de l'édit du mois de mai 1711, portant
règlement général pour les duchés-pairies, registré le 21 des mêmes mois
et an, porte ces mots\,: «\,Nos enfants légitimés et leurs enfants et
descendants mâles qui posséderont des pairies, représenteront
pareillement les anciens pairs au sacre des rois\,; après et au défaut
des princes du sang, et auront droit d'entrée et voix délibérative en
nos cours de parlement, tant aux audiences qu'au conseil à l'âge de
vingt ans, en prêtant le serment ordinaire des pairs, avec séance
immédiatement après les princes du sang, conformément à notre
déclaration du 5 mai 1694\,; et ils y précéderont tous les ducs et
pairs, quand même leurs duchés-pairies seront moins anciennes que celles
desdits ducs et pairs. Et en ce cas qu'ils aient plusieurs pairies et
plusieurs enfants mâles, leur permettons, en se réservant une pairie
pour eux, d'en donner une à chacun de leurs dits enfants si bon leur
semble, pour en jouir par eux aux mêmes honneurs, rangs, préséances et
dignité que dessus, du vivant même de leur père.\,»

\begin{enumerate}
\def\labelenumi{\arabic{enumi}.}
\setcounter{enumi}{49}
\item
  Brevets du 20 mai 1711, par lesquels le roi veut et entend que MM. le
  duc du Maine et le comte de Toulouse continuent à jouir leur vie
  durant à la cour, dans la famille royale, dans toutes les cérémonies
  publiques et particulières, aux audiences des ambassadeurs des princes
  étrangers, aux logements, et généralement en toutes rencontres et
  occasions, des mêmes honneurs qui sont et pourront être rendus aux
  princes du sang, et immédiatement après eux, le tout sans préjudice de
  l'édit du présent mois, que Sa Majesté veut être exécuté dans toute
  son étendue.
\item
  Brevet du 21 mai 1711 par lequel Sa Majesté, ayant égard aux très
  humbles supplications à lui faites par le duc du Maine, a déclaré et
  déclare, veut et entend que les princes et princesses, fils et filles
  de M. le duc du Maine et petits-fils de Sa Majesté, jouissent à
  l'avenir, ainsi qu'ils ont déjà fait, de tous tels et semblables
  honneurs et autres avantages dont ledit duc du Maine a ci-devant joui,
  et est en droit de jouir aux termes du brevet du 20 du présent mois,
  le tout sans préjudice de l'édit du présent mois que Sa Majesté veut
  être exécuté dans toute son étendue.
\end{enumerate}

Voilà l'usurpation de tout l'extérieur de prince du sang faite par le
père, puis par les enfants, de la tacite volonté du roi, non jamais même
verbalement exprimée, passée en titre bien clair et bien libellé par
écrit. Voilà sans doute un brave et succulent mois de mai. Monseigneur
était mort à Meudon le 24 avril précédent.

\begin{enumerate}
\def\labelenumi{\arabic{enumi}.}
\setcounter{enumi}{51}
\item
  Lettres d'érection du marquisat de Rambouillet, auquel sont unies les
  terres, seigneuries et forêt de Saint-Léger en duché-pairie en faveur
  du comte de Toulouse et de ses enfants tant mâles que femelles, etc.,
  données en mai 1711, registrées le 29 juillet même année.
\item
  Le prince de Dombes pourvu en survivance du gouvernement de Languedoc
  en mai 1712.
\item
  Le comte d'Eu pourvu du gouvernement de Guyenne en janvier 1713 vacant
  par la mort du duc de Chevreuse.
\end{enumerate}

Le Dauphin et la Dauphine étaient morts en février 1712, et M. le duc de
Berry en mai 1714. On se hâta d'en profiter.

\begin{enumerate}
\def\labelenumi{\arabic{enumi}.}
\setcounter{enumi}{54}
\item
  Édit du mois de juillet 1714, registré au parlement le 2 août même
  année, qui appelle à la succession à la couronne M. le duc du Maine,
  et M. le comte de Toulouse, et leurs descendants mâles au défaut de
  tous les princes du sang royal, et ordonne qu'ils jouiront des mêmes
  rangs, honneurs et préséances que lesdits princes du sang, après tous
  lesdits princes.
\item
  Prince de Dombes prend séance au parlement précisément en la manière
  des princes du sang à l'occasion de la réception du duc de Tallard au
  parlement le 2 avril 1715.
\item
  Déclaration du roi, 23 mai 1715, registrée au parlement le 24 des
  mêmes mois et an, portant que M. le duc du Maine et M. le comte de
  Toulouse, et leurs descendants en légitime mariage, prendront la
  qualité de princes du sang royal.
\end{enumerate}

On s'arrête ici, parce que ce que le roi fit dans la suite pour bien
assurer cette effrénée grandeur appartient à son testament, dont il ne
s'agit pas encore, et parce que, encore qu'il le fit en même temps, les
dispositions n'en furent sues qu'à l'ouverture de son testament et de
son codicille après sa mort. On ne sut même que quinze jours après qu'il
en avait un, comme on le verra incontinent, sans que personne se fût
douté qu'il y travaillât.

Pour peu qu'on examine ce groupe immense qui, du profond non-être des
doubles adultérins, les porte à la couronne, on sera moins frappé de
l'imagination des poètes qui ont fait entasser des montagnes les unes
sur les autres, à force de bras, par les Titans pour escalader les
cieux. En même temps, l'exemple que ces poètes offrent d'un Encelade et
d'un Briarée se présente aussi bien naturellement a l'esprit, comme le
los le plus juste de pareilles entreprises.

Que les rois soient les maîtres de donner, d'augmenter, de diminuer\,;
d'intervertir les rangs, de prostituer à leur gré les plus grands
honneurs, comme à la fin ils se sont approprié le droit d'envahir les
biens de leurs sujets de toutes conditions et d'attenter à leur liberté
d'un trait de plume à leur volonté, plus souvent à celle de leurs
ministres et de leurs favoris, c'est le malheur auquel la licence
effrénée des sujets a ouvert la carrière, et que le règne de Louis XIV a
su courir sans obstacle jusqu'au dernier bout, devant l'autorité duquel
le seul nom de loi, de droit, de privilège, était devenu un crime. Ce
renversement général, qui rend tout esclave, et qui, par le long usage
de n'être arrêté par rien, de pouvoir tout ce qu'on veut sans nul
obstacle, et de ne recevoir que des adorations à l'envi du fond des
gémissements les plus amers et les plus universels, et de la douleur la
plus sanglante de tous les ordres d'un État opprimé, accoutume bientôt à
vouloir tout ce qu'on peut. Un prince, arrivé et vieilli dans ce comble
extrême de puissance, oublie que sa couronne est un fidéicommis qui ne
lui appartient pas en propre, et dont il ne peut disposer\,; qu'il l'a
reçue de main en main de ses pères à titre de substitution, et non pas
de libre héritage (je laisse à part les conditions abrogées par la
violence et le souverain pouvoir devenu totalement despotique)\,;
conséquemment qu'il ne peut toucher à cette substitution\,; que, venant
à finir par l'extinction de la race légitime, dont tous les mâles y sont
respectivement appelés par le même droit qui l'en a revêtu lui-même, ce
n'est ni à lui ni à aucun d'eux à disposer de la succession qu'ils ne
verront jamais vacante\,; que le droit en retourne à la nation de qui
eux-mêmes l'ont reçue solidairement avec tous les mâles de leur race,
pendant qu'il y en aura de vivants\,; que les trois races ne l'ont pas
transmise par un simple édit, et par volonté absolue de l'une à
l'autre\,; que, si ce pouvoir était en eux, ils le pourraient exercer en
faveur de gui bon leur semblerait\,; que dès lors, il y a moins loin
d'en priver les mâles de leur race appelés solidairement avec eux à la
même substitution, pour en revêtir d'autres à leur gré, que d'usurper le
pouvoir de la disposition même, puisque, si ce pouvoir était en effet en
eux, rien ne pourrait les empêcher d'en user dans toute étendue, et avec
la même injustice, à l'égard des appelés à la substitution avec eux,
qu'ils en usent sans cesse avec tous leurs sujets pour les rangs, les
honneurs et les biens\,; que dès lors chaque roi serait maître de
laisser la couronne à qui bon lui semblerait\,; et que l'exemple de
Charles VI, qui n'est pas l'unique, quoique le plus solennel et le seul
accompli au moins pour le reste de son règne, fait voir qu'il ne serait
pas impossible de voir des rois frustrer de la couronne tous ceux qui y
sont appelés par la substitution perpétuelle, en faveur d'un étranger,
mais jusqu'à leurs propres enfants. On laisse moins à juger quelles
pourraient être les suites de l'exercice de cette usurpation, qui
sautent aux yeux d'elles-mêmes, qu'à considérer que, le premier pas
franchi par cet édit pour la première fois depuis tant de siècles que la
monarchie existe sous trois races, il ne sera pas impossible, pour en
parler avec adoucissement, d'en porter l'abus jusque-là, surtout si on
considère avec soin de quelles infractions légères est sorti
l'abattement entier de tous droits, lois, serments, engagements,
promesses, qui forme cette confusion générale et ce désordre universel
dans tous les biens et les conditions et états du royaume.

Que penser donc d'une créole, publique, veuve à l'aumône de ce poète
cul-de-jatte, et de ce premier de tous les fruits de double adultère
rendu à la condition des autres hommes, qui abusent de ce grand roi au
point qu'on le voit, et qui ne peuvent se satisfaire d'un groupe de
biens, d'honneurs, de grandeurs si monstrueux, et si attaquant de front
l'honnêteté publique, toutes les lois et la religion, s'ils n'attentent
encore à la couronne même\,? et se peut-on croire obligé d'éloigner
comme jugement téméraire la pensée que le prodige de ces édits, qui les
appellent à la couronne après le dernier prince du sang, et qui leur en
donnent le nom, le titre, et tout ce dont les princes du sang jouissent
et pourront jouir, n'aient pas été dans leur projet un dernier échelon,
comme tous les précédents n'avaient été que la préparation à ceux-ci, un
dernier échelon, dis-je, pour les porter-à la couronne, à l'exclusion de
tous autres que le Dauphin et sa postérité\,? Sans doute il y a plus
loin de tirer du non-être par état, et de porter après ces ténébreux
enfants au degré de puissance qu'on voit ici par leurs établissements,
et à l'état de rang entier des princes du sang, avec la même habileté de
succéder à la couronne\,; sans doute, il y a plus loin du néant à cette
grandeur, que de cette grandeur à la couronne. Le total est à la vérité
un tissu exact et continuel d'abus de puissance, de violence,
d'injustice, mais une fois prince du sang en tout et partout, il n'y a
plus qu'un pas à faire\,; et il est moins difficile de donner la
préférence à un prince du sang sur les autres pour une succession dont
on se prétend maître de disposer, puisqu'on se le croit, de faire des
princes du sang par édit, qu'il ne l'est de fabriquer de ces princes
avec de l'encre et de la cire, et de les rendre ainsi tels sans la plus
légère contradiction.

On a coté exprès le nombre des degrés qui ont porté les bâtards à ce
comble, pour n'être pas noyé dans leur nombre. Qu'on examine le
trente-neuvième et le cinquantième, on y trouvera les avantages qui y
sont accordés aux enfants du duc du Maine fondés, libellés, établis, et
causés, comme \emph{petits-fils du roi} ; le mot de naturel y est omis.
Ce n'est pas que cela se pût ignorer, mais enfin il ne s'y trouve point.
Voilà donc le fondement du droit qui leur est accordé en tant de choses
et de façons par ces articles\,! Ce fondement ainsi déclaré et réitéré
est le même qui très explicitement se suppose où il n'est pas exprimé,
pour tout ce qui leur est donné de nouveau\,; ainsi c'est comme
descendants du roi que les descendants de ses deux bâtards sont avec eux
appelés à la couronne après le dernier prince du sang. Mais nul autre
qu'eux, excepté l'unique Dauphin et la branche d'Espagne, ne descendait
du roi. Le Dauphin était unique et dans la première enfance\,; sans père
ni mère, morts empoisonnés\,; la branche d'Espagne avait renoncé à la
succession française\,; M. le duc d'Orléans, rendu odieux et suspect
avec grand art, n'avait qu'un fils et ne sortait que du frère du roi\,;
tous les autres princes du sang d'un éloignement extrême, sortis du
frère du père d'Henri IV, et remontaient jusqu'à saint Louis pour
trouver un aïeul du roi de France. Quelle comparaison de proximité avec
les petits-fils du roi, et combien de raisons, dès que droit et
possibilité s'en trouvent dans leur grand-père, de leur donner la
préférence et à leurs pères qui sont ses fils\,? Et voilà l'aveuglement
où conduit l'abandon aux femmes de mauvaise vie que Salomon décrit si
divinement. Il est vrai que la vie du roi ne fut pas assez longue pour
leur donner le loisir d'arriver à ce grand point.

Mais sans même comprendre cette vue dans le tissu de tant d'effrayantes
grandeurs, laissant à part l'amas d'une puissance si dangereuse dans un
État, et la subversion des premiers, des plus anciens, et des plus
grands rangs du royaume, se renfermant dans l'unique concession du nom,
titre, etc., de prince du sang, et de l'habileté après eux à la
couronne, quel nom donner devant Dieu à une telle récompense d'une
naissance tellement impure, que jusqu'à ces bâtards les hommes en pas un
pays n'ont voulu la connaître ni l'admettre à rien de ce qui a trait au
nom, à l'état, et à la société des hommes, sans s'être jamais relâchés
sur ce point, dans les pays même où l'indulgence est la plus grande à
l'égard des autres bâtards\,? et devant les hommes, y peut-on dissimuler
l'attentat direct à la couronne, le mépris de la nation entière dont le
droit est foulé aux pieds, l'insulte au premier chef à tous les princes
du sang, enfin le crime de lèse-majesté dans sa plus vaste et sa plus
criminelle étendue\,?

Quelque vénérable que Dieu ait rendu aux hommes la majesté de leurs rois
et leurs sacrées personnes, qui sont ses oints, quelque exécrable que
soit le crime d'attenter à leur vie qui est connu sous le nom de
lèse-majesté au premier chef, quelque terribles et uniques que soient
les supplices justement inventés pour le punir et pour éloigner par leur
horreur les plus scélérats de l'infernale résolution de le commettre, on
ne peut s'empêcher de trouver dans celui dont il s'agit une plénitude
qui n'est pas dans l'autre, quelque abominable qu'il soit, si on veut
substituer le raisonnement sur celui-ci au trouble et au soulèvement des
sens qui est un effet naturel de l'impression de l'autre. Cet autre, qui
ne peut être trop exagéré (et que Dieu confonde quiconque oserait le
vouloir exténuer le moins du monde), doit néanmoins, sans tomber dans
cette folie, être examiné tel qu'il est pour en faire une juste
comparaison avec celui dont l'invention est due à la perversité et au
désordre de nos temps, en l'examinant de même. Dans l'un il s'agit de la
vie de l'oint du Seigneur\,; mais quelque horrible que soit ce crime, il
n'attente que sur la vie d'un seul. L'autre joint à la fois la
subversion des lois les plus saintes, et qui subsistent depuis tant de
siècles que dure la monarchie, et en particulier la race heureusement
régnante, sans que l'ambition la plus effrénée ait osé y attenter\,; à
l'extinction radicale du droit le plus saint, le plus important, le plus
inhérent à la nation entière\,; et de cette nation si libre que, jusque
dans son asservissement nouveau, elle en porte encore le nom, et des
restes très évidents de marques, ce crime en fait une nation d'esclaves,
et la réduit au même état de succession purement, souverainement et
despotiquement arbitraire, fort au delà de ce que le czar Pierre Ier a
osé entreprendre en Russie, le premier de tous ses souverains, et qui a
été imité après lui, fort au delà, on le répète, puisqu'il n'y avait
point de maison nombreuse appelée à la couronne comme nos princes du
sang, et encore moins de loi salique, qui est la règle consacrée par
tant de siècles du droit unique à la succession à la couronne de France.
Et qu'on n'oppose point ici les funestes fruits de la guerre des
Anglais, qui, après s'être soumis au jugement rendu en faveur de la loi
salique, ne fondèrent leurs prétentions qu'en impugnant de nouveau cette
loi fondamentale. Qu'on n'allègue point non plus les infâmes desseins de
la Ligue\,; quand on n'aurait pas horreur de s'en protéger, au moins les
ligueurs couverts du manteau de l'hypocrisie, et voulant exclure Henri
IV comme hérétique relaps, respectèrent encore les droits de la nation,
et, supposant qu'il n'y avait plus de princes de la race d'Hugues Capet
en état de régner, après avoir échoué à usurper la couronne comme
prétendus descendants mâles et légitimes de la seconde race, ils
voulurent au moins une figure d'élection, et la tenir de la nation même.

Ici elle n'est comptée que pour une vile esclave, à qui, sans qu'on
songe à elle, on donne des rois possibles et une nouvelle suite de rois,
par une création de princes du sang habiles à succéder à la couronne,
qui ne coûte à établir que la volonté, et une patente à expédier et à
faire enregistrer. Dès lors, comme on l'a dit, une telle puissance,
établie et reconnue, disposera de la couronne non seulement dans un
lointain qui peut ne jamais arriver, mais d'une manière prompte, subite,
active, au préjudice des lois de tous les temps, de la nation entière,
de la totalité de la maison appelée à la couronne, des fils de France
même. Et que penser des désordres si nécessairement causés par un crime
de cette nature, de la vie des princes en obstacle, de celle du roi
même, duquel, de quelque façon que ce soit, douce ou violente, on aurait
arraché cette disposition\,?

Voilà donc un crime de lèse-majesté contre l'État qui entraîne très
naturellement celui qui est connu sous le nom du premier chef, qui égale
les princes du sang, et dans la partie le plus éminemment sensible, à la
condition de tous les autres sujets qui leur peuvent être préférés par
un roi pour lui succéder, et qui ne va pas à moins par une suite
nécessaire qu'à les écraser et à se défaire d'eux. Pendant la violence
de tels mouvements que devient un royaume, et que ne font pas ses
voisins pour achever de l'abattre et pour en profiter\,?

Ces considérations, qui sont parfaitement naturelles, et on ne peut
s'empêcher qu'elles ne sautent aux yeux, ne prouvent-elles pas avec
surabondance, ce qui fait peur à penser, mais qui n'en est pas moins une
vérité frappante, que le crime de se faire prince du sang et habile à
succéder à la couronne avec une patente qui s'enregistre tout de suite,
sans que qui que ce soit ose même en soupirer trop haut, est un crime
plus noir, plus vaste, plus terrible, que celui de lèse-majesté au
premier chef, et qui, outre tous ceux qui à divers degrés portent le nom
de lèse-majesté qu'il renferme, en présente sans nombre qui en aggravent
l'espèce énorme, et qui n'avaient jamais été imaginés.

Rapprochons d'autres temps à celui-ci, quelques-uns même qui n'en sont
pas fort éloignés, et qu'une courte mention en soit permise sans sortir
de ce qui s'en trouve épars dans ces Mémoires. Cette tendresse d'un roi
puissant pour les enfants de son amour, cultivée sans cesse par la
dépositaire funeste de son cœur qui avait été leur gouvernante, et qui
aimait M. du Maine comme son propre fils depuis le sacrifice entier
qu'il lui avait fait de sa propre mère\,; cette jalouse et superbe
préférence de sentiment des enfants de la personne, et qui n'étaient
rien que par elle, sur les enfants du roi, grands par cet être
indépendant de lui qui fut toujours un si puissant ressort dans l'âme de
Louis XIV, avaient bien pu l'engager en leur faveur aux premiers excès
sur l'extérieur des princes du sang tacitement usurpé, et à leur
prodiguer les charges et les biens, même à marier leurs sœurs dans les
nues. Mais on a vu qu'il résista longtemps au mariage des frères, et
qu'il ne feignit pas de dire et de répéter que ces espèces-là ne
devaient jamais se marier.

En effet ce fut à toutes peines et à la fin sous le seul prétexte de la
conscience, que M. du Maine arracha la permission de se marier. On a vu
que Longepierre fut honteusement chassé de chez le comte de Toulouse et
de la cour pour avoir parlé de son mariage avec M\textsuperscript{lle}
d'Armagnac dont il était amoureux, toute neuve encore, d'une naissance
plus que très sortable, et fille de l'homme de son temps à qui le roi a
témoigné l'amitié, la distinction, la considération la plus constante et
la plus marquée toute sa vie. On a vu que le comte de Toulouse, en tout
si heureusement différent de son frère, n'a osé songer à se marier tant
que le roi a vécu. On a vu par quels longs et artificieux détours le duc
de Vendôme parvint au commandement des armées, avec quelle sécheresse il
fut refusé d'y rouler d'égal avec les maréchaux de France, c'est-à-dire
de commander à ceux qui étaient ses cadets lieutenants généraux, en
obéissant aux autres plus anciens lieutenants généraux que lui. On a vu
encore en quels termes le roi répondit au maréchal de Tessé, qui allant
en Italie, y rencontrerait le duc de Vendôme, commandant les armées, car
il y en avait deux corps, et qui demandait les ordres sur sa conduite
avec lui, et de quel ton le roi lui dit qu'il ne devait ni éviter ni
balancer de prendre le commandement sur le duc de Vendôme, et de quel
air il ajouta qu'il ne fallait pas accoutumer ces petits messieurs-là,
ce fut son expression que Tessé m'a rendue à moi et à bien d'autres, à
ces sortes de ménagements. Enfin on ne peut avoir oublié la curieuse
scène du soir du cabinet du roi, lorsqu'il y déclara le rang qu'il
donnait aux enfants de M. du Maine, à combien peu il tint qu'il ne fût
révoqué deux jours après, la réduction ridicule de s'être appuyé de mon
compliment aussi simple que forcé, et de l'éclaircissement que
M\textsuperscript{me} la duchesse de Bourgogne m'en fit demander\,: que
de distance en peu d'espace de temps de façons de penser et de faire\,!

Mais le roi ne pensait pas autrement en se laissant tout arracher. Après
ce grand acte de succession à la couronne déclare, et avant
l'enregistrement de l'édit qui suivit de si près, le roi, accablé de ce
qu'il venait de faire, ne sut se contenter, tout maître de lui-même
qu'il était, de dire en soupirant à M. du Maine, en présence de ce peu
de courtisans intimes, et de ce nombre de valets principaux qui se
trouvaient dans son cabinet à Marly, qu'il avait fait pour eux,
entendant aussi son frère et ses fils, tout ce qu'il avait pu\,; mais
que plus il avait fait, plus avaient-ils à craindre et à travailler à
s'en rendre dignes, pour se pouvoir soutenir après lui dans l'état où il
les avait mis, ce qu'ils ne pouvaient attendre que d'eux-mêmes, par leur
propre mérite. C'était bien laisser échapper ce qu'il sentait et qu'il
ne disait pas, et cela fut incontinent su de tout le monde. Il n'est pas
temps encore de développer par quels moyens le roi fut amené à ce
dernier période, car il peut être confondu avec son testament, qui se
fabriquait en même temps. Nous y arrivons incessamment, puisque entre
les deux déclarations il n'y eut qu'une quinzaine. Délassons-nous
quelques moments par le récit de ce qui se passa entre-deux.

\hypertarget{chapitre-x.}{%
\chapter{CHAPITRE X.}\label{chapitre-x.}}

1714

~

{\textsc{Prostitution du maréchal d'Huxelles.}} {\textsc{- Embarras de
Maisons.}} {\textsc{- Enregistrement de l'édit.}} {\textsc{- Bâtards
traités en princes du sang au parlement.}} {\textsc{- Grand présent du
roi à M\textsuperscript{me} la duchesse de Berry.}} {\textsc{- Électeur
de Bavière et Peterborough à Marly.}} {\textsc{- Promenades nocturnes au
Cours à la mode.}} {\textsc{- Mort de M\textsuperscript{me} de
Vaudemont\,; son caractère.}} {\textsc{- Mort de la marquise de
Béthune-Harcourt.}} {\textsc{- Mort de Virville.}} {\textsc{- Mort de
l'abbé de Clérembault.}} {\textsc{- Sourches cède à son fils la charge
de grand prévôt.}} {\textsc{- Actions devant Barcelone.}} {\textsc{-
Marlborough retourne en Angleterre.}} {\textsc{- Mort de la reine
Anne.}} {\textsc{- L'électeur d'Hanovre proclamé.}} {\textsc{- Routes
profondes par lesquelles le duc du Maine parvient à l'état, nom et tout
droit de prince du sang, et au testament du roi.}} {\textsc{- Fortes
paroles du roi au duc du Maine.}}

~

La cour, Paris, le monde furent étrangement indignés de l'infâme
prostitution du maréchal d'Huxelles, qui vint remercier le roi, en forme
et comme de la plus grande grâce qu'il aurait personnellement reçue, de
ce qu'il venait de faire pour les bâtards. Il brigua de leur donner un
grand dîner, l'un des jours qu'ils devaient employer en sollicitations à
Paris pour la forme. Il n'osa en prier ni ducs ni gens distingués. Enfin
il se donna pour recevoir des compliments sur cette affaire. Il
petillait d'entrer dans le conseil, il séchait d'être duc\,; sa
prostitution ne lui valut ni l'un ni l'autre.

Mais ce qui me donna fort à penser, fut que l'un des deux jours de cette
sollicitation, le duc du Maine et le comte de Toulouse dînèrent à huis
clos chez le président de Maisons. Je ne sais comment un homme d'esprit
pouvait espérer que cela ne se saurait point. Il s'en flatta pourtant,
aussi n'y eut-il nuls convives. Il se trouva fort embarrassé quand je
lui en parlai. Je ne fis pas semblant de le remarquer, et pris pour bon
le hasard qu'il allégua, qu'ils étaient pressés de leurs sollicitations,
parce qu'ils ne couchaient point à Paris\,; qu'ils ne savaient où manger
un simple morceau, parce qu'ils ne voulaient pas s'arrêter à dîner.
Cette conduite me sembla mal ajustée avec les fureurs dont j'avais été
témoin il y avait si peu de jours, et ces messieurs, dans l'apogée de
leur faveur et de leur gloire, ne devaient pas être réduits à ne savoir
où faire un léger repas à la hâte, et avec chacun une maison dans Paris.
Maisons n'avait pas eu cette préférence et cette privance sans l'avoir
recherchée. C'est ce que je fis sentir à M. le duc d'Orléans, avec qui
Maisons se déployait tant en raisonnements contre les bâtards, et que je
crus toujours avoir eu grande part à la scène dont il me rendit
spectateur chez lui, qu'il se doutait bien que je rendrais à ce prince.

Les deux frères, seuls avec leur cortège rassemblé, sans avertir
personne de l'heure de leur visite, allèrent chez tous les pairs et chez
tous ceux des magistrats qui avaient séance à la grand'chambre. Si toute
voix avait été étouffée, et jusqu'aux soupirs retenus, on peut juger
quel crime c'eut été de manquer à cette invitation sous aucun prétexte
que de maladie bien effective et bien évidente. Le jeudi 2 août fut le
grand jour du possible couronnement de cet ordre nouveau de princes du
sang. M. le Duc et M. le prince de Conti, et une vingtaine de pairs,
c'est-à-dire tout ce qui y pouvait assister, s'y trouvèrent. J'y fus
témoin du frémissement public lorsque les deux bâtards parurent, et qui
augmenta avec une sorte de bruit suffoqué, lorsqu'ils se mirent à
traverser lentement le parquet.

L'hypocrisie était peinte sur le visage et sur toute la contenance de M.
du Maine, et une modestie honteuse sur toute la personne du comte de
Toulouse qui le suivait. L'aîné, courbé sur son bâton avec une humilité
très marquée, s'arrêtait à chaque pas pour saluer plus profondément de
toutes parts. Il redoublait sans cesse ses révérences, et y demeurait
plongé en pauses distinguées\,; je crus qu'il s'allait prosterner vers
le côté où j'étais\,; son visage contenu dans un sérieux doux, semblait
exprimer le \emph{non sum dignus} du plus profond de son âme, que ses
yeux, étincelants d'un ravissement de joie, démentaient publiquement, et
qu'il promenait sur tous, comme en les dardant à la dérobée. Il
multiplia encore ses révérences du corps de tous les côtés, arrivé en sa
place avant que s'asseoir, et il fut admirable à considérer pendant
toute la séance, et lorsqu'il en sortit.

Les princes du sang furent ceux qui parurent avoir le moins de part à
tant de courbettes\,; ils étaient trop jeunes pour qu'il en fit cas.

Le comte de Toulouse droit, froid à son ordinaire, avait les yeux
baissés, ses révérences mesurées, point multipliées\,; il ne levait les
yeux que pour les adresser. Toute sa personne témaignait qu'il se
laissait conduire, et sa confusion de ce qui se passait. Il fut immobile
et sans ouvrir la bouche tant qu'il fut en place, regardant comme point,
et l'air concentré, tandis qu'on apercevait le travail du duc du Maine à
contenir tout ce qui lui échappait. Il put jouir à son aise d'un silence
farouche, rarement interrompu par quelques ondulations de murmures
sourds et contenus avec violence, et de regards qui tous, sans exception
que du seul premier président, qui nageait aussi dans une indiscrète
joie, découvraient à plein l'horreur dont chacun était saisi.

Le premier président donna un grand dîner à ces nouveaux successeurs à
la couronne, où le maréchal d'Huxelles se surpassa\,; force domestiques
de ces deux messieurs, quelque magistrature avide du sac, d'Antin, nul
autre duc ni autres gens de marque, quelque peu de mortiers, Maisons
entre autres qui tint dans la séance une contenance fort grave, fort
sérieuse et fort compassée. Le soir, les deux bâtards retournèrent à
Marly.

Quelque peu de satisfaction que le roi eût de M\textsuperscript{me} la
duchesse de Berry, quelque fût son éloignement pour elle, et pour M. le
duc d'Orléans, dans lequel M\textsuperscript{me} de Maintenon
l'entretenait avec tant d'art et de soin sur ce prince, tout ce qu'il
venait de faire pour ses bâtards l'engagea à tâcher d'en émousser
l'amertume par un traitement dont il pût espérer cet effet. M. {[}le
duc{]} et M\textsuperscript{me} la duchesse de Berry avaient fait plus
de cinq cent mille livres de dettes depuis leur mariage\,; ils avaient
fait faire quantité de très beaux meubles, et acheté beaucoup de
pierreries quoiqu'ils en eussent déjà beaucoup, mais
M\textsuperscript{me} la duchesse de Berry en était insatiable. Le roi
lui fit payer pour quatre cent mille livres de dettes\,; et comme il n'y
avait point d'enfants, lui donna tous les meubles et toutes les
pierreries, même celles que M. le duc de Berry avait avant son mariage,
et celles qu'il avait eues de feu Monseigneur.

L'électeur de Bavière vint chasser, jouer et souper à Marly, comme il
avait fait plusieurs fois, sans voir le roi qu'à la chasse. Le comte de
Peterborough, si échauffé pour le service des alliés contre la France,
et qui avait tant fait de voyages et de personnages, de négociations et
de guerres, passa à Paris retournant à Londres de son ambassade de
Turin, et vint dîner à Marly, chez Torcy. Le roi ordonna au duc
d'Aumont, qui l'avait fort connu en Angleterre, et à d'Antin, de lui
faire voir les jardins de Marly, et d'y faire jouer les eaux. Il joignit
le roi à la promenade, qui le traita avec beaucoup de distinction. Il
s'en retourna coucher à Paris, et partit peu de jours après pour
l'Angleterre.

On se mit à Paris à s'aller promener au Cours à minuit, aux flambeaux, à
y mener de la musique, à danser dans le rond du milieu. Cette mode
emporta longtemps tout Paris, et beaucoup de personnes de la cour. Il en
naquit force histoires qui ne corrigèrent personne de continuer à y
aller. Il y avait presque autant de carrosses qu'aux plus beaux jours de
l'été. Cette folie eut son cours, et prit fin avec les derniers jours où
les nuits purent être supportables.

M\textsuperscript{me} de Vaudemont mourut d'apoplexie à Commercy\,; en
entrant le matin dans sa chambre on la trouva râlant, sans connaissance
qui ne revint plus. On a dit ailleurs qui elle était, et qu'elle n'avait
plus d'enfants. Ainsi le duc d'Elbœuf hérita de ce qu'elle avait eu de
son père, et M. de La Rochefoucauld du maternel. Le tout alla à peu de
chose. C'était une dévote précieuse, qui ne put s'accoutumer à n'être
plus une manière de reine, et qui sécha peu à peu de dépit et de douleur
d'avoir vu se dissiper en fumée ses folles prétentions de rang et ses
vastes chimères de faire à la cour et à Paris un grand personnage.
L'unisson avec toutes les dames titrées, dont tout l'art, la souplesse
et les appuis ne la purent distinguer en rien, et la solitude où son air
haut, sec, froid, mécontent, la jetèrent, lui avaient fait prendre
promptement le parti de se confiner à Commercy, où l'ennui acheva de la
tuer. M\textsuperscript{me} d'Espinoy y courut chercher et ramener son
cher oncle, qui, comme tous les grands princes, arriva consolé.

Le maréchal d'Harcourt perdit en même temps sa sœur, mère de la
maréchale de Belle-Ile aujourd'hui, pendant que son mari, le marquis de
Béthune, était allé de la part du roi recevoir à Marseille la reine
douairière de Pologne, sœur de sa mère.

Virville mourut aussi, qui laissa un grand héritage à sa sœur, mariée à
Senozan, riche financier, à qui on avait compté de s'en défaire pour
rien. Virville était sur le point de se marier, il avait une autre sœur,
mais imbécile, que Verderonne, frère de M\textsuperscript{me} de
Pontchartrain, ne laissa pas d'épouser, et dont il n'a point eu
d'enfants. J'ai parlé de la naissance de Virville dont le nom est
Groslée, à l'occasion de la mort de son père qui était frère de la femme
du maréchal de Tallard.

L'abbé de Clérembault mourut aussi. C'était un assez vilain bossu, qui
avait de l'esprit et de la science, et qui ne se produisait pas
beaucoup. Il laissa quatre abbayes. La maréchale de Clérembault, qui
n'avait plus d'autres enfants, ne crut pas que ce fût la peine de s'en
affliger.

En même temps le roi permit à Sourches, prévôt de son hôtel, dit par
abus grand prévot, de céder sa charge à Monsoreau, son fils aîné, ancien
lieutenant général. Sourches était fort vieux, fort menaçant ruine, et
grand dévôt, qui n'avait jamais pu se faire admettre nulle part à la
cour\footnote{M. Bernier a publié des Mémoires du marquis de Sourches
  dont il est ici question. Paris, 2 vol. in-8°.}. Son père y était
considéré dans la même charge, et fut de la promotion de l'ordre de
1661, sans qu'on y trouvât à redire. M. de Louvois empêcha Cavoye, ami
de M. de Seignelay, d'être de celle de 1688. Il n'y put jamais
revenir\,; et j'ai toujours ouï dire que cela avait empêché le grand
prévôt d'en être, le roi ne voulant pas faire Cavoye, ni lui donner le
déplaisir de voir l'ordre au grand prévôt.

Le duc de Berwick emporta le 30 juillet le chemin couvert de Barcelone
sans résistance ni perte. Un des bastions fut attaqué le 13, et fut
bravement défendu. Sauvebœuf et Polastron, colonels de Blésois et de La
Couronne, l'emportèrent\,; le premier y fut tué, l'autre très blessé. La
Couronne s'y maintint valeureusement, mais ayant été relevé le lendemain
par les gardes wallones, elles en furent rechassées.

Le périlleux état où la reine Anne se trouvait rappela le duc de
Marlborough en Angleterre, où la fortune se réconcilia incontinent avec
lui. Anne mourut le 1er août, à cinquante-trois ans, veuve et sans
enfants, après un règne de douze années, dont la fin fut traversée par
beaucoup de factions et de chagrins. On a cru qu'elle avait toujours eu
dessein de faire en sorte que le roi son frère lui succédât, qu'elle
avait sans cesse travaillé sur ce plan, qu'il fut le ressort secret du
changement entier du ministère d'Angleterre à la chute de Godolphin et
de Marlborough, et de la paix. Le roi y perdit une sincère amie, qui
avait ardemment désiré qu'il voulût bien prendre l'ordre de la
Jarretière, à l'exemple de ses pères et d'autres de ses prédécesseurs\,;
mais le roi, qui par amitié pour elle l'aurait accepté volontiers, ne
put se résoudre d'ajouter au préjudice du vrai roi d'Angleterre, et aux
yeux de la reine sa mère, dans Saint-Germain, une nouvelle marque et si
éclatante de sa reconnaissance du droit de la reine Anne. Il eut raison
de la regretter beaucoup. Le deuil fut de six semaines qu'il porta en
violet. L'électeur d'Hanovre fut proclamé aussitôt à Londres, et bientôt
après le ministère entièrement changé, et celui duquel nous tenions la
paix abandonné à la haine et aux recherches.

Il est temps maintenant de venir au testament du roi, qui va paraître
avec de si singulières précautions, tant pour la profondeur du secret de
tout son contenu, que pour l'inviolable sûreté de cette pièce. Le roi
vieillissait, et sans qu'il parût aucun changement à l'extérieur de sa
vie, ce qui le voyait de plus près commençait depuis quelque temps à
craindre qu'il ne vécût pas longtemps. Ce n'est pas ici le lieu de
s'étendre sur une santé jusque-là si forte et si égale\,; il suffit
maintenant de dire qu'elle menaçait sourdement. Accablé des plus
cuisants revers de la fortune, après une si longue habitude de la
dominer, il le fut bien davantage par les malheurs domestiques. Tous ses
enfants avaient disparu devant lui, et le laissaient livré aux
réflexions les plus funestes. Il s'attendait lui-même à tous moments au
même genre de mort. Au lieu de trouver du soulagement à cette angoisse
dans ce qu'il avait de plus intime, et qu'il voyait le plus
continuellement, il n'y rencontrait que peines nouvelles. Excepté le
seul Maréchal, son premier chirurgien, qui travailla sans cesse à le
guérir de ses soupçons, M\textsuperscript{me} de Maintenon, M. du Maine,
Fagon, Bloin, les autres principaux valets de l'intérieur vendus au
bâtard et à son ancienne gouvernante, ne cherchaient qu'à les augmenter,
et dans la vérité ils n'y pouvaient avoir grand'peine. Personne ne
doutait du poison, personne n'en pouvait douter sérieusement\,; et
Maréchal, qui en était aussi persuadé qu'eux, n'en différait d'avis
auprès du roi que pour essayer de le délivrer d'un tourment inutile, et
qui ne pouvait que lui faire un grand mal. Mais M. du Maine avait trop
d'intérêt à le maintenir dans cette crainte, et M\textsuperscript{me} de
Maintenon aussi pour sa haine et pour servir ce qu'elle aimait le mieux,
dont toute l'horreur par leur art en tombait sur le seul prince d'âge,
et de la maison royale, que pour se faire place ils avaient entrepris de
renverser, tellement que le roi, soutenu sans cesse dans ses pensées, et
ayant tous les jours sous ses yeux le prince qu'on lui donnait pour
l'auteur de ces crimes, et à sa table, et à certaines heures dans son
cabinet, on peut juger du redoublement continuel de ses sentiments
intérieurs.

Avec ses enfants il avait perdu, et par la même voie, une princesse
irréparable qui, outre qu'elle était l'âme et l'ornement de sa cour,
était de plus tout son amusement, toute sa joie, toute son affection,
toutes ses complaisances dans presque tous les temps qu'il n'était pas
en public. Jamais depuis qu'il était au monde il ne s'était familiarisé
qu'avec elle\,; on a vu ailleurs jusqu'à quel point cela était porté.
Rien ne pouvait remplir un si grand vide, l'amertume d'en être privé
s'augmentait par ne plus trouver de délassement. Cet état malheureux lui
en fit chercher où il put, en s'abandonnant de plus en plus à
M\textsuperscript{me} de Maintenon et à M. du Maine. Leur dévotion sans
lacune extérieure, leur renfermé continuel le rassurait sur eux. Ils
avaient eu de longue main l'art de lui persuader que M. du Maine,
quoique avec beaucoup d'esprit et de capacité pour les affaires, dans
l'opinion de laquelle il l'entretenait par les derniers détails de ses
charges, et les détails étaient un des grands faibles du roi, ils
l'avaient, dis-je, persuadé que M. du Maine était sans vues, sans
desseins, incapable même d'en avoir, occupé seulement de ses enfants en
bon père de famille, touché de grandeur uniquement par rapport à la
grandeur du roi dont il était par attachement suprêmement amoureux, tout
simple, tout franc, tout droit, tout rond, et qui, après avoir travaillé
tout le jour a ses charges par devoir et pour lui plaire, après avoir
donné bien du temps à la prière et à la piété, se délassait
solitairement à la chasse, et usait dans son petit particulier de la
gaieté et de l'agrément naturel de son esprit, sans savoir le plus
souvent quoi que ce soit de la cour ni de ce qui se passait dans le
monde.

Toutes ces choses plaisaient infiniment au roi, et le mettaient
parfaitement à son aise avec un fils d'ailleurs le bien-aimé, qui
l'approchait si continuellement de si près, et qui l'amusait fort par
ses contes et ses plaisanteries, où il y excellait plus qu'homme que
j'aie jamais connu, avec un tour charmant et si aisé qu'on croyait en
pouvoir dire autant, en même temps adroit à faire du mal, à toucher
cruellement le ridicule, et tout cela avec mesure, suivant le temps,
l'occasion, l'humeur du roi qu'il connaissoit à fond et {[}selon{]} que
les choses prenaient, poussant ou enrayant avec tant d'artifice, de
naturel et de grâce, qu'on aurait dit qu'il ne songeait à rien, et avec
cela, et toujours quand il voulait, le plus excellent pantomime. Que si
on rapproche de ceci son caractère, qui est touché ailleurs, on sentira
avec terreur quel serpent à sonnettes dans le plus intime intérieur du
roi.

Dans l'état où on vient de représenter qu'était le roi, établis l'un et
l'autre dans son esprit et dans son cœur au point où ils l'étaient, et
parfaitement d'accord ensemble, il fut question de profiter d'un temps
précieux qu'ils sentaient bien ne pouvoir plus être long. Si la couronne
même n'était pas leur but, comme il semble difficile d'en douter après
ce qui a été remarqué sur l'édit qui en rend les bâtards capables, au
moins voulaient-ils toutes les grandeurs dont on vient de parler, et
s'assurer en même temps, autant qu'il pouvait être possible, d'une
puissance qui les établit, à la mort du roi, dans un état assez
formidable pour les mettre en situation non seulement de se soutenir
entiers d'une manière durable, mais encore de forcer le régent de
compter sur tout avec eux.

Tout leur riait dans ce vaste dessein\,; eux mêmes en avaient préparé
les voies par les calomnies exécrables dont ils avaient eu l'art
profond, et si bien suivi, de noircir le seul prince à qui la régence ne
pouvait être contestée. Ils étaient parvenus, à force d'artifices et de
manéges obscurs, mais toujours vigilants, à persuader les ignorants et
les simples, à donner des soupçons aux autres, à le rendre au moins
suspect à tous dans Paris et dans les provinces, et plus à la cour
qu'ailleurs, où personne ne voulait ou n'osait approcher de M. le duc
d'Orléans. Ces bruits ne pouvaient pas toujours durer\,; on se lasse
enfin de dire et de parler de la même chose. Ils tombaient donc, mais
tôt après ils reprenaient une nouvelle vigueur. On n'entendait plus
s'entretenir d'autre chose, sans savoir pourquoi cela avait repris\,; et
ces bouffées d'ouragan reprenaient de la sorte et se soutenaient du
temps par les mêmes ressorts qui leur avaient donné le premier être. Ces
bouffées leur servaient infiniment pour réveiller toutes les horreurs du
roi par les récits de ce qu'ils feignaient d'apprendre, et pour
l'entretenir sur son neveu dans les pensées les plus sinistres, dont par
eux-mêmes, sans ces prétextes tirés du public, ils n'auraient osé lui
parler souvent. Par cette conduite soutenue par les valets intérieurs,
ils confirmaient le roi par le public, et le public par le roi, dont
l'éloignement pour son neveu devenait de plus en plus visible à sa cour,
et eux-mêmes le savaient faire répandre. Il n'en fallait pas davantage
pour froncer les courtisans importants, et les autres à leur exemple, à
l'égard de M. le duc d'Orléans, ou par soupçons ou par crainte de se
perdre, les mieux au fait encore plus timides parce qu'ils apercevaient
clairement M. du Maine et M\textsuperscript{me} de Maintenon dans
l'enfoncement de la cour. Le même esprit se répandait dans Paris, et
inondait les provinces. Ces ressorts, ils les faisaient jouer tout à
leur aise. Que pouvait y opposer un prince isolé, dans la cruelle
situation dans laquelle ils l'avaient mis\,? Comment prouver une
négative, et négative de cette espèce\,; et que faire d'ailleurs pour se
dénoircir aux yeux du roi paqueté de la sorte, et du monde ou sot, ou
méchant, ou timide\,? M. du Maine pouvait-il avoir plus beau jeu\,? Il
le sentit si bien, et M\textsuperscript{me} de Maintenon aussi, que, dès
qu'ils se furent assurés d'avoir mis les choses à ce point, ils ne
différèrent plus à se mettre en chemin d'en tirer tout ce qu'ils s'en
étaient proposé pour le présent et pour le futur.

Plus ils connaissoient parfaitement le roi, plus ils en avaient tiré de
choses jusque-là inouïes en faveur des bâtards, plus ils connaissoient
jusqu'à quelle faiblesse la tendresse et la superbe du roi l'avaient
jeté pour eux, mieux aussi ils avaient senti à chaque cran de succès
qu'il était moins un don qu'une conquête, à laquelle des idées anciennes
du roi, comme on l'a dit et on l'a vu, avaient fortement résisté, qu'ils
avaient conquis plutôt qu'obtenu, et qu'ils en étaient redevables à
l'adresse, à l'artifice, au pied à pied, si on peut hasarder ce terme, à
la persévérance, plus qu'à tout au malaise de refuser opiniâtrement les
désirs opiniâtrés de ce qu'on aime, de qui on veut être aimé, et avec
qui on passe uniquement les particuliers les plus libres.

Ces considérations, la dernière surtout, les conduisirent à d'autres. Il
ne s'agissait plus ici de charges, de gouvernements, de survivances,
encore moins d'honneurs, de distinction de rangs. L'affection avait
facilité les premiers\,; la superbe, aidée de leurs artifices, avait
arraché peu à peu les autres. Ils se souvenaient avec terreur de ce qui
s'était passé sur le rang donné aux enfants de M. du Maine, et de
combien près ils avaient frisé l'affront de se le voir révoquer sitôt
après l'avoir emporté. Toutes ces choses étaient épuisées parce qu'elles
étaient au comble. Les ducs, les rangs étrangers, les maréchaux de
France, les ambassadeurs même et les cardinaux, en avaient été
cruellement blessés, mais ce n'avait pas été de quoi les arrêter, et le
roi, malgré ses répugnances tant de fois marquées, s'était enfin laissé
forcer la main à tous ces égards.

Ce qu'ils voulaient maintenant était tout autre chose. Devenir par être
ce que par être on ne peut devenir\,; d'une créature quoique couronnée
en faire un créateur\,; attaquer les princes du sang dans leur droit le
plus sublime et le plus distinctif de toutes les races des hommes\,;
introduire le plus tyrannique, le plus inouï, le plus pernicieux de tous
les droits\,: anéantir les lois les plus antiques et les plus saintes\,;
se jouer de la couronne\,; fouler aux pieds toute la nation\,: enfin
persuader cet épouvantable ouvrage à faire à un homme qui ne peut
commander à la nature, et faire que ce qui n'est pas, soit\,; au chef de
cette race unique, et tellement intéressé à en protéger le droit qu'il
n'est roi qu'à ce titre, ni ses enfants après lui, et à ce roi de la
nation la plus attachée et la plus soumise\,; de la déshonorer et d'en
renverser tout ce qu'elle a de plus sacré, pour possiblement couronner
un double adultère, qu'il a le premier tiré du néant depuis qu'il y a
des François, et qui y est demeuré sans cesse jusqu'à cette heure
enseveli chez toutes les nations, et jusque chez les sauvages\,; la
tentative était étrangement forte, et si\footnote{Et pourtant.} ce
n'était pas tout, parce qu'elle ne pouvait se proposer seule sans
s'accabler sous ses ruines, et perdre de plus tout ce qu'on avait
conquis.

Ils ne virent donc qu'un testament du roi, dicté par eux-mêmes, dont ils
pussent espérer une stabilité de leur nouvel être par le respect du
testateur, et par les nouveaux degrés de puissance dans lesquels ils se
feraient établir. Ce n'était pas que M. du Maine pût ignorer le sort
ordinaire de pareilles précautions\,; mais il n'était pas aussi dans le
cas ordinaire à cet égard, par tout ce que de longue main il avait su
faire jouer d'artifices et de ressorts, toujours depuis si soigneusement
soutenus. Il avait su, comme on l'a expliqué, persuader au roi et au
gros du monde toutes les horreurs sur M. le duc d'Orléans qui lui
étaient les plus utiles\,; il s'agissait maintenant d'en recueillir le
fruit.

Ce fruit était de profiter des dispositions où il avait mis le roi pour
l'engager par conscience, pour la conservation de l'unique rejeton qui
lui succédait immédiatement, pour le salut du royaume, à énerver le plus
qu'il serait possible la puissance d'un prince rendu si suspect, et qui,
par les renonciations, n'avait entre la couronne et soi que ce rejeton
dans la première enfance\,; revêtir, à faute de princes du sang d'âge
raisonnable, ses bâtards de toute l'autorité soustraite au régent\,; de
rendre M. du Maine dépositaire et maître absolu de la personne de ce
rejeton si précieux\,; ne l'environner que des personnes livrées au
bâtard\,; et de lui donner sur elles, et sur toute la maison civile et
militaire, tout pouvoir indépendant du régent.

M. du Maine avait lieu de se flatter que l'impression prise par ses
soins dans la cour, dans Paris, dans les provinces, sur M. le duc
d'Orléans, serait puissamment fortifiée par ces dispositions si
déshonorantes, et que tout y applaudirait bien loin qu'on en fût
choqué\,; qu'il se trouverait ainsi montré et reçu comme le gardien et
le protecteur de la vie du royal enfant, à laquelle était attaché le
salut de la France, dont lui-même par là deviendrait l'idole\,; que la
possession indépendante du jeune roi, et de sa maison militaire et
civile, fortifierait avec l'applaudissement public la puissance dont il
se trouverait revêtu dans l'État, aux dépens de celle du régent, par ce
testament\,; que le régent, honni et dépouillé de la sorte, avec
l'horreur qu'on avait eu l'artifice de répandre sur sa personne et
d'entretenir, non seulement ne serait pas en état d'oser rien disputer,
mais même n'aurait pas de quoi se défendre de tout ce que le bâtard
voudrait entreprendre dans les suites contre lui, établi comme il se le
trouverait dans une posture si favorable et si puissante, qui lui
rallierait pour le présent et les personnages et les peuples, et pour
l'avenir ceux dont l'ambition songerait à être portés auprès du roi
majeur par celui auquel il aurait l'obligation de la vie et de la
couronne. Pour arriver lui-même à ce grand état qu'il atteignait dès
lors en projet pour le temps de la majorité, il lui était essentiel de
n'avoir en caractère auprès du jeune prince que des dépendants et des
affidés sur qui il pût entièrement compter, et les faire choisir et
nommer par le testament pour tous les emplois de l'éducation, et pour
les rendre invulnérables au régent par ces choix, et pour n'avoir l'air
de vouloir se rendre absolu s'il les faisait après lui-même, ne pas
s'exposer au mécontentement des aspirants, enfin pour éviter là-dessus
tout prétexte de lutte avec le régent, et avoir en même temps ses
propres choix autorisés du testament qui paraîtroit seul les avoir
faits.

À ce genre de domination, où, en cas de mort, et pour rendre le régent
plus suspect et plus odieux à toute la France par la multiplication des
précautions contre lui sur la conservation de l'enfant si précieux, et
les étendre en faveur de la bâtardise, il fallait substituer un frère à
l'autre, et pour en cacher la grossièreté un gouverneur à celui qui
serait nommé\,; à ce genre de domination, dis-je, M. du Maine n'oublia
pas de penser à un autre, toujours en flétrissant le futur régent de
plus\,: ce fut de ne lui en laisser que le nom, et de faire attribuer en
effet tout le pouvoir de la régence au conseil établi par le même
testament, avec l'application la plus exacte de le composer de façon que
les deux frères y fussent les maîtres par la pluralité des voix. Il
n'est pas temps encore d'expliquer combien M. du Maine sut bien faire
tous ces différents choix. Ils demeurèrent scellés tous sous le plus
impénétrable secret tant que le roi vécut. Il faut donc attendre à les
démêler jusqu'à ce que l'ouverture du testament les déclare.

Il restait encore un point, qui n'était pas le moins difficile, et qui,
comme les précédents, opérât plusieurs choses à la fois, c'était la
sûreté du testament lorsqu'on serait parvenu à le faire faire, une
sûreté qui fût entière, une sûreté qui augmentât le respect pour les
précautions par le bruit et la singularité, une sûreté qui emportât la
voix publique d'avance en faveur du testament, une sûreté enfin qui
rendît l'exécution de tout ce qui s'y trouverait contenu la chose propre
du parlement et de toute la magistrature du royaume. Mais quel moyen de
surmonter la prévention du roi à l'égard du parlement, prise dès les
temps de sa minorité, dont l'impression qui n'avait jamais pu
s'affaiblir l'avait engagé sans cesse à l'abattre avec jalousie, et
souvent indignation\,? esprit et sentiment que diverses difficultés sur
des édits bursaux avaient entretenus, et que les matières de Rome, et en
dernier lieu celles de la constitution, avaient fort aigris. Confier son
testament à la garde du parlement n'était pas, à la vérité, ajouter,
moins encore confirmer ses volontés par l'autorité du parlement, mais
c'était en quelque sorte la reconnaître pour la sûreté de l'instrument,
et même pour les protéger à son ouverture comme d'une pièce dont ils
étaient les dépositaires, et pour laquelle ils devaient s'intéresser. À
qui a connu le roi, la fermeté de ses principes, la force d'une habitude
sans interruption, l'excès de sa délicatesse sur tout ce qui pouvait
avoir le trait le plus imperceptible à son autorité, même dans le plus
grand lointain, cette dernière difficulté paraîtroit insurmontable.

Mais il était dit que, pour la punition du scandale donné au monde
entier par ce double adultère, celui qui, le premier de tous les hommes
et jusqu'à aujourd'hui l'unique, par un excès de puissance l'avait tiré
du néant, et enhardi par là ses successeurs à le commettre, sentirait à
chaque pas qu'il ferait après en sa faveur l'iniquité de ce pas, dans
toute sa force et sa honte\,; qu'il serait entraîné malgré lui à passer
outre\,; et que de degrés en degrés, tous sautés malgré lui, il en
viendrait enfin, en gémissant dans l'amertume de son âme et dans le
désespoir de sa faiblesse, à couronner son crime par la plus prodigieuse
et la plus redoutable apothéose.

Pour arriver à la fois à ce double but, qui ne se pouvait séparer, de
l'habilité de succéder à la couronne avec le nom, titre, état entier de
prince du sang, et du testament, la double place de Voysin était un coup
de partie, et un instrument dans la main de M. du Maine et de
M\textsuperscript{me} de Maintenon, toujours prêt, également nécessaire
et à portée de tout comme chancelier et comme secrétaire d'État, qui
avait prétexte de {[}voir le roi{]} et de travailler avec lui à toute
heure. Ce fut aussi sur lui que porta tout le faix. Il fallait être bien
esclave, bien valet à tout faire, pour oser se charger d'une pareille
insinuation\,; mais il fallait encore plus être instruit à fond de
l'incroyable faiblesse du roi pour l'un et pour l'autre, laissant à part
l'horreur de la chose, celle de ses suites, toute probité, toute
religion, tout honneur, tout lien à sa patrie, à laquelle il ne fallait
pas même tenir par le moindre petit filet. Que si on considère que
Voysin, qui avait marié ses filles, qui n'avait ni fils ni neveux, dont
le grand-père était un des greffiers criminels du parlement, qui au
double comble de son état ne pouvait plus avoir d'objet que de s'y
conserver, qui n'en pouvait tomber en démontrant la chose impossible à
tenter, et plus sûr encore de demeurer entier après le roi par ce trait
d'honneur et de prudence si utile au régent, on sera bien tenté de
croire aux possessions du démon, aussi effectives et réelles que peu
visibles au dehors. Que si de là on jette les yeux sur la mort de ce
malheureux homme, on n'en sera que plus persuadé.

Les deux consuls et leur licteur convinrent donc de tout ensemble, et du
personnage de chacun d'eux dans cette funeste tragédie. Ils ne doutèrent
pas de la résistance et de l'amertume que causerait une si étrange
insinuation et qui ne pouvait avoir de base que la mort peu éloignée à
présenter à un roi de soixante-seize ans, tout effarouché de la mort et
du genre de mort de tous ses enfants. Aussi arrêtèrent-ils qu'elle ne se
ferait que peu à peu et à sages reprises, de peur de se voir la bouche
fermée par une défense de plus revenir à une si dure matière. À chaque
fois que Voysin avait tentée, il rendait compte à ces deux commettants,
et puisait en eux des forces et des lumières nouvelles. Cette sape,
quoique si délicatement conduite, ne trouvant qu'un rocher vif qui
émoussait les outils, M\textsuperscript{me} de Maintenon et M. du Maine
changèrent de batteries, ils ralentirent les efforts de Voysin, qui
avait essayé de tourner ses insinuations en propositions, pour en venir
au plan qu'ils avaient arrêté entre eux\,; tandis qu'eux-mêmes ne se
montrèrent plus au roi que sous une forme entièrement différente de
celle qu'ils avaient constamment prise jusqu'alors devant lui.

Ils n'avaient jamais été occupés qu'à lui plaire et à l'amuser, chacun
en sa manière, à le deviner, à le louer, disons tout, à l'adorer. Ils
avaient redoublé en tout ce qui leur avait été possible, depuis que, par
la mort de la Dauphine, ils étaient devenus tous deux son unique
ressource. Ne pouvant l'amener à leurs volontés en ce qu'ils
considéraient comme si principalement capital, et à quelque prix que ce
fût le voulant arracher, ils prirent une autre forme dans l'entière
sécurité qu'ils n'y hasarderaient rien. Tous deux devinrent sérieux,
souvent mornes, silencieux jusqu'à ne rien fournir à la conversation,
bientôt à laisser tomber ce que le roi s'efforçait de dire, quelquefois
jusqu'à ne répondre pas même à ce qui n'était pas une interrogation
précise. De cette sorte, l'assiduité qui fut toujours la même de
M\textsuperscript{me} de Maintenon dans sa chambre tant que le roi y
était, de M. du Maine dans les cabinets aux temps des particuliers, ne
servait plus qu'à faire sentir au roi un poids d'autant plus triste
qu'il lui était plus inconnu\,; à contenir, par cet air de contrainte et
de tristesse, ce très petit nombre de diverses sortes de gens des
cabinets, et chez M\textsuperscript{me} de Maintenon ce peu de dames,
toujours les mêmes, admises aux dîners particuliers, aux musiques et au
jeu, les jours qu'il n'y avait point de travail de ministres\,; et à
tourner en ennui et en embarras tout ce qui était délassement et
amusement, sans que le roi eût aucun moyen d'en pouvoir chercher
ailleurs.

Ces dames étaient M\textsuperscript{me} d'O, M\textsuperscript{me} de
Caylus, M\textsuperscript{me} de Dangeau, et M\textsuperscript{me} de
Lévi, amie intime et de toute confiance de M\textsuperscript{me} de
Saint-Simon et de moi de tout temps. Elles se mesuraient toujours sur
M\textsuperscript{me} de Maintenon. Elles furent les dupes un temps du
voile de sa santé\,; mais voyant enfin que la durée passait les bornes,
qu'il n'y avait aucuns moments d'intervalle, que le visage n'annonçait
aucun mal, que la vie ordinaire n'était en rien dérangée, que le roi
devenait aussi sérieux, aussi triste, chacune se sondait, se tâtait. La
crainte de quelque chose qui les regardât troubla chacune d'elles, et
cette crainte les rendit encore de plus mauvaise compagnie que la
retenue ou le modèle de M\textsuperscript{me} de Maintenon les
contraignait.

Dans les cabinets, c'étaient pour toute ressource les froids récits de
chasses et de plants de Rambouillet que faisait le comte de Toulouse,
qui ne savait rien du complot, mais qui n'était pas amusant, quelque
conte de quelqu'un des valets intérieurs, qui se ralentirent dès qu'ils
s'aperçurent que M. du Maine ne ramassait plus rien et ne les faisait
plus durer et valoir à son ordinaire. Maréchal et tous les autres,
étonnés de ce morne inconnu du duc du Maine, se regardaient sans pouvoir
en pénétrer la cause. Ils voyaient le roi triste, ennuyé, ils en
craignirent pour sa santé, mais pas un d'eux ne savait et n'osait que
faire. Le temps coulait, et dans l'un et l'autre des deux particuliers
le morne s'épaississait. Voilà jusqu'où il a été permis aux plus
instruits de l'extérieur des particuliers de pénétrer, et ce serait
faire un roman que vouloir paraître l'être des scènes qui, sans doute,
se passèrent dans le tête-à-tête pendant le long temps que ce manège
dura sans se relâcher en rien. La vérité exige également d'avouer ce que
l'on sait, et d'avouer ce que l'on ignore\,; je ne puis donc aller plus
loin, ni percer plus avant dans l'épaisseur de ces mystères de ténèbres.

Ce qui est certain, c'est que les deux intérieurs se rassérénèrent tout
à coup, avec la même surprise des témoins que ce morne si continu leur
avait causée, parce qu'ils ne pénétrèrent pas plus la cause de la fin
que celle du commencement, et qu'ils n'arrivèrent que tout à la fois à
cette double connaissance, que quelques jours après que
M\textsuperscript{me} de Maintenon et M. du Maine eurent repris auprès
du roi, et avec une sorte d'usure, leur forme ordinaire, c'est-à-dire à
l'épouvantable fracas de la foudre qui tomba sur la France, et qui
étonna toute l'Europe. Il faut venir maintenant au noir événement qui
suivit l'autre de si près, et qui furent résolus ensemble.

On a déjà vu, par ce qu'il était échappé au roi de dire à M. du Maine,
sur ce qu'il venait de faire en sa faveur pour l'habilité de succéder à
la couronne, par l'air et le ton qui fut tant remarqué, combien malgré
lui cette énormité lui avait été forcément arrachée. Maintenant on va
voir encore que ce monarque, de tous les hommes le plus maître de soi,
ne se rendit pas moins transparent sur cela encore, et sur ce qui
regardait son testament. Quelques jours avant que cette nouvelle
éclatât, plein encore de l'énormité de l'état et droits entiers de
prince du sang, et d'habilité de succéder à la couronne qui venait de
lui être arrachée pour ses bâtards, il les regarda tous deux dans son
cabinet, en présence de ce petit intérieur de valets, et de d'Antin et
d'O, et d'un air aigre et qui sentait le dépit, il se prit tout à coup à
leur dire, adressant la parole et un oeil sévère à M. du Maine\,:
«\,Vous l'avez voulu, mais sachez que, quelque grands que je vous fasse,
et que vous soyez de mon vivant, vous n'êtes rien après moi, et c'est à
vous après à faire valoir ce que j'ai fait pour vous, si vous le
pouvez.\,» Tout ce qui était présent frémit d'un éclat de tonnerre si
subit, si peu attendu, si entièrement éloigné du caractère du roi et de
son habitude, et qui montrait si naïvement l'ambition extrême du duc du
Maine, et la violence qu'il avait faite à la faiblesse du roi, qui
semblait si manifestement se la reprocher, et au bâtard son ambition et
sa tyrannie.

Ce fut alors que le rideau se leva devant tout cet intérieur, jusque-là
si surpris, si étonné, si en peine des changements si marqués, si suivis
de M. du Maine dans cet intérieur, qui viennent d'être expliqués il n'y
a pas longtemps. Deux jours après, ce qui arriva acheva de lever le
rideau. La consternation de M. du Maine parut extrême à cette sortie si
brusque, et que nul propos qui vint à cela n'avait attirée. Le roi s'y
était abandonné de plénitude. Tout ce qui était là, les yeux fichés sur
le parquet, en étaient à retenir leur haleine. Le silence fut profond un
temps assez marqué\,; il ne finit que lorsque le roi passa à sa
garde-robe, et qu'en son absence chacun respira. Il avait le coeur bien
gros de ce qu'on lui avait fait faire\,; mais, semblable à une femme qui
accouche de deux enfants, il n'avait encore mis au monde qu'un monstre,
et il en portait encore un second dont il fallait se délivrer, et dont
il sentait toutes les angoisses, sans aucun soulagement des douleurs que
lui avait causées le premier.

\hypertarget{chapitre-xi.}{%
\chapter{CHAPITRE XI.}\label{chapitre-xi.}}

1714

~

{\textsc{Testament du roi.}} {\textsc{- Ses paroles en le remettant au
premier président et au procureur général pour être déposé au
parlement.}} {\textsc{- Paroles du roi à la reine d'Angleterre sur son
testament.}} {\textsc{- Lieu et précautions du dépôt du testament du
roi.}} {\textsc{- Édit remarquable sur le testament.}} {\textsc{-
Consternation générale sur le testament, et ses causes.}} {\textsc{- Duc
d'Orléans\,; sa conduite sur le testament.}} {\textsc{- Dernière marque
de l'amitié et de la confiance du roi pour le duc de Beauvilliers, et de
celles du duc pour moi.}} {\textsc{- Mort du duc de Beauvilliers.}}
{\textsc{- Sa maison\,; sa famille.}} {\textsc{- Son caractère et son
éloge.}} {\textsc{- Époque et nature de la charge de chef du conseil
royal des finances, que le duc de Beauvilliers accepte difficilement.}}
{\textsc{- Malin compliment du comte de Grammont au duc de
Saint-Aignan.}}

~

On était lors à Versailles. Le lendemain 27 août, Mesmes, premier
président, et Joly de Fleury\footnote{Saint-Simon a biffé le nom de Joly
  de Fleury et l'a remplacé par celui de d'Aguesseau. Voy. plus bas la
  note suivante.}, procureur général, que le roi avait mandés, entrèrent
dans son cabinet à l'issue de son lever\,; ils avaient vu le chancelier
chez lui auparavant, la mécanique de la garde du dépôt y avait été
arrêtée. On peut juger que dès que le duc du Maine avait été bien assuré
de son fait, il l'avait bien discutée avec le premier président, sa
créature. Seuls avec le roi, il leur tira d'un tiroir sous sa clef un
gros et grand paquet cacheté de sept cachets (je ne sais si M. du Maine
y voulut imiter le mystérieux livre à sept sceaux de l'Apocalypse, pour
diviniser ce paquet). En le leur remettant\,: «\,Messieurs, leur dit-il,
c'est mon testament\,; il n'y a qui que ce soit que moi qui sache ce
qu'il contient. Je vous le remets pour le garder au parlement, à qui je
ne puis donner un plus grand témoignage de mon estime et de ma
confiance, que de l'en rendre dépositaire. L'exemple des rois mes
prédécesseurs et celui du testament du roi mon père ne me laissent pas
ignorer ce que celui-ci pourra devenir\,; mais on l'a voulu, on m'a
tourmenté, on ne m'a point laissé de repos, quoi que j'aie pu dire. Oh
bien\,! j'ai donc acheté mon repos. Le voilà, emportez-le, il deviendra
ce qu'il pourra\,; au moins j'aurai patience et je n'en entendrai plus
parler.\,» À ce dernier mot, qu'il finit avec un coup de tête fort sec,
il leur tourna le dos, passa dans un autre cabinet et les laissa tous
deux presque changés en statues. Ils se regardèrent, glacés de ce qu'ils
venaient d'entendre, et encore mieux de ce qu'ils venaient de voir aux
yeux et à toute la contenance du roi, et dès qu'ils eurent repris leurs
sens ils se retirèrent et s'en allèrent à Paris. On ne sut que
l'après-dînée que le roi avait fait un testament, et qu'il le leur avait
remis. À mesure que la nouvelle se publia, la consternation remplit la
cour, tandis que les flatteurs, au fond aussi consternés que le reste de
la cour et que Paris le fut ensuite, se tuèrent de louanges et d'éloges.

Le lendemain lundi 28, la reine d'Angleterre vint de Chaillot, où elle
était presque toujours avec M\textsuperscript{me} de Maintenon. Le roi
l'y fut trouver. Dès qu'il l'aperçut\,: «\,Madame, lui dit-il en homme
plein et fâché, j'ai fait mon testament, on m'a tourmenté pour le
faire\,;» passant lors les yeux sur M\textsuperscript{me} de
Maintenon\,: «\,J'ai acheté du repos\,; j'en connais l'impuissance et
l'inutilité. Nous pouvons tout ce que nous voulons tant que nous
sommes\,; après nous, nous pouvons moins que les particuliers\,; il n'y
a qu'à voir ce qu'est devenu celui du roi mon père, et aussitôt après sa
mort, et ceux de tant d'autres rois. Je le sais bien, malgré cela on l'a
voulu, on ne m'a donné ni paix, ni patience, ni repos qu'il ne fût
fait\,; oh bien\,! donc, madame, le voilà fait, il deviendra ce qu'il
pourra, mais au moins on ne m'en tourmentera plus.\,»

Des paroles aussi expressives de la violence extrême soufferte, et du
combat long et opiniâtre avant de se rendre, de dépit et de guerre
lasse, aussi évidentes, aussi étrangement signalées, veulent des preuves
aussi claires, aussi précises qu'elles le sont elles-mêmes, et tout de
suite les voici. Je tiens celles que le roi dit au premier président et
au procureur général du premier qui n'avait eu garde de les oublier\,;
il est vrai que ce ne fut que longtemps après, car il faut être exact
dans ce que l'on rapporte. Je fus entre deux ans brouillé avec le
premier président jusqu'aux plus grands éclats\,; la durée en fut
longue. Il fit tant de choses pour se raccommoder avec moi après le
mariage de sa fille avec le duc de Lorges, sur quoi je me portai aux
plus grandes extrémités, qu'enfin le raccommodement se fit, et si bien
que je devins avec lui à portée de tout\,; et que sa sœur,
M\textsuperscript{me} de Fontenilles, femme d'une piété et d'un esprit
rare, devint une de nos plus intimes amies, de M\textsuperscript{me} de
Saint-Simon et de moi, sans que cela se soit démenti un moment depuis.
C'est alors que le premier président me raconta mot pour mot ce que le
roi leur dit en leur remettant le testament, que le procureur général me
raconta précisément et de même, tous deux chacun à part et en temps
différents\footnote{Cette phrase\,: \emph{que le procureur général me
  raconta précisément et de même, tous deux chacun à part et en temps
  différents}, a été biffée par Saint-Simon qui a ajouté la note
  suivante\,: «\,Je me suis ici trompé de nom et de mémoire, Fleury
  n'étoit pas lors procureur général, et ne sut que par le premier
  président et par le procureur général, qui étoit d'Aguesseau, ce que
  le roi leur avoit dit. Je fais cette note pour rendre raison de la
  rature de ce que j'écrivis avant hier.\,»}, tel exactement que je le
viens d'écrire. Il n'est pas temps de parler de cette brouillerie, moins
encore du raccommodement\,; mais il m'a paru nécessaire de faire ici
cette explication.

À l'égard de ce que le roi dit à la reine d'Angleterre, qui est encore
bien plus fort et bien plus expliqué, parce qu'il était plus libre avec
elle, peut-être encore parce que M\textsuperscript{me} de Maintenon
était en tiers, sur laquelle en plus grande partie tombaient les
reproches que le dépit d'être violenté lui arrachait, je le sus deux
jours après de M. de Lauzun, à qui la reine d'Angleterre le raconta,
encore dans sa première surprise. Nous le fûmes à tel point que
M\textsuperscript{me} de Lauzun, pour qui la reine avait beaucoup
d'amitié et d'ouverture, se hâta de lui aller faire sa cour, et elle la
voyait souvent et souvent en particulier tête à tête, pour se le faire
raconter. La reine ne s'en fit pas prier, tant elle était encore pleine
et étonnée, et lui rendit le discours que le roi lui avait tenu mot pour
mot, comme M. de Lauzun nous l'avait dit, et tel que je l'ai exactement
écrit ici.

Il parut à l'altération si fort inusitée du visage du roi, de toute sa
contenance, du bref et de l'air sec et haut de son parler plus rare
encore qu'à l'ordinaire, et de ses réponses sur tout ce qui se
présentait, à l'embarras extrême et peiné de M\textsuperscript{me} de
Maintenon que ses dames familières virent à plein, à l'abattement du duc
du Maine, que la mauvaise humeur dura plus de huit jours, et ne
s'évapora ensuite que peu à peu. Il est apparent qu'ils essuyèrent des
scènes\,; mais ils tenaient tout ce qu'ils avaient tant désiré, et ils
se trouvaient quittes à bon marché d'essuyer une humeur passagère, sûrs
encore par ce qu'ils venaient d'éprouver que, la souffrant avec patience
et accortise, et reprenant et redoublant même leurs manières accoutumées
avec lui, il se trouverait bientôt trop heureux de se rendre et de
goûter ce repos qu'il avait si chèrement acheté d'eux.

Aussitôt que le premier président et le procureur général furent de
retour à Paris, ils envoyèrent chercher des ouvriers, qu'ils
conduisirent dans une tour du palais, qui est derrière la buvette de la
grand'chambre et le cabinet du premier président, et qui répond au
greffe. Ils firent creuser un grand trou dans la muraille de cette tour,
qui est fort épaisse, y déposèrent le testament, en firent fermer
l'ouverture par une porte de fer, avec une grille de fer en deuxième
porte, et murailler encore par-dessus. La porte et la grille eurent
trois serrures différentes, mais les mêmes à la porte et à la grille, et
une clef pour chacune des trois, qui par conséquent ouvraient chacune
deux serrures. Le premier président en garda une, le procureur général
une autre, et le greffier en chef du parlement la troisième. Ils prirent
prétexte de la donner au greffier en chef sur ce que ce dépôt était tout
contre la chambre du greffe du parlement, pour éviter la jalousie entre
le second président à mortier et le doyen du parlement, et la division
que la préférence aurait pu causer. Le parlement fut assemblé en même
temps, à qui le premier président rendit le compte le plus propre qu'il
lui fut possible à flatter la compagnie, et à la piquer d'honneur sur la
confiance de ce dépôt et le maintien de toutes les dispositions qui s'y
trouveraient contenues.

En même temps les gens du roi y présentèrent un édit que le premier
président et le procureur général avaient reçu des mains du chancelier à
Versailles le même matin que le roi leur remit son testament, et y
firent enregistrer cet édit. Il était fort court. Il déclarait que le
paquet remis au premier président et au procureur général contenait son
testament, par lequel il avait pourvu à la garde et à la tutelle du roi
mineur et au choix d'un conseil de régence, dont, pour de justes
considérations, il n'avait pas voulu rendre les dispositions
publiques\,; qu'il voulait que ce dépôt fût conservé au greffe du
parlement jusqu'a la fin de sa vie\,; et qu'au moment qu'il plairait à
Dieu de le retirer de ce monde, toutes les chambres du parlement
s'assemblassent avec tous les princes de la maison royale et tous les
pairs qui s'y pourraient trouver, pour, en leur présence, y être fait
ouverture du testament, et après sa lecture, les dispositions qu'il
contenait être rendues publiques et exécutées sans qu'il fût permis à
personne d'y contrevenir, et les \emph{duplicata} dudit testament être
envoyés à tous les parlements du royaume, etc., par les ordres du
conseil de régence, pour y être enregistrés.

Il fut remarquable que dans tout cet édit il n'y eut pas un seul mot
pour le parlement, ni d'estime, ni de confiance, ni même un seul mot sur
le choix du greffe du parlement, pour que vaguement encore ce greffe
{[}fût{]} le lieu du dépôt, ni nommer rien qui pût avoir trait à la
garde des clefs. Il était pourtant bien naturel de gratifier le
parlement dans un édit de cette sorte, et si expressément fait sur ce
dépôt, en un mot de faire le moins et le gracieux, puisqu'on faisait le
solide et l'important. C'était bien encore le compte et l'esprit de M.
du Maine d'y flatter le parlement, qui, avec tout le public, fut surpris
de n'y rien trouver du tout qu'un silence sec et dur, et qui parut même
affecté, pour cette compagnie. Quoique ce que le roi avait dit à M. du
Maine sur la dernière grâce qu'il lui avait faite pour l'état de prince
du sang et l'habilité à la couronne, et au premier président, au
procureur général et à la reine d'Angleterre, sur son testament, ne fût
pas public, la surprise extrême des témoins de l'un, et l'étonnement
prodigieux des deux magistrats et de la reine, en avaient laissé
transpirer quelque chose. Le malaise du roi, précédent et long, avait
aussi un peu percé. On ignorait le fond et les détails, mais les gens de
la cour les mieux instruits, et d'autres par eux à la cour et à la
ville, savaient en gros la violence, le dépit, le chagrin marqués du
roi. La sécheresse singulière de l'édit confirma cette persuasion, et on
ne douta point que le roi ne se fût roidi à vouloir l'édit de cette
sorte par humeur, et qu'il n'en eût fallu passer par là.

On a dit en passant que la consternation fut grande à la nouvelle du
testament. C'était le sort de M. du Maine d'obtenir tout ce qu'il
voulait, mais avec la malédiction publique. Ce même sort ne l'abandonna
point sur le testament, et dès qu'il la sentit, il en fut accablé,
M\textsuperscript{me} de Maintenon indignée, et leurs veilles et leurs
soins redoublés pour enfermer le roi de telle sorte que ce murmure ne
pût aller jusqu'à lui. Ils s'occupèrent plus que jamais à l'amuser et à
lui plaire, et à faire retentir autour de lui les éloges, la joie,
l'admiration publique d'un acte si généreux et si grand, en même temps
si sage et si nécessaire au maintien du bon ordre et de la tranquillité
publique, qui le ferait régner si glorieusement au delà même de son
règne.

Cette consternation était bien naturelle, et c'est en cela même que le
duc du Maine se trouva bien trompé et bien en peine. Il avait cru tout
préparer, tout aplanir en rendant M. le duc d'Orléans si suspect et si
odieux\,; il y était en effet parvenu, mais il croyait l'être encore
plus qu'il n'était véritable. Ses désirs, ses émissaires lui avaient
tout grossi\,; et il se trouva dans l'étonnement le plus accablant,
quand, au lieu des acclamations publiques dont il s'était flatté que la
nouvelle du testament serait accompagnée, ce fut précisément tout
l'opposé.

Ce n'était pas qu'on ne vît très clairement que ce testament ne pouvait
avoir été fait que contre M. le duc d'Orléans, puisque, si on n'eût pas
voulu le lier, il n'était pas besoin d'en faire, il ne fallait que
laisser aller les choses dans l'ordinaire et dans l'état naturel. Ce
n'était pas, non plus, que les opinions et les dispositions semées et
inculquées avec tant d'artifice et de suite contre ce prince eussent
changé\,; mais quoi qu'on en pensât, de quelque sinistre façon qu'on fût
affecté à son égard, personne ne s'aveuglait assez pour ne pas voir
qu'il serait nécessairement régent par le droit incontestable de sa
naissance\,; que les dispositions du testament ne pouvaient l'affaiblir
que par l'établissement d'un pouvoir qui balançât le sien\,; que c'était
former deux partis dans l'État, dont chaque chef serait intéressé à se
soutenir, et à abattre l'autre par tout ce que l'honneur, l'intérêt et
le péril ont de plus grand et de plus vif\,; que personne alors ne
serait à l'abri de la nécessité de choisir l'un ou l'autre\,; que ce
choix des deux côtés aurait mille dangers, et nulle bonne espérance pour
soi-même, raisonnable.

Tous les particuliers trouvèrent donc à gémir sur leur fortune, sur
eux-mêmes, sur l'État livré ainsi à l'ambition des partis. Le chef du
plus juste, ou plutôt du seul juste en soi, on l'avait mis en horreur.
Le chef de l'autre, et il n'y avait personne qui n'y reconnût M. du
Maine, qui n'en faisait pas moins par son ambition effrénée qui l'avait
porté où il était à l'égard de la succession à la couronne, qui avait
outré tous les cœurs, et qui, aux dépens des suites qu'on en prévoyait,
voulait après le roi faire contre au régent, et élever autel contre
autel. On comparait les droits sacrés en l'un, nuls en l'autre. On
comparait les personnes, on les trouvait toutes deux odieuses\,; mais la
valeur, la disgrâce, le droit du sang l'emportaient encore sur tout ce
que l'on voyait en M. du Maine. Je ne parle pas du gros monde peu
instruit, et de ce qui se présentait naturellement de soi-même\,;
combien plus dans ce qui l'était davantage, et qui n'avait point de
raison de sortir de neutralité\,!

Ces considérations, dont plus ou moins fortement selon l'instruction et
les lumières, mais l'universalité était frappée, formaient ces plaintes
et ces raisonnements à l'oreille, d'où naissait le murmure qui, bien
qu'étouffé par la crainte, ne laissa pas de percer, et qui partout perça
enfin de plus en plus.

Ce que la raison dictait, ce que les plus considérables voulaient, ce
qui entrait même dans les têtes communes qui font le plus grand nombre
dans ce qu'on appelle le public, n'était rien moins qu'un testament
scellé, qui tenait tout en crainte, et jetait en partialité. Le défaut
de ces hommes illustres par leurs exploits, par leur capacité, par une
longue et heureuse expérience, par là reconnus supérieurs aux autres, et
en possession de primer et d'entraîner par leur mérite et leur
réputation\,; le défaut d'âge de tous les princes du sang\,; les idées
si fausses, mais si fort reçues, qui défavorisaient celui à qui de droit
et de nécessité inévitable les rênes de l'État se trouveraient dévolues,
faisaient souhaiter que le roi mît ordre au gouvernement qui succéderait
au sien, mais non pas dans les ténèbres.

On souhaitait que le roi établît de son vivant le gouvernement tel qu'il
le voulait laisser après lui\,; qu'il mît actuellement dans son conseil
et dans ses affaires ceux qu'il y destinait après lui, et dans les
places et les fonctions qu'ils devaient remplir\,; que lui-même,
gouvernant toujours avec la même autorité, réglât publiquement celle qui
devait succéder à la sienne, dans les limites et dans l'exercice qu'il
avait résolu qu'elle eût\,; qu'il dressât le futur régent, et ceux qui
en tout genre entreraient après lui dans l'administration, à celle que
chacun devait avoir\,; qu'il en formât l'esprit et l'harmonie en se
servant d'eux dès lors en la même façon qu'ils devaient servir après
lui, chacun respectivement au gouvernement de l'État\,; qu'il eût le
temps de voir et de corriger, de changer, d'établir ce qu'il trouverait
en avoir besoin\,; qu'il accoutumât à ce travail, et qu'il instruisit
ceux qu'il ne faisait qu'y destiner, et le reste de ses sujets à voir
ceux-là en place, et à les honorer\,; en un mot à tout exécuter
lui-même, de manière qu'il n'y eût aucun changement à sa mort, qu'elle
n'interrompît pas même la surface des affaires, et qu'il n'y eût qu'à
continuer tout de suite et tout uniment ce qu'il aurait établi lui-même,
dirigé et consolidé.

Mais ce qui était le voeu public, celui même des plus sages, le bien
solide de l'État, n'était pas celui du duc du Maine\,; il craignait trop
le cri public de tout ce qu'il emblait au régent, et le prince qui
devait l'être, qui avec honneur et sûreté n'aurait pu s'y soumettre\,;
le parallèle de la loi et de la faveur aveugle et violente\,; celui de
leur commune base, le sang légitime des rois, dont M. le duc d'Orléans
était petit-fils et neveu, avec le ténébreux néant d'une naissance si
criminelle que jusqu'au duc du Maine elle était inconnue de la société
des hommes\,; enfin la comparaison militaire dans une nation toute
militaire\,; et de la nudité entière du petit-fils de France, avec ce
prodigieux et monstrueux amas de charges, de gouvernements, de troupes,
de rangs et d'honneurs inouïs dont le groupe effrayant servait de
piédestal au double adultère pour fouler aux pieds tous les ordres de
l'État, et y mettre pour le moins tout en confusion pour peu qu'ils
voulussent se servir de la puissance qu'il avait su arracher.

M. du Maine redoutait les réflexions qui naîtraient de ces trop fortes
considérations, et le repentir du roi trop annoncé par la violence qu'il
avait soufferte, dont il n'avait pu retenir ses plaintes\,; et qu'il ne
saisit l'indignation publique accrue par l'exercice des fonctions, pour
détruire ce qu'il avait eu tant de peine à édifier. Enfin il eut peur,
et peut-être le roi plus que lui, des plaintes de ceux qui n'étaient pas
des élus\,: l'un de s'en faire des ennemis qui dès lors se jaindraient à
M. le duc d'Orléans, l'autre de l'importunité des mécontents et des
visages chagrins. Ainsi on était bien éloigné de voir révéler des
mystères que leurs auteurs avaient tant d'intérêt de cacher.

M. le duc d'Orléans fut étourdi du coup\,; il sentit combien il portait
directement sur lui\,; du vivant du roi il n'y vit point de remède. Le
silence respectueux et profond lui parut le seul parti qu'il pût
prendre\,; tout autre n'eût opéré qu'un redoublement de précautions. On
en demeurera là maintenant sur cet article\,; il n'est pas temps encore
d'entrer dans les mesures et dans les vues de ce prince pour l'avenir.
Le roi évita avec lui tout discours sur cette matière, excepté la simple
déclaration après coup\,; M. du Maine de même. Il se contenta d'une
simple approbation monosyllabe avec l'un et avec l'autre, en courtisan
qui ne se doit mêler de rien, et il évita même d'entrer là-dessus en
matière avec M\textsuperscript{me} la duchesse d'Orléans, et avec qui
que ce fût. J'étais le seul avec qui il osât se soulager et raisonner à
fond\,; avec tout le reste du monde un air ouvert et ordinaire, en garde
contre tout air mécontent et contre la curiosité de tous les yeux.
L'abandon inexprimable où il était au milieu de la cour et du monde lui
servit au moins à le garantir de tout propos hasardé sur le testament,
dont personne ne se trouva à portée de lui parler\,; et ce fut en vain
que Maisons, qui affecta de laisser passer quelque temps sans le voir,
essaya par Canillac et par lui-même de le faire parler là-dessus. Ce ne
fut que dans la suite que le duc de Noailles et lui le firent avec plus
de succès, lorsque la santé plus menaçante du roi engagea à s'élargir
sur les mesures à prendre.

Il fallait qu'il y eût déjà du temps que le roi songeât à pourvoir à
l'éducation du Dauphin après lui. Il était bien naturel que, pensant sur
tout comme on le faisait penser de M. le duc d'Orléans, il ne voulût pas
lui en laisser la disposition, et songeât à la faire lui-même. Peut-être
fut-ce par ce point que M\textsuperscript{me} de Maintenon et M. du
Maine firent ouvrir la tranchée devant lui par Voysin, pour de l'un à
l'autre le conduire à tout le reste. Quoi qu'il en soit, étant allé à
Vaucresson fort peu après la mort de M. le duc de Berry, où M. de
Beauvilliers était dans son lit un peu incommodé, il voulut être seul
avec moi. Là il me dit sans préface et sans que la conversation
conduisit, car ce fut tout aussitôt que nous fûmes seuls, qu'il avait
une question à me faire, mais qu'avant de me dire ce que c'était, il
exigeait ma promesse que j'y répondrais sans complaisance, sans
contrainte, mais naturellement, suivant ce que je pensais, et que ce
n'était que sur ce fondement assuré qu'il pouvait me parler.

Je fus surpris de ce propos et je le lui témoignai. Je lui demandai si
depuis tant d'années de bontés et de confiances intimes de sa part pour
moi, et pendant lesquelles il s'était traité et passé tant de choses si
importantes entre nous, l'ouverture, la franchise, la liberté entière de
ma part avec lui, ne devaient pas lui répondre qu'il trouverait toujours
en moi les mêmes. Il me répondit avec toute l'amitié que je lui
connaissois pour moi, et il ajouta que si je lui donnais la parole qu'il
me demandait, je verrais, par ce qu'il avait à me dire, qu'il aurait eu
raison de vouloir s'en assurer. Je la lui donnai donc, encore plus
surpris de cette recharge et plus curieux de ce qui la lui faisait
faire.

Il me dit que le roi n'espérant guère voir le Dauphin en âge de passer
entre les mains des hommes, se croyait être obligé de pourvoir lui-même
à son éducation\,; que le roi l'en voulait charger et de tout ce qui la
regardait comme il l'avait été de celle de Mgrs son père et ses
oncles\,; qu'il s'était excusé sur son âge et ses infirmités qui ne lui
permettaient point les assiduités nécessaires, ni d'espérer même
d'achever l'éducation jusqu'à l'âge qui la termine\,; que le roi,
persistant à vouloir l'en charger, consentait qu'il ne fît que ce qu'il
pourrait et voudrait\,; et tout de suite fixant son regard plus
attentivement sur moi\,: «\,Vous êtes, me dit-il, duc et pair, mon
ancien\,; auriez-vous de la peine à être gouverneur conjointement avec
moi, à suppléer à tout ce que je ne pourrais faire, à agir dans cette
fonction dans un concert entier, en un mot, quoique égaux en fonctions
et plus ancien pair que moi, à n'être pas le premier\,? C'est sur cela
que je vous conjure de me répondre naturellement, sans complaisance, sûr
que je ne serai blessé de rien. Vous voyez, ajouta-t-il, que j'avais
raison de vous en demander votre parole\,; vous me l'avez donnée,
tenez-la-moi à présent.\,»

Je lui répondis que je la lui tiendrais en effet sans peine, que
j'entendais bien que sous un nom pareil c'était être gouverneur sous lui
en tout et partout\,; que je ne connaissois qui que ce fût sans
exception autre que lui, avec qui je l'acceptasse\,; mais que pour lui
que j'avais toute ma vie regardé comme mon père, qui m'en avait servi,
dont je connaissois les talents et la vertu avec une vénération aussi de
toute ma vie, et la confiance et l'amitié par une expérience de même
durée, je serais avec lui et sous lui, en tout et partout, sans en avoir
la moindre peine, et que mon cœur lui était attaché de manière que je
trouverais ma joie à lui marquer sans cesse respect, déférence, et un
abandon dont je lui avais donné une preuve plus difficile sur les
renonciations. Il m'embrassa, me dit que je le soulageais infiniment et
mille choses touchantes.

Il me demanda un profond secret, et de la façon qu'il me parla, j'eus
lieu de croire que, lorsqu'il aurait pesé et fait tous ses arrangements
et ses choix pour la totalité de l'éducation, le roi ne tarderait pas à
les déclarer après qu'il les lui aurait proposés. Je ne laissai pas de
repasser d'autres sujets avec lui par l'importance dont la chose me
parut. Sur deux qui étaient fort en sa main, je lui dis que la vérité
exigeait de moi que je lui avouasse que l'un y était plus propre que
moi\,; que pour l'autre je m'y croyais plus propre. Il ne fit que
glisser sur eux comme sur les autres dont nous parlâmes, ce n'était que
conversation\,: il s'était fixé sur moi. Cela n'était pas nouveau,
puisque Mgr le Dauphin était pleinement déterminé à me demander au roi
pour gouverneur du frère aîné du roi d'aujourd'hui, que je ne l'ignorais
pas, et que ce prince ne pouvait avoir pris et s'être affermi dans cette
résolution que par le duc de Beauvilliers qui ne voulait pas être du
tout gouverneur de ce jeune prince, chargé comme il l'était déjà, et
comme il l'eût été de plus en plus, de fonctions auprès du Dauphin qui
le demandaient tout entier pour la totale confiance de ce prince, et
pour les affaires de l'État.

Telle fut la dernière marque que M. de Beauvilliers me donna de son
estime, de son amitié, de sa confiance\,; tel fut aussi le dernier
témoignage qu'il reçut de celle du roi, malgré la haine persévérante de
M\textsuperscript{me} de Maintenon. Son peu de santé dura trop peu après
cette conversation pour que la matière en pût subsister. Elle était en
soi délicate\,; une vie entièrement partagée entre les exercices de
piété, les fonctions de ses charges dont il ne manquait aucune de celles
qui ne se croisaient pas, et les affaires, ne lui laissait que de courts
délassements, dans le plus intime intérieur de sa famille la plus
étroite, et de moins encore d'amis, et ne contribuait pas à former une
santé bien établie. La perte de ses enfants l'avait foncièrement
pénétré\,; on a vu avec quel courage et quelle insigne piété lui et
M\textsuperscript{me} de Beauvilliers en firent sur l'heure même le
sacrifice, mais ils ne se consolèrent ni l'un ni l'autre. La mort du
Dauphin lui fut encore tout autrement sensible\,: il me l'a avoué bien
des fois. Toute sa tendresse s'était réunie dans ce prince, dont il
admirait l'esprit, les talents, le travail, les desseins, la vertu, les
sacrifices, et la métamorphose entière que la grâce avait opérée en lui
et y confirmait sans cesse\,; il était sensiblement touché de sa
confiance sans réserve, et de leur réciproque liberté à se communiquer,
à discuter et à résoudre toutes choses\,; il était pénétré de l'amour de
l'État, de l'ordre, de la religion qu'il allait voir refleurir, et comme
renaître sous son règne, et en attendant, par sa prudence, sa sagesse,
sa justice, sa modération, son application, et par l'ascendant que le
roi se plaisait à lui laisser prendre sur la cour, sur les affaires, et
sur lui-même. Quelque convaincu qu'il fût de sa sainteté et de son
bonheur, sa mort l'accabla de telle sorte, qu'il ne mena plus qu'une vie
languissante, amère, douloureuse, sans relâche, sans consolation. Enfin,
la mort du duc de Chevreuse, son cœur, son âme, le dépositaire et
souvent l'arbitre de ses pensées les plus secrètes, même de piété, enfin
depuis toute leur vie un autre lui-même, lui donna le dernier coup.

Il fut malade près de deux mois à Vaucresson, où peu auparavant il
s'était retiré et renfermé à l'abri du monde, même de ses plus
familiers, pour ne songer plus qu'à son salut et y consacrer tous les
instants de sa solitude. Il y mourut le vendredi, dernier août, sur le
soir, de la mort des justes, ayant conservé toute sa tête jusqu'à la
fin. Il avait près de soixante-six ans, environ trois ans moins que le
duc de Chevreuse, étant né le 24 octobre 1648 d'une maison fort ancienne
et très noblement alliée, surtout en remontant.

Il était fils de M. de Saint-Aignan qui, avec de l'honneur et de la
valeur, était tout romanesque en galanterie, en belles-lettres, en faits
d'armes. Il avait été capitaine des gardes de Gaston, et tout à la fin
de 1649, acheta du duc de Liancourt la charge de premier gentilhomme de
la chambre du roi, lors duc à brevet. Il commanda ensuite en Berry
contre le parti de M. le Prince, lors prisonnier, puis {[}fut{]}
lieutenant général de l'armée destinée contre MM. de Bouillon et de
Marsillac en Guyenne. Il eut le gouvernement de Touraine à la mort du
marquis d'Aumont, et le crédit de le vendre fort cher à Dangeau encore
jeune, lorsqu'à la disgrâce de M. et de M\textsuperscript{me} de
Navailles, il s'accommoda avec lui du gouvernement du Havre de Grâce en
1664. Il fut chevalier de l'ordre à la promotion de 1661 et duc et pair
en 1663, de cette étrange fournée des quatorze\footnote{Voir. t. Ier,
  p.~449, note de la fin du volume.}. Il fut chef et juge du camp des
derniers carrousels du roi, et mourut à Paris 16 juin 1687. Il avait
épousé une Servien, parente du surintendant des finances, qu'il perdit
en 1679. Au bout de l'an, il se remaria à une femme de chambre de sa
femme qui y était entrée d'abord pour avoir soin de ses chiens. Elle fut
si modeste et lui si honteux que le roi le pressa souvent et toujours
inutilement de lui faire prendre son tabouret. Elle vécut toujours fort
retirée et avec tant de vertus, qu'elle se fit respecter toute sa vie
qui fut longue. Du premier mariage, le comte de Seri et le chevalier de
Saint-Aignan qui fut tué au duel de MM. de La Frette, et l'aîné mourut à
vingt-six ans survivancier de premier gentilhomme de la chambre et
distingué à la guerre, deux fils morts enfants\,; des filles abbesses,
et une qui ne voulut point être religieuse, qu'on maria à Livry, premier
maître d'hôtel du roi, pour s'en défaire. M. de Beauvilliers demeura
seul de ce lit. Du second, deux fils dont l'aîné fut évêque-comte de
Beauvais, l'autre duc de Saint-Aignan, comme on l'a vu en leur lieu, et
une fille aussi romanesque que le père, mais en dévotion, qui épousa un
fils de Marillac, conseiller d'État, tué avancé à la guerre sans
enfants, puis M. de L'Aubépine, mon cousin germain, dont elle a un fils
qui sert et qui est gendre du duc de Sully.

Je ne sais quel soin M. et M\textsuperscript{me} de Saint-Aignan prirent
de leurs aînés. Pour M. de Beauvilliers, ils le laissèrent jusqu'à six
ou sept ans à la merci de leur suisse, élevé dans sa loge, d'où ils
l'envoyèrent à Notre-Dame de Cléry, en pension chez un chanoine, dont
tous les canonicats étaient à la nomination de M. de Saint-Aignan. Ils
ne sont pas gros. Tout le domestique du chanoine consistait en une
servante, qui mit le petit garçon coucher avec elle, lequel y couchait
encore à quatorze et quinze ans, sans penser à mal ni l'un ni l'autre,
ni le chanoine s'aviser qu'il était un peu grand. La mort du comte de
Seri le fit rappeler par son père, qui en même temps lui fit donner la
survivance de sa charge, et remettre deux abbayes qu'il avait. C'était
tout à la fin de 1666. Il servit avec distinction à la tête de son
régiment de cavalerie, et fut brigadier.

Il était grand, fort maigre, le visage long et coloré, un fort grand nez
aquilin, la bouche enfoncée, des yeux d'esprit et perçants, le sourire
agréable, l'air fort doux, mais ordinairement fort sérieux et concentré.
Il était né vif, bouillant, emporté, aimant tous les plaisirs. Beaucoup
d'esprit naturel, le sens extrêmement droit, une grande justesse,
souvent trop de précision\,; l'énonciation aisée, agréable, exacte,
naturelle\,; l'appréhension vive, le discernement bon, une sagesse
singulière, une prévoyance qui s'étendait vastement, mais sans
s'égarer\,; une simplicité et une sagacité extrêmes, et qui ne se
nuisaient point l'une à l'autre\,; et depuis que Dieu l'eut touché, ce
qui arriva de très bonne heure, je crois pouvoir avancer qu'il ne perdit
jamais sa présence, d'où on peut juger, éclairé comme il était, jusqu'à
quel point il porta la piété. Doux, modeste, égal, poli avec
distinction, assez prévenant, d'un accès facile et honnête jusqu'aux
plus petites gens\,; ne montrant point sa dévotion, sans la cacher
aussi, et n'en incommodant personne, mais veillant toutefois ses
domestiques, peut-être de trop près\,; sincèrement humble, sans
préjudice de ce qu'il devait à ce qu'il était, et si détaché de tout,
comme on l'a vu sur plusieurs occasions qui ont été racontées, que je ne
crois pas que les plus saints moines l'aient été davantage. L'extrême
dérangement des affaires de son père lui avait néanmoins donné une
grande attention aux siennes (ce qu'il croyait un devoir), qui ne
l'empêchait pas d'être vraiment magnifique en tout, parce qu'il estimait
que cela était de son état.

Sa charité pour le prochain le resserrait dans des entraves qui le
raccourcissaient par la contrainte de ses lèvres, de ses oreilles, de
ses pensées, dont on a vu les inconvénients en plusieurs endroits. Le
ministère, la politique, la crainte trop grande du roi, augmentèrent
encore cette attention continuelle sur lui-même, d'où naissait un
contraint, un concentré, dirai-je même un pincé, qui éloignait de lui,
et un goût de particulier très resserré, et de solitude qui convenait
peu à ses emplois, qui l'isolait, qui, excepté ses fonctions, parmi
lesquelles je range sa table ouverte le matin, lui faisait un désert de
la cour, et lui laissait ignorer tout ce qui n'était pas les affaires où
ses emplois l'engageaient nécessairement. On a vu où cela pensa le
précipiter plus d'une fois, sans la moindre altération de la paix de son
âme, ni la plus légère tentation de s'élargir là dessus\,; son coeur
droit, bon, tendre, peu étendu\,; mais ce qu'il aimait, il l'aimait
bien, pourvu qu'il pût aussi l'estimer.

Sa crainte du roi, celle de se commettre, ses précisions,
engourdissaient trop son désir sincère de servir ses amis. Il fut tout
autre, comme on l'a vu, sur cela comme sur tout le reste, après la mort
de Monseigneur, et on ne put douter alors qu'il se plaisait à servir ses
amis en petites et en grandes choses.

Dans les particuliers où il était libre, comme chez lui les soirs,
surtout chez le duc de Chevreuse, et à Vaucresson, il était gai, mettait
au large, plaisantait avec sel, badinait avec grâce, riait volontiers.
Il aimait qu'on plaisantât aussi avec lui\,; il n'y avait que le coucher
de la servante du chanoine dont sa pudeur se blessait, et je l'ai vu
quelquefois embarrassé de ce conte que M\textsuperscript{me} de
Beauvilliers faisait, en rire pourtant, mais quelquefois aussi la prier
de ne le point faire.

Il l'épousa en 1671\,; le triste état des affaires de sa maison que son
père avait ruinée, les engagea à faire cette alliance de la troisième
fille de M. Colbert avec de grands biens. L'aînée avait épousé quatre
ans auparavant le duc de Chevreuse, et huit ans après la dernière fille
mariée au duc de Mortemart. Les ducs de Chevreuse et de Beauvilliers et
leurs femmes se trouvèrent si parfaitement faits l'un pour l'autre, que
ce ne fut qu'un cœur, qu'une âme, qu'une même pensée, un même sentiment
toute leur vie, une amitié, une considération, une complaisance, une
déférence, une confiance réciproques. Elle était pareille entre les deux
sœurs, et la devint bientôt entre les deux beaux-frères. Vivant tous
deux à la cour, attachés par leurs charges, et par la place de dames du
palais de leurs femmes, ils se voyaient sans cesse, et mangeaient par
semaine l'un chez l'autre, ce qui dura jusqu'à ce que les grands emplois
du duc de Beauvilliers l'obligèrent à tenir une table publique\,; ils ne
s'en voyaient guère moins, rarement une seule fois par jour tant qu'ils
vécurent. Il était rare aussi d'être ami de l'un à un certain point sans
l'être aussi de l'autre et de leurs épouses.

La piété du duc de Beauvilliers, qui commença de fort bonne heure, le
sépara assez de ceux de son âge. Étant à l'armée, à une promenade du
roi, dans laquelle il servait, il marchait seul un jour un peu en
avant\,; quelqu'un le remarquant se prit à dire qu'il faisait là sa
méditation. Le roi qui l'entendit se tourna vers celui qui parlait, et
le regardant\,: «\,Oui, dit-il, voilà M. de Beauvilliers qui est un des
plus sages hommes de la cour et de mon royaume.\,» Cette subite et
courte apologie fit taire et donna fort à penser, en sorte que les
gloseurs demeurèrent en respect devant son mérite.

Il fallait que le roi en fût dès lors bien prévenu pour le charger de la
commission la plus délicate en 1670. Madame venait d'être si
grossièrement empoisonnée\footnote{Voy., t. III, p.~448, note de la fin
  du volume.}, la conviction en était si entière et si générale qu'il
était bien difficile de le pallier. Le roi et le roi d'Angleterre, dont
elle venait tout nouvellement d'être le plus intime lien par le voyage
qu'elle venait de faire en Angleterre, en étaient également pénétrés de
douleur et d'indignation, et les Anglais ne se contenaient pas. Le roi
choisit le duc de Beauvilliers pour aller faire ses compliments de
condoléance au roi d'Angleterre, et sous ce prétexte tacher que ce
malheur n'altérât point leur amitié et leur union, et calmer la furie de
Londres et de la nation. Le roi n'y fut pas trompé\,; la prudente
dextérité du duc de Beauvilliers ramena entièrement la bouche égarée du
roi d'Angleterre, et adoucit même Londres et la nation.

Le maréchal de Villeroy mourut à Paris en sa quatre-vingt-huitième
année, le 28 novembre 1685. M. Colbert, intendant du cardinal Mazarin,
en même temps intendant des finances à sa mort, avait été recommandé au
roi par ce tout puissant premier ministre comme l'homme le plus capable
qu'il connût pour l'administration des finances, en même temps qu'après
avoir sucé le surintendant Fouquet jusqu'au sang, il le lui avait rendu
plus que suspect. Il ne fut donc pas difficile à Colbert, après la mort
de son maître, de s'introduire auprès du roi, et de s'établir sur les
ruines de Fouquet. Il connaissoit parfaitement le roi sur ce qu'il en
avait ouï dire si souvent à Mazarin. Il le prit par les détails et par
la capacité et par l'autorité de tout faire\,; il acheva de concert avec
Le Tellier la ruine de Fouquet, glissa en la place de contrôleur général
suffoquée jusqu'alors par celle de surintendant. Il persuada au roi le
danger de cette grande place, et, comme il n'osait y aspirer, il fit
accroire au roi de s'en réserver toutes les fonctions. Le roi crut les
faire par les \emph{bons} et les signatures dont Colbert, souple commis,
l'accabla, tandis qu'il saisit toute l'économie et tout le pouvoir des
finances, et qu'il s'en rendit le maître plus qu'aucun surintendant\,;
mais ne se trouvant pas d'aloi à exercer cette autorité sans voile, il
en imagina un de gaze en persuadant au roi de créer une charge toute
nouvelle de chef du conseil des finances qui aurait l'entrée dans ceux
que le roi tiendrait, dans les grandes directions\footnote{Voy., sur les
  conseils du roi, t. Ier, p.~445. Il y est question des conseils de
  finances, connus sous le nom de \emph{grande} et \emph{petite
  direction}.}, qui présiderait chez lui aux petites, qui ferait des
signatures d'arrêts en finances, et qui avec un nom et une
représentation ne ferait rien en effet dans les finances, et lui
laisserait l'autorité entière d'y tout faire et d'y tout régler.

Cette charge fut donc créée lors de la catastrophe de Fouquet, et donnée
au maréchal de Villeroy, qui avait été gouverneur de la personne du roi
sous le cardinal Mazarin, chef de son éducation, et qui avec cette ombre
ne fut jamais ministre d'État. Cela valait quarante-huit mille livres de
rente avec d'autres choses encore, en sorte que cette vacance eut tout
ce qu'il y avait de grand et de plus considérable à la cour pour
aspirants\,: le duc de Montausier, qui avait été gouverneur de
Monseigneur\,; le duc de Créqui, gouverneur de Paris, premier
gentilhomme de la chambre, dont l'ambassade à Rome et la fameuse affaire
des Corses de la garde du pape avait fait tant de bruit, et dont la
femme était dame d'honneur de la reine, et plusieurs autres dans la
privance du roi et dans la première considération.

Le roi leur préféra le duc de Beauvilliers qui avait trente-sept ans et
qui n'avait garde d'y songer. Il en était si éloigné que la délicatesse
de sa conscience, alarmée de tout ce qui sentait les finances, ne put se
résoudre à l'accepter, lorsque le roi la lui donna. La surprise du roi
d'un refus de ce qui faisait l'ambition des plus importants de sa cour
ne servit qu'à le confirmer dans son choix. Il insista et il obligea le
duc à consulter des personnes en qui il pouvait prendre confiance, et de
tirer parole de lui qu'il le ferait de bonne foi, avec une droite
indifférence, et qu'il se rendrait à leur avis s'il allait à le faire
accepter. Le duc s'y engagea et consulta. Au bout de sept ou huit jours
le roi lui en demanda des nouvelles, et le poussa jusqu'à lui faire
avouer qu'il avait trouvé tous les avis de ceux qu'il avait consultés
pour qu'il ne refusât pas davantage. Le roi en fut fort aise, le somma
de sa parole, et le déclara deux heures après, au grand étonnement de sa
cour.

Le comte de Grammont, qui était sur le pied de se divertir de tout aux
dépens de qui il appartenait, et qui savait que le duc de Saint-Aignan
s'était mis aussi sur les rangs pour cette charge, le rencontra dans la
galerie une heure après la déclaration. Il alla droit à lui, et lui dit
«\,qu'il lui faisait ses compliments d'être d'une race si heureuse
qu'elle donnait tous les chefs que le roi choisissait\,: que s'il en
fallait un aux carrousels, il prenait le père\,; s'il y en avait un à
nommer pour le conseil des finances, il choisissait le fils,\,» et sans
attendre de réponse, le laissa là, avec une révérence et une pirouette,
outré de dépit de son compliment.

\hypertarget{chapitre-xii.}{%
\chapter{CHAPITRE XII.}\label{chapitre-xii.}}

1714

~

{\textsc{Duc de Beauvilliers\,; quel sur le cardinal de Noailles, Rome,
Saint-Sulpice, les jésuites.}} {\textsc{- Mesures futures pour
l'archevêque de Cambrai.}} {\textsc{- Ambition de ce prélat.}}
{\textsc{- Grandeur d'âme et de vertu du duc de Beauvilliers.}}
{\textsc{- Comparaison des ducs de Chevreuse et de Beauvilliers.}}
{\textsc{- Mot plaisant et vrai du chancelier de Pontchartrain.}}
{\textsc{- Caractère de la duchesse de Beauvilliers.}} {\textsc{-
Fortune et conduite des Saumery.}} {\textsc{- Épreuve et action de vertu
héroïque de la duchesse de Beauvilliers.}} {\textsc{- Mort de la
duchesse de Beauvilliers en 1733.}}

~

M. de Beauvilliers fut duc en se mariant sur la démission de son père
dont il eut les gouvernements à sa mort, et chevalier de l'ordre de la
promotion de 1688. En 1689 le roi lui demanda s'il ferait autant de
difficultés pour être gouverneur de Mgr le duc de Bourgogne, qu'il
allait ôter d'entre les mains des femmes, qu'il en avait apporté pour la
place de chef du conseil des finances. Il n'en fit aucune et l'accepta.
Il le fut des deux autres fils de France, à mesure qu'ils quittèrent les
femmes\,; et ce fut avec tant de confiance de la part du roi, qu'à
l'exception de Moreau, un de ses premiers valets de garde-robe qu'il fit
premier valet de chambre de ce prince et de deux ou trois valets qu'il y
voulut placer, il laissa tout le reste au choix du duc de
Beauvilliers\,: précepteur, sous-gouverneur et tout le reste, sans faire
de perquisition sur aucun. On a vu ailleurs que ce fut aussi avec tant
de désintéressement de la part du duc qu'il refusa absolument les
appointements pour les deux autres princes\,: quarante-huit mille livres
pour chacun par an, c'est-à-dire quatre-vingt-seize mille livres.

La mort de Louvois, qui rendit le roi libre sur bien des choses, fit
rappeler Pomponne dans le conseil d'État en 1691 aussitôt après, et y
fit entrer le duc de Beauvilliers en même temps. Ce fut un prodige, et
l'unique gentilhomme qui y ait été admis en soixante-douze ans de
règne\,; je dis l'unique, parce que les deux maréchaux de Villeroy qui
ne l'étaient guère plus qu'il ne fallait, le père ne fut jamais
ministre, et le fils, qui ne l'a été qu'un an depuis la mort de M. de
Beauvilliers jusqu'à celle du roi, ne peut être compté en un si court
espace. M. de Beauvilliers n'y songeait pas plus qu'il avait fait à ses
deux autres places.

Quelque excessivement que le roi lui imposât, quelque faible qu'il parût
à loi parler pour des grâces par une timidité qui était en lui, il
n'était pas reconnaissable au conseil, à ce que j'ai ouï dire à
Chamillart son ami, et au chancelier de Pontchartrain son ennemi si
longtemps, lorsqu'il s'agissait d'affaires de justice, ou d'affaires
d'État importantes. Il opinait alors avec fermeté, embrassait toute
l'étendue de l'affaire avec netteté et précision, la développait avec
lumière, prenait son parti avec fondement, et le soutenait avec
modestie, mais avec une force que le penchant montré du roi n'ébranlait
point. Dans les autres il se laissait assez aller à son naturel doux et
timide. Son exactitude, ou, pour parler plus juste, sa ponctualité à ses
diverses et continuelles fonctions, était sans le plus léger relâche,
qui, je crois, avait augmenté sa précision naturelle jusqu'aux minutes,
et jusqu'à savoir ce qu'il lui en fallait pour aller de chez lui chez le
roi.

On a vu ailleurs avec quelle grandeur d'âme, quel détachement, quelle
soumission à Dieu, quelle délicatesse de totale dépendance à son ordre,
il soutint l'orage du quiétisme, la disgrâce de l'archevêque de Cambrai,
de ceux qui y furent enveloppés, et le péril extrême qu'il y courut\,;
avec quelle noblesse il s'y conduisit\,; et avec quelle soumission il
reçut la nouvelle de la condamnation du livre de M. de Cambrai à Rome.
Toutefois les plus rares tableaux ont des ombres, et la vérité m'oblige
à ne pas dissimuler celles de ce modèle de toutes les vertus. En les
considérant on ne l'en estimera pas moins si on est équitable, mais on
tremblera à la vue des profondeurs de Dieu, et on s'humiliera jusqu'en
terre à la vue de ce que sont les hommes les plus parfaits.

Celui-ci, avec la probité la plus innée, l'amour et la soif de la vérité
la plus ardente et la plus sincère, la pureté la plus scrupuleuse, une
présence de Dieu sensible, habituelle dans toutes les diverses fonctions
et situations de ses journées, à qui il rapportait avec une sainte
jalousie ses plus importantes et ses plus légères actions, son travail,
ses fonctions, ses amitiés, ses liaisons, ses vues, ses bienséances, et
jusqu'aux délassements et aux besoins de l'esprit et du corps\,; cet
homme, si droit, si en garde contre lui-même, et d'une attention si
active, se laissa tellement enchanter, lui et M. de Chevreuse aux
charmes de l'archevêque de Cambrai, que sans l'avoir jamais vu depuis sa
disgrâce, ce prélat ne cessa d'être l'âme de son âme et l'esprit de son
esprit, que tout ce qu'il pratiquait dans son intérieur de conscience et
dans son domestique était réglé souverainement par M. de Cambrai,
qu'enchanté d'après lui de M\textsuperscript{me} Guyon, il ne la vit
jamais que sainte, et qu'excellent docteur, enfin que s'étant hasardée à
faire des prophéties claires qu'il vit toutes manquées, le bandeau ne
put jamais lui tomber des yeux. Disons tout et ne retenons point la
vérité captive, on a vu en son lieu la grande et sainte action par
laquelle le cardinal de Noailles le sauva et le maintint dans ses places
aux dépens de son frère, à qui elles étaient destinées de leur su, et
avec lequel il en fut brouillé plusieurs années. Tombé lui-même en
disgrâce par l'affaire de la constitution, jusqu'à la défense de voir le
roi, jusqu'à voir poursuivre la privation de son chapeau et la
déposition de son siège, jusqu'au plus juste soupçon que le roi l'allait
faire enlever et conduire à Rome, j'étais peiné de savoir M. de
Beauvilliers des plus ardents contre lui, et que l'objet si cher de M.
de Cambrai, de la doctrine et du livre duquel le cardinal de Noailles
avait été un de plus grands adversaires, dépouillât cette âme si vraie,
si droite, si candide, de reconnaissance et d'humanité en divinisant ses
préventions.

Je ne pus m'empêcher de lui en parler un jour qu'il vint causer avec moi
dans ma chambre à Versailles comme il faisait assez souvent pour y être
plus en liberté. Après quelque peu de propos\,: «\,Mais vous, monsieur,
lui dis-je à brûle-pourpoint, ne songez-vous jamais que sans la rare
vertu et la pureté d'âme du cardinal de Noailles vous étiez chassé, et
que, de son su, son frère avait toutes vos places\,? Il était sûr de
leur destination, le maréchal et la maréchale de Noailles ont été bien
des années à le lui pardonner. Vous n'ignorez pas qu'il ne vous
raffermit pas sans peine, et qu'il se rendit même votre caution auprès
du roi, et aujourd'hui vous pousseriez un homme à qui vous devez tout,
et depuis si longtemps, et sans lequel vous seriez depuis tant d'années
hors de mesure\,!» Le duc demeura quelques moments sans repartie,
rougit, convint après quelque silence par un seul «\,il est vrai,\,» se
défendit sur sa conscience, mais mollement, et fut toujours depuis fort
mesuré avec moi sur le cardinal de Noailles, lorsque nous traitions ces
matières, où d'ailleurs nous n'étions jamais d'accord. Ce n'était pas
certainement défaut de sentiment dans un homme qui en avait de si
délicats, moins encore ingratitude. Il était très reconnaissant par
nature et par principe, mais telle fut en lui la force d'un abandon
aveugle divinisé en lui pour M. de Cambrai par religion.

Cette même disposition le mettait toujours du côté de Rome sur ses
diverses entreprises, et le rendait industrieux à les exténuer et à les
pallier. Nous en avions souvent des disputes vives. Sa préface était
toujours la même en ces occasions\,: les droits sacrés des rois de
France que saint Louis même avait soutenus contre les papes avec plus de
force qu'aucun autre roi\,; mais le cas dont il s'agissait n'était
jamais, selon lui, de ceux qu'on devait défendre.

Saint-Sulpice où il avait toujours eu sa principale confiance, et non
les jésuites avec qui il vivait bien, mais qu'il connaissoit, et à qui
lui et M. de Chevreuse auraient voulu ôter la feuille et le
confessionnal des rois\,; Saint-Sulpice, dis-je, l'avait gâté de bonne
heure sur Rome, et l'archevêque de Cambrai qui avait ses raisons, qu'il
se gardait bien de lui montrer, avait achevé.

De ces matières et de celles de la constitution, il m'en parlait
toujours le premier, soit confiance, soit espérance de me convertir,
jusqu'à ce que tout à la fin de sa vie disputant là-dessus, tous deux
seuls dans ma chambre à Versailles, il me pria que nous ne nous en
parlassions plus, parce que cela l'agitait trop, et depuis en effet nous
ne nous en sommes jamais parlé.

Avec cet abandon à M. de Cambrai, qui le liait à tout ce petit troupeau
d'une chaîne si forte, il eut la fidélité de n'entretenir son commerce
avec lui que du su du roi, et de ne voir qu'à Vaucresson fort à la
dérobée, mais avec sa permission, ceux que son affaire avait fait ôter
d'auprès des princes, et chasser de la cour. Jamais, comme on le voit,
je n'avais été initié dans ces mystères, mais je les voyais librement à
Vaucresson\,; on y parlait tout librement aussi devant moi\,; et depuis
la mort du Dauphin, M. de Beauvilliers et M. de Chevreuse, ces exilés me
parlaient ouvertement de leur désir extrême du retour de Fénelon.
Jusqu'aux plus petites choses qui pouvaient toucher ce prélat étaient
leur grand ressort à tous, et le plus infailliblement puissant. Les deux
ducs, et je ne l'ai jamais compris, qui demeurèrent toujours dans le
plus parfait silence avec moi sur une doctrine et des principes dont
l'enchantement les avait absorbés, parce qu'ils ne m'en crurent pas
capable ou qu'ils sentirent que je n'y prendrais point, n'en furent non
seulement pas le moins du monde en contrainte avec moi sur toute espèce
de confiance, comme on l'a pu voir par tant de choses qui ont été
racontées, mais ils s'ouvrirent toujours à moi sur leur attachement à M.
de Cambrai, et à ceux qui tenaient à lui par les mêmes liens, et sur
tout ce qui les regardait.

Ils me parlèrent donc franchement après la mort du Dauphin, pour
m'engager à lui être favorable auprès de M. le duc d'Orléans, pour le
rappeler, et l'employer grandement à la mort du roi\,; ils voyaient bien
que ce prince mènerait aisément M. le duc de Berry, sur lequel ils
n'avaient pas lieu de compter avoir grand crédit, comme il a été
remarqué ailleurs, et qui ne se souciait de son précepteur en nulle
sorte\,; je ne m'en souciais pas intérieurement davantage, mais je ne
pouvais rien refuser à M. de Beauvilliers. Je m'engageai donc à lui et à
M. de Chevreuse, et j'eus d'autant moins de peine à réussir, que M. le
duc d'Orléans était naturellement porté d'estime et d'inclination pour
Fénelon. Cette espérance fondée que je leur donnai les combla. Par les
discours du duc de Chevreuse, je compris qu'il l'informait de ce qu'il
se passait à son égard. Je le dis au duc, qui me l'avoua et qui m'en
parla depuis ouvertement, jusqu'à me dire franchement que l'archevêque,
certain de ce que je faisais pour lui, ne laissait pas de me craindre.
Cela me revint encore par d'autres endroits.

Je ne le connaissois que de visage\,; trop jeune quand il fut exilé, je
ne l'avais pas vu depuis. Ainsi il ne pouvait aussi me connaître que par
autrui, et à la façon dont j'étais avec les deux ducs, et à ce que je
voyais librement de cette faciende\footnote{Cabale.} à Vaucresson, il ne
pouvait lui être revenu rien qui lui inspirât cette frayeur. Mais
accoutumé comme il était à régner à la divine sur son royal pupille, sur
les deux ducs, sur tout ce petit troupeau, il craignait de ne régner pas
de même sur M. le duc d'Orléans, de me trouver entre ce prince et lui,
et de ne me pas rencontrer facile à son joug, autant que ceux qu'il y
avait assujettis. Sa persuasion, gâtée par l'habitude, ne voulait point
de résistance\,; il voulait être cru du premier mot\,; l'autorité qu'il
usurpait était sans raisonnement de la part de ses auditeurs, et sa
domination sans la plus légère contradiction\,; être l'oracle lui était
tourné en habitude, dont sa condamnation et ses suites n'avaient pu lui
faire rien rabattre\,; il voulait gouverner en maître qui ne rend raison
à personne, régner directement de plain-pied. Pour peu qu'on se rappelle
ce qui se trouve en son lieu de son caractère et de sa conduite à la
cour, et depuis qu'il en fut chassé, on le reconnaîtra à tous ces
traits. C'est ce qui excita sa crainte à mon égard, dont tout ce que je
fis pour lui, et tout ce qu'il apprenait de moi par les deux ducs, ne
purent le guérir. Son ambition ignorait qu'il ne vivrait pas assez pour
être satisfaite, pas même pour s'en voir dans le chemin.

Quelque solidement humble que fût le duc de Beauvilliers, quelque
déférence qu'il se fût accoutumé d'avoir pour les sentiments du duc de
Chevreuse, il était fort loin de ne penser jamais que comme lui, et de
se rendre à lui sur toutes choses. On en a vu en leurs lieux plusieurs
exemples, un entre autres sur les renonciations où il fut pour moi
contre lui, et où je fus dans une honte et dans une surprise égale,
parce que cela regardait mon avis. L'humilité n'altérait point en lui la
dignité\,; plus il était sincèrement détaché de tout, plus il se tenait
à sa place, sans soins bas ou superflus. Jamais il ne fit un seul pas
vers Monseigneur ni aucun de son intrinsèque qui ne l'aimaient pas, ni
vers M\textsuperscript{me} de Maintenon depuis l'orage du quiétisme, qui
ne lui pardonna jamais d'avoir échappé à tous ses efforts pour le
perdre, qu'elle redoubla, comme on a vu, de temps en temps, et qu'elle
n'abandonna que par en sentir enfin l'impuissance.

Elle haïssait encore plus le duc de Chevreuse, et ne fut pas plus
heureuse contre lui. Il est plaisant qu'avec cela elle aimât assez
M\textsuperscript{me} de Chevreuse, et fort sa fille,
M\textsuperscript{me} de Lévi, qui néanmoins était toute franche et un
avec son père et sa mère et M. et M\textsuperscript{me} de Beauvilliers.
Pour celle-ci, M\textsuperscript{me} de Maintenon ne la pouvait
souffrir. M\textsuperscript{me} de Beauvilliers ne s'en souciait guère,
ne lui rendait aucun devoir, n'était point comme sa soeur des
particuliers du roi, dont elle était pourtant fort bien traitée, et ne
la voyait jamais, sinon rarement par hasard à des promenades, où le roi
la menait et où M\textsuperscript{me} de Maintenon se trouvait
quelquefois, et alors très poliment, également, mais d'une politesse
sèche de part et d'autre. Il n'y eut que les énormités de la campagne de
Lille et leurs suites qui rejoignirent M. de Beauvilliers à
M\textsuperscript{me} de Maintenon, qui en fit les premiers pas. Le
concert fut entier entre eux et le commerce vif, mais qui cessa tout
court avec la matière qui l'avait causé, et ils demeurèrent pour
toujours depuis comme ils étaient auparavant qu'elle fût née.

Quoique inaccessible à ce qui n'était pas de devoir étroit et de
bienséance nécessaire, sans commerce à la cour, et fort volontiers à
l'écart chez le roi, et cela sans proportion plus que M. de Chevreuse,
il est surprenant jusqu'où il imposait chez le roi, et partout ailleurs
dès qu'il paraissait quelque part\,; M\textsuperscript{me}s de Chevreuse
et de Beauvilliers de même, mais un peu plus mêlées dans la cour,
quoique avec grande réserve. Les princes du sang, les bâtards même, les
plus considérables seigneurs, les ministres ne l'approchaient qu'avec un
air de respect, de déférence, fort souvent d'embarras. On regardait à
qui il parlait\,; je me suis souvent diverti des instants à voir les
yeux des principaux de la cour, ce qui arrivait assez souvent à Marly,
fichés sur moi, assis à l'écart auprès de lui qui me parlait à
l'oreille. Je n'ai vu personne sur un si grand pied à la cour, et, à
quelques semaines près de l'orage du quiétisme, tant qu'il a vécu, même
après la mort du Dauphin.

Depuis cette fatale époque, i1 se retira de plus en plus, et il ne se
soutint qu'à force de piété, de courage, d'abandon à Dieu, de conformité
à sa volonté. Quelque musique d'airs tristes, quelques soupers chez moi,
plus rares néanmoins qu'avant cette plaie, faisaient tout son
délassement. Il était fait exprès pour être capable et en même temps
digne de former un excellent roi, bon, saint, grand devant Dieu et
devant les hommes. Il y avait mis tous ses talents et tous ses soins, et
il voyait avec ravissement et actions de grâces continuelles, que le
succès passait de loin ses plus flatteuses espérances. Il se trouvait le
conseil intime, le cœur, l'esprit, l'âme de ce prince, qui en avait
infiniment. Il en attendait tout pour le rétablissement de l'ordre, de
la justice, du bonheur des sujets de tous les états, et le
rétablissement du royaume, parce qu'il en savait les vues, les projets,
les désirs, que lui-même avait inspirés\,; et il en voyait assez par
l'expérience pour ne pas craindre la corruption du cœur ni
l'étourdissement de l'esprit par le souverain pouvoir. Enfin il
considérait un âge qui dans sa fleur avait vaincu toutes les plus
formidables passions\,; une vertu solidement fondée, et qui avait passé
par d'étranges épreuves, enfin un long cours d'années à donner tout
loisir aux sages et lentes opérations au dedans et au dehors, dont
lui-même, après les plus promptes, pouvait se flatter de voir les
commencements\,; et tout à coup il voit enlever ce prodige de talents et
de grâce dont nous n'étions pas dignes, qui ne nous fut montré que pour
nous faire admirer la puissance de la droite de Dieu, et nous faire
sentir l'excès de nos péchés par la profondeur de notre chute.

Alors, si on ose hasarder ce terme, les jointures de son âme avec son
corps furent ébranlées, il aperçut d'un coup d'oeil les funestes suites
qui résultaient sur la France, il éprouva les plus horribles effets de
la tendresse, il entra dans le néant que cet horrible vide laissait, il
en vivifia son plein sacrifice, il dompta la nature éperdue par un
effort si terrible qu'il m'a souvent avoué que celui de ses enfants ne
lui avait en comparaison presque rien coûté. Tout fut mis au pied de la
croix. Avide de profiter de toute l'amertume d'un calice si exquis, on a
vu qu'il n'en perdit pas une seule goutte dans ses affreuses fonctions à
Saint-Denis, à Notre-Dame, auprès du roi, avec une supériorité sur
soi-même qui passait la portée de l'homme. La mort du duc de Chevreuse
combla en lui la destruction de l'homme animal. Sa solitude la fut moins
qu'une prison. Des sacrifices sanglants devinrent le tissu de sa vie.
L'épurement sublime de son âme sans cesse lancée vers Dieu acheva la
dissolution de la matière, et fit de sa mort un holocauste. Que si ce
que la vérité m'a forcé de rapporter sur M. de Cambrai et sur le
cardinal de Noailles était capable de répandre quelques nuages
trompeurs, qu'on se souvienne sur le dernier de saint Épiphane avec
saint Jean Chrysostome\,; et sur le premier et sa Guyon, du célèbre
Grenade, des lumières et de la sainteté dont personne n'a douté, et qui,
pour un entêtement semblable, plus surprenant encore, n'a pu être
canonisé\,; et de nos jours, du savant Boileau de l'archevêché, et de M.
Duguet, dont les nombreux ouvrages de piété font admirer l'étendue et la
sublimité de son érudition et de ses lumières, qui tous deux ont été les
admirateurs et les dupes jusqu'à leur mort, de cette
M\textsuperscript{lle} Rose, cette étrange béate qui fut enfin chassée,
sans que leurs yeux pussent s'ouvrir sur elle, et dont on a parlé en son
temps.

J'avais eu la douceur de goûter toute la joie de la réconciliation
parfaite, qu'on a vu en son lieu que j'avais faite entre le duc de
Beauvilliers et le chancelier de Pontchartrain, et le déplaisir
véritable du premier de la retraite de l'autre\,; et j'eus la
consolation de voir le chancelier sincèrement affligé de la mort du duc.
Dès auparavant cette réconciliation, le chancelier, quoique ami du duc
de Chevreuse, me disait quelquefois plaisamment des deux beaux-frères
«\,qu'il était merveilleux, liés comme ils l'étaient par l'habitude de
toute leur vie, jusqu'à n'être tous deux qu'un cœur, une âme, un esprit,
un sentiment, {[}que{]} M. de Beauvilliers eût un ange qui à point nommé
l'arrêtait, et ne manquait jamais de le détourner de tout ce que M. de
Chevreuse avait de nuisible et quelquefois d'insupportable, l'un dans sa
conduite, qui ruinait ses affaires et sa santé, l'autre dans ses
raisonnements\,; un ange qui lui faisait pratiquer tout l'opposé, qui
dans tout le reste ne troublait en rien leur union, et par cela même ne
l'altérait pas. » En effet, rien de plus opposé que le désordre et le
bon état des affaires de l'un et de l'autre, avec toute l'application de
l'un, et une plus générale de l'autre\,; que l'austérité de la sobriété
de l'un, et l'ample nourriture de l'autre\,; l'un persuadé par
philosophie et par le livre de Cornaro, l'autre par Fagon\,; la
précision jusqu'à une minute des heures de M. de Beauvilliers, l'homme
le plus avare de son temps, et qui faisait des excuses à son cocher s'il
n'arrivait pas avec justesse au moment qu'il avait demandé son carrosse,
et l'incurie de M. de Chevreuse de se faire toujours attendre, dont on a
vu en leur lieu des exemples plaisants, et son ignorance des heures,
quoique jaloux aussi de son temps\,; enfin l'exactitude de l'un à tout
faire et finir avec justesse, tandis que l'autre faisait sans cesse et
paraissait ne jamais finir. Aussi M. de Beauvilliers, qui voulait le
bien en tout, s'en contentait\,; et M. de Chevreuse, qui cherchait le
mieux, manquait bien souvent l'un et l'autre.

M. de Beauvilliers voyait les choses comme elles étaient\,; il était
ennemi des chimères, pesait tout avec exactitude, comparait les partis
avec justesse, demeurait inébranlable dans son choix sur des fondements
certains. M. de Chevreuse, avec plus d'esprit, et sans comparaison plus
de savoir en tout genre, voyait tout en blanc et en pleine espérance,
jusqu'à ce qui en offrait le moins, n'avait pas la justesse de l'autre,
ni le sens si droit. Son trop de lumières point assez ramassées
l'éblouissait par de faux jours, et sa facilité prodigieuse de concevoir
et de raisonner lui ouvrait tant de routes qu'il était sujet à
l'égarement, sans s'en apercevoir et de la meilleure foi du monde. Ces
inconvénients n'étaient jamais en M. de Beauvilliers, qui était
préférable dans un conseil, et M. de Chevreuse dans toutes les
académies. Il avait aussi une élocution plus naturellement diserte,
entraînante, et dangereuse aussi par les grâces qui y naissaient
d'elles-mêmes, à entraîner dans le faux à force de chaînons, quand on
lui avait passé une fois ses premières propositions en entier faute
d'attention assez vigilante, et de donner par cet entraînement dans un
faux qu'à la fin on apercevait tout entier, mais déjà dans le branle
forcé de s'y sentir précipité. Enfin, pour achever ce contraste de deux
hommes si unis jusqu'à n'être qu'un, le duc de Chevreuse ne pouvait se
lever ni se coucher\,; M. de Beauvilliers, réglé en tout, se levait fort
matin, et se couchait de bonne heure, c'est-à-dire qu'il sortait de
table au commencement du fruit, et qu'il était couché avant que le
souper fût fini.

Ils furent tous deux, comme on l'a vu ailleurs, les protecteurs et le
soutien de leurs frères et sœurs du second lit et des femmes de leur
père. M. de Beauvilliers eut le moyen et la funeste occasion d'y être
plus magnifique que son beau-frère\,; il y fut aussi plus heureux, et
M\textsuperscript{me} de Beauvilliers s'y surpassa. Elle but à loisir le
calice de la chute de l'évêque de Beauvais, que M. de Beauvilliers n'eut
pas le loisir de voir. Elle logeait ce beau-frère\,; elle lui donnait\,;
et persuadée de sa piété, il faisait toute sa consolation. Elle porta
seule la douleur de ses premiers désordres, qu'elle essaya d'ensevelir
dans le plus grand secret. Ils étaient de nature à n'y pouvoir pas
demeurer longtemps. Elle n'oublia ni soins, ni caresses, ni mesures, et
les moins selon son cœur, puisqu'elle employa le cardinal de Noailles,
qui s'y prêta comme son propre frère. Je fus témoin de tout ce qui s'y
passa, de la charité vraiment tendre et agissante, de la douleur la plus
amère de M\textsuperscript{me} de Beauvilliers. L'éclat affreux, qu'ils
ne purent jamais empêcher par la folie de ce déplorable évêque, fut peu
à peu porté à son comble, qui fut celui des douleurs de la duchesse de
Beauvilliers, et une nouvelle et forte épreuve de sa vertu, qui
néanmoins eût été ici supprimée, si la cour, Paris, toute la France, et
par un reflet devenu nécessaire, Rome même, n'avaient pas retenti de ce
malheur rendu si peu commun, et si étrangement public, par
l'extravagance d'une conduite qui fut le sceau de l'affliction de
M\textsuperscript{me} de Beauvilliers.

Il n'y eut point de femme à la cour qui eût plus d'esprit que celle-là,
plus pénétrant, plus fin, plus juste, mais plus sage et plus réglé, et
qui en fût plus maîtresse. Jamais elle n'en voulait montrer, mais elle
ne pouvait faire qu'on ne s'en aperçût dès qu'elle ouvrait la bouche,
souvent même sans parler. Il était naturellement rempli de grâce, avec
une si grande facilité d'expression, qu'elle en était parée, jusqu'à en
faire oublier sa laideur, qui, bien que sans difformité ni dégoût, et
avec une taille ordinaire et bien prise, était peu commune. Il y avait
même un tour galant dans son esprit. Elle aimait à donner, et je n'ai vu
qu'elle et la chancelière qui eussent l'art de le faire avec un tour et
des grâces aussi parfaites. Son goût était exquis et général\,: meubles,
parures de tout âge, table, en un mot sur tout\,; fort noble, fort
magnifique, fort polie, mais avec beaucoup de distinction et de dignité.
Elle aurait eu du penchant pour le monde. Une piété sincère dès ses
premières années, et le désir de plaire à M. de Beauvilliers, la
retenait, mais elle y était fort propre\,; et indépendamment de commerce
avec elle, on le sentait à la manière grande, noble, aisée\,;
accueillante avec discernement, dont elle savait tenir sa maison ou la
cour\,; et les étrangers qualifiés abondaient à dîner.

Son esprit qui échappait quelquefois, quoique toujours avec grande
circonspection, se montrait, malgré elle, assez pour faire regretter
qu'elle ne lui laissât pas plus de liberté. Sa conversation était
agréable, charmante en liberté, avec des traits vifs, fins, perçants,
après lesquels il était plaisant de la voir quelquefois courir. Ailleurs
il y avait du contraint, et qui communiquait de la contrainte\,; et en
tout il est vrai que fort peu de gens, même des plus familiers, se
trouvaient avec elle pleinement à l'aise, au contraire de
M\textsuperscript{me} de Chevreuse qui, avec autant de piété, avait
beaucoup moins d'esprit. D'ailleurs, M\textsuperscript{me} de
Beauvilliers était parfaitement droite et vraie, tendre amie et parente
excellente. Les aumônes et les bonnes œuvres que M. de Beauvilliers et
elle ont faites se peuvent dire immenses\,; c'était leur premier soin,
et, avec la prière, leur plus chère occupation.

Une en tout avec M. de Beauvilliers, on a vu ailleurs comment elle en
usa à la mort de ses enfants pour ceux du second mariage du vieux duc de
Saint-Aignan qu'elle combla de biens, de soins, de tendresse, et à qui
elle ne laissa jamais sentir quel poignard ce lui était que ce souvenir
perpétuel de ses pertes.

Celle de M. de Beauvilliers fut un glaive qui ne sortit plus de son
cœur, qui le perça. Elle resta aussi riche que la duchesse de Chevreuse
était demeurée pauvre\,; aussi le chancelier de Pontchartrain
prétendait-il «\,que c'était toujours l'effet du jeu de ce même ange en
faveur de l'un pour confondre la philosophie de l'autre.\,»

M\textsuperscript{me} de Beauvilliers, si tendrement et si pieusement
une avec son époux toute leur vie, demeura inconsolable, mais en
chrétienne et en femme forte. Il voulut être enterré à Montargis, dans
le monastère de bénédictines où huit de ses filles avaient voulu faire
profession, et dont l'aînée était supérieure perpétuelle, sans qu'aucune
ait voulu ouïr parler d'abbaye\,; M\textsuperscript{me} de Beauvilliers
y alla, et, par un acte de religion qui fait la plus terrible horreur à
penser, elle voulut assister à son enterrement. Ce fut aussi le lieu de
sa plus chère retraite depuis, toutes les années de sa vie, et longtemps
et souvent plus d'une fois l'an, vivant au milieu de ses filles, et
d'autres fort proches dont le couvent était rempli, dans la plus
poignante douleur, et la pénitence la plus austère, sans que rien en
parût aux heures du délassement de la communauté. À Paris, dans sa vaste
maison, fort loin de ses sœurs (et c'était un autre sacrifice, surtout à
l'égard de M\textsuperscript{me} de Chevreuse), elle ne se crut pas
obligée à vivre comme les autres veuves, n'ayant ni enfants ni besoins.
Sa retraite fut totale\,; ni table, ni le plus léger amusement d'aucune
espèce. Tout ce qui put y avoir le moindre trait fut banni, tout
commerce fut rompu avec le monde. Elle se borna à sa plus étroite
famille, et à un nombre le plus court d'amis qui l'étaient de M. de
Beauvilliers aussi, avec qui tout lui était commun. Sa solitude était
entière, rarement interrompue par quelqu'un de ce petit nombre. Ses
journées n'étaient que prières chez elle ou à l'église, quelquefois chez
ses sœurs, et chez M\textsuperscript{me} de Saint-Simon depuis que nous
fûmes à Paris\,; nulle autre part, ou comme jamais. Assez l'été dans ses
terres pour y faire de bonnes œuvres, où elle était, s'il se peut,
encore plus seule qu'à Paris. Un trait d'elle que je ne puis me refuser
montrera jusqu'où elle porta la vertu.

Les fouille-au-pot de la cuisine d'Henri IV, avant qu'il eût recueilli
la couronne de France, furent heureux comme l'a témoigné la fortune de
La Varenne et de sa postérité. Deux autres, qui vinrent de Béarn en
cette qualité, s'appelaient Joannes et Beziade. Ce dernier serait bien
étonné de voir d'Avaray, son petit-fils, chevalier de l'ordre. Joannes,
c'est-à-dire Jean, nom fort commun aux laquais basques, fut mis
jardinier à Chambord, devint par les degrés jardinier en chef, ne
travaillant plus, et concierge du château. Il s'enrichit pour son état
et pour son temps, acheta des terres, fit porter à son fils le nom de
celle de Saumery\,; et de Joannes il ôta l'\emph{s}, en fit Joanne pour
le nom de sa maison. Ce fils se trouva un honnête homme, brave et
d'honneur, servit avec distinction, devint capitaine et concierge de
Chambord, comme les autres le sont des maisons royales, et se maria à
Blois avec une fille de Charron, bourgeois du lieu, qui avait donné
l'autre à Colbert avant tout commencement de fortune de cette sœur de
M\textsuperscript{me} Colbert\footnote{Ce passage, depuis \emph{Deux
  autres, qui vinrent de Béarn}, a été supprimé dans les précédentes
  éditions. Voy., sur Saumery, t. II, p.~331-333, 452, et t. VII, p.~204
  et 448.}. Saumery qui est mort très vieux, que j'ai vu venir faire de
courts voyages à Versailles, de Chambord où il s'était retiré, qu'on
accueillait par son âge et parce qu'il ne s'était jamais méconnu, eut
plusieurs enfants, dont l'aîné fort bien fait, audacieux et impudent à
l'avenant, quitta le service de bonne heure pour une blessure qui lui
estropia légèrement un genou, dont il sut se parer et s'avantager mieux
que blessé que j'aie vue de ma vue.

Il était retiré à Chambord, dont il avait la survivance, et avec une
fille de Besmaux, gouverneur de la Bastille, qu'il avait épousée, plus
impertinente et plus effrontée encore que lui\,: il faisait le gros dos
dans la province, décoré d'une charge de maître des eaux et forêts. Il
était donc cousin germain des enfants de M. Colbert, qui l'y avait
laissé, jusqu'à ce que M. de Beauvilliers l'en tira, lorsque M. le duc
d'Anjou, depuis roi d'Espagne, passa des femmes aux hommes, pour le
faire sous-gouverneur. Il avait plusieurs enfants et bon appétit. Sa
place lui parut avec raison le comble d'une fortune inespérée, mais
bientôt, il n'y trouva que le chemin de la faire

Ce n'était ni un esprit ni un sot, mais un drôle à qui toute voie fut
bonne, et qui fureta partout. Il fit des connaissances, disait le
bonjour à l'oreille, parlait entre ses doigts, et montait cent escaliers
par jour. Pour le faire court, il s'initia chez le duc d'Harcourt et
chez les plus opposés à M. de Beauvilliers, qui avaient apparemment
leurs raisons pour l'accueillir. Il en fit l'important de plus en plus,
et se fourra tant qu'il put. Je ne sais s'il se douta de quelque chose,
mais il évita, même scandaleusement, la campagne de Lille par un voyage
à Bourbonne. Il en revint à la cour dans le temps des plus grands cris
contre Mgr le duc de Bourgogne, et de tous les mouvements qui ont été
racontés. Il vit de quel côté venait le vent, et n'eut pas honte d'être
un des grands prôneurs de M. de Vendôme, et de tomber sur Mgr le duc de
Bourgogne, auprès duquel il avait été mis, et y était. Cette infamie le
déshonora, mais elle fut bien récompensée par les patrons qu'elle lui
valut. Il est mort bien des années depuis avec plus de quatre-vingt
mille livres de rentes de grâces de Louis XIV, sans compter les
militaires pour ses enfants. Le même crédit le fit sous-gouverneur du
roi d'aujourd'hui, dont son fils aîné eut la singulière survivance et
l'exercice.

Celui-là était un fort honnête homme, avec de la valeur, du sens et de
la modestie, et n'a pas survécu son père longtemps. Il avait un cadet
qui faisait le beau fils et l'homme à bonne fortune\,; et c'est celui
dont il va être question.

M. et M\textsuperscript{me} de Beauvilliers avaient toujours reçu
Saumery à peu près à l'ordinaire, qui s'y présentait aussi dégagé que
s'il n'avait eu quoi que ce fût à se reprocher, bien que très informés
de toute sa conduite. Je les avais inutilement attaqués là-dessus, et je
ne m'étais pas contraint dans le monde de ce que je pensais de Saumery
et de ses procédés. Ses fils s'étaient aussi enrichis. Le cadet
longtemps depuis, ce beau fils dont j'ai parlé, avait acheté des terres,
une entre autres qui convenait à M\textsuperscript{me} de Beauvilliers
pour des mouvances\footnote{La mouvance d'un fief était comme on l'a
  déjà dit, la dépendance d'un fief inférieur par rapport au fief
  dominant ou suzerain. Il y a eu de très longues contestations pour
  savoir si la Bretagne était un fief mouvant du duché de Normandie.}
qui l'auraient jetée en beaucoup d'embarras, et qu'il lui avait
soufflée. Elle était peu considérable, elle ne l'était pas même pour
Saumery, qu'on appelait Puyfonds, qui n'avait pas les mêmes raisons.
Elle résolut de la retirer, et lui en fit faire toutes les civilités
possibles. Le compagnon trouva plaisant qu'elle imaginât d'exercer son
droit sur un homme de son importance\,; et n'eut pas honte de demander
«\,qui était donc cette M\textsuperscript{me} de Beauvilliers qu'il ne
connaissoit point, et qui prétendait qu'on eût des égards pour elle\,?»
Il tint ferme à contester le droit contre tout ce qui lui parla de la
famille.

Dans l'embarras d'un procès, et de procédés de même impudence que les
propos, M\textsuperscript{me} de Beauvilliers trouva, par des raisons de
terres et de mouvances, qu'il n'y avait que d'Antin qui pût lui imposer
et lui faire quitter prise\,; nul moyen en elle d'approcher d'Antin
jusqu'à lui faire prendre fait et cause. On a vu souvent combien il
avait toujours été éloigné de M. de Beauvilliers, et M. de Beauvilliers
de lui. Je ne l'avais pas été moins\,; mais vers les fins de la vie du
roi, il s'était fort jeté à moi, et depuis encore davantage.
M\textsuperscript{me} de Beauvilliers, avec qui je vivais toujours dans
la plus étroite union, crut qu'il n'y avait que moi qui pût faire que
d'Antin se prêtât à elle. Elle se garda bien de me parler de cette
affaire que j'ignorais, mais elle vint la conter à M\textsuperscript{me}
de Saint-Simon, et prit exprès son temps que j'étais au conseil de
régence. Après lui avoir expliqué la chose et les procédés, et ce que
j'y pouvais faire, elle lui dit que c'était à elle à voir si je pourrais
être capable de la servir sans éclater contre Puyfonds\,; qu'elle se
souvint de la façon dont j'avais mené le père à leur occasion\,; qu'elle
craignait que je ne tombasse sur le fils, et en discours violents et en
choses, avec le crédit que j'avais\,; que, pour peu que je ne fusse pas
maître de moi là-dessus, elle la priait instamment de ne m'en jamais
parler, parce que pour rien elle ne me voulait faire offenser Dieu et le
prochain, et aimait mieux perdre et ruiner son affaire que d'en être
cause. Il fallut donc entrer en négociation avec moi pour le service
qu'on en désirait, sans expliquer rien ni nommer personne que
M\textsuperscript{me} de Beauvilliers, jusqu'à ce qu'on m'eût fait
convenir des conditions. Je les passai toutes, dans le désir de lui être
utile, et avec grande curiosité de développer de si rares conditions et
des précautions si singulières. Je vins à bout très promptement de
l'affaire, mais non si aisément de moi sur ce que j'avais promis, sans
que le pied m'y glissât un peu, ni sans grand effort ni mérite de me
retenir autant.

Cet ingrat et impudent Puyfonds fut bien heureux, au temps où nous
étions, d'avoir eu affaire à une vertu aussi sublime qu'il força
M\textsuperscript{me} de Beauvilliers à se montrer. Ce trait est si fort
au-dessus de la nature et de la vertu même plus qu'ordinaires, il
caractérise si nettement la duchesse de Beauvilliers que j'aurais cru
commettre plus aussi qu'un larcin de le laisser périr dans l'oubli,
trait d'autant plus héroïque qu'elle avait naturellement une grande
sensibilité.

Son extrême solitude la rongea lentement, et augmenta beaucoup le poids
de sa pénitence\,: elle n'y était pas accoutumée, rien ne put l'engager
à l'adoucir. La mort du duc de Rochechouart, son petit-fils, qui donnait
les plus grandes espérances, et qui la consolait de tout ce que le duc
de Mortemart lui donnait de souffrances par sa conduite et ses procédés
avec elle, et la perte de la duchesse de Chevreuse, qui arrivèrent coup
sur coup, achevèrent de l'accabler. Elle combla de biens le duc de
Saint-Aignan jusque par son testament, qui fut également sage, juste,
pieux, et succomba enfin sous les plus dures épreuves d'une longue
paralysie qu'elle porta avec une patience et une résignation parfaite,
et depuis que la tête commença à s'attaquer, il n'y avait que les choses
de Dieu qui la rappelassent, et dont elle pouvait être occupée, vivement
même, dont j'ai été souvent témoin. Elle et M. de Beauvilliers en
étaient si remplis, que ce qui leur échappait quelquefois avec moi
là-dessus, mais toujours courtement, était rempli d'une onction et d'un
feu admirable. Elle vécut presque vingt ans dans la plus solitaire et la
plus pénitente viduité, moins d'un an après M\textsuperscript{me} de
Chevreuse\,; et mourut en 1733, à soixante-quinze ans, infiniment riche
en aumônes et en toutes sortes de bonnes œuvres.

\hypertarget{chapitre-xiii.}{%
\chapter{CHAPITRE XIII.}\label{chapitre-xiii.}}

1714

~

{\textsc{Ma situation à la cour.}} {\textsc{- Conduite étrange de
Desmarets.}} {\textsc{- Brutalité avec moi, qui lui est fatale.}}
{\textsc{- Maréchal de Villeroy chef du conseil royal des finances.}}
{\textsc{- Son fils archevêque de Lyon.}} {\textsc{- Continuation de ma
situation à la cour.}} {\textsc{- Macañas\,; quel.}} {\textsc{- Cardinal
del Giudice fait fonction à Marly de grand inquisiteur d'Espagne\,;
choque les deux rois\,; est rappelé\,; donne part publique du mariage du
roi d'Espagne\,; part à grand regret\,; se morfond longtemps à Bayonne
avec défense de passer outre.}} {\textsc{- Moyens en Espagne contre les
entreprises de Rome.}} {\textsc{- Repentir inutile de la princesse des
Ursins du mariage de Parme.}} {\textsc{- Mariage à Parme de la reine
d'Espagne, qui part pour l'Espagne\,; sa suite.}} {\textsc{- Mariage du
fils du prince de Rohan avec la fille de la princesse d'Espinoy.}}
{\textsc{- Mariage du comte de Roye avec la fille d'Huguet, conseiller
au parlement.}} {\textsc{- Voyage de Fontainebleau par Petit-Bourg.}}
{\textsc{- Le roi de fort mauvaise humeur.}} {\textsc{- Électeur de
Bavière à Fontainebleau.}} {\textsc{- Amusements du roi redoublés et
inusités chez M\textsuperscript{me} de Maintenon.}} {\textsc{- Paix de
l'empire et de l'empereur signée à Bade.}} {\textsc{- Le roi
d'Angleterre donne part au roi de son avènement à cette couronne, passe
en Angleterre et y fait un entier changement.}} {\textsc{- Maréchal de
Villeroy arrive à Fontainebleau\,; est fait ministre.}} {\textsc{-
Ministres ne prêtent point de serment.}} {\textsc{- Ineptie parfaite du
maréchal.}} {\textsc{- Retour du maréchal de Villars.}} {\textsc{- Duc
de Mortemart apporte au roi la nouvelle de l'assaut général de
Barcelone, qui se rend à discrétion avec Mont-Joui et Cardone.}}
{\textsc{- La Catalogne soumise.}} {\textsc{- Broglio, gendre de Voysin,
apporte le détail de la prise de Barcelone.}} {\textsc{- Vues et
conduite domestique du roi de Pologne, qui fait voyager son fils
incognito.}} {\textsc{- Il arrive à Paris et à la cour\,; très-bien
reçu.}} {\textsc{- Ce qu'on en trouve.}} {\textsc{- Ses conducteurs.}}
{\textsc{- Sa conversion secrète.}} {\textsc{- Électeur de Bavière voit
le roi en particulier et retourne à Compiègne.}}

~

J'avoue que j'ai peine à m'arracher à des objets qui me furent si chers,
et qui me le seront toute ma vie. Il est temps de reprendre une nouvelle
idée de ma situation à la cour, bien différente de celle où je m'étais
trouvé. La perte du Dauphin et de la Dauphine, la dispersion de ses
dames qui ne figuraient plus, la disgrâce de Chamillart, la retraite du
chancelier de Pontchartrain, la mort du maréchal de Boufflers, du duc de
Chevreuse, enfin celle du duc de Beauvilliers, me laissèrent dans un
vide (je ne parle pas du cœur, dont ce n'est pas ici le lieu), que rien
ne pouvait, non pas remplir, mais même diminuer. J'étais dans
l'intimité, la confiance la plus étroite de ces ministres et de ces
seigneurs si principaux, je l'étais de plusieurs dames très instruites
et très importantes qui en diverses façons avaient disparu. Ces
liaisons, surtout ce qui, malgré les plus sages précautions, ne laissa
pas de transpirer de celles du Dauphin tout à la fin de sa vie, et plus
encore depuis, m'avaient attiré tous les regards. La jalousie devançait
de loin ma fortune de perspective. On regardait si peu comme une chimère
que je pusse dès lors entrer dans le conseil, à quoi je ne songeai
jamais\,;car, après le roi, personne n'en doutait du temps du Dauphin et
depuis, que la peur qu'on en eut fit que Bloin, vendu à M. du Maine, le
lâcha au roi, qui était la façon la plus propre à m'écarter. Il le lui
dit comme un discours qu'il croyait ridicule, mais que la cour ne
regardait pas comme tel et qu'elle craignait. Toutefois il ne parut pas
que cet honnête office fît d'impression.

De tout cet intérieur du roi de toute espèce, je n'avais que Maréchal,
qui rompit plus d'une fois des lances pour moi contre les autres qui
m'attaquaient devant le roi, et qui avaient de bons garants pour le
faire. Dans le ministère je n'eus plus qui que ce fût\,: Desmarets, sans
cause aucune, s'était éloigné de moi, et dès que je m'en aperçus je m'en
éloignai de même. MM. de Chevreuse et de Beauvilliers le remarquèrent\,;
ils me pressèrent de le voir et d'excuser un homme accablé d'aussi
difficiles affaires, et voyant enfin qu'ils ne me persuadaient pas, ils
me forcèrent d'y aller dîner avec eux, chose qui ne leur arrivait
presque jamais. Tout s'y passa à la glace pour moi de la part de
Desmarets, dont les deux ducs furent tellement scandalisés qu'ils me
dirent qu'ils ne m'en demanderaient plus davantage. C'était à
Fontainebleau, un an juste avant la mort du duc de Chevreuse. Dans la
suite, lorsqu'il fallait parler à Desmarets pour quelque mangerie de
financiers dans mes terres, ou pour être payé d'appointements, je priais
toujours M\textsuperscript{me} de Saint-Simon d'y aller. Bientôt elle
n'en fut pas plus contente que moi. Elle laissait accumuler plusieurs
choses pour lui parler de toutes en même temps\,; à la fin elle ne put
se résoudre à y retourner. Différents payements d'appointements
s'étaient accumulés\,; je différais toujours à aller les demander,
jusqu'à ce qu'un jour M\textsuperscript{me} de Saint-Simon m'en pressa
tant que j'y fus après le dîner, qui était assez l'heure de lui parler.

Elle ne faisait que finir lorsque j'entrai dans son cabinet, à
Versailles, qui était grand. Il venait de se mettre à son bureau. Dès
que je parus il vint à moi d'un air ému, me coupa au premier mot la
parole, disant qu'il était bien malheureux d'être la victime du public,
et d'autres plaintes dont le ton s'élevait. Voyant ainsi la marée monter
à vue d'œil, je voulus essayer de reprendre la parole, il m'interrompit
à l'instant\,; le rouge lui monta, ses yeux s'enflammèrent, ses plaintes
aigres, mais vagues et sans rien que je pusse prendre pour moi,
redoublèrent d'une voix fort élevée, et tout d'un coup se jetant sur des
papiers que je tenais à la main, que je m'étais proposé de lui expliquer
en deux mots avant de les lui laisser\,: «\,Voyons donc, dit-il, ce que
c'est que tout cela\,», d'un ton qui, dans mon extrême surprise, me
détermina à n'en pas attendre davantage. Il était venu à moi jusque fort
près de la porte, je l'ouvris, et sans regarder derrière moi, je cours
encore.

J'allai conter mon aventure à M\textsuperscript{me} de Saint-Simon, et à
des personnes de nos amis qui avaient dîné avec nous, et que je
retrouvai encore, et me promis bien de ne parler plus que par lettres à
un animal si ingrat et si bourru, quand j'aurais très nécessairement
affaire à lui. La vérité est que, de ce moment, je me promis bien de ne
rien oublier pour le mettre hors d'état d'avoir à brutaliser personne,
et j'y parvins, comme on le verra dans la suite.

Dès le lendemain un commis me renvoya les expéditions faites sur les
papiers dont je viens de parler et les payements se firent, mais ces
payements étaient dus, et cette insolence ne me l'était pas, ainsi nous
en demeurâmes en ces termes, et quand il fallait passer par lui je lui
envoyais un mémoire.

Il était si enivré de sa place et de sa faveur inespérée, si en proie à
son humeur et aux flatteries des nouveaux amis qui ne voulaient que
faire des affaires, qu'il oublia les leçons de sa longue disgrâce et ses
vrais et anciens amis désintéressés. M. de Beauvilliers et M. de
Chevreuse n'étaient plus alors\,; il s'était refroidi de même avec eux
jusqu'à la cessation du commerce, et brouillé fortement avec
M\textsuperscript{me} de Croissy qui, pendant sa disgrâce, avait été
toute sa ressource, depuis qu'il put demeurer à Paris, par conséquent
très froidement avec Torcy. Tel était cet ogre.

Torcy, on a vu que je n'avais jamais eu aucun commerce avec lui, et sur
quel pied gauche j'étais resté avec Pontchartrain\,; Voysin, chancelier
et secrétaire d'État, je n'y avais jamais eu la plus légère
connaissance, et il était d'ailleurs l'âme damnée de
M\textsuperscript{me} de Maintenon et de M. du Maine.

Ainsi, tous les successeurs de mes plus intimes amis m'étaient fort
opposés, ou pour le moins parfaitement indifférents\,; encore avais-je
lieu de ne pas m'en croire quitte à si bon marché avec pas un, jusqu'au
successeur de M. de Beauvilliers, comme on l'a vu épars en plusieurs
endroits\,; en dernier lieu même nous étions demeurés assez mal ensemble
depuis les belles prétentions des maréchaux de France, lors de l'affaire
du duc d'Estrées et du comte d'Harcourt, qu'il avait fort soutenues, et
sur lesquelles je m'étais espacé sur lui sans ménagement.

On comprend assez que c'est le maréchal de Villeroy dont j'entends
parler\,; il venait d'obtenir l'archevêché de Lyon pour son fils, et
commandement dans tout le gouvernement, comme l'archevêque son
grand-oncle, malgré ses mœurs et son ignorance, l'un et l'autre
parfaitement connus. À peine la place de chef du conseil des finances
fut-elle vacante que le roi lui manda, à Lyon où il était encore, qu'il
la lui donnait. Outre la façon dont nous étions ensemble, c'était encore
un homme vendu à M\textsuperscript{me} de Maintenon, et par conséquent
au moins pour lors au duc du Maine. Tallard, Tessé, d'autres courtisans
importants, nous avions toujours marché sous différentes enseignes, et
quoique Harcourt m'eût souvent rapproché, ce que j'étais au duc de
Beauvilliers m'avait empêché de m'y jamais prêter au delà de la simple
et indispensable bienséance.

En un mot je ne tenais plus à personne\,; Charost, malgré sa charge,
n'était rien, et Noailles avec tous ses dehors, et le cancer interne de
sa disgrâce couverte, avait plus besoin de moi pour le futur, que moi de
lui pour le présent. J'avais donc sans nul appui le ministère et
l'intérieur du roi contre moi, et dans la cour force piques baissées sur
moi par la peur et la jalousie qu'on avait prise, et sur l'idée encore
d'un avenir peu éloigné par la régence de M. le duc d'Orléans.

La liaison entre lui et moi était de toute notre vie\,; on n'ignorait
plus que sa séparation d'avec M\textsuperscript{me} d'Argenton, son
raccommodement avec M\textsuperscript{me} la duchesse d'Orléans, l'union
dans laquelle ils vivaient depuis, le mariage de M\textsuperscript{me}
la duchesse de Berry, ne fût mon ouvrage. La disgrâce du roi si marquée,
si approfondie, les dangers de l'affaire d'Espagne, les vacarmes tant
renouvelés des poisons, la fuite générale de sa présence qui durait
toujours, les avis, les menaces secrètes qu'on avait pris soin de me
faire revenir, n'avaient pu me séparer de lui, ni d'être le seul homme
de la cour qui le vît publiquement, et qui publiquement parût avec lui
dans les jardins de Marly, et jusque sous les yeux du roi. L'uniformité
de cette conduite ne pouvait être imputée aux espérances, puisqu'elle
avait été la même du temps de Monseigneur et des princes ses fils, où je
n'en pouvais attendre que des disgrâces. Alors même ce peu de ménagement
était considéré comme une singulière hardiesse dans la situation où ce
prince se trouvait avec le roi et M\textsuperscript{me} de Maintenon que
personne n'ignorait, et dont le testament du roi devenait dans son
obscurité une preuve manifeste qui portait tous les pas vers le duc du
Maine.

Celui-ci n'avait pas oublié l'inutilité de tous les siens vers moi, ni
mon extrême horreur des rangs qu'il avait obtenus. Ma conduite avec M.
le duc d'Orléans démentait avec force l'imputation exécrable faite à ce
prince si importante au duc du Maine, dont il avait si habilement su
profiter, et que pour l'avenir il entretenait et ressuscitait avec tant
d'art et de manège, toujours M\textsuperscript{me} de Maintenon de
moitié avec lui.

J'avais conservé une réputation entière de vérité, de probité et
d'honneur, que les jaloux, les querelles de rang, les divers orages
n'avaient jamais attaquée\,; M\textsuperscript{me} de Saint-Simon était
de toute sa vie sur le plus grand pied de réputation en tout genre\,;
personne n'ignorait, quoique en gros, que nous avions infiniment perdu
au Dauphin et en la Dauphine pour le présent et pour l'avenir, ni
l'amertume de notre douleur. Je n'avais jamais passé pour savoir me
contraindre, il était donc évident que j'aurais rompu avec M. le duc
d'Orléans, sans ménagement, et sans égard aucun sur l'avenir, si je
l'avais soupçonné le moins du monde\,: cela même était universellement
avoué, et je le voyais trop journellement, trop intimement pour, à la
fin, n'avoir rien soupçonné pour peu qu'il y eût à le faire. Voilà ce
qui m'avait tant détaché d'avis et de menaces de toutes parts pour
m'obliger à changer de conduite avec ce prince, dont l'inutilité
retombait en rage sur moi de la part de M\textsuperscript{me} de
Maintenon et de M. du Maine qui, outre ce principal objet que je remets
ici devant les yeux quoique je l'aie touché ailleurs, s'y proposaient
encore de priver M. le duc d'Orléans du seul homme qui le vît et avec
qui il pût raisonner et consulter.

Les croupiers de ces deux personnes si prodigieusement principales ne
leur manquaient pas en ce genre. À eux se jaignaient d'ailleurs un
groupe toujours nombreux d'envieux et de jaloux, qui étaient bien
persuadés que, dès que M. le duc d'Orléans serait régent, je ferais
auprès de lui la première figure en confiance et en crédit, et qui s'en
désespéraient d'avance. Cela même était encore une des frayeurs de M. du
Maine et de M\textsuperscript{me} de Maintenon.

La réputation d'esprit qu'on m'avait donnée pour me perdre auprès du
roi, lorsqu'il me choisit, en 1706, pour l'ambassade de Rome, et qui
réussit si fort au gré des honnêtes gens qui l'imaginèrent, comme on l'a
vu alors, était demeurée dans la tête de M. du Maine, de
M\textsuperscript{me} de Maintenon, du roi même\,; le gros du monde, qui
y avait donné, avait eu plus tôt fait de le croire que d'y aller voir,
et c'est ainsi que s'établissent et que durent mille fausses idées qu'on
se forme tous les jours. J'avais soutenu beaucoup d'aventures,
d'affaires de rang, et d'autre nature avec des princes du sang et des
plus grands et accrédités de la cour, des orages même, toutes choses que
pour la plupart on a vues ici en leurs places. Je ne m'étais effrayé
d'aucunes, j'étais toujours bien sorti de toutes. Ce tout, joint
ensemble par l'envie et la jalousie, épouvantait et me livrait aux
effets de ces passions cruelles.

Quoiqu'il parût que le roi commençait à se flétrir, rien au dehors ne
menaçait encore, et je me voyais un long trajet de mer à me conduire
seul parmi ces écueils et ces gouffres\,; je les voyais tous paraître ou
s'ouvrir devant moi\,; je sentais à quel point je pensais à M. du Maine
et à M\textsuperscript{me} de Maintenon, dans l'intimité unique du
prince qui leur était en butte, et lui et moi sans la moindre défense\,;
combien je leur paraissais dangereux auprès de lui après le roi\,;
enfin, combien d'envieux, de jaloux, d'ennemis tourmentés de ces mêmes
pensées par différents regards. Plus de conseil principal et intime, et
plus personne en crédit pour m'appuyer et me défendre. Dieu permit que
je ne me troublai point\,; je me résolus à une conduite sage, mais sans
rien changer à mes allures, sans rechercher personne, surtout à vivre
avec M. le duc d'Orléans entièrement comme j'avais accoutumé en
particulier et en public, et à ne donner le plaisir à personne de me
voir faiblir et chercher à m'accrocher. Cette courte exposition était
nécessaire pour ce qui suivra, quoique ce ne soit pas encore le temps de
parler de ce qui se passait entre M. {[}le duc{]} et
M\textsuperscript{me} la duchesse d'Orléans et moi. Retournons en
attendant dans le monde qu'il y a trop longtemps que nous avons quitté.

Il faut se souvenir que ce fut le dimanche 26 août que le roi remit son
testament au premier président et au procureur général à Versailles,
qu'ils reçurent le même matin du chancelier l'édit qui l'accompagna,
qu'il fut enregistré le mardi suivant 28, et le testament enfermé le
même jour dans le lieu de son dépôt\,; que le lendemain mercredi le roi
alla coucher à Petit-Bourg, qu'il arriva le jeudi 30 août à
Fontainebleau, et que le lendemain vendredi dernier août, le duc de
Beauvilliers mourut à Vaucresson. Revenons maintenant un instant sur nos
pas, et voyons de suite le rappel du cardinal del Giudice.

Quelque soumise que l'Espagne paroisse à Rome, les entreprises de cette
cour qui cherche sans cesse à augmenter son pouvoir forment souvent de
petits orages. Son joug est trouvé trop pesant pour le laisser augmenter
encore\,; on s'y défend fortement de son accroissement\,; et quand Rome
s'emporte, la cour de Madrid la range par famine, et la force de se
rendre à la raison. C'est ce qui s'exécute aisément en y fermant la
nonciature dont le tribunal est extrêmement étendu, et vaut plus de deux
cent mille écus à la cour de Rome, tous les officiers payes, et le nonce
même qui tire gros. Les mœurs des pays d'inquisition sont si différentes
des nôtres, et ce détail mènerait si loin, que je m'abstiendrai d'entrer
dans l'affaire émue par la cour de Rome, qui blessa la cour de Madrid.

Macañas revêtu d'une charge dans le conseil de Castille, et homme fort
savant et fort attaché aux droits et à la personne du roi d'Espagne, fut
chargé d'écrire contre cette entreprise. Il le fit par un ouvrage si
bien prouvé que Rome ne put répondre que par l'abus auquel elle a si
souvent recours. L'inquisition d'Espagne fit un décret furieux contre la
personne et l'ouvrage de Macañas, et l'envoya en France au cardinal del
Giudice, grand inquisiteur d'Espagne, qui l'expédia et le data de Marly
le dernier juillet. Le roi fut fort choqué de cet exercice de sa charge
dans sa propre maison, hors de son territoire d'Espagne, et dans son
royaume, qui ne reconnaît point d'inquisition ni d'inquisiteurs.
Néanmoins il n'en voulut rien témoigner au dehors, sinon légèrement par
Torcy, qui par ordre du roi se paya aisément des excuses qu'il prodigua,
et qui ne coûtent rien aux ministres de Rome, pourvu qu'ils aient fait
ce qu'ils ont voulu, et que les excuses n'arrêtent point ce qu'ils ont
fait.

En Espagne on fut fort irrité de la conduite d'un grand inquisiteur, qui
était en même temps dans le conseil d'État, qui se pouvait si aisément
excuser à Rome sur son absence d'Espagne, et se porter si convenablement
par ses deux emplois en amiable compositeur du différend qui en juge
aussi partial et aussi sévère. M\textsuperscript{me} des Ursins fut
ravie d'une occasion si naturelle de se délivrer en Espagne du poids
incommode du cardinal. Elle avait eu cette vue pour un temps en
l'envoyant si indécemment en France\,; mais l'autre vue qu'elle avait
eue pour ce voyage n'était pas encore remplie, et qui regardait le
mariage du roi d'Espagne\,; elle se contenta donc d'aigrir le roi
d'Espagne contre le cardinal, mais de temporiser jusqu'à ce que sa
commission fût accomplie. Il l'acheva en effet le matin même que le roi
partit l'après-dînée de Versailles pour aller coucher à Petit-Bourg, et
lui donna part publique du mariage du roi d'Espagne, dont jusqu'alors il
ne lui avait donné part qu'en particulier, par respect et confiance de
son petit-fils, qui toutefois l'avait conclu avant de lui en avoir fait
dire un mot.

Le roi continua à dissimuler sur l'entreprise du cardinal grand
inquisiteur et sur le mariage. Il avait invité le cardinal de venir à
Fontainebleau où il lui avait donné un beau logement. Mais la princesse
des Ursins, qui savait le jour précis que cette part publique du mariage
serait donnée, s'était ajustée là-dessus, de façon que dès le lendemain
le cardinal reçut un ordre précis qui le rappelait en Espagne
sur-le-champ. Giudice en fut consterné. Il vint le lundi 3 septembre à
Fontainebleau, vit longtemps le roi le lendemain dans son cabinet à
l'issue de son lever, prit congé de lui, et s'en retourna à Paris. Il ne
se cacha à personne du chagrin de son départ\,; ni assez de son
inquiétude, car il ne se contraignit pas de dire qu'il quittait un
paradis terrestre pour retourner dans un pays où il ne trouverait que
des épines, et pas un homme à qui se fier, et qu'il quitterait avec
plaisir tous les emplois qu'il avait en Espagne, si le roi son maître
lui voulait faire la grâce de le nommer son ambassadeur en France pour y
demeurer toujours. Deux jours après, le roi lui envoya un diamant de dix
mille écus, et il partit aussitôt après avec Cellamare son neveu, pour
retourner en poste en Espagne.

En arrivant à Bayonne, il trouva un ordre qui lui défendait d'entrer en
Espagne, et qui lui enjaignait d'en attendre de nouveaux à Bayonne. Il
en parut fort abattu. Il envoya son neveu à Madrid et il demeura à
Bayonne. Nous l'y laisserons parce qu'il y demeura longtemps. Il y eut
le dégoût de recevoir défense de voir la nouvelle reine d'Espagne, qui y
entra tandis qu'il se morfondait à Bayonne. On verra en son temps ce
qu'il devint et Macañas.

Je ne sais ce qui était revenu à la princesse des Ursins sur les
dispositions de la princesse de Parme, mais elle entra dans de tels
soupçons de son esprit haut et entreprenant, qu'elle se repentit d'avoir
fait ce mariage, et qu'elle eut envie de le rompre. Elle fit donc naître
je ne sais quelles difficultés, sur lesquelles elle fit dépêcher un
courrier à Rome au cardinal Acquaviva qui y faisait les affaires du roi
d'Espagne, avec ordre de différer son voyage à Parme, où il avait ordre
d'aller faire la demande et d'y voir épouser la princesse par le duc de
Parme, frère cadet du feu père de la princesse, qui avait épousé sa mère
peu de temps après avoir succédé au duché. M\textsuperscript{me} des
Ursins avait changé d'avis trop tard. Le courrier ne trouva plus
Acquaviva à Rome\,: ce cardinal était en chemin et près d'arriver à
Parme, de sorte qu'il n'y eut pas moyen de reculer.

Il fut reçu avec de grands honneurs et une grande magnificence\,: il fit
la demande, mais il différa les épousailles comme il put, et ce
retardement fit beaucoup parler. En attendant, la dépense était pesante
à Parme\,; le mariage, qui se devait célébrer le 25 août, ne le fut que
le 16 septembre, par le cardinal Gozzadini, légat \emph{a latere} pour
cette fonction, et pour complimenter la reine d'Espagne au nom du pape.
Elle partit incontinent après pour aller s'embarquer à Gènes et aller
par mer à Alicante, accompagnée du marquis de Los Balbazès et de la
princesse de Piombino, femme de beaucoup d'esprit, et amie particulière
de la princesse des Ursins. Albéroni, qu'elle avait envoyé à Parme dès
les commencements de cette affaire du mariage, retourna de la part du
duc de Parme à son emploi d'Espagne, à la suite de la nouvelle reine.

Deux mariages moins importants se firent en même temps. La princesse
d'Espinoy, intimement liée, comme on l'a vu en plus d'un endroit avec
feu M\textsuperscript{me} de Soubise et ses fils, donna sa fille, qui
était fort riche, au fils unique du prince de Rohan, qui de son côté
devait l'être infiniment. Il n'y eut point de fiançailles chez le roi,
et quelques jours après M\textsuperscript{me} d'Espinoy présenta sa
fille, qui prit le tabouret au souper.

L'autre mariage ne fut pas si égal en biens et en naissance. Le comte de
Roucy s'était détaché de faire le mariage de M\textsuperscript{lle} de
Monaco pour son fils, malgré M\textsuperscript{me} de Monaco et M. le
Grand. Il le maria à la fille d'Huguet, conseiller au parlement, unique
et fort riche, dont le comte de Roye avait fort grand besoin.

Le roi, qui avait été de fort mauvaise humeur durant le chemin, jusqu'à
se fâcher de bagatelles contre son ordinaire, à casser le cocher qui le
menait, et à tomber sur le premier écuyer qu'il aimait, à ce que me dit
M\textsuperscript{me} de Saint-Simon, qui alla à Fontainebleau et en
revint seule dans son carrosse avec les princesses, n'était apparemment
pas revenu du tourment qu'il avait reproché au duc du Maine, et dont il
avait parlé si ouvertement et si amèrement au premier président, et au
procureur général, et à la reine d'Angleterre, sur tout ce qu'on lui
avait fait faire si fort contre son gré. Il trouva son appartement à
Fontainebleau tout à fait changé. Je ne sais s'il fut plus commode, mais
il n'en parut pas plus beau.

L'électeur de Bavière y vint peu de jours après, et s'y établit chez
d'Antin avec une table et le plus gros jeu du monde qui commençait dès
le matin. Il ne laissait pas d'aller jouer chez M\textsuperscript{me} la
Duchesse, et elle quelquefois chez lui. Elle le menait d'ordinaire dans
sa gondole sur le canal lorsque le roi, suivi de toute la cour, s'y
promenait en carrosse. L'électeur fut de toutes les chasses, où il
voyait le roi, d'ailleurs fort rarement dans son cabinet.

M\textsuperscript{me} de Maintenon chercha fort à amuser le roi chez
elle par des dîners, des musiques, quelque jeu dans leur intrinsèque. On
avait pratiqué une tribune sur la salle de la comédie en face du
théâtre. On allait à cette tribune de chez M\textsuperscript{me} de
Maintenon. Le roi, qui depuis longues années n'allait plus aux
spectacles, y parut quelquefois pendant quelques actes avec quelques
dames choisies outre celles des dîners. J'y vis une fois
M\textsuperscript{me} d'Espinoy. Il ne laissa pas d'en voir
quelques-unes entières de Molière chez M\textsuperscript{me} de
Maintenon jouées par les comédiens, avec des intermèdes de musique.

Le fils du comte du Luc y arriva le matin du mercredi 12 septembre, avec
la nouvelle que la paix de l'empereur et de l'empire avec le roi avait
été signée le 7 à Bade, sur le modèle signé et convenu entre l'empereur
et le roi à Rastadt.

Prior y donna aussi part au roi dans une audience particulière, de la
part du nouveau roi d'Angleterre, de son avènement à cette couronne, de
son prochain départ d'Hanovre pour se rendre à Londres, et de son
dessein d'entretenir la paix et un bon voisinage. Il fit son entrée fort
magnifique à Londres le 1er octobre\,; ôta au duc d'Ormond, au lord
Bolingbroke et à plusieurs seigneurs leurs emplois\,; changea tout le
ministère de la reine Anne, en prit un tout opposé qui poursuivit le
dernier sur la paix de l'Angleterre avec la France, et sur des affaires
intérieures\,; rétablit Marlborough dans toutes ses charges et
commandements\,; éleva les whigs aux dépens des torys. Cela ne
témaignait rien de favorable à la France, aussi était-il tout à
l'empereur.

Le maréchal de Villeroy arriva de Lyon à Fontainebleau, le mardi 18
septembre, heureux de s'être trouvé absent lors du dernier comble des
bâtards et du testament, et hors de portée de ces temps si orageux dans
l'intime intrinsèque où il était admis. Il fut reçu en favori tout
nouvellement comblé des plus grandes grâces, déclaré ministre d'État,
dont il prit place le lendemain au conseil d'État. Il est plaisant que
cet emploi, le plus important de tous, soit l'unique qui ne prête aucun
serment, fondé sur ce qu'à chaque conseil d'État l'huissier va le matin
même avertir tous ceux qui en sont de s'y rendre, de manière que si l'un
d'eux n'est point averti, il n'y va point, et comprend qu'il est
remercié. Cela n'arrive pourtant jamais de la sorte\,; leur disgrâce se
déclare par un ordre de se retirer, ou en un lieu marqué pour exil, ou
hors de la cour seulement. Longtemps depuis, Torcy m'a conté que le roi
prenait la parole avant le maréchal de Villeroy dans les commencements,
pour lui mieux faire entendre de quoi il s'agissait, que le maréchal
opinait si pauvrement et disait ou demandait des choses si étranges que
le roi rougissait, baissait les yeux avec embarras, quelquefois
interrompait ses questions pour répondre d'avance, et qu'il ne
s'accoutuma jamais, mais comme un gouverneur qui couve son élève, à
l'ignorance, aux \emph{sproposito}, à l'ineptie du maréchal, qui par le
grand usage de la cour et du commandement des armées dans les derniers
temps des affaires et de la confiance du roi, les surprenait tous par ne
savoir jamais ce qu'il disait, ni même ce qu'il voulait dire. J'en fus
étonné moi-même au dernier point après la mort du roi.

Le maréchal de Villars arriva de Bade le lendemain de l'autre dont il se
trouva fort obscurci.

Le duc de Mortemart arriva le jeudi 20 septembre à Fontainebleau,
dépêché par le duc de Berwick, qui fit commencer à la pointe du jour du
11 septembre une attaque générale à Barcelone, à laquelle les assiégés
ne s'étaient point attendus. Ils défendirent mal leurs brèches, et on
demeura maître de trois bastions et de deux courtines. Ils ne se
défendirent point dans le bastion de Saint-Pierre qui était le quatrième
attaqué à la fois. Mais on n'y put demeurer par le grand feu qui sortait
d'un couvent qui le commandait. Ce fut où on perdit le plus, et en tout
l'action a beaucoup coûté de part et d'autre. Ils se retirèrent derrière
l'ancienne enceinte qui sépare les deux villes, et le maréchal de
Berwick en fut bien aise, pour leur donner lieu de capituler, et à lui
d'empêcher le pillage de la ville\,; Talleyrand et Houdetot brigadiers y
furent tués. À la fin les assiégés se rendirent à discrétion la vie
sauve, mais sans aucune mention de leurs biens\,; le Mont-Joui se rendit
de même en même temps\,; et Cardone quelques jours après, comme on en
était convenu.

Cet assaut général, où Dillon commandait comme lieutenant général de
tranchée, et Cilly lieutenant général avec la nouvelle tranchée qui
devait le relever, fut donné par trente et un bataillons et trente-huit
compagnies de grenadiers commandés par le marquis de La Vère, frère du
prince de Chimay, et par Guerchy, lieutenants généraux, et Châteaufort
avec six cents dragons attaqua en même temps une redoute vers la mer,
soutenu par Armendaris, avec trois cents chevaux, qui a été depuis
vice-roi du Pérou. Tout fut attaqué en même temps\,; il se trouva un
grand retranchement derrière tout le front de l'attaque où les assiégés
chassés des trois bastions et des deux courtines firent plus ferme. Les
assiégeants s'étendirent et les emportèrent\,; ils s'emparèrent aussitôt
de beaucoup de maisons et de quelques places, et s'y maintinrent malgré
plusieurs recharges des assiégés. Berwick y fut toujours au milieu du
plus grand feu, y donnant ses ordres avec le même sang-froid que s'il
eût été dans sa chambre. Il fit faire une coupure au rempart pour faire
de nouvelles dispositions, et au moyen des maisons se porter en avant.
Le feu fut très violent de toutes parts et dura jusqu'à quatre heures
après midi, que les ennemis firent rappeler. Leurs députés sortirent, il
y eut plusieurs allées et venues. Enfin le lendemain 12, il se rendirent
à discrétion, comme on l'a dit. La cavalerie monta sur la fin de
l'action par les brèches dans la ville.

On souffrit assez de plusieurs mines et fougasses qu'ils firent jouer
pendant l'attaque\,; et on compta environ quinze cents hommes tués ou
blessés de chaque côté à cette attaque, avec beaucoup d'officiers. La
place avait tenu soixante et un jours de tranchée ouverte, avec une
résolution et une opiniâtreté extrêmes des troupes et des habitants,
enragés de l'abandon de l'empereur et de la perte pour toujours de leurs
privilèges par leur réduction, et de ceux de leur province dont ils ont
été de tout temps si jaloux, et dont ils avaient si étrangement abusé.
Les moines de tous ordres, surtout les capucins, et tous les autres de
Saint-François, les jésuites même, signalèrent leur rage par les
fatigues et les périls où ils s'exposèrent sans cesse, et par leurs
vives exhortations soutenues de leur exemple.

Berwick mit un si grand ordre à tout que, dès le lendemain qu'ils se
furent rendus, tout parut si tranquille par toute la ville que les
boutiques y furent ouvertes à l'ordinaire. Il fit rendre les armes aux
bourgeois, changea toute l'ancienne forme du gouvernement, cassa la
députation, fit de nouveaux magistrats, établit une nouvelle forme de
gouvernement, sous le nom de junte, en attendant les ordres du roi
d'Espagne, auquel il dépêcha le prince de Lanti, neveu de la princesse
des Ursins. Les miquelets et les volontaires de la campagne vinrent se
rendre en foule. La Catalogne fut soumise. Villaroël, dont on a parlé à
l'occasion de l'affaire d'Espagne de M. le duc d'Orléans, commandait à
Barcelone. Il fut embarqué avec Basset et une vingtaine d'autres
principaux chefs de la rébellion, tous militaires, et conduits au
château d'Alicante, pour y demeurer le reste de leurs jours, ou être
distribués en d'autres prisons.

Le duc de Berwick demeura un mois à Barcelone pour régler toutes les
affaires militaires et civiles de la ville et de la province, et s'en
alla ensuite à Madrid. Cette conquête, qui couvrit de gloire sa valeur,
sa capacité, sa prudence, fut le sceau de l'affermissement de la
couronne d'Espagne sur la tête de Philippe V et de la tranquillité
publique, dont l'empereur ne put cacher son extrême déplaisir malgré la
paix.

Broglio, gendre de Voysin, arriva le 23 septembre à Fontainebleau avec
tout le détail. On sut par lui qu'il n'y avait eu ni capitulation ni
aucuns articles signés, que le duc de Berwick ne l'avait pas voulu
souffrir, et qu'il avait mis quatorze bataillons français dans Barcelone
avec quelque cavalerie espagnole. Pour Cardone, Montemar, qui a tant
fait parler depuis de lui en Italie, en prit possession pour le roi
d'Espagne\,; il permit à la garnison, à toute laquelle il accorda le
pardon, de se retirer à leur choix hors de la domination d'Espagne, ou
chez eux ceux qui avaient du bien.

Le roi de Pologne, qui s'était fait catholique pour obtenir cette
couronne si bien séante à la situation de son électorat, s'y trouvait
assez affermi depuis le désastre du roi de Suède, pour se flatter d'y
pouvoir avoir son fils pour successeur. Mais le premier pas à faire pour
y parvenir était que le prince électoral embrassât aussi la religion
catholique, et il s'y trouvait de grandes difficultés. Comme électeur de
Saxe il était chef et protecteur né des luthériens d'Allemagne\,;
c'était à lui que s'adressaient tous leurs griefs sur leur religion, il
était chargé de les faire redresser par l'empereur et par l'empire, et
de l'exécution de tous les traités faits là-dessus. Cette qualité lui
donnait un grand poids dans l'empire, et il en était si bien persuadé,
que tout catholique qu'il était devenu, il avait trouvé moyen de se
conserver cette dictature. Il n'avait point d'autres enfants que ce fils
à qui il voulait aussi transmettre cette même autorité dans l'empire.
Toute la Saxe était rigidement luthérienne, ses autres États l'étaient
en partie, deux électeurs catholiques de suite ne pouvaient que causer
une grande alarme aux luthériens et les porter du moins à se choisir un
autre protecteur.

Il trouvait de plus un grand obstacle dans la personne de Christine
Éverardine son épouse et mère du prince électoral, fille de
Christian-Ernest, marquis de Brandebourg-Bareith, princesse altière,
courageuse, luthérienne zélée, qui avait publiquement détesté son
changement de religion, l'ambition qui l'y avait porté, qui n'avait
jamais voulu mettre le pied en Pologne, ni prendre le nom, les marques
et le rang de reine. Elle avait même poussé les choses jusqu'à ne
vouloir pas le voir dans les séjours qu'il allait faire en Saxe, où elle
se retirait dans un château éloigné dès qu'elle apprenait qu'il partait
de Pologne, et s'y tenait jusqu'à ce qu'il fût retourné.

Tant d'obstacles ne furent pas capables de le rebuter. Il gagna l'esprit
de son fils dans ses séjours en Saxe, il glissa sourdement auprès de lui
quelques domestiques sûrs et de sa confiance\,; et pour le tirer
d'auprès de l'électrice en son absence, et d'une cour toute luthérienne,
il le fit voyager avec peu d'accompagnement dans un entier incognito
sous le nom de comte de Lusace.

Il choisit le palatin de Livonie pour lui confier le prince et son
secret, et il était difficile de trouver un seigneur qui eût toutes les
qualités de celui-là, et aussi capable de conduire aussi dignement et
aussi convenablement un jeune prince dans les différentes parties de
l'Europe qu'il lui fit voir. Le roi de Pologne y joignit un habile
jésuite travesti qui en eut permission de son général et du pape, et qui
conduisit la conversion du prince, et ses affaires à lui si heureusement
et avec tant de dextérité, qu'il en fut fait cardinal lorsqu'on jugea
qu'il était temps de rendre la conversion publique. C'est lui qui a
figuré si longtemps depuis sous le nom de cardinal de Salerne, et mort à
Rome au bout de neuf ou dix ans de son cardinalat.

Le prince électoral avec ce peu de suite vit l'Italie entière, après
avoir parcouru une partie de l'Allemagne. Il séjourna longtemps à Rome
où il fit secrètement son abjuration. Le pape lui accorda un bref qui
lui permit de la tenir cachée, en sorte que jusqu'à ses domestiques y
furent trompés. Deux ou trois domestiques affidés gardèrent un secret
impénétrable, par le moyen desquels il entendait la messe, dans sa
chambre, du P. Salerne, et y approchait souvent des sacrements avant
qu'on fût levé chez lui. Il vint en France en ce temps-ci, et prit toute
une maison garnie sur le quai Malaquais, au coin de la rue des
Petits-Augustins.

Il arriva le 26 septembre à Fontainebleau, ayant passé quelques jours à
Paris. Il vit Madame en arrivant, qui le présenta au roi sous le nom du
comte de Lusace au sortir de son souper. Il parut un grand et gros
garçon de dix-huit ans, bien frais, blond, avec de belles couleurs, et
faisant fort souvenir de M. le duc de Berry, l'air sage, modeste,
attentif à tout, fort poli mais avec mesure et dignité, et qui, sous un
incognito qui ne prétendit jamais rien, montrait sentir fort ce qu'il
était, et sans embarras. Son palatin plut extrêmement à tout le monde
par son esprit, sa sagesse, le discernement qu'on lui remarqua, l'air du
grand monde, et une aisance mesurée à propos dans sa liberté, et qui ne
laissait jamais apercevoir au dehors qu'il fût le mentor du jeune
prince.

Il dîna le vendredi 28 septembre chez l'électeur de Bavière, qui avait
vu le roi dans son cabinet après sa messe, et qui s'en alla le soir à
Saint-Cloud et de là à Compiègne. Le lendemain le roi courut le cerf. Il
fit donner de ses meilleurs chevaux au prince électoral et au palatin,
et d'autres aux principaux de sa suite. Il eut pendant son séjour toutes
les attentions pour lui que l'incognito permit, et traita aussi le
palatin avec distinction. Les principaux de la cour leur en firent fort
bien les honneurs. Le roi le convia souvent aux chasses, et sur ce qu'il
versa dans Paris, envoya un gentilhomme ordinaire savoir de ses
nouvelles.

\hypertarget{chapitre-xiv.}{%
\chapter{CHAPITRE XIV.}\label{chapitre-xiv.}}

1714

~

{\textsc{Mort et famille de M\textsuperscript{me} de Bullion\,; son
caractère.}} {\textsc{- Mort et caractère de Sézanne\,; sa famille.}}
{\textsc{- Mort et caractère du bailli de La Vieuville et de la comtesse
de Vienne.}} {\textsc{- Le bailli de Mesmes lui succède et ne le
remplace pas dans l'ambassade de Malte.}} {\textsc{- Mort, caractère,
famille, testament de la marquise de Saint-Nectaire.}} {\textsc{- La
reine d'Espagne débarque à Monaco et va par terre en Espagne.}}
{\textsc{- Sa dot.}} {\textsc{- Sa réception incognito.}} {\textsc{-
Béthune, premier gentilhomme de la chambre de M. le duc de Berry en
année à sa mort, reporte sa Toison en Espagne, et l'obtient.}}
{\textsc{- Le duc de Saint-Aignan porte un médiocre présent du roi à la
reine d'Espagne à son passage.}} {\textsc{- Chalais grand d'Espagne avec
exclusion d'en avoir en France le rang et les honneurs.}} {\textsc{-
Prince de Rohan et prince d'Espinoy ducs et pairs.}} {\textsc{- Manéges
qui les font.}} {\textsc{- Ruse orgueilleuse du prince de Rohan.}}
{\textsc{- L'autre prend le nom de duc de Melun.}} {\textsc{- Voyage et
retour de Sicile de son nouveau roi.}} {\textsc{- Maffei\,; ses
emplois\,; son caractère.}} {\textsc{- Retour de Fontainebleau par
Petit-Bourg\,; le roi chagrin pendant le voyage.}} {\textsc{- Embarras
sur la constitution.}} {\textsc{- Amelot envoyé à Rome pour la tenue
d'un concile national en France.}} {\textsc{- P. Tellier me propose
d'être commissaire du roi au concile\,; son ignorance\,; surprise de mon
refus.}} {\textsc{- Mort singulière de Brûlart, évêque de Soissons\,;
son caractère.}} {\textsc{- Mort de M. de Saint-Louis retiré à la
Trappe.}} {\textsc{- Avary ambassadeur en Suisse.}} {\textsc{- Comte du
Luc ambassadeur à Vienne et conseiller d'État d'épée.}} {\textsc{-
L'impératrice couronnée reine de Hongrie à Presbourg.}} {\textsc{-
Électeurs de Cologne et de Bavière voient le roi à Marly.}} {\textsc{-
Saumery fils envoyé du roi près l'électeur de Bavière.}} {\textsc{-
Pompadour et d'Alègre vainement ambassadeurs en Espagne et en
Angleterre.}} {\textsc{- Retour du duc de Berwick avec une épée de
diamants donnée par le roi d'Espagne.}} {\textsc{- Taxe du prix des
régiments d'infanterie.}} {\textsc{- Pension de dix milles livres au
prince de Montbazon.}} {\textsc{- Cent cinquante mille livres
d'augmentation de brevets de retenue sur ses charges à Torcy.}}
{\textsc{- Dix mille écus à Amelot pour son voyage.}} {\textsc{- Procès
d'impuissance intenté au marquis de Gesvres par sa femme\,; accommodé.}}
{\textsc{- M. le duc d'Orléans se trouve assez mal.}} {\textsc{- Grand
témoignage du roi sur moi.}} {\textsc{- Apophtegme du roi sur M. le duc
d'Orléans.}}

~

M\textsuperscript{me} de Bullion mourut à Paris. Elle était de ces
Rouillé des postes, et point vieille\,; c'était une femme d'esprit, mais
dominante dans sa famille\,; habile, altière, ambitieuse, et qui ne se
consolait point d'être Rouillé et femme de Bullion, enfermé chez lui à
la campagne, et qui aurait dû l'être beaucoup plus tôt qu'il le fut. On
a parlé ailleurs d'elle. Ses soeurs eurent des maris plus complaisants.
Le marquis de Noailles, frère du cardinal, et Bouchu, conseiller d'État,
leur donnèrent lieu, après leur mort, d'épouser le duc de Richelieu et
le duc de Châtillon. M\textsuperscript{me} de Bullion serait morte
d'étonnement et de suffocation de joie, si elle avait vécu jusqu'en
1724, et qu'elle eût vu son fils chevalier de l'ordre.

Sézanne mourut à Rouen en ce même temps. Il était frère de père du duc
d'Harcourt, et frère de mère de la duchesse d'Harcourt, lieutenant
général, et encore fort jeune. C'était un, grand bellâtre, fort prévenu
de son mérite et de sa capacité, qui en prévenait fort peu les autres,
et fort gâté par le brillant état de son frère, qui l'avait élevé comme
son fils. Sa maladie fut une langueur de plusieurs années qui le
consuma, où la médecine ne connut rien. Il était persuadé, et on le crut
aussi, que sa galanterie en Italie avec des maîtresses que le duc de
Mantoue entretenait publiquement et à grand marché, mais dont il était
fort jaloux, lui avait fait donner un poison lent. Il ne laissa point
d'enfants de son mariage avec la fille unique fort riche de Nesmond,
lieutenant général fort distingué des armées navales\,: Harcourt lui
avait fait donner en Espagne la Toison qui lui était destinée. Il
l'obtint à sa mort pour son second fils. Ce fils mourut quelque temps
après\,; elle fut donnée au troisième. Il mourut aussi de fort bonne
heure. Mais les temps étaient changés, et cette Toison si successive
sortit de chez les Harcourt.

Le bailli de La Vieuville, ambassadeur de Malte, mourut aussi de
l'opération de la taille, universellement regretté. C'était un des
hommes que j'aie vus des plus aimables, et un fort honnête homme, noble
et magnifique autant qu'il le put dans son emploi, sans faire tort à
personne. Il était fils de feu M. de La Vieuville, duc à brevet, mort
gouverneur de M. le duc d'Orléans, dans ce temps-là duc de Chartres, un
mois après avoir été reçu chevalier du Saint-Esprit, en la promotion de
1688. Sa belle-sœur la comtesse de Vienne, qui jouait fort, et beaucoup
à Paris du grand monde, mourut bientôt après chez la duchesse de Nemours
à Paris, à qui elle était allée rendre une visite. Le bailli de La
Vieuville fut mal remplacé\,; M. du Maine n'avait garde de manquer cette
occasion de s'attacher le premier président de plus en plus par son
endroit le plus sensible. Il engagea le roi de s'intéresser pour le
bailli de Mesmes son frère, et il fut ambassadeur. C'était un homme sans
esprit et sans mine, étrangement débauché, grand panier percé, assez
obscur, qui fit honte à son emploi en plus d'une sorte, et qui courut
risque de le perdre plus d'une fois.

La marquise de Saint-Nectaire mourut à Paris, à soixante et onze ans.
Elle avait de l'esprit et de l'intrigue, avait été fille d'honneur de la
reine, et fort jolie sans avoir jamais fait parler d'elle\,; elle était
Longueval et riche par la mort de son frère, tué lieutenant général en
Italie, sans avoir été marié. Elle avait épousé en 1668 le cousin
germain du duc de La Ferté fils des deux frères. Il tua à Vienne en
Autriche le comte du Roure en duel, dont il demeura manchot. Il eut de
grands démêlés avec sa mère qui était Hautefort, étrangement remariée à
Maupeou, président à mortier au parlement de Metz. Il fut assassiné à
l'occasion de ces démêlés à Privas en 1671, n'ayant que vingt-sept ans.
Sa mère en fut fort soupçonnée, et son second fils, le chevalier de
Saint-Nectaire\,; d'y avoir eu tant de part, qu'il en fut plus de
vingt-cinq ans en prison, et n'en sortit que par un accommodement. Il
parut depuis dans le monde avec un air fort hébété.
M\textsuperscript{me} de Saint-Nectaire n'eut qu'une fille, dont la
beauté fit tant de bruit, qui mourut avant sa mère, et qui laissa de
Florensac, frère du duc d'Uzès, un fils qui n'a pas vécu et une fille
qui épousa le beau comte d'Agenois, que la princesse de Conti et le
parlement ont fait duc et pair d'Aiguillon. M\textsuperscript{me} de
Saint-Nectaire laissa tout son bien à Cani, par amitié pour Chamillart
son père, en cas que les enfants de sa fille n'en laissassent point.

L'envoyé de Parme eut audience du roi, le 11 octobre, à Fontainebleau,
sur le mariage de la princesse de Parme. C'était un peu tard. Elle eut
cent mille pistoles de dot, et pour trois cent mille livres de
pierreries. Elle s'était embarquée pour Alicante à Sestri di Levante.
Une forte tempête la dégoûta de la mer. Elle débarqua à Monaco pour
traverser par terre la Provence, le Languedoc et la Guyenne, pour gagner
Bayonne et y voir la reine d'Espagne, veuve de Charles II, sœur de sa
mère. Desgranges, maître des cérémonies, la fut trouver en Provence avec
ordre de la suivre, et de la faire accompagner et servir de tout par les
gouverneurs lieutenants généraux, et par les intendants des provinces
par où elle devait passer, quoiqu'elle fût dans le parfait incognito.

Le marquis de Béthune, aujourd'hui duc de Sully, premier gentilhomme de
la chambre de M. le duc de Berry en année à sa mort, reporta sa Toison
en Espagne. Il était gendre de Desmarets, et M\textsuperscript{me} des
Ursins ne manqua pas cette occasion de la lui faire donner. Le roi
consola le duc de Saint-Aignan, qui était l'autre premier gentilhomme de
la chambre, et qui aurait fort voulu aller porter la Toison, dans
l'espérance de l'obtenir, en l'envoyant à la reine d'Espagne, à son
passage, lui porter ses compliments et un présent de sa part. Il
consistait en son portrait garni de quatre diamants avec quelques
bijoux. Il se ressentit du peu de satisfaction du mariage, car il ne
valait guère que cent mille francs.

La princesse des Ursins fit faire en même temps grand de la première
classe Chalais, son homme de toute confiance, fils du frère de son
premier mari, qu'on a vu en plus d'un endroit ici employé par elle à
bien des choses secrètes. Il fallut en demander la permission au roi,
qui ne la voulut accorder qu'à condition de ne revenir plus en France,
ou de se résoudre à n'y jouir d'aucun rang ni honneurs, non plus que
s'il n'était pas grand d'Espagne. Cette nouveauté, non encore arrivée
depuis l'avènement de Philippe V à la couronne d'Espagne, dut donner à
penser à M\textsuperscript{me} des Ursins. C'était un coup de fouet qui
portait directement sur elle. Chalais ne laissa pas d'être grand, et
certes il était temps pour lui\,; on verra dans la suite qu'il n'est
rien tel que d'obtenir ces grandes grâces.

Le roi, sortant de dîner le samedi 20 octobre, fit entrer le prince de
Rohan dans son cabinet. Il lui dit qu'il le faisait duc et pair, et le
prince d'Espinoy aussi, qu'il ne pouvait refuser cette grâce au mérite
de sa mère, à laquelle il commanda au prince de Rohan d'en porter la
nouvelle de sa part. La princesse d'Espinoy vint remercier le roi, à son
retour de la chasse, qui la combla d'honnêtetés, et lorsque le prince
d'Espinoy le remercia, il lui dit qu'il avait grande obligation à sa
mère, et qu'il ne pouvait trop lui témoigner de reconnaissance, de
respect et d'attachement.

Le prince de Rohan désirait ardemment d'être duc et pair, et l'avait
souvent demandé\,; jamais aussi je ne vis homme si aise, ni qui le
témoignât plus franchement, bien que la franchise ne fût pas sa vertu
favorite. Lui et M\textsuperscript{me} d'Espinoy venaient de marier
leurs enfants. Il faut se souvenir de là liaison intime qu'on a vue en
son lieu\,; que l'habile M\textsuperscript{me} de Soubise, dans la vue
de Monseigneur et de l'avenir, forma avec M\textsuperscript{me} de
Lislebonne et ses deux filles, qui, à cause du présent, s'y prêtèrent
volontiers\,; que ce fut pour cela que M\textsuperscript{me} de Soubise
fit le mariage du feu prince d'Espinoy, fils de sa sœur, avec la seconde
fille de M\textsuperscript{me} de Lislebonne\,; et que la liaison devint
telle que M\textsuperscript{lle} de Lislebonne, abbesse de Remiremont,
après la mort de Monseigneur, et sa sœur, M\textsuperscript{me}
d'Espinoy, ne furent qu'un avec le prince et le cardinal de Rohan, ce
qui subsista toute leur vie.

M\textsuperscript{me} de Soubise, avant sa mort, avait tiré parole du
roi de faire le prince de Rohan duc et pair. Tout princes que sa beauté
avait su faire les Rohan, elle avouait très librement que cela ne tenait
qu'à un bouton, et qu'il n'y avait en France de vraie et solide grandeur
pour les maisons que le duché-pairie. La maison de Lorraine, à qui la
principauté véritable ne peut être disputée, l'avait pensé ainsi dans sa
plus haute puissance. Elle en accumula dix ou douze à la fois dans ses
diverses branches. Ce fut par ce degré qu'elle monta depuis à tout ce
qu'elle osa entreprendre sur les rangs, et de là aux choses les plus
hautes qui furent si près de renverser l'État, et d'ôter la couronne à
la postérité de saint Louis et d'Hugues Capet, rangs et distinctions
qu'elle a su se conserver dans la chute de la Ligue, et dont la
jouissance jusqu'à aujourd'hui fait l'admiration d'étonnement de tout ce
qui pense et réfléchit. Ce que M\textsuperscript{me} de Soubise avait si
sagement comme assuré, le cardinal son fils l'acheva. Devenu avec le P.
Tellier une seule et même personne pour la ruine du cardinal de Noailles
et pour tous les vastes et pernicieux desseins de cet effroyable
jésuite, auquel, comme on l'a vu ailleurs, il s'était enfin abandonné
totalement, il ne laissa pas échapper une conjoncture pour sa maison
aussi favorable pour lui que l'affaire actuelle de la constitution, et
voulut en même temps profiter de si puissants appuis pour le prince
d'Espinoy, fils de son cousin germain, et dont la sœur venait d'épouser
son neveu. M\textsuperscript{me} d'Espinoy, comme on l'a vu ailleurs,
avait depuis longtemps avec M\textsuperscript{me} de Maintenon
d'étranges et d'invisibles liaisons, si fortes et si intimes qu'il était
bien difficile qu'elle ne la servît pas à souhait, tellement que cette
complication de choses fit ces deux nouveaux ducs et pairs. On verra
bientôt une troisième pairie de la même façon de cette féconde
constitution. Joyeuse fut le duché-pairie érigé pour le prince
d'Espinoy, qui, préférant le nom de sa maison véritablement fort grande,
prit le nom de duc de Melun.

Le prince de Rohan, transporté du solide qu'il avait si longuement
poursuivi, rusa et voulut faire plus que pas un de la maison de
Lorraine, de celle de Savoie, ni des autres vrais princes étrangers qui
ont été ducs, excepté l'unique comte de Soissons, mari de cette toute
puissante nièce du cardinal Mazarin, pour qui fut inventée la charge de
surintendante de la reine. Il fit ériger Frontenay en duché-pairie, dont
Soubise, ce fameux rebelle, avait été fait duc à brevet par Louis XIII.
Mais le prince de Rohan lui fit changer son nom, et donner le sien
redoublé de Rohan-Rohan, à l'exemple de quelques branches de maisons
d'Allemagne, comme Baden-Baden, pour se distinguer des autres de même
nom\,; lui pour se distinguer du duché-pairie de Rohan, qui a passé dans
la maison Chabot, mais en effet pour continuer à porter le nom de prince
de Rohan sous le spécieux prétexte de la cacophonie continuelle des noms
de duc de Rohan et de duc de Rohan-Rohan tous deux existants. Avec cette
adresse il conserva son nom de prince de Rohan, et laissa croire aux
sots qu'il n'avait pas daigné porter un titre, après lequel il ne se
cachait pas même d'avoir si ardemment et si longuement soupiré, et
d'être comblé de joie d'en être enfin revêtu.

Le duc de Savoie, nouveau roi de Sicile par la paix, alla avec la reine
son épouse se faire couronner dans son île, la connaître par lui-même,
et y établir son gouvernement. Il passa plusieurs mois à Messine et à
Palerme, au milieu d'une nombreuse cour, des plus grands seigneurs et de
la première noblesse de Sicile. Il revint à Turin en ce temps-ci, ayant
laissé le comte Maffei vice-roi, homme de beaucoup d'esprit et délié,
fort dans sa confiance, et chargé souvent par lui d'affaires délicates
et secrètes.

Ce fut lui qu'il envoya au Pont-Beauvoisin lors du mariage de
M\textsuperscript{me} la duchesse de Bourgogne, pour voir comment elle
serait reçue en France. Il fut depuis en diverses ambassades
importantes, enfin à Paris, où il reçut l'Annonciade, qui est le suprême
honneur de la cour de Savoie, en la dernière promotion de cet ordre que
fit son maître. Maffei était souple, avisé, insinuant, capable des plus
grandes affaires et des plus adroites exécutions, comme on le verra en
son temps en Sicile. Avec cela gaillard, même fort débauché, et
d'excellente compagnie, vivant toujours avec la meilleure partout. Il
savait beaucoup et avait fort servi à la guerre. Il mourut fort vieux,
fort suspect au nouveau roi, et fort abandonné depuis la catastrophe du
premier roi, auquel il était uniquement attaché.

Le roi revint de Fontainebleau, le mercredi 23 octobre, coucher à
Petit-Bourg, et le lendemain à Versailles. M\textsuperscript{me} de
Saint-Simon, qui était dans son carrosse, me dit qu'il n'était pas de
meilleure humeur qu'en allant, et qu'à le voir ainsi de suite sa santé
paraissait diminuer. Ce fut aussi son dernier voyage de Fontainebleau.

Il était aussi fort tourmenté de l'affaire de la constitution où le P.
Tellier lui avait fait mettre sa conscience et son autorité. Il y avait
eu force négociations avec le cardinal de Noailles. Le cardinal
d'Estrées, qui, par ordre du roi, s'en était mêlé d'abord, s'en était
retiré presque aussitôt, indigné des friponneries continuelles du P.
Tellier et de Bissy, dont il ne se tut pas. Le cardinal de Polignac s'y
fourra longtemps après. Le succès fut pareil\,; il en demeura mal avec
le roi, et rompit avec tant d'éclat avec le cardinal de Rohan qu'il ne
lui fit aucun compliment sur le duché-pairie de son frère. Tout ce qui
était savant et de bonne foi suivait le cardinal de Noailles dans
l'épiscopat, les fameuses universités entières, les ordres religieux et
réguliers, les chapitres et les curés de Paris, et une infinité de
toutes les provinces, enfin les parlements et tous les laïques instruits
qui n'étaient pas esclaves des jésuites\,; jusque dans la cour, il n'y
avait sourdement qu'une voix.

Parmi les acceptants, pas l'ombre d'uniformité\,: les uns évêques et
autres adhéraient en petit nombre à ce qu'avait fait l'assemblée des
quarante, et ceux-là encore avec des diversités chacun\,; la plupart des
acceptants, sans y adhérer, avaient tous entre eux des explications
différentes\,; les quarante même se mirent à varier sur le sens de leur
mandement d'acceptation\,; c'était un chaos et une tour de Babel, ainsi
que le montra un extrait tiré de la totalité des mandements des évêques
qui se contredisaient tous en acceptant, sans qu'aucun s'accordât avec
un autre.

On vit donc plus clairement que jamais que, sans les menaces et les
promesses, les récompenses et les plus durs châtiments et les plus
étendus, l'artifice et la violence ouverte, la constitution aurait été
universellement rejetée, et qu'il n'était question parmi les acceptants
que de trouver le moyen de ne recevoir que des mots, et de rejeter tout
le sens.

Le pape de plus, très mécontent de n'avoir pas trouvé la soumission
aveugle et uniforme dont le P. Tellier lui avait tant répondu, et sans
quoi il ne se serait jamais embarqué dans cette détestable affaire,
avait fait sentir aux quarante évêques en particulier, par un bref
public, la colère où il était de leur audace d'avoir osé interpréter sa
bulle, et de ne l'avoir pas acceptée aveuglément\,; en sorte que ceux
qui avaient le plus fait n'irritèrent pas le pape moins que les autres,
parce qu'il veut prononcer des oracles, ne les point expliquer dans la
crainte de quelque brèche à la prétendue infaillibilité, et que, voulant
être le seul évêque et l'unique juge souverain de la foi, et regardant
les autres évêques comme ne tenant leur autorité que de lui seul, non de
Jésus-Christ immédiatement, contre le texte formel, clair et répété de
l'Évangile, et la foi de tous les siècles, et des papes, qui ne s'en
sont écartés que dans les derniers, il réputait à crime tout ce qui
n'était pas l'obéissance la plus aveugle et l'acceptation la plus
soumise de tout ce qu'il daigne prononcer de plus absurde et de plus
inintelligible, et à crime encore plus grand de chercher à l'entendre, à
l'expliquer, et à oser même lui en demander l'explication, comme dans
tous les siècles elle a été demandée aux papes dans ce qui émanait d'eux
d'obscur, qui l'ont toujours donnée, et ont toujours excité les évêques
à la leur demander, à l'exemple même de Jésus-Christ, comme tant
d'endroits clairs et exprès de l'Évangile le prouvent si manifestement.

Tant d'embarras firent donc résoudre de faire faire au roi un effort
auprès du pape pour obtenir de lui quelque explication, ou de souffrir
qu'il se tînt en France un concile national, qu'on peut juger par ce qui
vient d'être dit être la bête de Rome. Amelot, ami des jésuites, mais
homme d'honneur et de grand talent pour la négociation et les affaires,
comme il y a tant paru en ses diverses ambassades, fut donc nommé pour
aller à Rome sans caractère que de simple ministre du roi. Il
l'entretint deux ou trois fois dans son cabinet, et il partit dans les
premiers jours de décembre.

Le roi arrivé donc de Fontainebleau à Versailles, le 25 octobre, nomma
Amelot le 29. La Toussaint se trouva le jeudi, et le lendemain il alla à
Marly, jusqu'au samedi Ier décembre. Vers les commencements du voyage,
le P. Tellier qui toujours me courtisait, et qui ne se lassait point de
me parler de la constitution, quelque peu content qu'il dût être de ses
conversations avec moi là-dessus, me parla fort du concile national, et
me fit une proposition, que pour un homme d'autant d'esprit et de
connaissance en manèges et en artifices, je n'ai jamais pu comprendre.
Après force propos pour me faire goûter ce concile, que j'aurais en
effet fort approuvé, s'il eût été possible qu'on l'eût laissé pleinement
libre, il me dit qu'il était résolu de le tenir à Senlis\,; qu'il était
impossible que ce fût dans Paris, par beaucoup de raisons qu'il
m'allégua, et toutes tendantes à se rendre bien maître et tyran du
concile\,; qu'il fallait une ville pour que tout le monde pût être logé,
et près de Paris pour en tirer les lumières d'une part, c'était à dire
ses ordres, et la subsistance de l'autre\,; assez loin de Paris pour
ôter la possibilité d'y aller souvent, assez loin de la cour aussi pour
ne pas donner lieu de croire qu'elle gênât la liberté, et empêcher aussi
les prélats de la fréquenter\,; puis me regardant d'un air affable mais
vif\,: «\,Vous êtes, ajouta-t-il, gouverneur de Senlis\,; il faut que
vous soyez le commissaire du roi au concile\,; personne n'en est plus
capable que vous, et rien ne convient mieux. --- Moi, mon père, saisi
d'effroi, m'écriai-je, commissaire au concile\,! pour rien dans le monde
je ne l'accepterai, ne vous avisez pas d'y penser.\,»

La surprise du confesseur fut inexprimable, et pour un homme d'autant
d'esprit, je le répète encore, la lourdise de sa réponse inexprimable
aussi. «\, Comment, monsieur\,! me dit-il d'un ton doux qui cherchait à
me ramener, croiriez-vous la commission au-dessous de vous parce que
vous êtes duc, et que les empereurs la donnaient à leurs comtes d'Orient
ou de leur palais pour les conciles de leur temps\,?» Je me mis à rire,
et lui répondis que je n'avais jamais cru nos ducs aller à la cheville
du pied d'un comte d'Orient, même les ducs de Bourgogne. Que je les
croyais aussi fort au-dessous de l'autorité et de la puissance de ces
comtes du palais des grands empereurs\,; que j'étais donc fort éloigné
de me comparer à eux, et fort aussi de ne pas trouver la commission de
commissaire du roi au concile un emploi extrêmement honorable\,; mais
qu'il était si au-dessus de ma capacité et si entièrement contradictoire
à mon goût, que je le suppliais que la pensée qui lui était venue
n'allât pas plus loin, parce que je serais au désespoir de déplaire par
un refus, que toutefois je ne ferais pas moins.

L'étonnement redoubla dans le bon père, qui ne me répondit rien. Je
cherchai à adoucir la rudesse de mon exclamation et de ce qui l'avait
suivie, pour ne pas irriter inutilement un si dangereux homme, que je
vis clairement qui avait follement, après tout ce qu'il avait si
nettement vu dans toutes nos conversations, jeté son coussinet sur moi
pour en faire le bourreau du concile, et l'exécuteur de toutes ses
volontés portant le nom du roi\,; il ne me parla plus de moi pour cet
emploi, mais d'ailleurs toujours à son accoutumé.

Dans ces conjonctures, il arriva un événement qu'on étouffa avec tout le
soin qu'il fut possible, mais que l'artifice et l'autorité ne put
empêcher de faire grand bruit malgré toute la crainte de la puissance et
de l'autorité. Brûlart, évêque de Soissons, mourut à Paris point vieux,
au milieu d'une ferme et constante santé. Il était frère de Puysieux,
chevalier de l'ordre, dont on a parlé plus d'une fois, et de Sillery,
écuyer de feu M. le prince de Conti jusqu'à sa mort. Il fut longtemps
évêque d'Avranches, où, pétri d'orgueil et d'ambition, il était outré de
se voir, comme disait M. de Noyon, un évêque du second ordre, reculé de
tous les moyens de se faire valoir. Huet, si connu par son rare savoir,
et qui avait été sous-précepteur de Monseigneur, était évêque de
Soissons, et ne faisait cas que de ses livres. Brûlart lui proposa de
troquer d'évêché, et lui montra du retour. Huet y consentit, et l'autre
crut avoir déjà fait sa fortune de s'être si fort rapproché de Paris, de
Sillery et de l'église de Reims, dont il se flattait que sa nouvelle
qualité de premier suffragant lui faciliterait la translation. Pour y
arriver, il se donna tout entier à la cour et aux jésuites, fit main
basse sur les meilleurs livres, sacrifia le repos des communautés de son
nouveau diocèse.

La rage le surmonta quand il vit ses espérances frustrées, surtout après
avoir eu l'imprudence de s'être vanté tout haut, et publiquement compté
sur l'archevêché de Reims. Il fut assez follement vain pour en montrer
sa douleur, même à Mailly transféré d'Arles à Reims, et depuis cardinal,
et d'en faire des plaintes publiques. Le repentir suivit de près
l'impétuosité de sa douleur, et d'un dépit qui avait été plus fort que
lui\,; il en craignit les suites pour sa fortune\,; il prodigua les
bassesses, et s'attacha de plus en plus aux jésuites, et à tout ce qu'il
imagina qui pouvait plaire à la cour. C'était bouillir du lait aux bons
pères. Ils l'en méprisèrent davantage, et trouvèrent en lui ce qu'ils
aiment le mieux, un valet à tout faire par l'espoir de ce qui n'arrive
jamais, et qui jamais n'ose se fâcher, ni cesser d'être entièrement en
leur main, de peur de perdre les services passés.

Brûlart avait beaucoup d'esprit et du savoir, mais l'un et l'autre fort
désagréables par un air de hauteur, de mépris des autres, de
transcendance, de pédanterie, d'importance, de préférence de soi, de
domination, répandu dans son parler et dans toute sa personne, jusque
dans son ton et sa démarche, qui frappait et qui le rendait de ces
hommes qui ont tellement le don de déplaire et d'aliéner, que dès qu'ils
ouvrent la bouche on meurt d'envie de leur dire non. Il joignait à tout
cela l'arrogance et ce rogue des La Rochefoucauld, dont était sa mère,
et la fatuité des fils de ministres, quoique son père ne fût que le fils
d'un ministre chassé. Il se piquait encore de beau monde, de
belles-lettres, de beau langage\,: enfin, il était de l'Académie
française et de celle des inscriptions.

L'affaire de la constitution lui parut propre à lui faire faire une
grande fortune. Il s'y livra à tout, et eut la douleur de n'y être pas
des premiers. Il avait été de diverses commissions où sa chaleur et son
travail avait fort plu, lorsqu'il tomba malade. Les réflexions l'y
saisirent sur l'aveuglement de la fortune à son égard, d'où naquirent
d'autres sur son aveuglement pour elle. De là les regrets, puis les
horreurs, les remords qui se tournèrent en hurlements, en protestations
à haute voix contre la constitution, et en confession publique de
l'avoir soutenue contre sa lumière et sa conscience. Sa tremblante
famille ne sut mieux faire que de le cacher, et d'écarter les valets non
nécessaires des chambres voisines, d'où on l'entendait crier ses
repentirs, ses confessions sur la constitution, ses protestations. Ce
qui l'environnait espéra le calmer par les discours des prélats avec qui
il avait le plus travaillé dans cette affaire\,: il s'écria aux
séducteurs et n'en voulut voir aucun. On fut réduit à lui faire recevoir
les sacrements avec les plus grandes précautions d'entière solitude,
excepté quelques valets affidés, dont on ne pouvait se passer, dans la
crainte d'une amende honorable publique contre la constitution, et sur
ce qu'il avait fait pour elle. Ses plaintes, ses reproches contre
lui-même, ses cris ne cessèrent point, et il mourut ainsi, toujours en
pleine connaissance, dans les angoisses et les éclats du plus vif
repentir, et dans les frayeurs les plus terribles des jugements de Dieu.

Quelques soins que sa famille eût pu prendre pour cacher une fin si
parlante, et dont les élans avaient duré presque autant que la maladie,
trop de médecins ou gens de santé, trop de valets, trop encore de
famille et d'amis même au commencement de cette surprise, avaient été
témoins de ces choses. Ils en avaient été trop effrayés pour que de l'un
à l'autre elles ne devinssent pas bientôt publiques. On nia, on étouffa
tant qu'on put, mais en vain. Trop de gens avaient vu et entendu, et
n'avaient pu, dans leur premier émoi, se contenir de le raconter.
L'autorité fit qu'on n'osa guère en parler tout haut après les premiers
jours\,; mais le fait n'en demeura pas moins certain, constaté et
public. On mit au moins bon ordre que le roi n'en sut rien, et avec cela
tout fut gagné. Ce déplorable évêque fut la première victime de la
constitution, qui s'en immola bien d'autres, et s'en immole encore tous
les jours depuis trente ans.

Détournons nos yeux d'un spectacle si terrible, pour nous consoler par
l'heureuse fin d'un prédestiné. M. de Saint-Louis était un gentilhomme
de bonne noblesse, dont le nom était Le Loureux, qui parvint à avoir un
régiment de cavalerie. Il servit même de brigadier avec grande
distinction, honoré de l'estime, de l'amitié et de la confiance des
généraux sous lesquels il servit, particulièrement de M. de Turenne\,;
et le roi, sous les yeux duquel il servit aussi, lui a toujours marqué
de l'estime et de la bonté, et en a souvent parlé en ces termes, même
plusieurs fois depuis sa retraite. Il était des pays d'entre le Perche
et le comté d'Évreux\,; il y allait quelquefois les hivers, et cette
situation lui fit connaitre M. de la Trappe, à qui, sans l'avoir jamais
vu et sur la seule réputation de la réforme qu'il entreprenait, {[}il
alla{]} offrir ses services dans un temps où il n'était pas en sûreté à
la Trappe de la part des anciens religieux, qui jusqu'alors y avaient
étrangement vécu, et qui ne se cachaient pas de vouloir s'en défaire.

Cette action toucha M. de la Trappe\,; tout ce que Saint-Louis remarqua
en lui le charma. Il ne fit plus de voyage chez lui qu'il n'allât voir
M. de la Trappe. Il avait eu un œil crevé du bout d'une houssine en
châtiant son cheval. La fluxion gagna l'autre œil, qu'il fut en danger
de perdre, lorsque le roi conclut cette trêve de vingt ans, que la
guerre de 1688 rompit. Ces circonstances rassemblées déterminèrent
Saint-Louis à se retirer du service. Il vendit son régiment au fils aîné
de Villacerf, pour lequel on le fit Royal-Anjou, et qui fut tué à la
tête. Saint-Louis eut une assez forte pension du roi, qui témoigna le
regretter. Les réflexions lui vinrent dans son loisir. Dieu le toucha\,;
il résista. À la fin, la grâce plus forte le conduisit à la Trappe.

M. de la Trappe le mit dans le logis qu'il venait de bâtir au dehors de
l'enceinte de son monastère, pour y loger les abbés commendataires, dans
un lieu d'où ils ne pussent troubler la régularité. Saint-Louis, vif et
bouillant, qui aimait la société, qui, sans avoir jamais abusé de la
table, en aimait le plaisir, qui n'avait ni lettres, ni latin, ni
lecture, se trouva bien étonné dans les commencements d'une si grande
solitude. Il essuya de cruelles tentations contre lesquelles il eut
besoin de tout son courage, et de ce don admirable de conduite que
possédait éminemment celui qui avait bien voulu se charger de la sienne,
quoique si occupé de celle de sa maison et des ouvrages qu'il s'était vu
dans la nécessité d'entreprendre pour en défendre la régularité. Il
disait toujours à Saint-Louis de se faire une règle de vie et de
pratiques si douces qu'il voudrait, pourvu qu'il y fût fidèle. Il se la
fit, et y fut fidèle jusqu'à la mort, mais la règle qu'il se fit aurait
paru bien dure à tout autre. Il y persévéra trente et un ans dans toutes
sortes de bonnes œuvres, et y mourut saintement vers ces temps-ci, à
quatre-vingt-cinq ans, parfaitement sain de corps et d'esprit, jusqu'à
cette maladie qui l'emporta sans lui brouiller la tête.

Tout ce qui allait d'honnêtes gens et de gens distingués à la Trappe se
faisaient un plaisir de l'y voir\,; plusieurs même lièrent amitié avec
lui. Je n'ai point connu d'homme avoir le cœur plus droit, être plus
simple ni plus vrai, avoir un plus grand sens et plus juste en tout,
avec fort peu d'esprit, que réparait l'usage qu'il avait eu du monde, et
qu'il n'avait point perdu, et beaucoup de politesse. J'étais le seul de
tout le pays qu'il vînt voir quelquefois à la Ferté, et il allait
rarement chez lui et y demeurait fort peu. Il fut singulièrement aimé,
estimé, regretté à la Trappe, où il était d'un grand exemple, et de tous
ceux qui le connaissoient. Il avait été marié autrefois et n'avait point
eu d'enfants.

Le roi nomma d'Avaray, lieutenant général, pour relever dans l'ambassade
de Suisse le comte du Luc, à qui il donna celle de Vienne, et une place
vacante de conseiller d'État d'épée.

L'empereur faisait en même temps couronner reine de Hongrie, avec
beaucoup de magnificence, à Presbourg, l'impératrice sa femme, et
tâchait d'y obtenir des états qu'ils voulussent déclarer les filles
capables de succéder à leur couronne. Cela était bien loin de l'élection
même pour les mâles, dont ils avaient eu une si longue possession, et
qu'ils prétendaient encore\,; mais la maison d'Autriche s'était si
puissamment établie en autorité, qu'il n'y eut rien à quoi elle ne crût
pouvoir atteindre.

L'électeur de Cologne, arrivé depuis quelques jours à Paris, en
magnifique équipage, y avait été retenu par la goutte. Il vint le 11
novembre à Marly, sur les trois heures, fut un quart d'heure seul avec
le roi dans son cabinet, et retourna à Paris. L'électeur de Bavière,
arrivé aussi de Compiègne en sa petite maison de Saint-Cloud, vint le 15
courre le cerf avec le roi à Marly, qui le mena dans ses jardins après
la chasse. L'électeur soupa chez d'Antin, joua dans le salon avant et
après, et s'en retourna à Saint-Cloud. Le fils aîné de Saumery fut nommé
pour suivre l'électeur lorsqu'il partirait pour ses États, en qualité
d'envoyé du roi près de lui.

Pompadour et d'Alègre furent aussi nommés\,: le premier à l'ambassade
d'Espagne, où le roi était bien assuré qu'il n'irait point\,; et
d'Alègre à celle d'Angleterre, où il n'alla point non plus, mais par
d'autres raisons. Tous deux acceptèrent avec joie. Pompadour surtout
parut transporté. De sa vie il n'avait été de rien\,; on a vu en son
lieu qu'après une longue vie fort obscure, lui et sa femme avaient vendu
leur fille à Dangeau, pour s'accrocher à la cour. Par eux et par la
protection qu'ils en avaient tirée de M\textsuperscript{me} de
Maintenon, plus de mine que d'effet, ils s'étaient jetés à corps perdu à
la princesse des Ursins. C'était la leurrer d'un ambassadeur tout à
elle, et par ce choix la persuader que ses fautes sur sa souveraineté et
sur le mariage du roi d'Espagne étaient effacées, et que le roi voulait
plus que jamais qu'elle gouvernât absolument en Espagne. Pompadour et sa
femme, les Dangeau même, y voyaient les cieux ouverts, les ordres et les
dignités pleuvoir sur Pompadour, dont la grandesse sûre passerait à sa
fille et à Courcillon, et Pompadour de plus avec la confiance de la cour
et celle de M\textsuperscript{me} des Ursins devenir un personnage. Ce
pot au lait de la bonne femme les ravissait\,; déjà Pompadour faisait
l'important et Dangeau en était tout bouffi. Malheureusement cette
fortune n'eut que la perspective\,; aussi le choix ne fut-il que pour la
spéculation.

Le duc de Berwick arriva, et fut reçu du roi comme il le méritait, qui
lui donna le surlendemain une longue audience à Marly dans son cabinet.
Il demeurait toujours à Saint-Germain, et, comme on l'a remarqué
ailleurs, n'avait jamais de logement à Marly\,; mais il avait la liberté
d'y venir faire sa cour sans la demander, et tous les voyages que le roi
y faisait il y venait tous les matins. Il n'avait passé que huit jours à
Madrid. Le roi d'Espagne l'y avait régalé d'une épée de diamants qui lui
venait de Monseigneur.

Le roi taxa les régiments d'infanterie qui étaient montés à un prix
excessif. Cette vénalité de l'unique porte par laquelle on puisse
arriver aux grades supérieurs est une grande plaie dans le militaire, et
arrête bien des gens qui seraient d'excellents sujets. C'est une
gangrène qui ronge depuis longtemps tous les ordres et toutes les
parties de l'État, sous laquelle il est difficile qu'il ne succombe, et
qui n'est heureusement point ou fort peu connue dans tous les autres
pays de l'Europe.

Les Rohan, trop florissants et trop alertes pour ne pas tirer parti de
tout, firent si bien que leur prince de Montbazon, qui perdait quarante
mille livres par cette taxe sur le régiment de Picardie quand il
deviendrait maréchal de camp et qu'il le vendrait, eut une pension de
dix mille livres. Torcy eut en même temps cinquante mille écus de brevet
de retenue d'augmentation sur ses deux charges, de manière que cela lui
fit six cent cinquante mille livres sur celle de secrétaire d'État, et
deux cent mille livres sur celle de chancelier de l'ordre. Amelot eut
dix mille écus pour son voyage.

Le marquis et la marquise de Gesvres divertissaient le public par leurs
dissensions depuis quatre ans\,; elle n'avait ni père, ni mère, ni
belle-mère. Le duc de Tresmes logeait chez lui sa sœur la comtesse de
Revel, il lui avait confié sa belle-fille\,; elle se trouva tenue de si
court qu'elle s'en ennuya, et qu'elle résolut d'attaquer son mari
d'impuissance afin de faire casser son mariage. Elle n'en était venue là
qu'après bien des scènes domestiques. Sa grand'mère et ses parents
l'appuyèrent\,; les Caumartin frères de sa mère s'en brouillèrent
ouvertement avec les Gesvres, dont ils étaient intimes de tout temps, et
qui avaient fait le mariage. La cause, portée à
l'officialité\footnote{L'officialité était le tribunal de l'évêque.
  L'official, ou juge d'église, avait juridiction sur tous les
  ecclésiastiques du diocèse, et dans certains cas sur les laïques, par
  exemple pour les procès relatifs aux mariages, hérésie, simonie, etc.
  L'official ne pouvait prononcer que des peines canoniques. Quand il
  s'agissait de peines corporelles, il devait en référer au juge
  séculier. Il y avait près de chaque official un promoteur qui
  remplissait les fonctions du ministère public.}, y assembla tout Paris
aux audiences\,; les factums ne furent pas ménagés, et volèrent partout.
On juge aisément de toutes les sottises qui abondèrent dans les
plaidoyers, dans les écritures, et dans les propos qui s'en tinrent, qui
à reprises furent la conversation de la cour et de la ville. Ils furent
visités juridiquement l'un et l'autre plusieurs fois, avec la honte et
les dérisions qui sont les suites inséparables de pareilles aventures.
Les Gesvres en mouraient de douleur. Enfin la marquise de Gesvres, qui
avait beaucoup d'esprit, se lassa de cet infâme vacarme, et donna un
désistement en bonne forme de ce vilain procès au cardinal de Noailles,
moyennant un accommodement aussi bien assuré de n'avoir plus de
dépendance, de loger avec son mari dans une maison particulière, eux
deux seuls, qu'elle ne pourrait être à la campagne qu'avec lui, qu'on
lui entretiendrait chevaux, carrosses, femmes de chambre et laquais pour
sortir et aller où il lui plairait, et huit mille livres par an bien
payées à elle pour ses habits et ses menus plaisirs. De part et d'autre
ils furent fort aises, avec un peu de sens ils l'auraient été plus tôt
et n'auraient point donné la farce au monde.

Le mercredi 28 novembre j'avais été une heure dans l'après-dînée avec le
duc d'Orléans, qui se portait fort bien à son ordinaire\,;
M\textsuperscript{me} la duchesse d'Orléans, qui avait eu quelques
légers accès de fièvre, était à Versailles\,; j'allai de là trouver le
roi qui était dans ses jardins. Après avoir été quelque temps à sa
promenade, le froid m'en chassa vers la fin du jour, et je vins me
chauffer dans le petit salon qui séparait son appartement de celui de
M\textsuperscript{me} de Maintenon, en attendant que le roi vint chez
lui changer d'habit et passer chez elle. Au bout d'un demi-quart d'heure
que je fus là tout seul, j'entendis crier M. Fagon, M. Maréchal, et
d'autres noms de cette sorte, qu'on supposait dans le cabinet du roi,
attendant qu'il rentrât. À l'instant les cris redoublèrent, des garçons
bleus coururent en même temps à travers ce salon. Je leur demandai ce
que c'était. Ils me dirent que M. le duc d'Orléans se trouvait
extrêmement mal. J'y courus aussitôt. Je le trouvai traîné plutôt
qu'appuyé sur deux de ses gens, tout déboutonné, sans cravate, qui le
promenaient le long de son appartement, toutes les fenêtres ouvertes. Il
était plus rouge encore qu'à l'ordinaire, mais rien de tourné dans le
visage, les yeux un peu fixes et étonnés, la parole libre sans
changement. Il me dit d'abord que cela lui avait pris tout à coup par un
étourdissement\,; qu'il croyait que ce ne serait rien. Peu après Fagon
vint, Maréchal, etc., qui le laissèrent encore promener, lui firent
prendre quelques essences, et lui conseillèrent après de se mettre au
lit, mais d'éviter d'y dormir. Ils voulaient le saigner, mais il y
répugna\,; ils s'y rendirent pour quelques heures. Je restai seul auprès
de lui. Il me dit que, dans l'incertitude de ce que ce pouvait être, et
ayant la tête libre, et ne sentant d'engagement nulle part, il voulait
se tâter, s'écouter, et se sentir avant de se déterminer à la saignée,
parce qu'il y a des poisons où elle est mortelle sans retour.

Dès que le roi fut rentré chez lui, il envoya Maréchal savoir de ses
nouvelles, et lui dire que, comme il savait par Fagon que ce ne serait
rien, et qu'il avait peine à monter, il ne viendrait point le voir. J'y
demeurai toujours jusqu'à plus de minuit presque toujours seul. Il y
vint très peu de monde, la plupart ensemble par pelotons qui ne firent
qu'entrer et sortir. M\textsuperscript{me} la duchesse de Berry et
Madame étaient allées à Versailles voir M\textsuperscript{me} la
duchesse d'Orléans, à qui j'écrivis deux fois dans la soirée. La saignée
se fit tard. Maréchal y vint quatre ou cinq fois jusqu'au coucher du
roi, qui me conta deux jours après qu'à chaque fois le roi lui demandait
qui il avait trouvé avec M. le duc d'Orléans, qu'il me nommait toujours,
et qu'une des dernières que cela arriva, le roi, qui n'avait rien
répondu aux précédentes, lui dit\,: «\,Il est fort des amis de mon
neveu, M. de Saint-Simon\,; je voudrais bien qu'il n'en eût jamais eu
d'autres, car il est fort honnête homme, et ne lui donne que de bons
conseils. Je ne suis point en peine de ceux-là, je voudrais qu'il n'en
suivît pas d'autres.\,»

Ce récit ne laissa pas de me soulager. J'avouerai sans orgueil, mais
avec droiture, que je ne pouvais pas être en peine de ma réputation\,;
mais M. le duc d'Orléans était si cruellement persécuté auprès du roi
par ce qu'il avait de plus intime\,; on m'avait tant fait pleuvoir
d'avis et de menaces sur mon commerce étroit avec lui, que, sans
craindre sur ma réputation du coté du roi non plus que d'aucun autre,
j'avais tout lieu de juger que cette liaison si intime lui déplaisait et
lui était fort désagréable, et je me sentis fort à mon aise de ne
pouvoir douter que cela n'était pas. Cette réponse du roi à Maréchal me
mit au net avec une nouvelle et très claire évidence d'où me venait tant
d'avis redoublés sans cesse, et tant de menaces sur ma façon d'être avec
M. le duc d'Orléans, et les raisons pressantes qu'on avait de m'écarter
de lui, que j'ai expliquées plus d'une fois.

Je cherchai d'où le roi avait pu prendre un sentiment si flatteur, j'ose
dire si vrai, en même temps si opposé à ce qu'on ne cessait de chercher
à me persuader. Il était plus que manifeste que le ne le devais pas à
M\textsuperscript{me} de Maintenon, à M. du Maine, à l'intérieur de leur
dépendance, à aucun des ministres. Peut-être à Maréchal\,; mais il me
l'aurait dit dans le temps et à quelle occasion, et cela ne parut pas à
la réponse que le roi lui fit sans qu'il l'eût attirée\,; peut-être à M.
de Beauvilliers\,; ce qui m'a paru de plus vraisemblable, c'est en gros
de n'avoir jamais été soupçonné d'aucune des choses si graves qui
avaient été si fort jetées sur M. le duc d'Orléans, non pas même la plus
légère idée parmi tant d'ennemis et d'envieux si peu ménagés de ma
part\,; et ma séparation entière et constante dans tous les temps de
tout ce qui était non seulement maîtresses, débauches, soupers, mais de
tous les amis de plaisir et de Paris de M. le duc d'Orléans\,; en
particulier de ce que le roi à la fin avait su que c'était moi qui avais
séparé M. le duc d`Orléans de M\textsuperscript{me} d'Argenton, qui
l'avais raccommodé avec M\textsuperscript{me} la duchesse d'Orléans, qui
entretenais leur union et en étais le lien continuel\,; et peut-être
M\textsuperscript{me} la duchesse d'Orléans elle-même, qui se trouvait
très heureuse que je fusse continuellement avec M. le duc d'Orléans,
avait eu occasion de dire quelque chose au roi là-dessus. Elle ne me l'a
toutefois jamais dit ni laissé entendre.

Maréchal m'ajouta que, ayant pris occasion ce même soir au petit
coucher, lorsque les courtisans qui ont ces entrées furent sortis, de
reparler encore de M. le duc d'Orléans de chez qui il descendait de
nouveau, pour faire parler le roi sur ce prince, qui lui avait paru fort
sec à tous les comptes qu'il lui en avait rendus toute cette
demi-journée, il se mit à le louer sur son esprit, sur ses diverses
sciences, sur les arts qu'il possédait, et à dire plaisamment que, s'il
était un homme à avoir besoin de gagner sa vie, il aurait cinq ou six
moyens différents de la gagner grassement. Le roi le laissa causer un
peu, puis, après avoir souri de cette idée par laquelle Maréchal avait
comme terminé son discours, il reprit un air sérieux, regarda
Maréchal\,: «\, Savez-vous, lui dit-il, ce qu'est mon neveu\,? il a tout
ce que vous venez de dire\,: c'est un fanfaron de crimes.\,» À ce récit
de Maréchal je fus dans le dernier étonnement d'un si grand coup de
pinceau\,; c'était peindre en effet M. le duc d'Orléans d'un seul trait,
et dans la ressemblance la plus juste et la plus parfaite. Il faut que
j'avoue que je n'aurais jamais cru le roi un si grand maître. M. le duc
d'Orléans se trouva si bien qu'il fut le lendemain au lever du roi, et
de là à Versailles où il demeura. Il n'y avait plus que deux ou trois
jours de Marly. Il fit quelques légers remèdes et il n'y parut plus.

\hypertarget{chapitre-xv.}{%
\chapter{CHAPITRE XV.}\label{chapitre-xv.}}

1714

~

{\textsc{Le roi de Suède arrivé de Turquie à Stralsund.}} {\textsc{-
Croissy ambassadeur vers lui.}} {\textsc{- Entrevue des deux reines
d'Espagne.}} {\textsc{- Maison de la régnante.}} {\textsc{- Duc de
Saint-Aignan l'y joint et l'accompagne à Madrid.}} {\textsc{- Mort
d'Alex. Sobieski à Rome.}} {\textsc{- Van Holl, riche financier\,; ce
que devient son fils.}} {\textsc{- Mort de la comtesse de Brionne.}}
{\textsc{- Mort de Jarnac\,; son caractère.}} {\textsc{- Mort,
extraction, famille, fortune, caractère du cardinal d'Estrées.}}
{\textsc{- Bon mot de l'abbé de la Victoire.}} {\textsc{-
Distractions.}} {\textsc{- Cardinal d'Estrées se démettant de l'évêché
de Laon, cardinal depuis dix ans, obtient le premier un brevet de
continuation du rang et des honneurs de duc et pair.}} {\textsc{- Trait
de l'évêque-comte de Noyon au festin de la réception au parlement de
l'évêque-duc de Laon chez le cardinal d'Estrées.}} {\textsc{- Trait du
cardinal d'Estrées pour se délivrer de ses gens d'affaires.}} {\textsc{-
Bon mot du cardinal d'Estrées.}} {\textsc{- Projet constant et suivi des
jésuites d'établir l'inquisition en France.}} {\textsc{- Mariage du fils
de Goesbriant avec la fille du marquis de Châtillon.}} {\textsc{- Prince
électoral de Saxe au lever du roi.}} {\textsc{- Bergheyck prend congé
pour sa retraite.}} {\textsc{- Électeur de Bavière voit le roi en
particulier.}} {\textsc{- Albergotti de retour d'Italie.}} {\textsc{-
Divers envoyés nommés.}} {\textsc{- Bissy abbé de Saint-Germain des
prés.}} {\textsc{- Rohan et Melun reçus ducs et pairs, Melun avec
dispense et condition.}} {\textsc{- Folies de Sceaux.}} {\textsc{-
Inquiétude du duc du Maine\,; mot plaisant qui lui échappe là-dessus.}}
{\textsc{- Noir dessein du duc du Maine.}} {\textsc{- Digression
nécessaire en raccourci sur la dignité de pair de France, et sur le
parlement de Paris et autres parlements.}}

~

Le roi de Suède arriva enfin, lui troisième, le 22 novembre, à
Stralsund. Je m'abstiendrai d'en dire davantage sur un prince qui a fait
tant et un si singulier bruit dans le monde, et sur lequel tant de
plumes ont travaillé. Croissy, frère de Torcy, fut aussitôt nommé
ambassadeur vers lui, et partit bientôt après.

La reine d'Espagne\,? en arrivant à Pau, trouva à quelque distance la
reine d'Espagne douairière, sa tante, qui venait à sa rencontre. Elle
était arrivée de Bayonne exprès pour la voir. À l'approche de leurs
carrosses, elles mirent toutes deux pied à terre en même temps, et après
les salutations elles montèrent toutes deux seules dans une belle
calèche que la reine douairière avait amenée à vide, et dont elle fit un
présent à la reine sa nièce. Elles soupèrent seules ensemble. La reine
douairière la conduisit jusqu'à Saint-Jean Pied-de-Port (car en ce
pays-là comme en Espagne les passages des montagnes, à leur entrée,
s'appellent des ports). Elles s'y séparèrent, la reine douairière lui
fit beaucoup de présents, entre autres d'une garniture de diamants. Le
duc de Saint-Aignan joignit la reine d'Espagne à Pau, et l'accompagna,
par ordre du roi, jusqu'à Madrid. Elle envoya Grillo, noble génois,
qu'elle fit depuis faire grand d'Espagne, remercier le roi de l'envoi du
duc de Saint-Aignan, et du présent qu'il lui avait apporté. Le roi
d'Espagne avait nommé sa maison\,: le marquis de Santa-Cruz majordome
major, il l'a été jusqu'à sa mort, je l'ai fort connu en Espagne, et
j'aurai occasion d'en parler\,; le marquis de Castanaga grand écuyer\,;
la princesse des Ursins camarera-mayor, qui choisit toute cette
maison\,; la duchesse d'Havré, les princesses de Masseran, Santo-Buono,
Robecque et Lanti, dames du palais, dont la première et la dernière
étaient fille et belle-fille de la feue duchesse de Lanti, sœur de la
princesse des Ursins. On en ajouta d'autres dans la suite.

Alex. Sobieski, chevalier du Saint-Esprit, second fils du roi Jean
Sobieski, roi de Pologne, mourut à Rome, sans avoir été marié. Il avait
mené une vie assez obscure et assez errante, par des prétentions
d'aucune desquelles il n'avait pu jouir nulle part. Le pape crut
apparemment l'en dédommager par de magnifiques obsèques qu'il voulut
voir passer sous les fenêtres de son palais.

Van Holl, riche financier, trésorier général de la marine, puis grand
audiencier\footnote{Officier de la grande chancellerie. Voy. sur les
  grands audienciers le règlement fait par Louis XIV pour la tenue du
  sceau, t. X, p. 451.}, qui donnait grand jeu et grande chère à Paris,
et à sa belle maison d'Issy, à beaucoup de gens de la cour, et que le
prince et le cardinal de Rohan voyaient et aimaient fort, {[}ainsi
que{]} le maréchal de Villeroy et quantité d'autres, dérangea si fort
ses affaires, qu'il fit une entière banqueroute qu'il jugea à propos de
ne pas voir. On dit qu'on l'avait trouvé mort dans son lit à Issy, et on
se hâta d'enterrer ou lui ou une bûche. On prétendit qu'il avait fait sa
main pour aller vivre inconnu quelque part. Il était Hollandais. Son
fils, protégé par les Rohan et par quelques autres, n'osa se montrer
d'abord\,; peu à peu il parut, fut maître des requêtes, et a passé par
diverses intendances. Il n'est pas sans esprit ni sans talents. De Van
Holl il s'est fait M. de Vanolles, le de est plus noble et le nom plus
français.

La comtesse de Brionne, riche héritière de la maison d'Épinay en
Bretagne, mourut en ce même temps, une des plus malheureuses femmes qui
aient vécu, sans l'avoir mérité. Elle laissa une fille morte plusieurs
années depuis sans avoir été mariée, et le prince de Lambesc.

Jarnac mourut en même temps, à Paris, de la petite vérole. Il s'était
distingué à la guerre et avait beaucoup d'esprit et orné, qui lui avait
fait beaucoup d'amis. Il ne laissa point d'enfants de l'héritière de
Jarnac-Chabot qu'il avait épousée. De lui, il n'avait rien\,; c'était un
dernier cadet de Montendre La Rochefoucauld. Il savait et voulait faire,
et avec une figure de paysan malgré sa naissance il eût été loin. Ce fut
dommage, il fut fort regretté.

Le cardinal d'Estrées mourut à Paris, dans son abbaye de Saint-Germain
des Prés, à quatre-vingt-sept ans presque accomplis, ayant toujours joui
d'une santé parfaite de corps et d'esprit, jusqu'à cette maladie qui fut
fort courte, et qui lui laissa sa tête entière jusqu'à la fin. Il est
juste et curieux de s'arrêter un peu sur un personnage toute sa vie
considérable, et qui à sa mort était cardinal, évêque d'Albano, abbé de
Longpont, du Mont-Saint-Éloi, de Saint-Nicolas aux Bois, de la Staffarde
en Piémont, où Catinat gagna une célèbre bataille avant d'être maréchal
de France, de Saint-Claude en Franche-Comté, dont l'abbé d'Estrées son
neveu était coadjuteur, et dont on a fait un évêché depuis quelques
années \footnote{L'évêché de Saint-Claude fut érigé le 22 janvier 1742.},
d'Anchin en Flandre, et de Saint-Germain des Prés dans Paris. Il était
aussi commandeur de l'ordre, de la promotion de 1688.

Le mérite aidé des hasards de la fortune, l'un et l'autre aux quatre
dernières générations, ont fait, de gentilshommes obscurs et assez
nouveaux du pays de Boulonais, une race infiniment et très
singulièrement illustrée, dont il ne reste plus que
M\textsuperscript{lle} de Tourbes, sœur du dernier maréchal d'Estrées.
Le cardinal leur oncle ne s'en faisait point accroire là-dessus, et
disait fort naturellement qu'il connaissoit ses pères, jusqu'à un qui
avait été page de la reine Anne, duchesse de Bretagne, mais que par delà
il n'en savait rien, et qu'il ne fallait pas chercher. Or ce page, qui
ne fit pas grande fortune, et qui épousa une La Cauchie, était le
grand-père du sien, dont le père était fils d'un bâtard de
Vendôme-Bourbon, et sa femme était Babou, fille de La Bourdaisière et
d'une Robertet, gens de beaucoup de faveur. Cette Babou avait six sœurs.
Elles étaient belles, mariées, intrigantes\,; on les appelait de leur
temps les Sept péchés mortels. Voilà ce qui commença à apparenter et à
mettre dans le monde le grand-père du cardinal d'Estrées. La Babou sa
grand'mère était aussi déterminée qu'intrigante. Il est remarquable
qu'elle fut tuée à Issoire, où elle s'était jetée et qu'elle défendait,
le dernier de l'année 1593, contre les ligueurs.

Elle laissa deux fils et six filles, dont trois figurèrent. Le fils aîné
fut tué, un an après sa mère, au siège de Laon, l'autre est le premier
maréchal d'Estrées, père du cardinal. Des filles, l'aînée fut seconde
femme du maréchal de Balagny, bâtard du célèbre évêque de Valence, frère
du maréchal de Montluc. Le maréchal de Balagny s'était fait, par les
armes et par adresse, souverain de Cambrai. Il n'y put résister
longtemps aux Espagnols, sur qui il avait usurpé le pays et la place. Sa
première femme, sœur du fameux Bussy d'Amboise, et qui n'avait pas moins
de courage que lui, mourut de rage et de dépit, peu de moments après
être sortie de Cambrai, en 1595. Balagny mourut en 1603, et sa seconde
femme deux ans après. Gabrielle d'Estrées fut la seconde, dont la beauté
fit la fortune de son père, et dont l'histoire est trop connue pour s'y
arrêter. Elle était sœur du père du cardinal, mais morte près de trente
ans avant sa naissance. La troisième, qui figura, épousa le premier duc
de Villars, à la fortune duquel elle contribua beaucoup. Pour revenir à
leur père, Gabrielle, dès lors pleinement et publiquement maîtresse
d'Henri IV, le fit gouverneur de Paris et de l'Ile-de-France après d'O,
et grand maître de l'artillerie après M. de Saint-Luc. Il en avait déjà
fait les fonctions fort longtemps auparavant pendant une longue maladie
de La Bourdaisière, son beau-père, qui l'était. M. d'Estrées avait été
chambellan du duc d'Alençon, gouverneur de ses apanages en partie, fort
bien avec lui\,; et ce prince, qui par mépris pour Henri III son frère
porta toujours l'ordre de Saint-Michel, sans avoir jamais voulu de celui
du Saint-Esprit, l'avait fait donner à d'Estrées, en la première
promotion de 1579\,; il se démit de l'artillerie en 1599, qui fut donnée
à M. de Rosny, depuis premier duc de Sully, lors en pleine faveur,
lequel obtint pour vin du marché de faire passer le gouvernement de
l'Ile-de-France du père au fils, qui est demeuré chez MM. d'Estrées,
jusqu'à la quatrième et dernière génération. L'artillerie alors n'était
qu'une charge. Elle ne devint office de la couronne qu'entre les mains
de M. de Sully, qui le fit ériger en 1601. C'est le dernier de tous, n'y
en ayant point eu d'érigé depuis.

La mère du cardinal d'Estrées était nièce de ce premier et célèbre duc
de Sully, fille du comte de Béthune son frère, si connu par sa capacité
et par ses grandes ambassades à Rome et ailleurs, et par ce grand nombre
de manuscrits qu'il ramassa, que son fils augmenta, et qu'il donna au
roi. Ainsi elle était sœur de ce second comte de Béthune, chevalier
d'honneur de la reine, qui fut connu aussi par ses ambassades, et du
comte de Charost, qui fut capitaine des gardes du corps, puis duc à
brevet, grand-père du duc de Charost, gouverneur de la personne du roi.
Ces choses ont maintenant vieilli\,; il est bon d'en rafraîchir la
mémoire, mais sans s'y étendre davantage.

Le père du cardinal d'Estrées fut un personnage toute sa vie par ses
grands emplois, son mérite, sa capacité, et l'autorité qu'il conserva
toute sa vie. Il fut maréchal de France en 1626, et il est unique que
lui, son fils et son petit-fils ont été non seulement maréchaux de
France, et le dernier du vivant de son père, mais tous trois doyens des
maréchaux de France, et longtemps. Le premier maréchal avait
quatre-vingt-douze ans, lorsqu'en 1663 il fut fait duc et pair dans
cette cruelle fournée des quatorze \footnote{Voy. t. Ier, p.~449.}, et
qu'il en prêta le serment. Il en avait quatre-vingt-dix-huit en 1670
lorsqu'il mourut. Il eut trois fils de ce premier mariage\,: le duc
d'Estrées mort en janvier 1687 à Rome, où il était ambassadeur depuis
quatorze ou quinze ans\,; le second maréchal d'Estrées\,; et le cardinal
d'Estrées. Ce second duc d'Estrées fut père du troisième, mort avant
cinquante ans, de la pierre, à Paris en 1698, et de l'évêque-duc de
Laon, mort en 1694. Le troisième duc d'Estrées fut père du dernier, mort
sans postérité en juillet 1723, à quarante ans passés\,; et le second
maréchal d'Estrées fut père du troisième, qu'il vit grand d'Espagne et
maréchal de France, et qui recueillit la dignité de duc et pair\,; et de
l'abbé d'Estrées, commandeur de l'ordre, mort nommé archevêque de
Cambrai, dont il attendait les bulles, et qui avait eu plusieurs
ambassades, ainsi que ses deux oncles et son grand-père. On voit par ce
court abrégé cinq ducs et pairs laïques, deux ducs-pairs
ecclésiastiques, un cardinal, un grand d'Espagne, trois doyens des
maréchaux de France, deux commandeurs et cinq chevaliers du
Saint-Esprit, trois ambassadeurs, un ministre d'État et deux
vice-amiraux, outre les gouverneurs des provinces\,; et voilà comme les
beautés élèvent des familles qui savent en profiter\,!
M\textsuperscript{me} de Soubise et la belle Gabrielle en sont des
exemples pour la postérité. Venons maintenant au cardinal d'Estrées.

Né en 1627, il avait vécu quarante ans avec son père, et sut profiter de
ses leçons et de sa considération. La liaison la plus intime fut toute
sa vie constante entre ses neveux, et petits-neveux de Vendôme, et lui
dont il fut le conseil toute sa vie, et le cardinal y participa dès sa
jeunesse. C'était l'homme du monde le mieux et le plus noblement fait de
corps et d'âme, d'esprit et de visage, qu'on voyait avoir été beau en
jeunesse, et qui était vénérable en vieillesse, l'air prévenant mais
majestueux, de grande taille, des cheveux presque blancs, une
physionomie qui montrait beaucoup d'esprit, et qui tenait parole, un
esprit supérieur et un bel esprit, une érudition rare, vaste, profonde,
exacte, nette, précise, beaucoup de vraie et de sage théologie,
attachement constant aux libertés de l'Église gallicane et aux maximes
du royaume, une éloquence naturelle, beaucoup de grâce et de facilité à
s'énoncer, nulle envie d'en abuser, ni de montrer de l'esprit et du
savoir, extrêmement noble, désintéressé, magnifique, libéral, beaucoup
d'honneur et de probité, grande sagacité, grande pénétration, bon et
juste discernement, souvent trop de feu en traitant les affaires. Il
avait été galant dans sa jeunesse, et il l'était demeuré sans blesser
aucunes bienséances. Parmi un courant d'affaires, la plupart de sa vie
continuelles, réglé en tout, aumônier, et très homme de bien. C'était
l'homme du monde de la meilleure compagnie, la plus instructive, la plus
agréable, et dont la mémoire toujours présente n'avait jamais rien
oublié ni confondu de tout ce qu'il avait su, vu et lu\,; toujours gai,
égal, et sans la moindre humeur, mais souvent singulièrement distrait\,;
qui aimait à faire essentiellement plaisir, à servir, à obliger, qui s'y
présentait aisément, et qui ne s'en prévalait jamais\,; il savait haïr
aussi et le faire sentir\,; mais il savait encore mieux aimer. C'était
un homme très-généreux\,; il était aussi fort courtisan et fort attentif
aux ministres et à la faveur, mais avec dignité, un désinvolte qui lui
était naturel, et incapable de rien de ce qu'il ne croyait pas devoir
faire. Jamais les jésuites ne purent l'entamer sur rien, ni le roi sur
eux, ni sur ce qu'on lui faisait passer pour jansénisme, ni en dernier
lieu, comme on l'a vu sur la constitution, ni l'empêcher d'agir, et même
de parler sur toutes ces matières avec la plus grande liberté, sans que
sa considération en ait baissé auprès du roi.

Tant de grandes et d'aimables qualités le firent généralement aimer et
respecter\,; sa science, son esprit, sa fermeté, sa liberté, le perçant
de ses expressions quand il lui plaisait, une plaisanterie fine et
quelquefois poignante, un tour charmant, le faisaient craindre et
ménager, et cela jusqu'à sa mort, par ceux qui étaient devenus la
terreur de tout le monde\,; avec beaucoup de politesse mais distinguée,
il savait se sentir, il était quelquefois haut, quelquefois colère. Ce
n'était pas un homme qu'il fit bon tâtonner sur rien. Ce tout ensemble
faisait un homme extrêmement aimable et sûr, et lui donna toujours un
grand nombre d'amis.

Il fut évêque-duc de Laon à vingt-cinq ans, sacré à vingt-sept, et
brilla fort cinq ans après en l'assemblée du clergé de 1660. Il eut la
principale part à finir l'affaire fameuse des quatre évêques par ce
qu'on a nommé la paix de Clément IX. Entré par son père dans l'intimité
de la maison de Vendôme, il traita et conclut en 1665 le mariage de
M\textsuperscript{lle} de Nemours avec le duc de Savoie, et en 1666
celui de sa sœur cadette avec Alphonse, roi de Portugal. L'une a été
mère du premier roi de Sardaigne, si connue sous le nom de Madame Royale
qu'elle usurpa au mariage de son fils\,; l'autre, illustre par sa
courageuse résolution, où le cardinal d'Estrées eut grande part, de
changer de mari, et de demeurer reine régnante. Toutes deux étaient
filles du pénultième duc de Nemours, tué en duel par le duc de Beaufort
son beau-frère, et de la fille de César duc de Vendôme, bâtard d'Henri
IV et de la belle Gabrielle, soeur du père du cardinal d'Estrées. Il en
eut la nomination de Portugal avec l'agrément du roi, et les malins
l'accusèrent d'avoir fait dans la vue du chapeau le mariage de son neveu
avec la fille du célèbre Lyonne, ministre et secrétaire d'État des
affaires étrangères, sur quoi il courut d'assez plaisantes chansons dont
il se divertit le premier.

Ce chapeau traîna et l'inquiétait. L'abbé de La Victoire, qui avait
beaucoup d'esprit et qui était fort du grand monde, était fort de ses
amis, et la mode alors était de faire force visites. Un soir qu'il
arriva fort tard pour souper dans une maison où il était attendu avec
bonne compagnie, on lui demanda avec impatience d'où il venait, et qui
pouvait l'avoir tant retardé\,: «\,Hélas\,! répondit l'abbé d'un ton
pitoyable, d'où je viens\,? j'ai tout aujourd'hui accompagné le corps du
pauvre M. de Laon. --- Comment M. de Laon\,! s'écria tout le monde, M.
de Laon est mort\,! il se portait bien hier, cela est pitoyable\,;
dites-nous donc\,: qu'est il arrivé\,? --- Il est arrivé, reprit l'abbé
toujours sur le même ton, qu'il m'est venu prendre pour faire des
visites, que son corps a toujours été avec moi, et son esprit à Rome,
que je ne fais que le quitter et fort ennuyé.\,» À ce récit la douleur
se changea en risée.

On a vu en son lieu ce grand dîner pour le prince de Toscane à
Fontainebleau, qui fut le seul qu'il oublia de prier, pour qui seul la
fête était faite. Il avait de ces distractions dans le commerce, qui
n'étaient que plaisantes parce qu'elles ne portaient jamais sur les
affaires, ni sur rien de sérieux.

Il fut cardinal de Clément X en 1671 mais \emph{in petto}, déclaré enfin
l'année suivante\,; protecteur des affaires de Portugal, et se trouva en
1676 au conclave où Innocent XI fut élu\,; six mois après il fut à
Munich pour le mariage de Monseigneur. Il se démit en 1681 en faveur de
son neveu, fils du duc d'Estrées, de son évêché\,; et tout cardinal
qu'il était depuis dix ans, il demanda et obtint un brevet de
conservation du rang et honneurs de duc et pair. C'en est le premier
exemple, et si je l'ai fixé à la même grâce accordée à d'Aubigny
transféré de Noyon à Rouen, c'est que je n'ai pas compté celle-ci faite
à un cardinal, et qui n'a jamais eu d'autre évêché qu'un des six
attachés aux six premiers cardinaux, qu'il opta pour son titre quand il
en eut l'ancienneté.

Ce fut au festin qu'il donna le jour de la réception de son neveu au
parlement, où étaient M. le Prince, M. le Duc, depuis connu le dernier
sous le nom de M. le Prince, et M. le prince de Conti l'aîné, avec
beaucoup de pairs, que lorsqu'on vint se mettre à table, M. de Noyon
avisa la sottise des valets de la maison, dont le cardinal fut après
bien en colère contre eux, qui avaient mis trois cadenas pour les trois
princes du sang. Il alla les ôter tous trois l'un après l'autre, puis
les regardant tous trois et se mettant à rire\,: «\,Messieurs, leur
dit-il, c'est qu'il est plus court d'en ôter trois que d'en faire
apporter une vingtaine.\,» Ils en rirent aussi comme ils purent parce
que le droit très reconnu y est, et qu'il n'y avait pas moyen de s'en
fâcher. J'en ai ouï faire le conte à plusieurs des convives, et à M. de
Noyon même, qui ne le faisait jamais sans un nouveau plaisir.

Le cardinal d'Estrées retourna à Rome pour l'affaire de la régale et
pour divers points des libertés de l'Église gallicane qu'il sut très
bien soutenir. On disait pourtant qu'on les entendait crier et se
quereller des pièces voisines, lui et don Livio Odescalchi, et qu'ils
traitaient les affaires à coups de poing. Il fut à Rome plusieurs années
chargé des affaires de France, conjointement avec le duc son frère, qui
y demeura quatorze ans ambassadeur, logeant et mangeant ensemble dans la
plus grande union. Le duc y mourut en 1687, et le cardinal demeura seul
avec tout le poids à porter. Il eut après à soutenir tout celui de
l'étrange ambassade du marquis de Lavardin, et toutes les fureurs de ce
même pape, peu habile, très entêté et tout dévoué aux ennemis de la
France, dont il se démêla avec grande capacité et dignité, conservant
une grande considération personnelle dans une cour où on se piquait
alors de manquer au roi en tout. Il vit enfin mourir cet inepte pape à
qui l'empereur Léopold dut tant, et l'Angleterre, et le prince d'Orange
sa révolution et sa couronne, dont il n'a pas tenu aux Romains de faire
un saint. Après l'élection d'Alexandre VIII, Ottobon, que la France fit,
et qui se moqua d'elle, le cardinal d'Estrées revint à la cour après
1689. Il n'y fut pas deux ans qu'il retourna au conclave où Innocent
XII, Pignatelli, fut élu en 1691. Il demeura deux ans à Rome, chargé des
affaires conjointement avec le cardinal de Janson, à terminer les
affaires du clergé. Il revint après en France jusqu'en 1700, qu'il
retourna au conclave de Clément XI, Albane, d'où il alla à Venise et à
Madrid. On a vu en son temps ce qu'il fit en ces deux villes, et son
dernier retour en France en 1703.

Devenu abbé de Saint-Germain des Prés, il vécut avec ses religieux comme
un père, et tous les soirs il avait deux, trois ou quatre moines savants
qui venaient l'entretenir de leurs ouvrages jusqu'à son coucher, qui
avouaient qu'ils apprenaient beaucoup de lui.

Il ne pouvait ouïr parler de ses affaires domestiques. Pressé et
tourmenté par son intendant et son maître d'hôtel de voir enfin ses
comptes qu'il n'avait point vus depuis un très grand nombre d'années, il
leur donna un jour. Ils exigèrent qu'il fermerait sa porte pour n'être
pas interrompus\,; il y consentit avec peine, puis se ravisa, et leur
dit que, pour le cardinal Bonzi au moins, qui était à Paris, son ami et
son confrère, il ne pouvait s'empêcher de le voir, mais que ce serait
merveille si ce seul homme, qu'il ne pouvait refuser venait précisément
ce jour-là. Tout de suite il envoya un domestique affidé au cardinal
Bonzi le prier avec instance de venir chez lui un tel jour entre trois
et quatre heures, qu'il le conjurait de n'y pas manquer, et qu'il lui en
dirait la raison\,; mais, sur toutes choses, qu'il parût venir de
lui-même. Il fit monter son suisse dès le matin du jour donné, à qui il
défendit de laisser entrer qui que ce fût de toute l'après-dînée,
excepté le seul cardinal Bonzi, qui sûrement ne viendrait pas\,; mais,
s'il s'en avisait, de ne le pas renvoyer. Ses gens, ravis d'avoir à le
tenir toute la journée sur ses affaires sans y être interrompus,
arrivent sur les trois heures\,; le cardinal laisse sa famille et le peu
de gens qui ce jour-là avaient dîné chez lui, et passe dans un cabinet
où ses gens d'affaires étalèrent leurs papiers. Il leur disait mille
choses ineptes sur la dépense où il n'entendait rien, et regardait sans
cesse vers la fenêtre, sans en faire semblant, soupirant en secret après
une prompte délivrance. Un peu avant quatre heures, arrive un carrosse
dans la cour\,; ses gens d'affaires se fâchent contre le suisse, et
crient qu'il n'y aura donc pas moyen de travailler. Le cardinal ravi
s'excuse sur les ordres qu'il a donnés. «\,Vous verrez, ajouta-t-il, que
ce sera ce cardinal Bonzi, le seul homme que j'aie excepté et qui tout
juste s'avise de venir aujourd'hui.\,» Tout aussitôt on le lui
annonce\,; lui à hausser les épaules, mais à faire ôter les papiers et
la table, et les gens d'affaires à s'en aller en pestant. Dès qu'il fut
seul avec Bonzi, il lui conta pourquoi il lui avait demandé cette
visite, et à en bien rire tous deux. Oncques depuis ses gens d'affaires
ne l'y rattrapèrent, et de sa vie n'en voulut ouïr parler.

Il fallait bien qu'ils fussent honnêtes gens et entendus. Sa table était
tous les jours magnifique, et remplie à Paris et à la cour de la
meilleure compagnie. Ses équipages l'étaient aussi, il avait un nombreux
domestique, beaucoup de gentilshommes, d'aumôniers, de secrétaires. Il
donnait beaucoup aux pauvres, à pleines mains à son frère le maréchal et
à ses enfants qui lors n'étaient pas à leur aise\,; et il mourut sans
devoir un seul écu à qui que ce fût.

Sa mort, à laquelle il se préparait depuis longtemps, fut ferme, mais
édifiante et fort chrétienne\,; la maladie fut courte, et il n'en avait
jamais eu, la tête entière jusqu'à la fin. Il fut universellement
regretté, tendrement de sa famille, de ses amis dont il avait beaucoup,
des pauvres, de son domestique, et de ses religieux qui sentirent tout
ce qu'ils perdaient en lui, et qui trouvèrent bientôt après qu'ils
avaient changé un père pour un loup et pour un tyran. L'abbé d'Estrées
devint abbé de Saint-Claude dont il était coadjuteur.

Avec toute sa franchise sur sa naissance, les mésalliances lui
déplaisaient. La maréchale d'Estrées sa belle-sœur, fille de Morin le
juif, qui avait tant d'esprit et de monde, en remboursait souvent des
plaisanteries, qui, sans rien de grossier, la démontaient au moment le
plus inattendu. Il disait de l'abbé d'Estrées qu'il était sorti de
Portugal sans y être entré\,: c'est qu'il y avait été ambassadeur et
n'avait point fait d'entrée. Il se divertissait volontiers à le désoler.

Il se moquait du vieux duc de Charost, son cousin germain, qui, depuis
qu'il fut pair, se plaisait à aller juger au parlement, et y menait le
duc d'Estrées. «\,Mon cousin, disait le cardinal à Charost, cela sent
son Lescalopier.\,» On a vu ailleurs ce qui fit Lescalopier président à
mortier, et le mariage de sa fille héritière avec le père de Charost.

Sur M\textsuperscript{me} des Ursins, le cardinal était excellent\,: il
ne finissait point sur elle, il y était toujours nouveau et avec une
liberté qui ne se refusait rien.

Un mot de lui au roi dure encore. Il était à son dîner, toujours fort
distingué du roi dès qu'il paraissait devant lui\,; le roi, lui
adressant la parole, se plaignit de l'incommodité de n'avoir plus de
dents. «\,Des dents, sire, reprit le cardinal, eh\,! qui est-ce qui en
a\,?» Le rare de cette réponse est qu'à son âge il les avait encore
blanches et fort belles, et que sa bouche fort grande, mais agréable,
était faite de façon qu'il les montrait beaucoup en parlant\,; aussi le
roi se prit-il à rire de la réponse, et toute l'assistance et lui-même
qui ne s'en embarrassa point du tout. On ne tarirait point sur lui\,; je
finirai ce qui le regarde par quelque chose de plus sérieux.

On a vu légèrement en son lieu, je dis légèrement parce que ce n'est pas
mon dessein de m'arrêter sur cette vaste matière, que l'affaire de la
constitution se traita un moment chez lui. Les chefs du parti de la
bulle ne purent parer ce renvoi que le roi donna à son estime pour la
capacité du cardinal d'Estrées, et à son désir de la paix. Ils
s'aperçurent bientôt qu'il savait trop de théologie pour eux, et trop
exactement, et trop aussi d'affaires du monde. Celui qui dans son
premier âge avait si bien su finir l'affaire des quatre évêques n'était
pas dans son dernier l'homme qu'il leur fallait, avec l'expérience et
l'autorité qu'il avait acquise. Ils prirent donc le parti de rompre des
conférences auxquelles le cardinal d'Estrées n'avait garde de prendre
goût, parce qu'il y voyait trop clairement la droiture et la vérité
d'une part, la fascination, le parti, les artifices, la violence de
l'autre.

Ce fut dans le court espace de temps de ces conférences que le P.
Lallemant, un des principaux boute-feu des jésuites, allait écumer le
plus souvent qu'il pouvait ce qui se passait à l'abbatial de
Saint-Germain des Prés. S'y trouvant un jour avec le maréchal d'Estrées
qui y logeait avec son oncle, et parlant tous deux de la matière qui
était sur le tapis pendant que le cardinal travaillait dans son cabinet,
le P. Lallemant se mit à vanter l'inquisition, et la nécessité de
l'établir en France. Le maréchal le laissa dire quelque temps, puis le
feu lui montant au visage, lui répondit vertement sur cette exécrable
proposition, et finit par lui dire que, sans le respect de la maison où
ils étaient, il le ferait jeter par les fenêtres.

Ce projet, qui est depuis longtemps le projet favori des jésuites et de
leurs principaux abandonnés, comme celui dont l'accomplissement mettrait
le dernier comble à leur puissance deçà et delà les monts, est celui
auquel ils n'ont cessé de loin d'aplanir toutes les voies, et à
l'avancement duquel ils n'ont cessé de travailler depuis l'espérance et
les moyens que leur en fournissent l'anéantissement de la paix de
Clément IX, et leur chef-d'oeuvre, l'affaire de la constitution, qui ont
établi une inquisition effective par la conduite que depuis sa naissance
on y tient de plus en plus tous les jours, qui est un prélude et un bon
préparatif pour y accoutumer le monde.

Leur P. Contencin, revenu en Europe pour leurs affaires de la Chine où
il en a été un des plus grands ouvriers, et y retournant en 1729, ne put
s'empêcher de dire, en s'embarquant au Port-Louis, que dans peu on
verrait l'inquisition reçue et établie en France, ou tous les jésuites
chassés. Ce mot fit grand bruit et retentit bien fortement jusqu'à
Paris.

En 1732, le P. du Halde, qui a donné les artificieuses relations de
leurs missions diverses, sous le titre de \emph{Lettres édifiantes et
curieuses}, et depuis une histoire et des cartes de la Chine, très bien
faites, mais où il n'y a pas moins d'art, me vint voir comme il y venait
quelquefois depuis que je l'avais connu secrétaire du P. Tellier. J'en
avais été content pour une affaire qui regardait la Trappe du temps du
roi, et à sa mort je lui procurai une bonne pension qui l'établit pour
toujours à leur maison professe de Paris, avec commodité et distinction.
Il tourna fort son langage, et à la fin me tint le même propos que
quinze ans auparavant le P. Lallemant avait tenu au maréchal d'Estrées,
et avec un miel jésuitique me voulut prouver que rien n'était meilleur
ni plus nécessaire que d'établir l'inquisition en France. Il est vrai
que je le relevai si brutalement que de sa vie il n'a osé m'en reparler.
C'est ainsi que ces bons pères vont sondant et semant sans se rebuter
jamais, jusqu'à ce que, la force à la main, ils y parviennent par
l'aveuglement du gouvernement à quelque prix que ce soit et par toutes
sortes de voies. Il y aurait, du reste, de quoi s'étendre sur une
matière si curieuse et si étrangement intéressante\,; il doit suffire
ici de l'avoir effleurée assez pour en constater le dessein, le projet,
et le travail constant et assidu pour arriver à cette abominable fin.

Goesbriant, qui passait pour fort riche, appuyé du crédit de Desmarets
son beau-père, maria son fils à une des filles du marquis de Châtillon,
éblouis, l'un de l'alliance, l'autre des biens, et de se défaire pour
rien d'une de ses filles, dont il avait quantité, et point de fils.

Le prince électoral de Saxe vit le roi à son lever, qui lui fit beaucoup
d'honnêtetés. Bergheyck prit ensuite congé du roi, qui lui donna force
louanges, jusqu'à lui dire qu'il plaignait le roi son petit-fils de ne
l'avoir plus à la tête de ses finances. Il se retira en Flandre, l'été
dans une terre, l'hiver à Valenciennes, et conserva des amis et beaucoup
de réputation et de considération.

L'électeur de Bavière tira des faisans dans le petit parc de Versailles,
vit après le roi seul dans son cabinet, joua chez M\textsuperscript{me}
la Duchesse, soupa chez d'Antin et retourna à Saint-Cloud. Il n'y avait
que le roi qui tirât dans ce petit parc, et fort rarement feu
Monseigneur pendant sa vie.

Albergotti revint de Florence et de quelques autres petites cours
d'Italie, où on crut qu'il avait été chargé de quelque commission du
roi. Il nomma en même temps Rottembourg pour être son envoyé près du roi
de Prusse, et divers autres pour Ratisbonne et les cours d'Allemagne.

Bissy, évêque de Meaux, nommé par le roi au cardinalat, eut l'abbaye de
Saint-Germain des Prés, et le gratis entier comme si déjà il avait été
cardinal. Ce morceau avait toujours été pour des cardinaux ou des
princes. Cette fortune d'un si mince sujet était bien duc à la
constitution.

Les deux ducs et pairs qu'elle venait de faire furent reçus au parlement
le mardi 18 décembre. On a vu ailleurs que c'est le roi qui a fixé le
premier leur âge à vingt-cinq ans pour y entrer, et ce qui a causé cette
nouveauté. Le duc de Melun qui ne les avait pas, et qui craignit qu'on
en fît d'autres qui les auraient, et de tomber avec eux dans le cas de
M. de La Rochefoucauld avec moi, obtint la permission d'être reçu avant
cet âge, et d'opiner cette fois, mais à condition de n'aller plus au
parlement qu'il n'eût vingt-cinq ans. Il fut donc reçu avec le prince de
Rohan, qui donna moins un grand repas qu'une fête dans sa superbe
maison. Ainsi finit cette année, dont je n'ai pas cru devoir interrompre
le cours par le commencement d'une affaire qui continua dans l'année où
nous allons entrer, et qui eut d'étranges suites.

Sceaux était plus que jamais le théâtre des folles de la duchesse du
Maine\,; de la honte, de l'embarras, de la ruine de son mari, par
l'immensité de ses dépenses, et le spectacle de la cour et de la ville
qui y abondait et s'en moquait. Elle y jouait elle-même \emph{Athalie}
avec des comédiens et des comédiennes, et d'autres pièces, plusieurs
fois la semaine. Nuits blanches\footnote{Voy., sur les \emph{nuits
  blanches} de Sceaux, t. V, p.~2, note.} en loterie, jeux, fêtes,
illuminations, feux d'artifices, en un mot fêtes et fantaisies de toutes
les sortes et de tous les jours. Elle nageait dans la joie de sa
nouvelle grandeur, elle en redoublait ses folies, et le duc du Maine,
qui tremblait toujours devant elle, et qui craignait de plus que la
moindre contradiction achevât entièrement de lui tourner la tête,
souffrait tout cela, jusqu'à en faire piteusement les honneurs, autant
que cela se pouvait accorder avec son assiduité auprès du roi dans ses
particuliers, sans s'en trop détourner.

Quelque grande que fût sa joie, à quelque grandeur et la moins
imaginable qu'il fût arrivé, il n'en était pas plus tranquille.
Semblable à ces tyrans qui ont usurpé par leurs crimes le souverain
pouvoir, et qui craignent comme autant d'ennemis conjurés pour leur
perte tous leurs concitoyens qu'ils ont asservis, il se considérait
assis sous cette épée que Denys, tyran de Syracuse, fit suspendre par un
cheveu au-dessus de sa table, sur la tête d'un homme qu'il y fit asseoir
parce qu'il le croyait heureux, auquel il voulut faire sentir par là ce
qui se passait sans cesse en lui-même. M. du Maine, qui exprimait si
volontiers les choses les plus sérieuses en plaisanteries, disait
franchement à ses familiers qu'il était comme un peu entre deux ongles
(des princes du sang et des pairs), dont il ne pouvait manquer d'être
écrasé, s'il n'y prenait bien garde. Cette réflexion troublait l'excès
de son contentement, et celui des grandeurs et de la puissance où tant
de machines l'avaient élevé. Il craignait les princes du sang dès qu'ils
seraient en âge de sentir l'infamie et le danger de la plaie qu'il avait
porté dans le plus auguste de leur naissance, et le plus distinctif de
tous les autres hommes\,; il craignait le parlement qui, jusqu'à ses
yeux, n'avait pu dissimuler l'indignation du violement qu'il avait fait
de toutes les lois les plus saintes et les plus inviolables, sans se
pouvoir rassurer par le dévouement sans mesure du premier président,
trop décrié par son ignorance, trop déshonoré par sa vie et ses mœurs,
pour oser espérer de tenir sa compagnie par lui. Enfin il craignait
jusqu'aux ducs, tant la tyrannie et l'injustice sont timides.

Sa frayeur lui fit donc concevoir le dessein de brouiller si bien ses
ennemis, de les armer si ardemment les uns contre les autres, qu'ils le
perdissent de vue, et qu'il leur échappât dans le cours de leur longue
et violente lutte, qui leur ôterait tout moyen de réunion contre lui,
qui était la chose qui lui semblait la plus redoutable. Pour entendre
comment il parvint à ce grand but, il faut expliquer certaines choses
entre les pairs et le parlement. On se contentera du nécessaire, ce lieu
n'étant pas celui de traiter cette matière à fond, mais ce nécessaire ne
peut être aussi court qu'on le désirerait ici.

Il faut d'abord voir ce qu'est la dignité de pair de France, si elle
n'est pas la même aujourd'hui qu'elle a été dans ces puissants
souverains, ou presque tels, dont les duchés et les comtés-pairies ont
été en divers temps réunis à la couronne, et ce qu'est le parlement de
Paris et les autres parlements du royaume. C'est une connaissance
nécessairement préalable aux choses qu'il est temps de raconter.

\hypertarget{chapitre-xvi.}{%
\chapter{CHAPITRE XVI.}\label{chapitre-xvi.}}

~

{\textsc{Origine et nature de la monarchie française, et de ses trois
états.}} {\textsc{- Son gouvernement.}} {\textsc{- Champs de mars, puis
de mai.}} {\textsc{- Pairs de France sous divers noms, les mêmes en tout
pour la dignité et les fonctions nécessaires, depuis la fondation de la
monarchie.}} {\textsc{- Pairs de fief\,; leurs fonctions.}} {\textsc{-
Hauts barons\,; leur origine, leur usage, leur différence essentielle
des pairs de France.}} {\textsc{- Changement du service par l'abolition
de celui de fief et l'établissement de la milice stipendiée.}}
{\textsc{- Origine des anoblissements.}} {\textsc{- Capitulaires de nos
rois.}} {\textsc{- Légistes\,; quels\,; leur usage\,; leurs progrès.}}
{\textsc{- Conseillers\,; origine de ce nom.}} {\textsc{- Parlements\,;
origine de ce nom.}} {\textsc{- Progrès du parlement.}} {\textsc{-
Multiplication des magistrats et de cours ou tribunaux de justice.}}
{\textsc{- Sièges hauts et bas de grand'chambre des parlements.}}
{\textsc{- Parité, quant à la dignité de pairs de France et ce qui en
dépend, de ceux d'aujourd'hui avec ceux de tous les temps.}} {\textsc{-
Noms donnés aux pairs par nos rois de tous les âges.}} {\textsc{- Pairie
est apanage, témoin Uzès.}} {\textsc{- Réversibilité à la couronne.}}
{\textsc{- Apanage\,; ce que c'est.}} {\textsc{- Ducs vérifiés\,; Bar.}}
{\textsc{- Ducs non vérifiés.}} {\textsc{- Officiers de la couronne.}}
{\textsc{- Ducs non vérifiés en compétence continuelle avec les
officiers de la couronne.}}

~

On ne peut douter que les premiers successeurs de Pharamond n'aient
moins été des rois que des Capitaines qui, à la tête d'un peuple
belliqueux qui ne pouvait plus se contenir dans ses bornes, se répandit
à main armée et fit des conquêtes. Clovis donna le premier plus de
consistance à ce nouvel état, plus de majesté à sa dignité, et par le
christianisme qu'il embrassa, plus d'ordre et de police à ses sujets,
dont il fut peut-être le premier roi\,; et plus de règle et de commerce
avec ses voisins. La nouvelle monarchie conquise fut toute militaire,
jamais despotique. Les chefs principaux qui avaient aidé à la former
étaient appelés à toutes les délibérations de guerre, de paix, de lois à
faire, à soutenir, à toutes celles qui regardaient le dedans et le
dehors.

Les conquêtes s'étant multipliées, les Francs qui les firent donnèrent
leur nom de France à la Gaule qu'ils avaient soumise, et ils reçurent de
leurs rois des partages des terres conquises, à proportion de leurs
services, et de leur poids, et de leurs emplois. Ces portions leur
tinrent lieu de paye. Ils les eurent d'abord à vie, et, vers le déclin
de la première race, presque tous en propriété\footnote{On appelait
  \emph{bénéfices} les terres qui furent données aux guerriers francs
  après la conquête. Voy., sur la nature de ces terres, les \emph{Essais
  sur l'Histoire de France}, par M. Guizot.}. Alors, ceux qui avaient
les portions les plus étendues en divisèrent des parties entre des
Francs moindres qu'eux, sous les mêmes conditions qu'ils tenaient
eux-mêmes leurs portions du roi, c'est-à-dire de fidélité envers et
contre tous, d'entretenir des troupes à leurs dépens, de les mener à
celui qui leur avait donné leurs terres pour servir à la guerre sous
lui, comme lui-même était obligé envers le roi à la même fidélité et au
même service de guerre, toutes les fois que le roi le mandait. C'est ce
qui forma la seigneurie et le vasselage. Ceux qui avaient leurs portions
des rois s'appelèrent bientôt \emph{feudi}\footnote{Saint-Simon a écrit
  ainsi ce mot au lieu de \emph{leudi} qui se trouve plus bas. Le mot
  leudes vient de l'allemand \emph{leuten} (accompagner) et désignait
  les compagnons ou fidèles des rois barbares.} , et \emph{fidèles}, de
la fidélité dont ils avaient contracté et voué l'obligation en recevant
ces portions qui furent appelées \emph{fiefs}\footnote{Voy. notes à la
  fin du volume p.~401.}\,; et l'action de les recevoir en promettant
fidélité et service militaire au roi, \emph{hommage}\footnote{Voy. t.
  II. p.~449, note sur l'hommage.}. Ces premiers seigneurs furent donc
les grands \emph{feudataires} qui eurent d'autres feudataires sous eux,
comme il vient d'être dit, qui tenaient des fiefs d'eux sous la même
obligation à leur égard de fidélité et de service militaire. C'est d'où
est venue la noblesse connue longtemps avant ce nom sous le générique de
\emph{miles}, homme de guerre, ou noble, synonymes, lorsque le nom de
noble commença à être en usage, à la différence des peuples conquis, qui
de leur entière servitude furent appelés \emph{serfs}.

Cette noblesse, pour parler un langage entendu, ne put suffire à la
culture de ses terres. Elle en donna des portions aux serfs, chacun dans
sa dépendance, non à condition de service militaire, comme les fiefs,
mais à \emph{cens} et à \emph{rente}, et à diverses conditions, d'où
sont venus les divers droits des terres. Ainsi ce peuple serf, qui
n'avait rien, commença à devenir propriétaire en partie, tandis qu'en
partie il continua à ne posséder quoi que ce soit, et de ces deux sortes
de serfs dont les uns devinrent propriétaires et les autres ne le furent
pas, est composé le peuple ou ce qui a été appelé depuis le \emph{tiers
état}, et comme aujourd'hui se pouvait distinguer dès lors en
bourgeoisie et en simple peuple. Ces baillettes\footnote{Terres sans
  importance données à bail.}, qui furent données d'abord aux meilleurs
habitants des villes, s'étendirent aux meilleurs de la campagne. Elles
furent bientôt connues sous le nom de \emph{roture}, à la différence des
fiefs\,; et leurs possesseurs sous le nom de
\emph{roturiers}\footnote{Le mot \emph{roturier} vient du latin barbare
  \emph{ruptarius} (qui rumpit terram)\,: il s'appliquait primitivement
  aux paysans qui étaient condamnés aux travaux corporels ou corvée.}, à
la différence des seigneurs de fief, terme qui n'avait et n'eut très
longtemps que sa signification naturelle, et que l'orgueil a fait depuis
prendre en mauvaise part.

L'Église fit aussi ses conquêtes pacifiques\,; par la libéralité des
rois et des grands seigneurs les évêques et les abbés les devinrent
eux-mêmes. Ils eurent des portions de terre fort étendues, ils en
donnèrent en fief comme avaient fait les grands seigneurs, et de là sont
venus les grands bénéfices que nous voyons encore aujourd'hui, et alors
la fidélité et le service militaire qu'ils devaient aux rois et qui leur
était aussi dû à eux-mêmes par leurs vassaux, leur grand état temporel
les fit considérer comme les autres grands seigneurs. Parvenus à ce
point, l'ignorance de ceux-ci se fit une religion de leur laisser la
primauté par l'union de leur sacerdoce avec leurs grands fiels, en sorte
que la noblesse, qui était le corps unique de l'État, en laissa former
un second qui devint le premier\,; et tous deux en formèrent un autre
par leurs baillettes, qui rendirent force serfs propriétaires, lesquels,
avec les autres serfs qui ne l'étaient pas, et qui tous étaient le
peuple conquis, devinrent par la suite le troisième corps de l'État sous
le nom déjà dit de tiers état.

Cet empire tout militaire se gouverna tout militairement aussi par ce
qu'on appela les \emph{champs de mars} puis de \emph{mai}. Tous les ans
en mars, et ensuite non plus en mars mais en mai, le roi convoquait une
assemblée. Il en marquait le jour et le lieu. Chaque prélat et chaque
grand seigneur s'y rendait avec ses vassaux et ses troupes. Là,
deuxespèces de chambres\footnote{Voy. notes à la fin du volume.} en
plein champ étaient disposées, l'une pour les prélats, l'autre pour les
grands seigneurs, c'est-à-dire les \emph{comtes}, dès lors connus sous
ce nom\,; tout proche, dans l'espace découvert, était la foule
militaire, c'est-à-dire les troupes et les vassaux qui les commandaient.
Le roi, assis sur un tribunal élevé, attendait la réponse des deux
chambres à ce qu'il avait envoyé leur proposer. Lorsque tout était
d'accord, le roi déclarait tout haut les résolutions qui étaient prises,
soit civiles, soit militaires\,; et la foule militaire éclatait aussitôt
en cris redoublés de vivat, pour marquer son obéissance. Dans cette
foule, nul ecclésiastique, nul roturier, nul peuple\,; tout était gens
de guerre ou noblesse, ce qui était synonyme, comme on l'a remarqué.
Cette foule ne délibérait rien, n'était pas même consultée\,; elle se
tenait représentée par leurs seigneurs, et applaudissait pour tout
partage à leurs résolutions unies à celles du roi qui les déclarait.
C'était de là qu'on partait pour la guerre, quand on avait à la faire.
Il y aurait bien de quoi s'étendre sur ce court abrégé\,; mais c'est un
récit le plus succinct pour la nécessité, et non un traité qu'il s'agit
de faire.

Cette forme de gouvernement dura constamment sous la première race de
nos rois, et cette assemblée se nommait \emph{placita}, de
\emph{placet}, c'est-à-dire de ce qu'il lui avait plu de résoudre et de
décider.

Pépin, chef de la seconde race, porté sur le trône par les grands
vassaux, à force de crédit, de puissance, d'autorité qu'il avait su
s'acquérir, continua la même forme de gouvernement\,; mais en mai au
lieu de mars, qui fut trouvé trop peu avancé dans le printemps pour
tenir les \emph{placita}. Charlemagne son fils les continua de même
autant que ses voyages le lui permirent, mais jamais sans ses grands
vassaux il n'entreprit aucune chose considérable de guerre, de paix, de
partage de ses enfants, d'administration publique, tandis qu'en Espagne
et en Italie il agissait seul. L'usage ancien fut suivi par sa
postérité. Sous elle les grands vassaux s'accrurent de puissance et
d'autorité, tellement qu'ils ne le furent guère que de nom sous les
derniers rois de cette race, dont la mollesse, la faiblesse et
l'incapacité y donnèrent lieu.

Peu à peu les différends de fiefs n'allèrent plus jusqu'aux rois. Les
feudataires jugeaient les contestations que leurs vassaux n'avaient pu
terminer entre eux par le jugement de leurs pareils\,; et pour les
causes les plus considérables, elles se jugeaient par les grands
feudataires assemblés avec le roi. La multiplication de ces différends
vint de celle des inféodations dans leurs conditions différentes, dans
le désordre des guerres qui fit contracter des dettes, et qui obligea à
mettre dans le commerce les fiefs qui n'y avaient jamais été, qui de là
les fit passer par divers degrés de successions souvent disputées, enfin
aux femmes, sans plus d'égard sur ce point à la fameuse loi salique, qui
les excluait de toute terre salique\footnote{On doit entendre
  probablement par \emph{terre salique} la terre de conquête et surtout
  l'espace qui entourait le manoir principal. Voy. les notes de M.
  Pardessus sur la loi salique.}\,: loi qui n'ayant pour objet que cette
terre, c'est-à-dire celle qui avait été donnée pour tenir lieu de paye,
qui était la distinction du Franc conquérant d'avec le Gaulois conquis,
des fiefs et d'avec les rotures, de la noblesse d'avec le peuple,
demeura uniquement restreinte au fief des fiefs, qui est la couronne.

La seconde race sur le point de périr par l'imbécillité des derniers
rois, Hugues Capet, duc de France, comte de Paris, proche parent de
l'empereur, et dont le grand-père avait déjà contesté la couronne, fut
porté sur le trône par le consentement de tous les grands vassaux du
royaume, qui les confirma dans tout ce qu'ils en tenaient, et l'augmenta
ainsi que leur autorité\,; c'est là l'époque où les ducs et les comtes,
chefs des armées et gouverneurs de province à vie, inféodés après en de
grands domaines, de suzerains devinrent souverains, non seulement de ces
domaines, mais des provinces dont ils n'étaient auparavant que les
gouverneurs. Je dis souverains, parce qu'encore qu'ils fussent vassaux
de la couronne, pour ces mêmes domaines et ces mêmes provinces, leur
puissance était devenue si étendue et si grande qu'elle approchait fort
de la souveraineté.

Le nom de \emph{pair} de France, inconnu sous la première race,
longtemps sous la seconde, peut-être même au commencement de la
troisième, manqua seulement aux plus grands de ces premiers grands
feudataires ou grands vassaux de la couronne, puisque, comme l'avouent
les meilleurs auteurs, ils faisaient les mêmes fonctions que ceux qui
parurent sous le nom de pairs de France, firent tout de suite et
précisément le même, et tout en la même manière, et sans érections pour
les six premiers laïques et ecclésiastiques qui l'ont porté. Ce qui
suffit à prouver que, sans nom ou avec d'autres noms, l'essence est la
même sans changement ni interruption, et que ce qui a été connu alors
par le nom et titre de pair de France, s'est trouvé assis à côté du
trône dès l'origine de la monarchie, et sous le nom de pairs de France
et de pairie de France, en même temps que la race heureusement régnante
a été portée dessus.

Ce nom de pair s'introduisit insensiblement de ce que chacun était jugé
par ses pairs, c'est-à-dire par ses égaux. Ainsi chaque grand fief avait
ses pairs de fief, dont on voit les restes jusqu'à nos jours par les
pairs du Cambrésis et d'autres grands ou moindres fiefs, et le nom de
pairs de France demeura aux plus grands de ces grands feudataires qui
tenaient leurs grands fiefs du roi\,; et qui avec lui jugeaient les
causes majeures de tous les grands fiefs, directement ou par appel, et
lui aidaient dans l'administration de l'État, militaire ou intérieure,
et pour faire les lois, les changer et régler, et faire les grandes
sanctions de l'État dans ces \emph{placita conventa} ou assemblées de
tous les ans. Bientôt toutes les mouvances majeures des seigneurs
ressortirent au roi ou à ces pairs, dont l'étendue de domaine avait
envahi les autres principaux vassaux.

Nos rois, outre ceux de leur couronne qui n'étaient presque plus que ces
premiers grands pairs de France, en avaient aussi de particuliers comme
duc de France et comte de Paris, que Hugues Capet était avant de
parvenir à la couronne, et qu'il leur avait transmis. Ils voyaient les
anciens grands seigneurs s'éteindre, et les pairs de France s'accroître
de leurs grands fiefs. Ils pensèrent à leur donner des adjoints aux
\emph{placita} dont ils ne pussent se plaindre, et ils y admirent de ces
grands vassaux du duché de France qui relevaient aussi immédiatement
d'eux, non comme rois, mais comme ducs de France, afin que les pairs n'y
fussent pas seuls, faute de grands vassaux immédiats\footnote{Vassaux
  qui relevaient directement du roi.}. Ceux-ci furent appelés d'abord
\emph{hauts barons} du duché de France, plus \emph{hauts barons} de
France. Ils y appelèrent aussi quelques évêques, dont la diminution des
grands fiefs avait diminué ces assemblées\,; et par l'usage que prirent
nos rois d'y appeler de ces hauts barons, ils y balancèrent la trop
grande autorité du petit nombre de ces trop puissants pairs de France.
La différence fut, et qui a subsisté jusqu'à nous dans toutes les
différentes sortes d'assemblées qui ont succédé aux \emph{placita
conventa}, fut que tous les pairs y assistaient de droit, en faisaient
l'essence, qu'il ne s'y faisait rien sans leur intervention à tous ou en
partie, et qu'il leur fallait une exoine, c'est-à-dire une légitime
excuse et grave, pour se dispenser de s'y trouver, au lieu que la
présence des hauts barons n'y était pas nécessaire, qu'ils n'y pouvaient
assister que lorsque nommément ils y étaient mandés par le roi, que
jamais ni tous ni la plus grande partie n'y étaient mandés, ni presque
jamais les mêmes plusieurs fois de suite\,; ainsi ces hauts barons
appelés à ces assemblées, au choix et à la volonté des rois, n'y étaient
que des adjoints admis personnellement à chaque fois, et non
nécessaires\,; tandis que les pairs l'étaient tellement que tout se
faisait avec eux, rien sans eux.

On voit par cette chaîne non interrompue depuis la naissance de la
monarchie, cette même puissance législative et constitutive pour les
grandes sanctions de l'État, concourir nécessairement, et par une
nécessité résidante dans le même genre de personne, sous quelque nom que
ç'ait été, \emph{de grands vassaux, grands feudataires, leudi, fidèles},
mais toujours relevant immédiatement de la couronne, enfin de pairs,
laquelle était en eux seuls privativement à tous autres seigneurs,
quelque grands qu'ils fussent, sous les trois races de nos rois.

Les querelles, les contestations de fief pour successions, pour dettes,
pour partages, pour saisie faute d'hommage, de service, ou pour crimes,
se multipliant de plus en plus, ainsi que les affaires d'administration
civile, rendirent les grandes assemblées plus fréquentes et hors du
temps accoutumé du mois de mai. Comme les délibérations n'étaient pas
militaires, et qu'on n'en partait plus pour la guerre, la foule
militaire ne s'y trouvait plus. Le roi, les pairs et ceux des hauts
barons et quelques évêques que le roi y appelait, formaient ces
assemblées, d'où peu à peu il arriva que, le prétexte du désordre qui
résultait du service de fief multiplié par les fiefs devenus sans nombre
sous les grands et les arrière-fiefs, l'abus de ce service des vassaux
des grands fiefs, contre les rois même quand les grands vassaux leur
faisaient la guerre, fit que les rois, accrus d'autorité et de
puissance, parvinrent à abolir ce service de fiefs, tant pour les
suzerains de toute espèce que pour eux-mêmes, changèrent sous divers
prétextes la forme de la milice, et la réduisirent pour l'essentiel à
l'état de levées, de solde, de distribution par compagnies, à peu près
dans l'état où elle se trouve aujourd'hui. Ainsi les rois mirent en leur
main des moyens de puissance et de récompenses qui énervèrent tout à
fait la puissance et la force de tous les grands vassaux et de tous les
suzerains, qui ne furent plus suivis des leurs à la guerre\,; ainsi
cette foule militaire des champs de mai disparut, et bientôt n'exista
plus ensemble. D'autres que ces anciens Francs d'origine furent admis
dans la milice\,; de là les nobles factices qui accrurent encore le
pouvoir des rois.

Les assemblées purement civiles n'étaient pas inconnues du temps même
des \emph{placita conventa} ou champs de mai, comme le témoignent les
capitulaires de Charlemagne et de ses enfants. C'étaient des assemblées
convoquées par ces princes dans leurs palais, mais qui n'étaient
composées que de ces mêmes grands feudataires et des prélats consultés
aux champs de mai, où il se faisait des règlements qui regardaient
l'Église, la religion et les affaires générales, mais civiles, ce qui
n'empêchait pas la tenue ordinaire des champs de mai.

Mais lorsque ces champs de mai ou \emph{placita conventa} eurent disparu
par le changement de la forme de la milice dont on vient de parler, et
que les assemblées devinrent telles qu'on vient de l'expliquer un moment
avant de parler des capitulaires, l'excès des procès qui se
multiplièrent de plus en plus, et par même cause les ordonnances
diverses et les différentes coutumes des différentes provinces,
devinrent tellement à charge aux pairs et à ceux des hauts barons qui
étaient appelés à ces assemblées, que saint Louis, qui aimait la
justice, fit venir des légistes pour débrouiller ces procès et les
simplifier, et faciliter aux pairs et aux hauts barons le jugement par
la lumière qu'ils leur communiquaient.

Ces légistes étaient des roturiers qui s'étaient appliqués à l'étude des
lois, des ordonnances, des différents usages des pays, ce qui fut depuis
appelé \emph{coutumes}, qui conseillaient les feudataires particuliers
dans le jugement qu'ils avaient à rendre avec leur suzerain, d'où peu à
peu sont dérivées les justices seigneuriales ou hautes justices des
seigneurs, en images très imparfaites de celles qu'ils rendaient avant
que petit à petit les rois les eussent changées par leur autorité, après
le changement dans la forme de la milice et après la réunion de
plusieurs grands fiefs à leur couronne.

Ces légistes étaient assis sur le marchepied du banc sur lequel les
pairs et les hauts barons se plaçaient, pour leur donner la facilité de
consulter ces légistes sans quitter leurs places et sur-le-champ. Mais
cette consultation était purement volontaire, ils n'étaient point
obligés de la suivre, et ces légistes, bien loin d'opiner, n'avaient
autre fonction que d'éclaircir les pairs et les hauts barons à chaque
fois et sur chaque point qu'ils s'avançaient à eux, sans se lever pour
l'être, après quoi ou sans quoi ils opinaient comme il leur semblait, en
suivant ou au contraire de ce qu'ils avaient appris des légistes sur ce
qu'ils les avaient consultés. De là leur est venu le nom de
\emph{conseillers}, de ce qu'ils conseillaient les pairs et les hauts
barons quand ils voulaient leur demander éclaircissement, non de juges
qu'ils n'étaient pas\,; et ce nom de conseiller leur est demeuré en
titre, de passager qu'il était par leur fonction.

Peu à peu les pairs, occupés de guerre et d'autres grandes affaires, se
dispensèrent souvent de se trouver à ces assemblées, où il ne s'agissait
que d'affaires contentieuses qui ne regardaient point les affaires
majeures. Les rois aussi s'en affranchissaient. Les hauts barons y
étaient appelés en petit nombre, quelques-uns d'eux alléguaient aussi
des excuses, tellement que, pour vider ce nombre toujours croissant de
procès que la diversité des coutumes des lieux et des ordonnances
multipliait sans cesse, les rois donnèrent voix délibérative aux
légistes, et peu à peu ceux-ci, accoutumes à cet honneur, surent se le
conserver en présence des pairs mêmes. Mais il n'est encore personne qui
ait imaginé que, dès lors ni longtemps depuis, ces légistes aient ni
obtenu, ni prétendu voix délibérative pour les affaires majeures, ni
pour les grandes sanctions de l'État. Outre qu'il n'y en a point
d'exemple, il n'y a qu'à les comparer aux pairs et aux hauts barons de
ces temps-là. On verra dans la suite l'identité des pairs d'aujourd'hui
avec ceux-là pour la dignité, l'essence, les fonctions, comme on a
commencé à le faire voir. Suivons les légistes.

La même nécessité de vider cette abondance toujours croissante de procès
donna lieu à des assemblées plus fréquentes. Nos rois les indiquaient à
certaines fêtes de l'année, dans leurs palais, tantôt aux unes, tantôt
aux autres, et ces assemblées prirent le nom de parlements, de parler
ensemble\,; de la vinrent les parlements de Noël, de la Pentecôte, de la
Saint-Martin, etc. Les pairs s'y trouvaient quand il leur plaisait pour
y juger sans être mandés\,; les hauts barons qui y étaient
personnellement appelés par le roi en petit nombre\,; et ceux d'entre
les légistes qu'il plaisait au roi. Jamais ni haut baron ni légiste qui
ne fût pas nommé et appelé par le roi, jamais les mêmes en deux
assemblées de suite autant qu'il se pouvait.

Ces parlements subsistèrent dans cette forme jusqu'à Charles VI. Sous ce
malheureux règne, les factions d'Orléans et de Bourgogne les composaient
à leur gré, suivant qu'elles avaient le dessus pendant les intervalles
que le roi n'était pas en état de les nommer. Le désordre qui en résulta
fit que, dans les bons intervalles de ce prince, il fut jugé à propos de
laisser à vie ces commissions qui n'étaient que pour chaque assemblée.
Ainsi ces commissions se tournèrent peu à peu en offices\,; et les
assemblées venant à durer longtemps, il fallut opter entre l'épée et
l'écritoire, et les nobles qui étaient choisis pour en être avec les
légistes, n'en ayant plus le loisir par les guerres qui les occupaient,
quittèrent presque tous cette fonction, en sorte qu'il n'en demeura
qu'un très petit nombre, qui ont fait les familles les plus distinguées
du parlement de Paris, dont il ne reste plus. Tout ce récit est plutôt
étranglé que suffisamment exposé, mais la vérité historique et prouvée
s'y trouve religieusement conservée. Le mémoire sur les renonciations
dont il a été parlé plus haut, quoique fort abrégé aussi, et qui se
trouvera parmi les Pièces, explique d'une façon plus complète et plus
satisfaisante ce qui vient d'être exposé jusqu'ici et qui le sera dans
la suite.

Il reste un monument bien remarquable de l'état des légistes séants aux
pieds des pairs et des hauts barons sur le marchepied de leurs bancs,
depuis même que les parlements sont devenus ce qu'on les voit
aujourd'hui. Ils n'avaient qu'une chambre pour leur assemblée, qu'on
appelle \emph{la grand'chambre} depuis qu'il y en a eu d'enquêtes,
requêtes, tournelle, etc., qui sont nées de cette unique chambre. On y
voit encore les hauts sièges qui étaient le banc des pairs et des hauts
barons, et des bas sièges qui étaient le marchepied de ce banc sur
lequel les légistes s'asseyaient\,; d'un marchepied ils en ont enfin
fait un banc tel qu'on le voit aujourd'hui, et de ce banc après ils sont
montés aux hauts sièges. Voilà le commencement des usurpations que l'art
d'un côté, l'incurie et la faiblesse de l'autre, ont multipliées à
l'infini. Mais, nonobstant celle-là, la magistrature devenue ce qu'on la
voit n'a osé prétendre encore monter aux hauts sièges aux lits de
justice\footnote{Voy., sur les lits de justice, notes à la fin du volume
  p.~405.}. Le chancelier même, bien que second officier de la couronne,
le seul qui ait conservé le rang et les distinctions communes autrefois
à tous, et chef de la justice mais logiste et magistrat, y est assis
dans la chaire sans dossier aux bas sièges, tandis que non seulement les
pairs, mais que tous les autres officiers de la couronne, sont assis aux
hauts sièges des deux côtés du roi.

Enfin l'assemblée du parlement dont les membres légistes étaient devenus
à vie, comme on vient de l'expliquer, devint de toute l'année, et
sédentaire à Paris, par la multiplicité toujours croissante des procès
et l'introduction des procédures. Les pairs, qui y conservèrent leur
droit et leur séance, y jugeaient quand bon leur semblait, comme ils
font encore aujourd'hui\,; et de là ce premier parlement et plus ancien
de tous, a pris le nom de \emph{cour des pairs}, qui est devenue le
modèle des autres parlements que la nécessité des jugements de procès
multipliés à l'infini a obligé les rois d'établir successivement dans
les différentes parties du royaume, avec un ressort propre à chacun,
pour le soulagement des sujets.

Un lieu destiné à cette assemblée, où les pairs se trouvaient quand il
leur plaisait, lieu dans la capitale et dans le palais de nos rois,
devint le lieu propre et naturel pour les affaires majeures et les
grandes sanctions du royaume, et c'est de là encore qu'il a usurpé le
nom de cour des pairs. Je dis usurpé parce qu'il ne lui est pas propre,
et que, partout où il a plu à nos rois d'assembler les pairs pour y
juger des affaires majeures, ou faire les sanctions les plus
importantes, son cabinet, une maison de campagne, un parlement autre que
celui de Paris, tous ces lieux différents ont été pour ce jour-là la
cour des pairs\,; et de cela beaucoup d'exemples depuis que le parlement
de Paris s'en est attribué le nom.

Tels étaient les légistes, tels sont devenus les parlements, dont
l'autorité s'est continuellement accrue parles désordres des temps qui
ont amené la vénalité des offices et les ont après rendus héréditaires
par l'établissement de la \emph{paulette}\footnote{Cet impôt tira son
  nom du financier Paulet, qui en fut le premier fermier. La paulette
  datait de 1604\,; les magistrats, pour devenir propriétaires de leurs
  charges, payaient chaque année le soixantième du prix.}, à la fin ont
multiplié à l'infini les cours et leurs offices.

Il faut revenir maintenant à l'examen de la parité des anciens pairs,
quant à la dignité, aux fonctions nécessaires, au pouvoir législatif et
constitutif, avec les pairs modernes jusqu'à ceux d'aujourd'hui, et pour
cela se défaire des préventions d'écorce qu'on trouve si aisément et si
volontiers dans leur disparité si grande de naissance, de puissance et
d'établissements, mais qui ne conclut quoi que ce soit à l'égard de la
dignité en elle-même, et de tout ce qui appartient à la dignité de pair.

Pour s'en bien convaincre on n'a qu'à parcourir l'histoire, et en
exceptant les temps de confusion et d'oppression de l'État, tels que les
événements où il pensa succomber sous les bouchers, l'université, etc.,
du temps de Charles VI, plus haut pendant la prison du roi Jean, en
dernier lieu sous les efforts de la Ligue, et voir s'il s'est jamais
fait rien de grand dans l'État, sanctions, jugements de causes majeures,
etc., sans la convocation et la nécessaire présence et jugement des
pairs, depuis l'origine de la monarchie jusqu'aux renonciations
respectives de Philippe V et des ducs de Berry et d'Orléans aux
couronnes de France et d'Espagne sous le plus absolu de tous les rois de
France, le plus jaloux de son autorité, et qui s'est le plus
continuellement montré, en grandes et en petites choses, le plus
contraire à la dignité de duc et pair, et le plus soigneusement appliqué
à la dépouiller. Les preuves de ce très court exposé sont éparses dans
toutes les histoires de tous les temps, et on y renvoie avec assurance
ici, où ce n'est pas le lieu d'en faire des volumes en les y ramassant.
Le sacre seul, et la juste et sage déclaration d'Henri III en faveur des
princes du sang qui les rend tous pairs nés à titre de leur naissance,
fourniraient une foule de démonstrations.

Les pairs ecclésiastiques en sont une vivante à laquelle il n'est pas
possible encore de se dérober. On a vu comme les grands bénéfices se
sont établis, et comment les prélats, devenus grands seigneurs par la
libéralité des rois et de leurs grands feudataires, sont devenus grands
seigneurs, et quelques-uns grands feudataires eux-mêmes. L'Église, à
l'ombre de l'ignorance et de la stupidité des laïques, s'accrut lors au
point de se revêtir de toute la puissance temporelle par l'abus et la
frayeur de la spirituelle. On ne peut attribuer à d'autres temps
l'origine inconnue de la pairie attachée en titre de duché aux sièges de
Reims, Laon et Langres\,; et de comté à ceux de Beauvais, Châlons et
Noyon. Voilà donc six pairies ecclésiastiques sans érection, comme les
duchés de Bourgogne, Normandie et Guyenne, et les comtés de Toulouse, de
Flandre et Champagne\,; toutes douze en mêmes droits et fonctions quant
à la dignité, et, nonobstant la distance, sans mesure de naissance et de
puissance entre les six laïques et les six ecclésiastiques, en même
rang, distinctions, égalité. Ces six prélats n'étaient pas différents de
leurs successeurs jusqu'à nous, et s'ils cédaient le pas aux six
laïques, c'était à raison d'ancienneté, puisque tout était entre eux
parfaitement et entièrement égal. Excepté Reims et Beauvais, et encore
qu'était-ce en comparaison des pairs laïques de Bourgogne, etc., il n'y
a guère, à la dignité près, de plus petits-sièges que les quatre autres,
et on peut avancer\,: aucun qui ne vaille Laon et Noyon. Néanmoins,
quand les seigneurs eurent rappris à lire et repris leurs sens, et leurs
vassaux à leur exemple, ils revendiquèrent les usurpations de l'Église,
et quoiqu'elle conservât le plus qu'elle put des conquêtes qu'elle avait
faites sur la grossièreté des laïques, elle demeura comme dépouillée, en
comparaison de ce qu'elle s'était vue en puissance et en autorité. Il
n'y eut que ces six sièges qui, en perdant les abus ecclésiastiques, se
conservèrent dans l'intégrité de leur rang, de leurs fonctions, du
pouvoir législatif et consultatif, à la tête des plus grands, des plus
puissants et des plus relevés seigneurs du royaume, uniquement par le
droit de leur pairie.

Il n'y a pas même eu quelquefois jusqu'à des cérémonies tout à fait
ecclésiastiques où leur pairie leur a donné la préférence, comme il
arriva à la procession générale de tous les corps faite à Paris en
actions de grâces de la délivrance de François Ier. L'archevêque de Lyon
y était avec sa croix devant lui, comme reconnu par Sens dont Paris
était lors suffragant. L'évêque de Noyon prétendit le précéder. La
préséance lui fut adjugée par arrêt du parlement comme étant pair de
France. Il en jouit, et l'archevêque de Lyon céda et assista à la
procession.

Dans ces anciens temps où ces anciennes pairies laïques sans érection
subsistaient encore, au moins les plus puissantes, et possédées par les
plus grands princes, tels que les ducs de Bourgogne, les rois
d'Angleterre, etc., ces six pairies ecclésiastiques n'étaient pas plus
considérables en terres et en revenus qu'aujourd'hui\,; et les évêques
de ces sièges, dont on a la suite, ne l'étaient pas plus en naissance ni
en établissements que le sont ceux d'aujourd'hui, et s'il y a eu
quelques cardinaux et quelques autres du sang royal ou de maisons
souveraines à Reims et à Laon, cela n'a été que rarement, et bien plus
rare ou jamais dans les autres sièges\,; et toutefois on voit ces six
évêques en tout et partout égaux en rang, en puissance, et autorité
législative et constitutive dans l'État, et ces autres pairs si grands
par eux-mêmes, et si puissants par leurs États, en usant avec eux et
comme eux, sans la moindre différence, de l'autorité du pouvoir, du rang
des séances, assistances et jugements des causes majeures et usage du
même pouvoir législatif et constitutif pour les grandes sanctions du
royaume, avec eux et comme eux sans aucune ombre de différence, pareils
en tout ce qui était de la dignité et de l'exercice de la pairie et
aussi en rang, quoiqu'en tout d'ailleurs si entièrement disproportionnés
d'eux. C'est une suite et une chaîne que les histoires présentent dans
tous les temps les plus reculés jusqu'à nous, et qui montre en même
temps quels étaient ces évêques, quant à leur personne, par la suite
qu'elles en offrent\,; tandis que, quant à ce qui ne regarde que
l'épiscopat, ils n'avaient pas plus d'avantages que tous les autres
évêques de France, où, dans ces siècles et longtemps depuis, l'autorité
des métropolitains était pleinement exercée sur leurs suffragants. Par
quoi il demeure évident que la naissance et la puissance par la grandeur
de l'extraction et de la dignité personnelle, par le nombre et l'étendue
des États et des possessions, l'autorité, le degré, la juridiction
ecclésiastique, sont accessoires, totalement indifférents à la dignité,
rang, autorité, puissance, fonctions de pair de France, laquelle a de
tout temps précédé les plus grands personnages du royaume en extraction,
étendue de fiefs et d'États laïques, et les métropolitains les plus
distingués, comme il s'est continuellement vu dans ces évêques.
Conséquemment comme il sera encore éclairci plus bas, que les pairs
nouveaux et qui ont une érection à l'instar de ces premiers qui n'en ont
point que l'on connaisse, et qui ont été érigés pour les remplacer, et
de là pour en augmenter le nombre, et qui ont tous joui très
constamment, quant à cette dignité, de tout ce qui vient d'être dit de
ces premiers, ont été pairs comme eux en toute égalité quant à tout ce
qui appartient à pairie, et de main en main jusqu'à nous, dont la
naissance et les biens ne sont pas inférieurs à ces six pairs
ecclésiastiques dans tous les temps.

La brièveté sous laquelle gémit nécessairement une matière si abondante,
forcément traitée en digression, me fera supprimer une infinité de
passages existants par lesquels on voit ce que nos rois pensaient et
disaient de la dignité et des fonctions de pairs, tant dans les
érections des pairies qu'ils faisaient, qu'ailleurs, pour n'alléguer
qu'un passage de Philippe le Bel du temps duquel ces anciens pairs de
Bourgogne, etc., étaient dans tout leur lustre personnel de grandeur,
d'extraction et de puissance terrienne, si différent de l'état personnel
des évêques-pairs d'alors et d'aujourd'hui. C'est d'une lettre de
Philippe le Bel, de 1306, au pape, qui existe encore en original
aujourd'hui, par laquelle il le prie de remettre à leur prochaine
entrevue le choix d'un sujet pour remplir le siège de Laon vacant.
\emph{In Laudunensi Ecclesia, lui dit-il, quam licet in facultatibus
tenuem, intra ceteras nostri regni utpote paritate seu paragii regni
ejusdem dotatam excellentia, nobilissimam reputamus, ejusque honorem,
nostrum et, regni nostri proprium arbitranmur\ldots. Personam praefici
cupientes, quae honoris regii et regni zelatrix existat, et per quam
praefata Ecclesia debitis proficiat incrementis urgente causa
rationabili, Sanct. Ap. attentis precibus, supplicamus\ldots{} per quam
etiam sicut nobis et status nostri regni expedire conspicimus regimen
ipsius paritatis seu paragii, quod est honoris regii pars non modica,
poterit in melius augmentari}, etc. Les paroles de cette lettre, soit
dans leur tissu, soit séparément considérées, sont si expresses qu'elles
n'ont besoin d'aucun commentaire pour les faire entendre ni valoir. Ce
texte est si remarquable que l'exprimer ce serait l'affaiblir. Il n'y a
pas un mot qui ne porte, et qui ne montre ce qui est dit ci-dessus avec
la plus lumineuse clarté. Le voici en français. On y voit du même coup
d'œil la petitesse et plus que la médiocrité du siège de Laon, si on en
excepte la pairie, en même temps l'excellence de cette dignité qui rend
cette Église la plus noble et la plus excellente de toutes, dont
l'honneur est réputé l'honneur même du roi et du royaume, desquels il
est partie principale, et dont l'augmentation du temporel est regardée
comme importante au roi et à l'État, qui, à cet effet, supplie
instamment le pape, etc., et qui juge le choix d'un évêque pour cette
Église d'une conséquence si importante pour lui et pour son royaume, et
nomme cet évêché-pairie, par deux fois apanage\footnote{Il n'est pas
  question d'apanage dans la lettre de Philippe le Bel, mais de
  \emph{parage} (\emph{paragium}). Le mot parage indiquait l'égalité
  entre les nobles\,; de là l'expression de \emph{paritas} employée
  comme synonyme de \emph{paragium} dans cette même lettre.}.

Quoi de plus exprès pour prouver l'extrême disparité de puissance
terrienne et de dignité personnelle d'une part\,; et de l'autre la plus
entière identité, quant à la dignité de la pairie et à tout ce qu'elle
renferme, entre celle de Laon et ces grandes, anciennes et ces
premières\,; entre un sujet encore inconnu et ces anciens et premiers
pairs de France\,; conséquemment la futilité de se frapper de disparité
quant à tout ce qui est de la pairie, fondée sur tout ce qui lui est
entièrement étranger, comme l'extraction, la puissance terrienne, la
souveraineté\,; et, pour s'en mieux convaincre encore, s'il est
possible, il faut ajouter qu'en ces temps reculés, c'est-à-dire les 19
et 26 février 1410, le procureur général du roi fit proposer, en la
cause de l'archevêque et archidiacre de Reims, suivant l'ancienne
comparaison de saint Louis, que les «\, pairs furent créés pour soutenir
la couronne, comme les électeurs pour soutenir l'empire, par quoi on ne
doit souffrir qu'un pair soit excommunié, parce que l'on a à converser
avec lui pour les conseils du roi, qui le devrait nourrir s'il n'avait
de quoi vivre, si est-ce la différence grande entre lesdits pairs et les
électeurs de l'empire qui font l'empereur, et lesdits pairs ne font le
roi, lequel vient de lignée et plus proche degré.\,»

Il serait difficile de déclarer le pouvoir législatif et constitutif des
pairs avec plus de clarté et d'énergie que le fait ce passage. La
comparaison est empruntée de saint Louis par le procureur général en
jugement, qui, de peur de l'affaiblir, a soin de prévenir l'exception si
naturelle de l'élection des empereurs par les électeurs que les pairs ne
font point de nos rois, qui viennent à la couronne par un droit
héréditaire attaché à l'aîné de leur auguste race. Il s'agissait de
l'excommunication, qui, dans ces temps-là, faisait trembler les
souverains et les plus grands d'entre les sujets, et qui ébranlait la
fermeté des trônes. Un excommunié, de quelque rang qu'il fût, était
interdit de tout, jusqu'au conseil et au service. Quiconque lui parlait
encourait par cela seul la même excommunication. Les rois de France,
fils aînés de l'Église et fondateurs de la grandeur temporelle des papes
et de leur siège, se prétendaient exempts d'encourir l'excommunication.
Les conseillers qui se choisissaient dans leurs affaires, c'est-à-dire
leurs ministres, ne prétendaient pas participer à cette exemption. Le
procureur général, conservateur né des droits de la couronne, n'en fait
pas la moindre mention. Mais les conseillers nécessaires, ceux qui par
leur pairie, exerçaient de droit le pouvoir législatif et constitutif
pour les grandes sanctions du royaume avec le roi, eux du concours
desquels ces sanctions ne pouvaient se passer pour avoir force de loi,
ni les causes majeures des grands fiefs, ou de la personne des grands et
immédiats feudataires, pour être validement jugées et d'une manière
définitive, parties essentielles et intégrantes de la couronne, du
commerce desquels ils n'était pas possible de se passer pour tout ce qui
concernait l'État, ceux-là seuls ne pouvaient être excommuniés, ni
eux-mêmes, ni pour avoir traité avec un excommunié.

Voilà la différence essentielle des ministres des rois à leur choix et
volonté, d'avec les ministres nés par fiefs et dignité de pairie,
ministres indispensables du royaume, comparés par saint Louis aux
électeurs de l'empire, non au droit d'élection des empereurs dans un
royaume héréditaire, mais au droit égal, pareil et semblable des
électeurs dans l'empire et des pairs de France en France, où l'empereur
ni le roi ne pouvaient faire loi, sanction, décision de cause majeure
sans leur intervention et leur avis, qui donnaient seuls force de loi ou
d'arrêt souverain à la sanction ou à la décision de la cause majeure.

Et sur qui le procureur général s'explique-t-il de la sorte\,? Sur
l'exemption de droit de l'excommunication si étendue, si reconnue, si
redoutable alors par les plus grands, sur une exemption nécessaire et
d'un droit inhérent à la couronne\,; c'est sur un pair de France comme
pair de France, quoique pair de France à titre de son siège,
c'est-à-dire à un titre qui, sans le respect de la pairie qui y est
unie, serait, comme évêque, plus en la main du pape et plus soumis à ses
censures que nul autre, sur un pair de naissance incertaine, puisque
c'est un évêque, si loin de l'extraction héréditaire de ces grands
princes et souverains revêtus de pairie, sur un pair qui n'a de commun
avec eux que la dignité de pair, et qui, en proportion de l'étendue des
fiefs et de la puissance territoriale, ne serait à peine que l'aumônier
et le domestique de ces grands et puissants pairs, et toutefois par
cette dignité commune avec eux, le même qu'eux, égal en tout à eux,
pareil à eux en droits, en rang, en pouvoir législatif et constitutif,
en assistance nécessaire aux grandes fonctions de l'État, et par cela
même aussi inviolable qu'eux, et aussi affranchi, par le même et commun
droit, de pouvoir être excommunié, même son archidiacre agissant pour
lui et par ses ordres.

Le procureur général achève de démontrer combien la grandeur de la
dignité de pair si parfaitement semblable, égale, pareille en tout à
celle de ces grands et puissants pairs laïques, est indépendante de
cette grandeur et de cette puissance purement personnelle, lorsqu'il
ajoute que «\,si un pair de France n'avait pas de quoi vivre, le roi
serait obligé de le nourrir.\,» On s'espacerait en vain à prouver qu'il
est jour lorsqu'on voit luire le soleil, on s'efforcerait de même en
vain, après des démonstrations si transcendantes, à vouloir prouver que
les pairs les plus pauvres, les plus dénués d'États et de puissance
territoriale, les plus éloignés de l'extraction illustre de ces grands
et puissants pairs, même souverains, sont leur compairs en tout ce qui
est de la dignité, rang, honneurs, grandeurs, facultés, puissance,
autorité, fonctions de leur commune dignité de pairs de France,
conséquemment qu'en cela même les pairs d'aujourd'hui sont en tout et
partout pairs, tels que ces anciens pairs, d'ailleurs si supérieurs sans
comparaison à eux, puisque l'archevêque de Reims, l'évêque de Laon et
les quatre autres tels dans les anciens temps qu'on les voit
aujourd'hui, ont été sans difficulté égaux en dignité, rang, fonctions,
autorité, puissance législative et constitutive, en un mot, pareils en
tout et parfaitement compairs des ducs de Bourgogne, de Normandie, etc.,
et compairs aussi des pairs érigés depuis dans tous les temps jusqu'à
nous, et les uns et les autres sans aucune diminution de ce qui
appartient à la dignité de pair de France, quoique si dissemblables en
naissance et puissance, et en attributs extérieurs étrangers à la
pairie, à ces anciens pairs, si grands, si puissants, et quelques-uns
rois et souverains.

Les noms si magnifiques lesquels les rois dans leurs diverses érections
de pairies, et dans nombre d'autres actes, et les magistrats dont la
charge est de parler pour eux et en leur nom, donnent dans tous les
siècles aux pairs de France, sont une autre preuve de tout ce qui a été
avancé de la grandeur et des fonctions du rang et de l'être des pairs de
France, comme tels et indépendamment de toute autre grandeur étrangère à
cette dignité en ceux même qui l'ont possédée. Tout y marque le premier
rang dans l'État, et ce pouvoir inhérent et nécessaire en eux seuls, de
faire avec le roi les grandes sanctions du royaume et de juger les
causes majeures. On les voit sans cesse nommés, «\,tuteurs des rois et
de la couronne\,; grands juges du royaume et de la loi salique\,;
soutiens de l'État\,; portion de la royauté\,; pierres précieuses et
précieux fleurons de la couronne\,; continuation, extension de la
puissance royale\,; colonnes de l'État\,; administrateurs, modérateurs
de l'État\,; protecteurs et gardes de la couronne (expression de
l'avocat général Le Maître en un lit de justice de 1487)\,; le plus
grand don et le plus grand effort de la puissance des rois\,» (comme l'a
encore dit et reconnu Louis XIV en propres termes)\,; on ne finirait
point sur ces dénominations dont l'énergie épuise toute explication, et
qui est la plus expresse sur la grandeur du rang, sur l'exercice du
pouvoir législatif et constitutif, et sur l'identité de pairies et de
pairs de tous les siècles et de tous les temps, puisque ces expressions
n'en exceptent aucuns, et qu'elles ne sont que pour les pairs, comme
tels, par la dignité de leur pairie, sans qu'il soit question en eux
d'aucune autre sorte de grandeur, et ce serait tomber en redites, moins
supportables en une digression qu'ailleurs, que s'étendre en preuves sur
une chose si claire et si manifeste.

On se contentera de remarquer que les temps de ces expressions étaient
encore exacts et purs sur ce qu'on voulait faire entendre. Il n'y avait
que la vérité qui portât nos rois et leurs organes à un langage si
magnifique\,; toute exagération, au moins en actes publics et portant le
nom du roi, était encore heureusement inconnue\,; rien que de vrai,
d'exact, de légitime, n'y était donné à personne, et personne n'avait
encore osé y prétendre au delà\,; rien n'y était donc inséré par
flatterie, par faveur, par faiblesse, rien pour fleur, pour éloquence,
pour l'oreille, tout pour réalité effective, existante, tout à la lettre
pour vérité, exactitude, usage\,; et ce n'est que bien des années depuis
que la corruption a commencé à se glisser dans les actes, les
prétentions à y primer, la faiblesse à y mollir, et finalement ce n'est
guère que de nos jours que ceux qui obtiennent des patentes y font
insérer tout ce qui leur plaît de plus faux et de plus abusif à leur
avantage, encore personnel et non de la dignité ou de l'office qui leur
est accordé par la patente. Ainsi les érections ne se sont expliquées
qu'avec justesse, et les magistrats parlant au nom du roi et sous leur
autorité, devenus responsables en leur propre nom aux rois et aux
tribunaux de leurs expressions et de leurs qualifications, se seraient
bien gardés de s'éloigner de la justesse, de la vérité, de la précision
la plus exacte, que les tribunaux ne leur auraient pas passé, et dont
les rois leur auraient fait rendre un compte rigoureux, s'agissant
surtout de termes et d'expressions si intéressant leur personne et leur
couronne, si ces termes et ces expressions n'avaient pas contenu
l'ingénuité et la vérité la plus consacrée, la plus existante et la plus
scrupuleuse.

Il est fâcheux d'allonger tant une digression\,; il le serait encore
plus, sinon de ne pas tout dire, puisque cela est bien éloigné d'être
possible ici, mais de ne pas montrer au moins et indiquer, pour ainsi
dire, ce qu'il est essentiel de ne laisser pas ignorer.

Tout apanage n'est pas pairie, mais toute pairie est tellement apanage,
qu'on voit que pairie et apanage \footnote{Voy., note précédente.} sont
comme synonymes dans la lettre citée de Philippe le Bel sur l'évêché de
Laon, où cela est et se trouve par deux fois. Or nulle différence
d'étendue, ni de puissance de fief entre la pairie de Laon et toutes les
pairies d'aujourd'hui, ni de grandeur personnelle de l'évêque de ce
siège à des pairs d'aujourd'hui.

Cette vérité d'apanage n'a jamais été contestée. Louis XI, si jaloux de
sa couronne et de tout ce qui y appartenait, déclare nettement en 1464,
en l'érection d'Angoulême\,: \emph{Que de toute ancienneté les pairs
tiennent leurs pairies en apanages}\,; et pour couper court là-dessus
d'une manière invincible, il ne faut que jeter les yeux sur l'érection
d'Uzès.

Uzès est une terre ordinaire, son seigneur est seigneur ordinaire\,; ce
n'est ni l'Anjou ni un fils de France, etc. C'est une pairie et un pair
de France qui, par son fief ou son personnel n'a rien que d'autres pairs
existants et postérieurs à lui n'aient pas, et on ne peut s'attacher à
son égard à cette écorce étrangère à la pairie, dont l'éclat éblouit
dans ces anciens pairs si grands en naissance et en puissance, et qui
sert à tromper ceux qui, ne faisant de ce total qu'une seule chose,
voudraient mettre de la différence jusque dans la dignité de pairs et
ses attributs, entre ces pairs si grands par eux-mêmes et leurs compairs
d'aujourd'hui. L'érection d'Uzès manifeste bien expressément l'égalité
parfaite, en dignité de pairie et tout ce qu'elle emporte, dans les
pairs d'aujourd'hui, avec ces anciens pairs d'ailleurs si dissemblables
à eux par des grandeurs et une puissance étrangère à leur dignité de
pair de France, et qui leur était purement personnelle. Uzès par son
érection est donné en apanage au duc d'Uzès, à quoi elle ajoute ces
termes\,: «\,Qu'avenant, à faute de mâle, réversion de cette pairie à la
couronne, ledit duché-pairie pourra tenir lieu d'une partie d'apanage
pour les derniers enfants de France, et être convenable à leur grandeur
et dignité.\,»

Je ne sais quelle expression pourrait être employée pour être plus
positive que celle-ci. Uzès érigé en duché-pairie est donc par cela seul
devenu apanage, et apanage convenable aux derniers enfants de France,
convenable, dis-je, à leur grandeur et dignité, si, à faute de mâle,
Uzès retourne à la couronne. Ainsi rien d'oublié ni pour la qualité et
l'essence d'apanage, ni pour la dignité d'un apanage, puisqu'il est
déclaré convenable à la grandeur et à la dignité des fils de France. Il
n'y a pas d'apparence qu'on puisse objecter qu'il est dit dans
l'érection, \emph{pour partie d'apanage}, puisqu'il ne peut être partie
d'apanage qu'il ne soit apanage par essence, et d'essence à être
convenable à la grandeur et à la dignité des fils de France. Mais
pourquoi partie d'apanage\,? c'est que le duché d'Uzès qui a toute la
dignité convenable à la grandeur d'un fils de France, n'a ni l'étendue
ni le revenu qui puisse suffire à former tout son apanage, comme en plus
grand le duché de Chartres, etc., sont, non l'apanage, mais une partie
de l'apanage qui fut formé à Monsieur frère de Louis XIV, et ainsi de
ceux de tous les fils de France. Et il faut dire des apanages de ces
princes ce qui a été démontré des anciens pairs, dont la grandeur
personnelle a été étrangère à leur dignité de pair de France, et à tout
ce que cette dignité emporte. Aussi un apanage de fils de France est
apanage, mais il a des extensions étrangères à l'apanage, comme des
revenus, des présentations d'offices et de bénéfices, des droits et des
dispositions de commissions qui ne viennent pas de l'apanage, qui ne
sont pas apanage, mais qui sont personnellement attribués à ces princes
pour la grandeur de leur naissance et pour l'entretien de leur cour\,:
toutes choses personnelles à ces princes, et tout à fait étrangères à la
nature et qualité propre de l'apanage.

Enfin il résulte bien nettement que les pairies de France ont toujours
été données aux pairs et possédées par eux dans tous les siècles jusqu'à
aujourd'hui, en apanage, et comme les propres apanages des fils de
France, et cette chaîne, plus d'une fois citée, se perpétue ainsi de
siècle en siècle jusqu'à nos jours pour la dignité, le rang, l'essence,
les fonctions de pairs de France de tous les âges comme tels,
indépendamment de la disparité de personne, de puissance et
d'extraction\,; sur quoi encore les ducs d'Uzès fourniraient des preuves
les plus transcendantes en rang, droits, etc., si on avait loisir de s'y
arrêter ici.

Mais pour ne rien retenir qui puisse laisser la plus petite couleur aux
cavillations les plus destituées même d'apparence, il faut dire que les
érections postérieures à celle d'Uzès portent pour la plupart une
dérogation à la réversion à la couronne de la terre érigée à faute
d'hoirs, et cette clause y est conçue avec tant d'indécence qu'elle
porte que\,: \emph{sans cette dérogation l'impétrant n'aurait voulu
accepter l'érection}. Toute exception de loi la confirme. La maxime
n'est pas douteuse\,; or il ne peut y avoir une exception de loi plus
précise que celle-ci, puisqu'elle est non seulement claire, précise,
formelle, mais puisqu'elle va jusqu'à en exprimer une cause et une
raison même très indécente.

Il est donc vrai que la loi y est nettement confirmée par cette
expression même, et que toutes les pairies dans l'érection desquelles
elle se trouve ne sont dissemblables en rien à toutes celles où elle ne
se trouve pas\,; conséquemment que toutes sont entièrement pareilles,
semblables, égales, et les mêmes par leur nature, et que ce {[}que{]}
Philippe le Bel et Louis XI, pour se contenter ici des citations qu'on y
a vues, ont dit du pair et de la pairie de Laon, est dit et se trouve
parfaitement et pleinement véritable de tous les pairs et de toutes les
pairies d'aujourd'hui\,; d'où il résulte d'une manière invincible que
tout ce qui a été dit, tenu et vu des premiers et plus anciens pairs
sous quelque nom qu'ils aient été connus d'abord, des premiers et plus
anciens pairs dont on n'a point d'érection, des premiers et plus anciens
pairs érigés après eux, et de leurs pairies, se peut et se doit dire des
pairs de tous les temps et de leurs pairies jusqu'à aujourd'hui, quant à
la dignité de pair et de pairie de France, et tout ce qu'elle emporte de
rangs, droits, pouvoir législatif et constitutif, sans exception, sans
distinction, sans différence, sans partage, en un mot dans tous les
temps compairs en tout, indépendamment de la grandeur personnelle
d'extraction et de puissance étrangère à la dignité, commune entre eux
tous, de la pairie de France, dont l'identité en eux tous se suit d'âge
en âge, sans la plus légère interruption de tout ce qui y appartient.

Qu'il y ait des apanages ou plutôt des parties d'apanages qui ne soient
pas pairie de France, car il y a eu peu d'apanages entiers donnés à des
fils de France qui n'eussent point de pairie, qu'il y ait des terres
réversibles à la couronne inféodées sous cette condition qui ne soient
point pairies ni apanages, ce sont choses entièrement étrangères à ce
que l'on traite ici, et qui n'y portent pas la moindre influence. On ne
s'est proposé que de montrer que les pairies d'aujourd'hui, non quant à
l'étendue de fief et à sa puissance, que les pairs d'aujourd'hui, non
quant à la grandeur de l'extraction et des possessions, mais quant à la
dignité de pair et à l'essence de la pairie et à tout ce qui y
appartient, sont égaux, pareils et compairs en tout et partout, sans
différence, exception ni dissemblance aucune, aux pairs de tous les
temps, et leurs pairies aux leurs\,; que ces pairies nouvellement
érigées le sont sur le modèle de toutes les précédentes\,; qu'elles sont
par nature apanage, et réversibles à la couronne, dont l'essence, au
dire de nos rois sur celle d'Uzès, est assez majestueuse pour être
convenable à devenir apanage des fils de France, convenable, dis-je, à
leur grandeur et dignité\,; qu'exception de loi la confirme\,; que Laon
pour les temps les plus reculés, Uzès pour les nôtres, n'ont rien
d'extérieur, même d'étranger à la pairie et aux pairs d'aujourd'hui, et
que conformes en tout, quant à la dignité de pair, à ceux de tous les
temps, tous ceux d'aujourd'hui ont avec eux et ceux de tous les âges une
pareille, semblable et entière conformité.

Or qu'est-ce qu'un apanage\,? Le voici en deux mots. Dans les plus
anciens temps, le royaume de France se partageait en autant d'États
souverains et indépendants que nos rois laissaient de fils, souvent même
de leur vivant. Le désordre et l'affaiblissement qui résulta de ces
partages en corrigèrent, et le fils aîné du roi succéda à la totalité du
royaume. Alors nos rois se trouvèrent à l'égard de leurs puînés dans la
même nécessité que les particuliers de pourvoir à leur subsistance, et
des enfants qui naîtraient d'eux. Nul patrimoine sur quoi la prendre,
puisque celui des rois est réuni à la couronne s'ils en ont lorsqu'ils y
viennent, et s'il leur arrive des héritages depuis qu'ils y sont
parvenus, ces héritages y sont pareillement et de droit réunis. Il faut
donc que les fils de la couronne soient nourris et pourvus par la
couronne, c'est-à-dire des biens de la couronne\,; et comme les biens de
la couronne sont par cela même inaliénables, la portion des biens qui
leur est donnée ne leur est que prêtée, c'est-à-dire qu'ils n'en peuvent
disposer, mais en jouir eux et leurs descendants de mâles en mâles,
pour, à faute enfin de mâle, retourner à la couronne, et c'est ce qui
est connu sous le nom d'apanage.

De là il est aisé de conclure de quelle dignité est un bien donné en
apanage, puisqu'il brille d'un rayon de la couronne même, qui se répand
sur son possesseur\,; et quel nouveau jour donne à ce qui a été dit
jusqu'ici de la dignité de pair et de la pairie de France, des noms
donnés aux pairs, etc., ce qu'on a cité de nos rois qui déclarent en
divers temps que pairie et apanage sont synonymes, et que de tous les
temps les pairies sont apanages, et récemment encore du duché d'Uzès.
Enfin, il faut ajouter à cette réflexion naturelle ce que nos rois
jusqu'à Louis XIV inclusivement ont dit des pairs et des pairies, et
leur aveu que c'est le plus grand effort de leur puissance et ce qu'ils
peuvent faire et donner de plus grand. Cela est dit par eux
indépendamment de la qualité d'apanage inhérente, comme on l'a vu, par
nature à la pairie. Joignant ensemble l'idée qui naît de la réunion de
ces deux choses en la même, quelle splendeur et quelle majesté\,! Aussi
nos rois n'ont-ils pu faire plus pour leurs fils puînés et pour leurs
frères jusqu'à aujourd'hui, ni pour les princes de leur sang, quoique si
singulièrement grands par le majestueux effet qu'ils reçoivent de la loi
salique, que de les faire et déclarer tous pairs de France par le droit
de leur naissance auguste, sans avoir même de pairie, et précédant tous
autres pairs. C'est ce que fit Henri III, avec d'autant plus de justice
qu'il était très indécent que des princes que leur naissance appelait à
la couronne, le cas en arrivant, fussent précédés par les aînés des
branches cadettes à la leur, qui ne pouvaient succéder qu'après eux, et
par des pairs qui pouvaient devenir leurs sujets sans à voir eux-mêmes
aucun droit de succession à la couronne.

Si, au lieu d'une digression forcée, et par là même si nécessairement
abrégée qu'elle en est comme mutilée, c'était ici un traité, l'occasion
deviendrait toute naturelle de parler des ducs non pairs vérifiés au
parlement, et apprendre à bien des gens qui se persuadent qu'ils sont de
l'invention du feu roi, que cette dignité est connue, dès 1354 au moins,
distinctement, par l'érection du duché de Bar en faveur de Robert, duc
de Bar, dont la maison est connue dès l'an 1044 par Louis, comte de
Montbéliard, de Mouson et de Ferrette, qui eut le comté de Bar par son
mariage avec Sophie, deuxième fille de Frédéric II, duc de la haute
Lorraine, et de Mathilde de Souabe dont la postérité prit le nom de Bar,
et dont le dixième descendant, Robert, épousa en 1364 Marie, fille de
notre roi Jean et de Bonne de Luxembourg.

Il en eut Henri, Philippe, Édouard, Louis, Charles et Jean, et quatre
filles, dont Yolande fut l'aînée. Henri fut père de Robert qui mourut
sans enfants, comme tous ses oncles, et fut comme le dernier de cette
maison. Louis fut évêque-duc de Langres, évêque-comte de Châlons, et
évêque de Verdun, et cardinal\,: il survécut tous ses frères et son
neveu. Yolande, l'aînée de ses sœurs, épousa Jean d'Aragon, fils de
Pierre IV, roi d'Aragon, et d'Éléonore de Portugal. Jean devint roi de
Portugal, et Yolande, sa femme, mourut à Barcelone en 1431. Elle laissa,
entre autres enfants, Yolande d'Aragon, qui, de son mariage avec Louis
II, duc d'Anjou, roi de Naples et de Sicile, eut le bon roi René, duc
d'Anjou, roi de Naples et de Sicile, auquel Louis, cardinal de Bar, son
grand-oncle maternel, duc de Bar et le dernier mâle de sa maison, fit
don du duché de Bar. Yolande d'Anjou, fille du roi René, et duchesse de
Lorraine par sa mère Isabelle, fille aînée et héritière de Charles Ier,
duc de Lorraine, et de Marguerite de Bavière, porta les duchés de
Lorraine et de Bar en mariage, en 1444, à Ferry de Lorraine, comte de
Vaudemont, son cousin, duquel mariage sont sortis tous les ducs de
Lorraine.

Ces ducs, quoique souverains et de maison si distinguée, tinrent
tellement à honneur la dignité de ducs de Bar, quoique comme tels
vassaux de la couronne de France, qu'ils en prirent les marques qu'ils
n'ont quittées que longtemps depuis, et on voit encore sur les portes de
Nancy leurs armes ornées du manteau ducal, que j'y ai vues et remarquées
moi-même.

Valentinois fut érigé de même sans pairie et vérifié en 1498, pour le
fameux César Borgia, si connu par ses crimes et par le feu que, pour son
agrandissement, le pape Alexandre VI, dont il était bâtard, alluma tant
de fois par toute l'Europe\,; Longueville en 1505, et d'autres en faveur
de princes de la maison de Savoie comme Nemours, et de princes du sang
comme Estouteville. On ne s'arrêtera pas à en citer davantage, mais on
remarquera qu'il y en a toujours eu depuis en existence, et que
Longueville, par exemple, etc., ne se sont éteints que depuis l'érection
pareille de La Feuillade et autres par Louis XIV.

Ainsi on voit deux choses\,: l'antiquité de ces sortes de duchés non
pairies vérifiés, et la grandeur de ceux en faveur de qui ils ont été
érigés, parmi lesquels, outre Bar, on compte des princes des maisons de
Lorraine et de Savoie, des bâtards de France et la maison de
Longueville, de très grands seigneurs français et étrangers, et plus que
tout cela un prince du sang. Aussi, quant à la dignité des fiefs et de
l'apanage, ces duchés sont égalés aux pairies, mais sans office, qui est
de plus en la pairie qui donne aux pairs ces grandes fonctions qu'on a
touchées, et leur a acquis ces grands noms que les rois leur ont
donnés\,: comme l'état de la dignité de duc vérifié est étrangère à la
cause de cette digression, on ne la grossira pas des raisons qui
montrent que les ducs vérifiés, et que l'usage nomme héréditaires, sont
ce qu'étaient les hauts barons.

Mais pour ne laisser aucune des trois sortes de ducs connus en France
sans quelque explication, puisqu'elle se présente si naturellement ici,
j'ajouterai un mot des ducs non vérifiés, que l'usage appelle mal à
propos à brevet, puisqu'ils n'ont point de brevet, mais des lettres
comme les autres qui ne sont point vérifiées, et qui, par conséquent,
n'opèrent rien de réel ni de successif, mais de simples honneurs de
cour, sans rang et sans existence dans le royaume. C'est à ceux-là
seulement que les officiers de la couronne disputent à raison de leurs
offices réels et existants dans l'État, contre de simples honneurs de
similitude, sans fief ni office, sans caractère, rang, ni existence dans
le royaume. C'est encore de ceux-là que le cardinal Mazarin disait
insolemment qu'il en ferait tant qu'il serait honteux de l'être et de ne
l'être pas, et néanmoins se le fit lui-même.

On est tombé dans la même erreur sur leur origine, qu'à l'égard des ducs
vérifiés, on les a crus de l'invention de la minorité de Louis XIV. À la
vérité, pour ceux-ci il serait peut-être difficile de les trouver plus
haut que François Ier\,; aussi ne sont-ils rien dans l'État, mais
Roannez fut duché de la sorte sous ce règne. On vit ensuite de même
Dunois pour la maison de Longueville, Albret en faveur d'Henri, roi de
Navarre\,; Brienne pour Charles de Luxembourg, beau-frère du duc
d'Épernon, et quantité d'autres pour de fort grands seigneurs français
et étrangers\,; et de ces ducs non vérifiés il y en a toujours eu
jusqu'à présent, et le duc de Chevreuse, grand chambellan, dernier fils
du duc de Guise tué à Blois, a été longues années duc de cette dernière
sorte avant d'être fait duc et pair.

Les officiers de la couronne n'ont aucune part à la cause de cette
digression, et ce serait en abuser que d'en parler ici. Quelque grands
que soient leurs offices, dès deux premiers surtout, ils n'ont ni
l'universalité ni la majesté de l'office de pair de France, et les
preuves n'en sont pas difficiles. Leur office de plus n'est qu'à vie, et
de fief comme office de la couronne ils n'en ont point, quoiqu'on trouve
des foi et hommage quelquefois rendus à nos rois pour ces offices, mais
sans nulle mention de fief.

Ainsi les pairs ont le plus grand fief et le plus grand office qu'un roi
de France puisse donner, et dont un vassal, même fils de France, encore
plus un sujet, puisse être revêtu. Un duc vérifié a le fief sans
l'office, ce qui met une grande distinction du pair à lui, et de lui à
l'officier de la couronne qui n'a qu'un office et à vie, et sans fief,
mais office très inférieur en tout à celui de pair de France, tellement
même que les ducs non vérifiés qui n'ont ni fief ni office, rien de réel
dans l'État, qui n'ont que des honneurs extérieurs et l'image des autres
ducs dont ils ne sont qu'une vaine et fictive écorce, ne cèdent point à
raison de cette image sans réalité qui est en eux, ne cèdent point,
dis-je, aux officiers de la couronne, qui n'ont pas comme eux cet
extérieur de ressemblance aux autres ducs, quoique vaine. Aussi ne
veulent-ils point céder à ces ducs non vérifiés à raison de leurs
offices et de ce qu'ils sont réellement dans l'État, tellement que la
compétence est entre eux continuelle, et qu'aux cérémonies de cour, car
ces ducs non vérifiés n'ont point de places aux autres, ils marchent
mêlés ensemble, comme le roi le prescrit, ce qui toujours, en tous les
temps, a été réglé de même.

Après avoir montré aussi brièvement qu'il a été possible quelle est la
dignité de duc et pair dans tous les âges de la monarchie jusqu'à ceux
qui en sont revêtus aujourd'hui, il faut essayer de faire voir aussi ce
que c'est que le parlement de Paris et les autres formés sur son modèle,
et tâcher de le faire avec la même évidence et la même brièveté, et
c'est l'autre partie de la digression indispensable pour faire entendre
ce qu'il s'agira ensuite de rapporter.

\hypertarget{chapitre-xvii.}{%
\chapter{CHAPITRE XVII.}\label{chapitre-xvii.}}

~

{\textsc{Parlement de Paris et les autres sur son modèle.}} {\textsc{-
Leur origine\,; leur nature\,; d'où nommés parlements.}} {\textsc{-
Récapitulation abrégée.}} {\textsc{- Ancien gouvernement.}} {\textsc{-
Légistes.}} {\textsc{- Conseillers\,; d'où ce nom.}} {\textsc{- Légistes
devenus juges.}} {\textsc{- Origine et monument des hauts et bas
sièges.}} {\textsc{- Parlement, par quels degrés prend la forme
présente.}} {\textsc{- Pairs seuls des nobles conservent voix et séance
au parlement toutes fois qu'ils veulent en user.}} {\textsc{- Préséance
des pairs en tous parlements\,; y entrent seuls de nobles avant le roi
lorsqu'il y vient, et pourquoi.}} {\textsc{- Le chancelier seul des
officiers de la couronne aux bas sièges aux lits de justice, et n'y
parle au roi qu'à genoux, seul d'entre eux non traité par le roi de
cousin, et seul de la robe parle et y opine assis et couvert.}}
{\textsc{- Pourquoi toutes ces choses.}} {\textsc{- Origine de la
présidence et de sa prétention de représenter le roi.}} {\textsc{-
Séance des présidents en tout temps à gauche de celle des pairs.}}
{\textsc{- Origine de l'enregistrement des édits, etc., aux
parlements\,; d'y juger les causes majeures, etc., et du titre de cour
des pairs affecté par celui de Paris.}} {\textsc{- Nécessité de la
mention de la présence des pairs aux arrêts des causes majeures et aux
enregistrements des sanctions.}} {\textsc{- Origine de la prétention des
parlements d'ajouter par les enregistrements un pouvoir nécessaire.}}
{\textsc{- Origine des remontrances, bonnes d'abord, tournées après en
abus.}} {\textsc{- Entreprises de la cour de Rome réprimées par le
parlement\,; ne lui donnent aucun droit de se mêler d'autres affaires
d'État ni de gouvernement.}} {\textsc{- Parlement uniquement compétent
que du contentieux entre particuliers\,; l'avoue solennellement sur la
régence de M\textsuperscript{me} de Beaujeu.}} {\textsc{- Cour des pairs
en tout lieu où le roi les assemble.}} {\textsc{- Enregistrements des
traités de paix faits au parlement uniquement pour raison purement
judicielle.}} {\textsc{- Régence de Marie de Médicis est la première qui
se soit faite au parlement, et pourquoi.}} {\textsc{- Époque de sa
prétention de se mêler des affaires d'État et de cette chimère de
tuteurs des rois, qui les ont continuellement réprimés à tous ces
égards.}} {\textsc{- Précautions de Louis XIII à sa mort aussi
admirables qu'inutiles, et pourquoi.}} {\textsc{- Régence d'Anne
d'Autriche\,; pourquoi passée au parlement.}} {\textsc{- Avantages
dangereux que la compagnie en usurpe, que Louis XIV réprime durement
depuis.}} {\textsc{- Régence de M. le duc d'Orléans au parlement se
traitera en son temps.}} {\textsc{- Duc de Guise qui fait tout pour
envahir la couronne, est le premier seigneur qui se fait marguillier, et
pour plaire au parlement, laisse ajouter à son serment de pair le terme
de conseiller de cour souveraine.}} {\textsc{- Dessein du parlement dès
lors à l'égard des pairs.}} {\textsc{- Le terme de conseiller de cour
souveraine ôté enfin pour toujours du serment des pairs.}} {\textsc{-
Nécessité d'exposer un ennuyeux détail.}} {\textsc{- Ordre et formes de
l'entrée et de la sortie de séance aux bas sièges.}} {\textsc{-
Présidents usurpent nettement la préséance sur les princes du sang et
les pairs à la sortie de la séance des bas sièges.}} {\textsc{- Ordre et
formes d'entrer et de sortir de la séance des hauts sièges.}} {\textsc{-
Séance, aux lits de justice, des pairs en haut qui opinent assis et
couverts, et les officiers de la couronne aussi\,; des présidents et
autres magistrats en bas, qui opinent découverts et à genoux, et du
chancelier en bas, qui ne parle au roi qu'à genoux, parce qu'il est
légiste, mais opine et prononce assis et couvert, parce qu'il est
officier de la couronne.}} {\textsc{- Présidents usurpent d'opiner entre
la reine régente et le roi\,; sont remis à opiner après le dernier
officier de la couronne en 1664\,; ce qui a toujours subsisté depuis.}}
{\textsc{- Changement par entreprise et surprise de la réception des
pairs, des hauts sièges où elle se faisait, aux bas sièges où elle est
demeurée depuis 1643.}} {\textsc{- Contraste de l'état originel des
légistes dans les parlements avec leurs usurpations postérieures.}}
{\textsc{- Efforts et dépit des présidents en 1664 et depuis.}}
{\textsc{- Novion, premier président, ôté de la place pour ses
friponneries, jaloux de l'élévation des Gesvres.}}

~

Pour prendre une idée juste de l'essence et de la nature de cette
compagnie, il faut se souvenir de ce qui a été dit des légistes, de la
façon de rendre les jugements, et des trois corps qui forment la
nation\,; que chacun était jugé par ses égaux\,; que les grands vassaux
jugeaient les leurs, chacun dans son fief avec les principaux
feudataires qui en relevaient\,; et que les grands et immédiats
feudataires de la couronne, connus dès la fondation de la monarchie et
sous divers noms, enfin pairs de France, jugeaient les grandes causes et
les affaires majeures avec le roi, et avec lui exerçaient le pouvoir
législatif et constitutif pour les grandes sanctions de l'État\,; ce que
c'était que les hauts barons et les grands prélats, et qu'ils y étaient
quelquefois, puis toujours appelés mais personnellement tantôt les uns,
tantôt les autres, par le roi, en sorte qu'ils ne tiraient leur droit
que de ce que le roi les mandait, ainsi que depuis les officiers de la
couronne dont on avait besoin pour ce qui regardait leurs offices, au
lieu que les pairs y venaient tous de droit, et que rien ne se pouvait
faire sans eux\,; que les procès se multipliant sans cesse depuis que
les fiefs eurent, contre leur originelle nature, passé aux femmes,
furent devenus susceptibles de partages, de successions, d'hypothèques,
et que les coutumes diverses sur toutes ces choses se furent introduites
par usages dans les différentes provinces, que les ordonnances se furent
accumulées, ce qui causa la multiplication des parlements aux
différentes fêtes, qui duraient huit, dix, quinze jours pour vider ces
procès\,; que saint Louis, qui aimait la justice, considérant le peu de
lumière que ces juges si nobles et si occupés de la guerre pouvaient
apporter au jugement de tant de questions embarrassées, et de coutumes
locales différentes, mit à leurs pieds des légistes pour être à portée
d'en être consultés en se baissant à eux, sans toutefois qu'ils fussent
obligés de le faire, ni, le faisant, de se conformer à leur avis ignoré
de toute la séance, et qu'ils ne disaient qu'à l'oreille du seigneur aux
pieds duquel ils se trouvaient assis quand il voulait les consulter, et
que c'est de là que ces logistes ont été dits conseillers\,; que le
peuple, esclave par sa nature, peu à peu affranchi, puis devenu en
partie propriétaire par la bonté des seigneurs dont ils étaient serfs,
forma la bourgeoisie et le peuple, et ceux qui eurent des fonds appelés
rotures, parce qu'ils ne pouvaient posséder de fiefs, furent de là
appelés roturiers\,; que de ce peuple affranchi, ceux que leur esprit et
leur industrie éleva au-dessus de l'agriculture et des arts mécaniques,
s'appliquèrent aux coutumes locales, à savoir les ordonnances et le
droit romain, qui demeura en usage en plusieurs provinces après la
conquête des Gaules, et y a été depuis toujours pratiqué. Ces gens-là se
multiplièrent avec les procès, s'en firent une étude, devinrent le
conseil de ceux qui en avaient, et des familles pour leurs affaires. De
leur application aux lois, dont ils se firent un métier, ils furent
appelés \emph{légistes}, et saint Louis en appela aux parlements pour
s'asseoir sur le marchepied des bancs des juges qui étaient tels qu'on
l'a expliqué, pour y être à portée de leur donner à l'oreille les
éclaircissements sur ce qu'il s'agitait devant eux, et former leur
jugement et leur avis, quand ces seigneurs croyaient en avoir besoin, et
se baissaient à eux pour le leur demander. {[}Il faut se souvenir{]} que
de là les procès se multipliant de plus en plus, et par conséquent ces
assemblées pour les juger qui de parler ensemble avaient comme les
grandes assemblées pour les causes majeures et pour les grandes
sanctions de l'État, et par même raison de \emph{parler ensemble},
avaient pris le nom de \emph{parlement}, les seigneurs, tant pairs qui y
étaient de droit, que ceux que le roi y appelait nommément, s'excusèrent
souvent par l'embarras des guerres ou de leurs affaires\,; alors la
nécessité de vider les procès fit donner voix délibérative en leur
absence en nombre suffisant à ces mêmes légistes, qui, profitant de
l'absence de vrais juges auxquels la nécessité les faisait suppléer,
usèrent des temps, et obtinrent voix délibérative avec eux, mais
néanmoins toujours séants à leurs pieds sur le marchepied de leurs
bancs.

Voilà comme de simples souffleurs, et consultés à pure volonté, et sans
parole qu'à l'oreille des juges seigneurs, ces légistes devinrent juges
eux-mêmes avec eux. De là, comme on l'a dit, cette humble séance leur
devenant fâcheuse, ils usurpèrent de mettre un dossier entre les pieds
des seigneurs et leur dos, puis d'élever un peu ce marchepied du banc
des seigneurs qui leur servait de siège, et d'en former doucement un
banc. Telle est l'origine des hauts sièges et des bas sièges de la
grand'chambre, et après elle des grand'chambres des autres parlements
formés dans les provinces sur ce premier modèle, qui tous n'eurent
d'abord qu'une seule chambre chacun, qui depuis la multiplication des
procès et des juges, ont multiplié les chambres, d'où la première,
auparavant unique, a été nommée en toutes {[}les cours{]} la
\emph{grand'chambre}, pour la distinguer des autres.

Il faut encore se souvenir que ces parlements, dont les juges légistes
changeaient à chaque parlement de Pâques, la Toussaint, etc., et les
seigneurs aussi qui n'étaient point pairs, et que le roi y mandait
nommément, seigneurs et légistes, durèrent jusqu'aux troubles des
factions d'Orléans et de Bourgogne sous Charles VI. Ses fréquentes et
longues rechutes, qui ne lui permettaient pas de choisir les membres de
ces parlements, en livraient la nomination à celle des deux factions qui
lors avait le dessus. Les désordres qui en naquirent firent changer
l'usage jusqu'alors observé\,; et pour ne retomber plus à chaque
parlement dans le même inconvénient, il fut réglé que les mêmes membres
le demeureraient à vie, et qu'il n'y en serait mis de nouveaux que par
mort de ceux qui s'y trouvaient, et que c'est l'époque qui a rendu les
légistes juges uniques de fait, parce que, ne s'agissant plus de donner
une quinzaine ou trois semaines en passant à juger des procès, les
seigneurs et les nobles que les rois y avaient jusque-là nommément
appelés à chaque tenue, tantôt les uns, tantôt les autres, ne purent
quitter l'exercice des armes, ni leurs affaires domestiques, pour passer
leur vie à juger en toutes ces diverses tenues de parlement, se
retirèrent presque tous, et laissèrent les légistes remplir leurs places
qui n'avaient rien mieux à faire. Parmi eux l'Église y conserva des
clercs, d'où sont venus les \emph{conseillers-clercs}, pour y veiller à
ses intérêts, mais de même étoffe que ces légistes, par ce que les
évêques et les grands prélats, occupés de leur résidence, souvent de
grandes affaires, et même de la guerre, ne purent donner leur temps à
ces fréquentes assemblées, et comme la noblesse les abandonnèrent.

Ainsi les légistes devenus juges, et par le fait seuls juges, juges à
vie, s'accréditèrent. Les malheurs de l'État et les pressants besoins
d'argent engagèrent nos rois à en tirer d'eux, pour d'une fonction à vie
en faire des offices, et finalement des offices héréditaires et vénaux.
Voilà donc ces juges devenus des magistrats en titre, et ces magistrats,
par les mêmes besoins de finances, ont été accrus et augmentés jusqu'à
la foule qu'on en voit aujourd'hui, qui peuplent Paris, les provinces
sous différents noms, en divers tribunaux supérieurs et subalternes.
Enfin le parlement, rendu sédentaire à Paris, agrandit ses membres
légistes, et jugeant non plus par convocations diverses dans l'année,
mais tout le long de l'année, acquit une dernière stabilité qui en fit
une compagnie de magistrats, modèle sur lequel la commodité des
plaideurs éloignés, et le nombre des procès accru à l'infini, fit former
les autres parlements les uns après les autres\,; et de là, comme on l'a
dit, par le besoin de finances, vint l'idée et l'exécution de tant de
créations de tribunaux partout, supérieurs et inférieurs de tant de
sortes, et de cette foule d'offices vénaux et héréditaires de la robe.

Les légistes devenus par tous ces divers degrés les seuls qui formèrent
le parlement, devenu perpétuel et sédentaire à Paris, et eux officiers
en titre vénal et héréditaire, délivrés des nobles qui avaient quitté
l'écritoire passagère dès qu'elle devint continuelle, et des
ecclésiastiques considérables qui comme les nobles n'y étaient plus
appelés par les rois comme avant Charles VI, n'eurent plus que les pairs
avec eux, qui de droit et sans y être appelés par les rois, à la
différence des hauts barons, des officiers de la couronne, des prélats
et des nobles en quelque nombre, et nommément à chaque parlement, et
jamais les mêmes, y entraient et y jugeaient toutes les fois qu'il leur
plaisait de s'y trouver. C'est de là qu'ils y ont conservé leur entrée
et leur voix délibérative toutes les fois qu'ils y veulent prendre
séance, tant au parlement de Paris que dans tous les parlements du
royaume, où ils précèdent sans difficulté le gouverneur de la province,
et l'évêque diocésain, s'ils s'y trouvent avec eux.

De là encore cette différence d'entrer en séance au parlement avant
l'arrivée du roi, lorsqu'il y vient\,; tandis que les officiers de la
couronne, et tous autres qu'il plaît au roi de mander pour son
accompagnement, ne peuvent entrer en séance qu'à sa suite et après lui,
encore que les officiers de la couronne y seoient aux hauts sièges, avec
voix délibérative, privativement aux gouverneurs et lieutenants généraux
des provinces, et aux chevaliers de l'ordre mandés par le roi, qui
seoient en bas, et n'ont point de voix, et c'est un reste de ce qui a
été dit de ces anciennes assemblées où les pairs seuls assistaient de
droit, longtemps seuls, puis ceux des hauts barons que les rois y
mandaient, etc. Et ce qu'il ne faut pas oublier, c'est qu'encore que les
officiers de la couronne aient leur séance aux hauts sièges, le seul
chancelier a la sienne en bas, comme il a été dit plus haut, parce
qu'encore qu'il soit le second officier de la couronne, et si
considérable en tout, et là même en son triomphe de chef de la justice
et de présider sous le roi, il n'est que légiste et maintenant
magistrat, et comme tel ne peut avoir séance aux hauts sièges. La même
raison le prive de traitement de \emph{cousin} que nos rois donnent non
seulement aux ducs-pairs et vérifiés, mais aussi aux ducs non vérifiés,
et à tous les autres officiers de la couronne.

Le parlement ainsi devenu sédentaire et perpétuel toute l'année, les
légistes, devenus à vie, puis en titre, et héréditaires, furent non
seulement juges et magistrats, mais les seuls qui composèrent le
parlement, à l'exclusion de tous autres nobles que les pairs, et comme
c'était une cour de justice, destinée aux jugements des procès devenus
sans nombre, les pairs ne s'y trouvèrent guère que pour des cas
extraordinaires\,; ainsi ces magistrats, seuls maîtres du lieu,
montèrent aux hauts sièges, dont l'usage se soutint insensiblement même
en la présence des pairs.

La forme des procédures se multiplia avec les procès, et la chicane, qui
la rendit d'abord nécessaire, se nourrit dans la suite de ses
diversités, dont l'une et l'autre se multiplia à l'infini, d'où naquit
un langage particulier dans les requêtes et dans les arrêts, qui rendit
le prononcé de ces derniers difficile souvent aux magistrats moins
experts, et à tous autres impossibles. De là le président de l'assemblée
continua d'en faire la fonction en présence des pairs, puis en titre,
comme les légistes de simples consulteurs étaient devenus magistrats.

De cette présidence en titre et de ce que la justice se rend au nom du
roi, vint l'idée de le représenter à celui qui exerçait cet office, puis
la prétention qui, à la longue, s'est consolidée, parce que personne n'a
pris garde à ce qui en pouvait résulter dans des personnes qui savaient
user au point qu'on le voit déjà de l'art de s'accroître et de s'élever.

Dans la suite les autres présidents que le besoin de finance fit créer,
et qui, du bonnet particulier qu'ils portaient et qu'ils ont accru
jusqu'à ne pouvoir plus le mettre sur leur tête et se contenter de le
tenir à la main, ont été connus sous le nom de \emph{présidents à
mortier}, ont prétendu ne faire avec le \emph{premier président} qu'un
seul et même président, ou un seul et même corps de présidence, et
conséquemment à lui, être tous ensemble les représentants du roi, et
avec le même succès.

Néanmoins avec toute cette représentation prétendue ils n'ont de banc
distingué des conseillers qu'en bas, où il n'y a qu'eux qui seoient\,;
car en haut les conseillers seoient de suite après eux sur leur même
banc\,; et tant en haut qu'en bas, ils n'occupent que le côté gauche, et
les pairs le côté droit. Lorsqu'il n'y a point de pairs séants, les
conseillers l'occupent entier, outre ceux qui sont sur le banc des
présidents, qui se sont bien gardés de changer de côté, pour éviter de
le céder aux pairs lorsqu'il en vient au parlement. Ces côtés droit et
gauche seront encore expliqués plus bas.

Voilà donc les magistrats présidents en titre, et qui exercent la
présidence en présence même du Dauphin, du régent quand il y en a, et
qui ne la cèdent qu'au chancelier de France, ou au garde des sceaux,
quand il y en a un, et que le chancelier ne s'y trouve pas. Ce progrès
suivit de fort près l'expulsion des prélats et des nobles.

L'ancienne forme d'être jugé chacun par ses pairs de fief, etc., étant
ainsi changée par l'établissement successif des parlements convoqués par
le roi en divers temps de l'année, puis peu à peu devenus tels par
degrés, de la manière qui vient d'être expliquée, les édits, ordonnances
et déclarations des rois ne purent plus être promulgués par les grands
feudataires, qui ne tenaient plus de cour de fief. Il fallait toutefois
qu'elles fussent connues pour être observées. Elles ne le pouvaient donc
plus être que par le moyen des assemblées de ces parlements en
différents temps de l'année, convoqués par les rois\,; et par leur
changement en parlement fixe, sédentaire, continuel, par ce tribunal\,;
et dans la suite par les autres parlements, chacun pour leur ressort,
qui furent érigés à l'instar de celui de Paris dans les différentes
provinces, pour le soulagement des plaideurs et l'expédition des procès.

De là vint l'usage de juger les causes majeures et de promulguer les
grandes sanctions au parlement de Paris, d'abord unique, puis devenu le
premier, séant dans la capitale, et le plus à portée des rois et des
grands du royaume. Les légistes qui le composaient, devenus juges et
magistrats, et, comme on l'a vu, juges même en présence des pairs et du
roi même, le demeurèrent dans ces grandes occasions\,; et de là ce
parlement, privativement aux autres du royaume, prit peu à peu le nom et
le titre de \emph{cour des pairs}.

Il est vrai qu'ils n'ont jamais prétendu être compétents des causes
majeures, ni de connaître des grandes sanctions seuls et sans
l'intervention des pairs, en qui seuls par nature en réside le droit,
mais par concomitance avec eux, et y participant par le bénéfice de leur
présence\,; et c'est ce qui en ces grandes occasions a fait charger les
arrêts et les enregistrements de ces paroles consacrées qui leur donnent
toute leur force et leur valeur, \emph{la cour suffisamment garnie de
pairs}, paroles qui ont assez souvent passé dans les arrêts et les
enregistrements communs lorsqu'il s'y trouvait des pairs.

De cet envoi des édits, ordonnances, déclarations des rois, lettres
patentes, etc., au parlement pour qu'elles fussent connues et observées,
et que le parlement y conformât ses jugements dans les affaires qui y
auraient trait, les troubles de l'État donnèrent lieu au parlement de
s'enhardir, et de prétendre qu'ils étaient un milieu entre le roi et son
peuple, qu'ils étaient les protecteurs, les gardiens et les
conservateurs de ce peuple, et que, lorsqu'il se trouvait foulé par des
édits, c'était au parlement à en faire au roi des remontrances.

L'usage qui s'en était introduit sur des matières de règlement purement
légales, où le parlement éclaircissait et redressait souvent par ses
représentations ce qui n'était pas assez clair, ou assez conforme au
droit commun ou public dans ces édits, etc., lui donna lieu aux
remontrances sur les édits bursaux, à former la prétention que je viens
de dire, à la confirmer, par l'usage où les rois avaient eux-mêmes peu à
peu mis le parlement de faire de son autorité, contre les entreprises de
la cour de Rome, et quelquefois même contre les entreprises de quelques
évoques du royaume, ce que la politique du temps ne leur permettait pas
de faire par eux-mêmes, d'où le parlement s'arrogea l'autorité
populaire, à laquelle celle de la police le conduisit comme par la main.
L'abus des favoris, la mauvaise administration des finances, la
faiblesse des règnes et des conjonctures, lui donnèrent beau jeu d'en
profiter, et de s'acquérir les peuples, pour le soulagement desquels il
semblait combattre en établissant son autorité.

De là ils vinrent à prétendre que les édits, etc., ne leur étaient pas
simplement envoyés pour être rendus notoires, pour que chacun les connût
et les observât, et pour que le parlement même y conformât ses
jugements. Ils osèrent prétendre un pouvoir concurrent, et prépondérant
à celui du roi dans l'effet des édits, ordonnances, déclarations,
lettres patentes, etc., qui leur étaient portées à \emph{enregistrer},
d'où ils changèrent ce terme dans l'usage de parler en celui de
\emph{vérifier}, et celui d'\emph{enregistrement} en
\emph{vérification}, parce que le parlement ne feignit plus de prétendre
que ce n'était que par l'autorité de leur enregistrement que ces lois
pouvaient avoir lieu, sans quoi elles demeuraient inutiles, caduques et
sans exécution, tellement que c'étaient eux qui par leur enregistrement
les rendaient vraies lois, et, les rendant telles, les rendaient vraies
et effectives, par conséquent les vérifiaient et en rendaient
l'exécution nécessaire, et en mettaient l'inobservation sous les peines
de droit, qui sans cela ne serait sujette à aucune peine, et la
désobéissance permise et soutenue comme à chose non intervenue ni
arrivée. Les édits bursaux forent d'un grand usage au parlement pour
établir cette autorité. En les refusant, ils s'acquirent les peuples,
qui trouvèrent une protection contre les impôts\,; ils s'assurèrent les
envieux des favoris et des ministres, ils se dévouèrent les ambitieux
qui voulurent brouiller l'État et faire compter avec eux.

Quoique les rois se soient toujours écriés contre ce prétendu concours
de puissance, les temps fâcheux la leur ont fait essuyer presque
continuellement dans le fait, et tout est plein dans les histoires de
cette lutte où les rois ne demeuraient vainqueurs que par adresse, par
manège, et souvent en gagnant les plus accrédités du parlement par des
grâces pécuniaires.

Cette nouvelle puissance, si hardiment usurpée, quoique sans être
consentie, mit les rois en brassière avec l'appui de tout ce qui
craignait l'abus des favoris et des ministres, et accoutuma les plus
grands de l'État à y recourir quand ils se croyaient lésés, dans les cas
les plus majeurs, et qui n'avaient aucun trait, je ne dis pas seulement
à la compétence du parlement, mais à ses usurpations.

Jamais il n'avait osé lever les yeux jusqu'à s'arroger rien sur les
régences. Le duc d'Orléans, depuis roi sous le nom de Louis XII, piqué
d'en être exclu quoique le plus prochain mâle du sang royal, et d'en
voir une femme revêtue par la volonté de Louis XI mourant et le
consentement de ceux à qui il appartenait de le donner, en faveur de la
dame de Beaujeu, sa fille, sœur fort aînée de Charles VIII, mineur,
adressa ses plaintes au parlement. Il lui répondit, par la bouche du
premier président de La Vacquerie, ces célèbres paroles si connues et si
exactement transcrites dans toutes les histoires\,: «\, que le parlement
était une cour de justice établie seulement pour administrer la justice
au nom du roi à ses sujets, non pour se mêler des affaires d'État et des
grandes sanctions du royaume, si ce n'était par très exprès commandement
du roi,\,» par quoi le duc d'Orléans ne put pas seulement se faire
écouter, et de là prit les armes avec le triste succès pour lui que
chacun sait\footnote{Louis d'Orléans fut vaincu et fait prisonnier à la
  journée de Saint-Aubin du Cormier, en 1468.}.

Ce témoignage si authentique du premier président de La Vacquerie en
plein parlement, et magistrat illustre par le poids de ses mœurs et de
sa doctrine, est une vérité dont l'évidence et la notoriété de droit et
de fait a paru trop pesante à ses successeurs, et à ceux qui dans les
suites ont succédé aux autres offices du parlement.

Les anciennes usurpations conviaient à de nouvelles, aussi le parlement
trouva-t-il bien mauvais de n'avoir nulle part aux régences de Catherine
de Médicis, et cria-t-il aussi haut que vainement de ce qu'elle fit au
parlement de Rouen, avec les pairs et les officiers de la couronne, la
déclaration de la majorité de Charles IX, et avec cette nouveauté que ce
prince ne faisait qu'entrer en sa treizième année, qui fut dès lors pour
toujours à l'avenir réputée révolue dès qu'elle serait commencée dans
les rois mineurs, ce qui était en effet moins une interprétation du
règlement de Charles V, approuvé et fait avec lui par tous les grands de
l'État, qui fixe la majorité à quatorze ans pour les rois, qu'un
changement et une nouvelle loi entée sur l'ancienne.

Le parlement de Paris députa. Il lui fut répondu que la cour des pairs
n'avait point de lieu, qu'elle était partout où il plaisait au roi
d'assembler les pairs, et comme il est vrai. Le parlement de Paris
demeura sans action comme sans réponse, et n'a osé renouveler depuis sa
prétention, lorsqu'il a plu au roi de juger des pairs, etc., dans leur
cabinet avec les pairs, en quelque part que ç'ait été, avec ceux qu'ils
y ont voulu appeler avec eux. Cela est arrivé plusieurs fois.

Le jugement du duc de La Valette rendu dans le cabinet de Louis XIII, à
Saint-Germain en Laye, après la levée du siège de Fontarabie, en est un
des derniers exemples. Le premier président y fut appelé avec quelque
peu de membres du parlement\,; et comme la séance était autour de la
table du conseil, les pairs en occupèrent les premières places aux deux
côtés, les officiers de la couronne ensuite, et le premier président
après eux, sans aucune difficulté.

La régence de Marie de Médicis est le premier exemple que le parlement
puisse alléguer d'être entré dans les matières d'État et de
gouvernement, si on excepte celles des différends avec Rome, où la
politique des rois a toujours voulu mettre le parlement entre eux et
cette cour, et lui faire faire ce qu'ils ne voulaient pas paraître faire
eux-mêmes. L'enregistrement des traités de paix n'est rien, puisque le
parlement ne fut jamais consulté pour les négocier et les conclure.
C'est, \emph{ut notum sit}, comme des édits, déclarations, ordonnances,
lettres patentes, et pour qu'il règle leurs jugements dessus entre
particuliers, si quelqu'un se plaint de contraventions et de pillage
contre d'autres particuliers. Le refus que François Ier lui fit faire
d'enregistrer le traité de Madrid ne fut qu'un acte d'obéissance
conforme au cri général de la nation, et son enregistrement, quand il
l'aurait fait, n'en eût pas servi davantage à Charles-Quint. C'est donc
à l'époque de la mort funeste d'Henri IV qu'il faut fixer la première
connaissance que le parlement a prise des affaires d'État et du
gouvernement.

Cet exécrable événement, du détail duquel toutes les histoires et les
Mémoires de ces temps-là soulageront ceux-ci, remplit toute la cour
d'horreur, et d'effroi toute la ville. Le prince de Condé était hors du
royaume et premier prince du sang\,; Monsieur\footnote{Il s'agit ici de
  Gaston, duc d'Orléans, frère de Louis XIII.}, plus jeune que le roi
mineur, et nul autre fils de France\,; les autres princes du sang, et il
n'y en avait que deux, le prince de Conti et le comte de Soissons, à
craindre pour la reine par plus d'une circonstance\,; peu de grands à
Paris, tellement que le duc d'Épernon, comptant de jouer un grand rôle
si la reine lui avait l'obligation de toute son autorité, ne pensa qu'à
la lui procurer de la manière la plus publique et la plus solennelle, et
à loi assurer le plus de gens qu'il pourrait, en les associant en un
acte que leur intérêt les engagerait après à soutenir, sans songer dans
cet instant subit aux conséquences.

Il se servit donc sur-le-champ de l'autorité de son office de colonel
général de l'infanterie, fit assembler le parlement quoiqu'il fût fête,
investit le palais en dehors, et la grand'chambre, en remplissant la
grande salle de milice, tout cela sur-le-champ, et, comme on dit, en un
tournemain, et y fit aller aussitôt tout ce peu qu'il y avait de pairs
et d'officiers de la couronne avec la reine, laquelle fut à l'instant,
du consentement de tous, déclarée régente et revêtue seule du pouvoir
souverain.

De là le parlement voulut profiter des troubles qui survinrent pour se
mêler du gouvernement, et c'est l'époque de leur chimère de se dire les
tuteurs des rois. Leurs tentatives ne réussirent à leur fournir aucun
acte sur lequel ils puissent rien fonder à cet égard, mais à faire voir
qu'il n'a pas tenu à eux, et qu'ils ont augmenté ces troubles.

Louis XIII, en quantité d'occasions, leur a bien su dire\,: «\,qu'ils ne
sont qu'une simple cour de justice pour juger les procès des
particuliers\,; et leur rendre la justice en son nom, sans droit aucun
par delà leur juridiction contentieuse\,;» et cela en plein parlement, y
séant, et d'autres fois à leurs députés\,; et pendant son règne a bien
su les contenir dans ces bornes.

Sa mort également héroïque, chrétienne et sainte, qui pour la France
combla trop tôt sa vaillance, ses exploits, sa justice, et le prodige de
tant de vertus dans un prince si expressément mal élevé, et né sur le
trône, donna un second titre de fait au parlement pour les régences. Ce
prince, qui n'avait pas lieu de compter sur le bon gouvernement de la
reine son épouse, encore moins sur une sage administration de Monsieur,
son frère, voulut les balancer l'un par l'autre\,; et tous les deux par
l'autorité qu'il voulut donner à M. le Prince, et au conseil de régence
qu'il nomma.

Se défiant avec raison de la puissance et de l'effet de la volonté des
plus grands, des plus sages et des plus justes rois, tel qu'il était,
après leur mort, il essaya d'y suppléer en persuadant l'équité et la
prudence de ses dispositions. Il assembla donc dans sa chambre son rang,
les pairs, les officiers de la couronne, les grands officiers de sa
maison, ses ministres, et les principaux d'entre les conseillers d'État
et des membres du parlement, et en leur présence fit faire la lecture de
son testament par un des secrétaires d'État. Tous le louèrent,
l'approuvèrent, l'admirèrent\,; mais la forme de le passer en sanction y
manqua, comme elle avait manqué à celui de Charles V qui l'avait ajoutée
au règlement de l'âge de la majorité des rois. Aussi celui-là, si
répugnant à la première inspection des choses, si contraire à l'intérêt
des régents et des plus puissants de l'État, est-il demeuré loi
constante jusqu'à cette heure, et les deux testaments si sages, si
prévoyants, si justes, l'un du même Charles V, l'autre de Louis XIII,
n'ont eu aucune exécution.

La reine, dont l'ambition fut excitée par ceux dont l'intérêt était
qu'elle fût pleinement maîtresse pour être eux-mêmes les maîtres sous
son nom, {[}se laissa persuader{]} d'imiter Marie de Médicis d'autant
plus aisément que le parlement était informé des dispositions du roi
pour la régence, puisqu'il en avait donné lecture à ses principaux
membres\,; que s'agissant de dépouiller Monsieur, M. le Prince, et ceux
qui étaient nommés au conseil de régence, pour se revêtir seule de leur
autorité, elle ne le pouvait espérer qu'en flattant le parlement, dont
les membres étaient bien plus indépendants de tout intérêt avec ces
princes et ces ministres que les grands de l'État, et par un accablement
de nombre en voix de gens qui espéreraient plus de grâces d'elle que du
concours du conseil, et dont aucun n'était en posture de les arracher
comme les grands du royaume par leur réputation, leurs alliances et
leurs emplois. Ce fut ce qui la détermina d'aller faire déclarer sa
régence au parlement, où en effet elle fut revêtue seule de toute
l'autorité royale par la pusillanimité des deux princes, à l'exemple
desquels ceux du conseil de régence n'osèrent se refuser.

Le parlement, dans la suite et dans les troubles de cette régence, sut
bien profiter de son avantage aux dépens de l'État et de l'autorité
royale, que Louis XIV eut grand'peine à reprendre, et à remettre le
parlement dans ses bornes, qu'il y a bien su contenir après tant qu'il a
vécu, jusqu'à être allé une fois en habit gris tenir son lit de justice
avec une houssine à la main, dont il menaça le parlement, en lui parlant
en termes répondant à ce geste\footnote{Voy., notes à la fin du volume.}.

La régence de M. le duc d'Orléans est un troisième exemple consécutif en
faveur du parlement pour les régences, dont je me réserve à parler en
son temps.

Les temps fâcheux sont toujours ceux des innovations et des entreprises.
Les commencements de la Ligue, qui en produisirent quantité en tout
genre, ne furent pas moins avantageux à celles du parlement. Le duc de
Guise, qui n'aspirait à rien moins qu'à mettre la couronne sur sa tête,
et de là dans sa maison, s'était proposé de gagner tous les cœurs. Il
était, comme par droit successif de ses pères, l'idole des troupes et du
parti catholique, de la cour de Rome, qui ne songeait qu'à profiter du
temps pour étendre son autorité en France et anéantir les libertés de
l'Église gallicane, monument de toute antiquité qui la blesse si
douloureusement. Il était plus que sûr de la maison d'Autriche, qui,
jusqu'à sa fin, n'a jamais manqué à la sienne, jusqu'à se la substituer
en tout ce qu'elle put\,; mais qui ne voulait que la subversion de la
France pour profiter de ses débris. Il avait séduit les ministres par
les charges de l'ordre, et le cabinet par les bienfaits et par la
crainte\,; il disposait des écoles de théologie et des prédicateurs,
presque de tous les prélats\,; il était adoré des peuples, et pour les
gagner davantage et se dévouer de plus en plus les curés, il est le
premier homme, je ne dis pas de son état, mais je dis de la noblesse la
moins distinguée, qui ait été marguillier de sa paroisse et qui en ait
fait la planche, qui à la française a été suivie depuis par les
seigneurs les plus distingués. Il n'oublia pas à chercher à gagner le
parlement. Ses pères et lui-même s'étaient élevés à la pairie, ils en
avaient accumulé dans leur maison. Leur puissance leur fit après
franchir toutes les bornes, et cette dignité dont lui-même dans ses
premiers commencements s'était si fort prévalu, à l'exemple de ses pères
et de ses oncles, il ne se soucia pas de la prostituer pour cheminer
vers son grand dessein.

Le serment des pairs à leur réception au parlement est «\,d'assister le
roi en ses hautes et importantes affaires, de tenir les délibérations de
la cour secrètes, et de se comporter en tout comme un bon, vertueux,
magnanime duc et pair de France doit faire.\,» Ce sont les termes
consacrés mot pour mot qui ont été en usage depuis l'introduction de la
prestation de serment par les pairs, la première fois que chacun d'eux
vient prendre séance au parlement. Il est le même pour les pairs
ecclésiastiques\,; on n'y change que le nom de comte au lieu de celui de
duc pour les laïques et les ecclésiastiques qui sont comtes-pairs. Dès
lors le parlement en regardait la dignité avec jalousie, et dans
l'impossibilité de se défaire d'eux comme des autres prélats et des
autres nobles, il cherchait à les dégoûter et à les écorner, sans
toutefois avoir osé le tenter.

L'occasion de la réception de M. de Guise se présenta, qui la saisit
pour laisser ajouter à ces mots du serment\,: «\,comme un bon, vertueux
et magnanime duc et pair,\,» ceux-ci\,: «\,et comme un bon conseiller de
cour souveraine doit faire.\,» Quelque monstrueux que fût l'accolement
de la dignité de pair de France avec la qualité de conseiller de cour
souveraine, et qu'il parût à tout le monde, l'indignation publique fut
étouffée sous le poids du duc de Guise, et son exemple passa longtemps
en loi.

Longtemps après il se trouva des pairs plus difficiles, qui refusèrent
cette étrange innovation, et les années coulèrent ainsi parmi plus de
soumis que de rénitents\footnote{Expression latine (\emph{renitentes})
  qui signifie \emph{luttant contre\,;} les précédents éditeurs l'ont
  remplacée par le mot \emph{résistants}.}\,; à la fin les pairs n'en
voulurent plus entendre parler. Le parlement sentit que la chose était
insoutenable, de quelque côté qu'on la prît\,; les mots ajoutés furent
peu à peu supprimés\,; mais ce ne fut qu'au commencement que le dernier
Harlay fut premier président qu'il fut décidé, sans que le roi y
intervînt autrement que de le trouver juste, que jamais plus il n'en
serait parlé.

Cette tentative, qui a duré si longtemps, met en évidence l'esprit des
magistrats de réduire peu à peu les pairs au parlement au niveau des
conseillers, et on va voir jusqu'où l'audace en a été depuis poussée et
la ténébreuse industrie dont ils ne se sont jamais lassés, ainsi que la
négligence et l'incurie incroyable des pairs.

Les princes du sang, si justement pairs nés depuis Henri III et
précédant tous autres, ne s'en étaient pas encore distingués comme ils
n'ont cessé de faire depuis par tout ce qu'il leur a plu d'entreprendre.
Il était donc difficile au parlement d'essayer de tenir les pairs dans
les séances, sans que cela portât aussi sur les princes du sang. Ils
l'avaient pu en ajoutant au serment des pairs la qualité de conseillers
de cour souveraine, parce que les princes du sang n'y en prêtent
point\,; mais il n'en était pas ainsi des autres entreprises qui se
couvaient.

Ce détail pourra être ennuyeux, mais il est indispensable pour ce qui
doit suivre du complot de M. du Maine, qu'on n'entendrait pas sans cela,
et il servira par un simple exposé de faits à découvrir l'esprit du
tiers état, je n'ose dire la sottise de la noblesse, ni la faiblesse du
sang royal, et la conduite des magistrats toujours tendante au même but
dans une si longue suite d'années. On donnera à la suite de ce récit un
plan de la grand'chambre avec des chiffres qui renverront aux
explications, lesquelles, avec l'inspection du plan, rendront clair ce
qui le serait difficilement par le simple discours.

Il y a deux manières différentes en général d'entrer et de sortir de
séance, l'une pour les bas sièges, l'autre pour les hauts. En bas, les
magistrats entrent par l'ouverture que laisse le barreau entre le siège
de l'interprète et le bureau du greffier. Cette ouverture est vis-à-vis
du coin du roi, en biais. En débouchant cette ouverture, les présidents
traversent le parquet pour gagner leurs bancs\,; les conseillers, au
contraire, longent le long des bancs de chaque côté, passent entre les
bancs et les petits bureaux répandus au-devant des bancs, et chacun va
ainsi gagner sa place.

Les princes du sang et les pairs n'entrent point que les magistrats ne
soient en place. Ils entrent et gagnent leurs places, les princes du
sang en traversant le parquet comme les présidents. On a vu ailleurs que
cela n'était pas, et l'époque de ce changement. Les pairs font le même
chemin que les conseillers.

Cette distinction des présidents dont ils veulent tirer une préférence
est en effet nulle, mais en est une pour les princes du sang par la
position des bancs. Les présidents seoient seuls sur celui qui est en
face de l'entrée, inutile par conséquent de décrire, pour y aller, les
deux côtés d'un carré, puisqu'ils remplissent celui auquel ils vont tout
droit chacun vis-à-vis de sa place. Les princes du sang, qui, comme les
pairs et les conseillers, ne remplissent qu'un banc de côté, trouvent
leurs places en le longeant, et traverseraient vainement le parquet,
excepté pour les premières places du banc des pairs qui joint en équerre
la place du premier président, tellement que c'était une affectation
contre eux que de leur faire faire l'équerre le long des bancs pour
aller en leurs places, dont M. le Prince le héros les a affranchis, et à
l'égard de passer entre les bancs et les petits bureaux, qui en petit
nombre, sont devant les bancs pour la commodité des rapporteurs et de
leurs papiers, c'est peut-être une affectation nouvelle pour mieux
distinguer le traversement du parquet des présidents, mais je ne
l'assurerai pas, parce que j'en ignore l'origine.

Pour sortir de séance, la chose a beaucoup varié. Anciennement, les
pairs sortaient les premiers à la tête de la magistrature. Depuis, les
présidents firent si bien qu'ils marchèrent de front avec les pairs, qui
de la sorte avaient la droite sur eux. Depuis que le serment fut changé
à la réception du duc de Guise, il parut aux présidents que leur dignité
était blessée de marcher de front avec des gens qui souffraient la
qualité de conseillers de cour souveraine. Ils ne laissèrent pas d'être
embarrassés des princes du sang qu'ils ne pouvaient séparer des pairs.

À la fin ils prirent courage\,: ils osèrent proposer aux princes du sang
de marcher à la sortie après le dernier des présidents, et ces princes y
consentirent, par quoi les pairs ne purent s'en dispenser. On s'en
tiendra au simple récit, et on laissera les réflexions aux lecteurs. On
verra dans la suite que ce joug à la fin a été secoué, et les deux
diverses façons de sortir qui ont été depuis en usage pour les pairs, et
une autre à part pour les princes du sang.

Aux hauts sièges, les princes du sang jusqu'à aujourd'hui et les pairs
sont à la cheminée, proche de la lanterne, tandis que les magistrats
sont à la buvette, où les princes du sang et les pairs ont droit
d'aller, mais où ils ne vont jamais pour n'entrer ni sortir avec les
magistrats, sinon quelqu'un qui leur veut dire un mot, et qui y va
lorsqu'ils y sont, et en sort avant qu'ils se mettent en état d'en
sortir eux-mêmes. Depuis qu'on a raccommodé la grand'chambre, et qu'on
en a déplacé la cheminée d'auprès de la lanterne, pour l'adosser à la
grande salle du palais, les princes du sang et les pairs continuent de
se tenir près de la même lanterne pendant la buvette. Ils ont soin
d'être avertis quand on en sort.

Le premier d'entre eux, suivi un à un de tous les autres en rang
d'ancienneté, débouche la lanterne en même temps que le premier
président débouche celle de la buvette. Le premier des princes du sang
ou, s'il n'y en a point, le premier des autres pairs mesure sa marche
sur celle du premier président, qui est suivi des autres présidents et
des conseillers, en telle sorte que, longeant les deux bancs, ils
marchent à même hauteur, et arrivent en même temps à leur place près du
coin du roi. On met un banc sans dossier couvert d'un tapis fleurdelisé
le long du banc du côté des pairs, au bas de leur marchepied, entre ce
marchepied et le débord du dossier des bas sièges. Là se mettent les
pairs qui, par leur ancienneté, n'auraient pas place sur le banc de
derrière, et les conseillers ensuite, outre ceux qui sont sur le banc
des présidents, et ceux-là font le tour des bas sièges hors le barreau,
et entrent par la lanterne de la cheminée après les pairs.

Pour sortir, tout se lève à la fois\,; et debout et découverts comme en
entrant, les pairs et les présidents se saluent, le premier président et
le premier des princes du sang, ou, en leur absence, le premier des
autres pairs se replie sur son banc, car il y a espace, le second de
chaque côté de même après que le premier a passé le long de lui, ainsi
du troisième et de tous les autres, et sortent ainsi en même rang et par
même chemin qu'ils sont entrés. Les pairs passent par la grande porte
qui donne immédiatement dans la grande salle, et les présidents suivis
des conseillers par la petite porte qui donne dans le parquet des
huissiers et de là dans la grande salle.

Ce parquet des huissiers est une manière de petite antichambre entre la
grande salle et la grand'chambre où les plaideurs attendent quand on
plaide à huis clos\,; et où la croix de l'archevêque de Paris et les
gardes du gouverneur de Paris s'arrêtent lorsque l'archevêque et le
gouverneur vont prendre séance au parlement. Je reviendrai après aux
huissiers d'accompagnement.

Les présidents étaient bien contents de précéder ainsi paisiblement, en
sortant de la séance des bas sièges, les pairs et les princes du sang
même, et toute la robe partageait cette gloire avec satisfaction\,; mais
plus ils s'y accoutumèrent, plus ils trouvèrent d'amertume dans le
changement que la présence du roi apportait à leur grandeur. Les bas
sièges {[}sont{]} alors la séance de toute magistrature, et les
présidents à mortier y sont aux pieds des pairs ecclésiastiques. Ils ne
se flattaient pas de pouvoir monter en haut, et ils s'en consolaient en
voyant le chancelier leur chef en bas comme eux. Mais d'opiner
découverts et à genoux leur était un grand crève-coeur, tandis qu'ils
voyaient les pairs et même les officiers de la couronne opiner assis et
couverts. Ils trouvaient bien en cela quelque similitude avec le
chancelier, qui prend l'avis du roi découvert, et à genoux à ses pieds,
et ne lui parle point dans une autre posture de toute la séance, tout
second officier de la couronne qu'il est, parce qu'il est légiste par
état et magistrat, mais quoique assis au même niveau des autres
magistrats dans la place que le greffier occupe aux grandes audiences,
il y parle et opine assis et couvert, et y prononce de même. Les
présidents négocièrent et obtinrent que, dès qu'ils seraient à genoux en
commençant de parler, le chancelier leur commanderait de la part du roi
de se lever, mais qu'en se levant ils mettraient un genou sur leur banc,
qu'ils opineraient ou parleraient toujours découverts en cette posture,
et qu'ils se mettraient à genou à terre en finissant de parler. C'est ce
qui s'observe encore aujourd'hui.

Je remarque exprès cette humiliante façon du tiers état de parler devant
le roi, et de sa séance en bas, à la différence du baronnage, par le
contraste inimaginable que les présidents osèrent entreprendre. Ils
prétendirent opiner devant les pairs et devant les princes du sang, ils
l'emportèrent. Encouragés par cet inespérable succès, ils voulurent
opiner avant les fils de France, et ils y réussirent. Enfin ils se
prévalurent si bien de la cassation du testament de Louis XIII que la
reine souhaitait si passionnément, et qui se laissa persuader de
s'adresser au parlement, qu'elle consentit, toute reine et régente
qu'elle était, que les présidents opinassent devant elle, et
immédiatement tous après le roi.

Cette énormité dura jusqu'en 1664\,; les pairs demandèrent enfin
justice, ce qui forma un procès où le parlement en corps se rendit
partie, avec toute la robe en croupe. Les pièces en sont entre les mains
de tout le monde, ainsi que l'arrêt contradictoire et très solennel par
lequel le roi les réduisit au rang d'opiner où ils devaient être, après
le dernier de tout ce qui est aux hauts sièges, ce qui s'est toujours
exécuté depuis jusqu'à aujourd'hui\,; ainsi je ne m'y étendrai pas, et
laisserai encore une fois le lecteur à ses réflexions.

L'ordre des temps étant préférable dans un récit historique à la suite
naturelle du discours, j'interromprai ici celle de l'arrêt de 1664, à
laquelle je reviendrai après pour parler du changement entier arrivé aux
réceptions des pairs au parlement.

Les pairs ont toujours été reçus au parlement jusqu'à la mort de Louis
XIII, à la grande audience à huis ouvert, la séance par conséquent aux
hauts sièges\,; un avocat présentant les lettres par un discours, un
avocat général parlant après et concluant. Le pair, après le serment
fait comme il se fait aujourd'hui, montait à sa place. On plaidait une
cause de nature à être jugée en cette audience même, pour que le nouveau
pair opinât, et l'audience finie on se retirait.

M. de Monaco, lassé de la domination des Espagnols, fit un traité avec
Louis XIII pour se donner à la France, qui fut secrètement ménagé par le
dernier duc d'Angoulême, gouverneur de Provence, qui s'y trouvait alors.
On a assez parlé de ces seigneurs de Monaco, à l'occasion du mariage du
dernier Monaco-Grimaldi avec la fille de M. le Grand, pour n'en pas
interrompre ici le fil du discours. Par un des articles du traité, il
fut stipulé que M. de Monaco serait fait duc et pair. Il l'exécuta avec
beaucoup d'adresse et de courage, mit la garnison espagnole hors de
Monaco, y en reçut une française, et le roi de son côté l'exécuta aussi
de sa part. Ces choses se passèrent en 1642. Dans cette année l'érection
nouvelle du duché de Valentinois avec la pairie fut faite et enregistrée
au parlement, et M. de Monaco a été le dernier duc et pair de Louis
XIII, et le dernier chevalier du Saint-Esprit aussi, dont il reçut le
collier des mains de ce monarque, au camp devant Perpignan, qui fut son
dernier exploit. M. de Monaco retourna de là à Monaco, où il demeura
jusqu'après la mort de Louis XIII, quelque temps après laquelle, mais la
même année, il vint à Paris, et il y profita de ce voyage pour se faire
recevoir au parlement.

C'était un temps de faiblesse, d'effervescence et de cantonnement\,;
c'en était un de triomphe pour cette compagnie, à qui pour la seconde
fois on venait d'avoir recours pour la régence, et de plus pour casser
le testament du roi, et donner toute puissance à la reine. Le parlement
comptait sur sa reconnaissance et plus encore sur sa crainte, et par
conséquent sur ses ménagements et ceux de ses ministres, à l'entrée
d'une minorité, dans le cours d'une forte guerre où le besoin d'argent
rendrait le concours du parlement nécessaire pour l'enregistrement des
édits, dans le pouvoir qu'on venait de lui reconnaître dans tout ce qui
venait de se passer, et où les grands de l'État, attentifs à leurs
intérêts particuliers, étaient presque tous aux frontières ou dans leurs
gouvernements\,; un temps enfin où chacun cherchait à s'appuyer, et ou
tout contribuait à rendre le parlement considérable, hardi et
entreprenant.

Cette compagnie n'avait jamais cessé de travailler à chercher à
approcher les pairs du niveau des conseillers, depuis que le duc de
Guise, tué à Blois, avait souffert, et les autres pairs après lui, le
changement au serment des pairs, qui a été expliqué, encore plus depuis
que les présidents à mortier étaient parvenus à se faire suivre, en
sortant de séance, par les princes du sang et les autres pairs,
quoiqu'il soit vrai que l'occasion ne s'en présentât guère, parce qu'il
était fort rare qu'il s'en trouvât aux petites audiences en bas, ou aux
procès par écrit qui s'y jugent, toutes les grandes causes et
jusqu'alors toutes les réceptions des pairs étant faites et plaidées aux
hauts sièges, où chacun entrait et sortait par son côté, comme il a été
expliqué.

Un temps si favorable aux entreprises du parlement le devint encore
davantage par la personne qui se présenta à faire le serment de pair de
France, et à en prendre la séance. M. de Monaco était un étranger qui
avait passé toute sa vie chez lui parmi des Espagnols et des Italiens,
qui n'avait jamais habité en France, qui en ignorait tout, et qui n'y
avait ni parents, ni amis, ni connaissances\,: M. d'Angoulême, avec son
traité et le voisinage, lui en aurait pu donner davantage, {[}mais il{]}
n'était point pair et n'en savait pas plus que lui sur les séances du
parlement. Cette compagnie n'en fit donc pas à deux fois\,; elle le
reçut aux sièges bas avant la petite audience du matin, avec un
rapporteur qui rapporta ses lettres, ce qui est la forme de recevoir les
conseillers. C'était une innovation bien hardie et bien étrange, et
toutefois l'inapplication, l'ignorance, l'incurie était déjà telle que
je ne sais si on s'en aperçut. Du moins M. de Monaco n'était pas pour
s'en douter, et si d'autres purent le remarquer, la faiblesse et
l'abandon fut tel aussi qu'on ne le releva pas.

Telle est la moderne époque de ce changement total de la réception des
pairs au parlement. Les troubles et l'autorité de cette compagnie qui
s'accroît toujours parmi les désordres, et la même faiblesse des pairs,
continuèrent sans bruit cette façon nouvelle des réceptions, qui
finalement s'est depuis soutenue jusqu'à aujourd'hui.

Les conquêtes que les parlements avaient faites devaient leur sembler
assez belles pour s'en contenter. Ils avaient fait l'étrange innovation
au serment des pairs qui a été expliquée, par laquelle ceux-ci
s'avouaient conseillers de cour souveraine\,; ils les avaient réduits
pour leur réception à la parité avec les conseillers\,; ils précédaient
les princes du sang, par conséquent les pairs à la sortie de la séance
des bas sièges, et l'occasion rare, jusqu'alors, en devenait plus
fréquente et plus solennelle depuis que les réceptions des pairs s'y
faisaient. Enfin ils opinaient entre le roi et la reine régente, par
conséquent avant elle, avant les fils de France, les princes du sang et
les pairs. C'était avoir fait un beau chemin pour des légistes
souffleurs du baronnage et assis sur son marchepied pour en être à
portée quand il plaisait à quelqu'un de ces seigneurs de les consulter à
l'oreille, sans toutefois y être astreints, ni de suivre l'avis qu'ils
leur disaient aussi à l'oreille.

On a vu que, quant à la dignité et aux fonctions de la pairie, ceux
d'aujourd'hui sont en tout les mêmes que dans tous les temps, et les
légistes eux-mêmes devenus tels qu'on les voit aujourd'hui ne se peuvent
dissimuler ni à personne leur état de légiste, et jusque dans leur
triomphe, leur séance aux pieds des pairs, et à ceux des officiers de la
couronne, nonobstant tout l'art et le temps qui a fait un banc de ce
marchepied et que comme tels ils n'opinent et ne parlent que découverts
et à genoux, ainsi que le tiers état dont ils sont membres par leurs
offices, quelque nobles que quelques-uns d'eux se voulussent prétendre,
et en quelque monstrueux rang qu'ils fussent parvenus à opiner, jusqu'à
y précéder la reine, mère de leur roi et régente du royaume. Quel
prodige pour des sujets d'entre le peuple, qui n'aurait pu entrer dans
l'esprit des premiers du royaume d'oser le prétendre, et quel monstre de
grandeur sur piédestal d'argile\,!

Les troubles domestiques et les embarras de la guerre au dehors en
maintinrent l'énormité. Mais après la paix des Pyrénées, les idées
revinrent, et la possibilité de remédier aux principaux désordres.
Celui-ci qui parut le plus suprême de tous, comme on l'a vu, fut abrogé
en 1664, et le premier président avec tous les autres, remis en son
premier rang d'opinion après le dernier de tout ce qui seoit aux hauts
sièges.

C'était tomber de bien haut après avoir opiné avant une reine régente de
n'opiner plus qu'après le dernier officier de la couronne, dont le
premier, c'est-à-dire le connétable quand il y en a un, ne seoit et
n'opine qu'après le dernier pair de France, ou s'il l'est lui-même, en
son rang d'ancienneté parmi eux. Le procès avait été contradictoirement
instruit, et les mémoires auxquels le duc de Luynes contribua beaucoup
par sa capacité, sont entre les mains de tout le monde ainsi que ceux
des présidents. Ils avaient eu l'adresse d'engager le parlement en corps
à se rendre partie avec eux\,; ils avaient épuisé l'art et le crédit
pour allonger l'instruction et retarder le jugement du roi.

Plus l'affaire avait fait de bruit, plus la rage de succomber fut
grande, et la passion de s'en venger. Mais ils n'ont osé rien tenter sur
le rang d'opiner qui est demeuré jusqu'à aujourd'hui dans la règle où
l'arrêt de 1664 l'a décidé. Ils se sont contentés à cet égard de rendre
les pièces et les mémoires imprimés en 1664 où on voit les signatures
des ducs de Guise et d'Elbœuf en leur rang d'ancienneté\,: le premier
après les pairs ecclésiastiques, l'autre après le duc d'Uzès. Leur
sensibilité a même été si passionnée là-dessus, qu'ils se sont portés
jusqu'aux menaces et jusqu'aux violences pour en empêcher la
réimpression, et ensuite la distribution et le débit, lorsqu'on fit
faire une édition pendant la régence, et qui fut faite et débitée
publiquement malgré leurs emportements si peu convenables à l'état de
légistes et à la gravité de magistrats.

Leur dépit les tint longtemps à chercher des dédommagements qu'ils
n'osèrent hasarder les premières années qui suivirent celle de 1664.
Lamoignon, premier président, mourut en 1677\,; Novion lui succéda, qui
fut chassé de cette belle place en 1689, pour ses friponneries et ses
falsifications d'arrêts qu'il changeait en les signant. Les rapporteurs
s'en aperçurent longtemps avant que d'oser s'en plaindre\,; à la fin,
les principaux de la grand'chambre lui en parlèrent, et l'obligèrent à
souffrir un témoin d'entre les conseillers à le voir signer. Il avait
encore une façon plus hardie pour les arrêts d'audience\,; il les
prononçait à son gré. Chaque côté de la séance dont il avait été prendre
les avis admira longtemps comment tout l'autre côté avait pu être d'un
avis différent de celui qui avait été le plus nombreux du sien, et cela
dura longtemps de la sorte. Comme cela arrivait de plus en plus souvent,
leur surprise fit qu'ils se la communiquèrent. Elle augmenta beaucoup
quand ils s'apprirent mutuellement qu'elle leur était commune depuis
longtemps, et que ces arrêts qui l'avaient causée n'étaient l'avis
d'aucun des deux côtés. Ils résolurent de lui en parler la première fois
qu'ils s'en apercevraient. L'aventure ne tarda pas, et le hasard fit que
la cause regardait un marguilliage\,; quelques-uns des plus accrédités
de la grand'chambre lui parlèrent comme ils en étaient convenus entre
eux, et tout modestement le poussèrent\,; se trouvant à bout, il se mit
à rire et leur répondit qu'il serait bien malheureux, étant premier
président, s'il ne pouvait pas faire un marguillier quand il en avait
envie. Ces gentillesses furent enfin portées au roi avec les couleurs
qu'elles méritaient, et il était chassé honteusement et avec éclat sans
le duc de Gesvres, premier gentilhomme de la chambre, et de tout temps
fort bien et fort libre avec le roi, qui en obtint qu'il donnerait sa
démission comme un homme qui veut se retirer, et il se chargea de
l'apporter au roi. La chose se passa de la sorte, et Harlay, lors
procureur général, fut premier président, et La Briffe, simple maître
des requêtes, procureur général.

\hypertarget{chapitre-xviii.}{%
\chapter{CHAPITRE XVIII.}\label{chapitre-xviii.}}

~

{\textsc{Les deux Novion, Harlay et Mesmes premiers présidents\,;
quels.}} {\textsc{- Affaire du bonnet.}} {\textsc{- Les princes du sang
et les pairs cessent de suivre les présidents à la sortie de la séance
des bas sièges.}} {\textsc{- Nouvelle forme pour les princes du sang et
deux autres successives pour les pairs.}} {\textsc{- Huissiers
d'accompagnement.}} {\textsc{- Nouveautés à cet égard et usurpations des
présidents.}} {\textsc{- Orgueil des présidents à l'égard des princes du
sang.}} {\textsc{- Nouvelle usurpation d'huissier très indécente.}}
{\textsc{- Princes du sang et pairs exclus de la tournelle par la ruse
et l'innovation des présidents.}} {\textsc{- Conseiller usurpe de couper
la séance des pairs, sans toutefois marcher ni opiner parmi eux.}}
{\textsc{- Nouvelle usurpation manquée.}} {\textsc{- Pairs ont partout à
la grand'chambre la droite très nettement sur les présidents.}}
{\textsc{- Distinction et préférence du barreau de la cheminée sur
l'autre.}} {\textsc{- Usurpation aussi singulière qu'indécente du
débourrage et surbourrage des places près le coin du roi.}} {\textsc{-
Nouvelle usurpation aux bas sièges d'un couvercle sur le banc des
présidents.}} {\textsc{- Saluts.}} {\textsc{- Origine de la séance du
grand chambellan sur les marches du trône au lit de justice.}}
{\textsc{- Nouveauté, en 1715, du passage des princes du sang par le
petit degré du roi pour monter à sa suite aux hauts sièges, au lit de
justice.}} {\textsc{- Siège unique du chancelier, et du garde des sceaux
en son absence, aux Te Deum et au lit de justice\,; en ce dernier
comment couvert.}} {\textsc{- Pairs ecclésiastiques rétablis en leur
préséance sur les cardinaux au parlement, le roi présent ou absent, par
la décision de Louis XIV, qui n'a point été enfreinte.}} {\textsc{-
Vaine tentative et honteuse du cardinal Dubois.}} {\textsc{- Nouveauté,
indifférente et consentie pour commodité, de la séance des officiers de
la couronne au-dessous des pairs ecclésiastiques, au lieu d'au-dessous
des pairs laïques, au premier lit de justice de Louis XV, qui subsiste
depuis.}} {\textsc{- Choix donné des deux côtés au duc de Coislin,
évêque de Metz\,; pourquoi il préfère le droit.}}

~

Ce préalable était nécessaire avant d'aller plus loin, tant pour les
dates que pour faire voir à quels premiers présidents les pairs eurent
affaire. Il serait en effet bien difficile d'en trouver trois de suite
en aucun tribunal aussi profondément corrompus que Novion, Harlay et
Mesmes, et de genres de corruptions plus divers par leur caractère
personnel, sans qu'on put dire néanmoins lequel des trois a été le plus
corrompu, quoique corrompus au dernier excès tous les trois, et chacun
différemment aussi, avec tous les talents et les qualités qui pouvaient
rendre leur corruption plus dangereuse. Novion laissa un petit-fils que
M. le Duc fit premier président presque aussitôt qu'il fut premier
ministre. Il n'y put durer longtemps et quitta. C'était un dangereux
maniaque, qui a laissé maints monuments de folie et de l'égarement de
son esprit.

Ce fut tant de honte pour les ducs, et un honneur si énorme pour les
Potier, d'en voir un fait duc et pair parmi les quatorze de 1663, qu'il
y avait lieu de croire que Novion comblé de l'un chercherait par sa
conduite à adoucir l'autre. Ce bourgeois ne pensa pas ainsi. Quoique
fort bien avec le duc de Gesvres, il était piqué de voir un cadet de sa
famille au rang des grands seigneurs et d'être demeuré dans celui de son
être, et quoique vivant en amitié avec les Gesvres, et se mettant à tout
pour eux, lui et son petit-fils, car son fils est mort jeune et obscur,
se sont toujours plu en des respects amers et ironiques pour les
Gesvres, et à se dire des bourgeois pour leur faire dépit. Telle fut
leur bizarrerie, ou plutôt leur ver rongeur, et la cause intime de leur
procédé avec les pairs, dont le petit-fils n'a pu que montrer la même
humeur en des occasions momentanées.

Novion, succédant à Lamoignon sans avoir pu remplir sa place, ne songea
donc qu'à seconder le dépit du parlement en suivant le sien particulier.
Il fut peu en cette place sans faire des tracasseries qui ne parurent
pas d'abord, qui après se firent sentir, et qui par leur opiniâtre durée
sont devenues des usurpations de la dernière indécence. Comme elles ne
furent introduites que peu à peu en tâtonnant, que les pairs ne s'en
aperçurent que tard, et que plus tard encore ils s'en plaignirent, je ne
puis fixer de date à chacune de ces apparentes ténuités, et je les
remets à la fin de cette digression, pour venir au point capital qui l'a
forcément engagée.

Ces tracasseries, que je remets à la fin, furent suivies de quelque
chose de bien plus sérieux, et qui commença à s'introduire par un air de
distraction et par de la variété. Aux audiences, le premier président se
lève pour aller prendre les opinions d'un côté, puis de l'autre, par
pelotons qui s'assemblent debout autour de lui\,; il est découvert du
moment qu'il se lève jusqu'à ce qu'il soit retourné à sa place, et
assis, pour prononcer couvert. Aux procès de rapport, qu'on appelle
autrement par écrit, où on est à huis clos (ou, comme au rapport de ce
qui regarde la réception d'un pair, on est censé y être), le premier
président, sans bouger de sa place, prend l'avis de toute la séance
ayant le bonnet sur sa tête\,; tous opinent découverts à mesure que le
premier président appelle le nom de chacun. Venant aux pairs, il se
découvrait en nommant le premier d'eux à opiner, de suite les princes du
sang opinaient sans être nommés, puis les présidents sans l'être non
plus\,; se couvrait après, puis prononçait.

Il faut dire en passant que cette différence de ne point appeler les
princes du sang ni les présidents par leur nom ne peut venir que de la
proximité du premier président d'eux, en sorte qu'il n'a besoin que de
les regarder l'un après l'autre pour leur faire entendre à qui c'est
d'opiner\,; au lieu que son éloignement des autres places l'oblige à
nommer le nom de chacun, que ses regards éloignés, et nécessairement peu
distincts entre quatre ou cinq voisins assis près les uns des autres,
seraient confusément reçus et ne leur laisseraient pas démêler l'ordre
de l'opinion. Cet usage, qui ne peut avoir d'autre origine, est devenu
une distinction des princes du sang et des présidents à mortier, qui, en
cela comme en d'autres qu'on remarquera à mesure, se sont égalés à eux.

Novion commença par mettre négligemment son bonnet sur le bureau, tantôt
au commencement, tantôt au milieu, quelquefois vers la fin de l'appel
des noms des conseillers, et il évita toujours de l'ôter au moment qu'il
nommait le premier à opiner des pairs. De là il poussa plus loin
l'affectation de son inadvertance, il demeura couvert en nommant les
premiers des pairs à opiner, puis se découvrait comme ayant oublié de le
faire, et achevait d'appeler le nom des autres. Les pairs furent quelque
temps assez simples pour n'y pas prendre garde. Leurs réceptions étaient
rares. Après s'en être aperçus cela s'oubliait jusqu'à la première qui
produisait la même surprise, et toujours avec la même incurie. Ce
prélude aurait néanmoins dû les réveiller, d'autant plus qu'ils ne
pouvaient penser que les présidents, ni la compagnie même, fussent
revenus du dépit de l'arrêt de 1664 sur la préopinion, et qu'ils avaient
eu depuis une autre occasion de pique dont j'expliquerai le fait après
celui-ci.

À la fin, l'évêque-comte de Châlons, si connu longtemps depuis sous le
nom de cardinal de Noailles, archevêque de Paris, fut reçu au parlement
en 1681, et ce fut à sa réception que Novion, levant le masque, demeura
couvert en appelant tous les noms des pairs, et ne se découvrit que
lorsqu'il en fut aux princes du sang. Le duc d'Uzès perdit patience,
enfonça son chapeau et opina couvert avec un air de menace. Les ducs
éclatèrent et se plaignirent au roi.

Le roi a, tant qu'il a pu, abaissé et diminué le rang des ducs en tout
ce qui lui a été possible\,; il n'était pas fiché des querelles de cette
nature, et il aimait à les faire durer en ne les jugeant point, pour
tenir les parties en division, et plus dans sa dépendance. Il prit
prétexte que le duc d'Uzès s'était fait justice lui-même, et aux pairs
avec lui, et ne voulut point s'en mêler. Il ne devait pas être difficile
de le mettre au pied du mur en tout respect\,: en le suppliant de
décider, et il n'était pas possible qu'il le fit en faveur d'une
indécence si poussée, et en même temps si nouvelle\,: ou, s'il
persistait à ne s'en point mêler, lui demander conséquemment la
neutralité de part et d'autre, et n'opiner plus aux procès par écrit que
couverts.

J'aurais peine à comprendre qu'on en fût demeuré là, et que les pairs
eussent retourné opiner découverts, le premier président restant couvert
depuis cette époque, si je n'avais vu de mes yeux de quoi rendre tout
croyable des pairs avec le parlement, pour ne parler que de ce dont il
s'agit ici, et du parlement avec eux en tout genre d'entreprise.

Je me contenterai de cette triste remarque et de dire que cette affaire,
dont la contestation dure encore au même état, et si connue sous le nom
de l'affaire du bonnet, est celle dont M. du Maine s'est servi avec tant
de noire profondeur et de fortune, qui donne lieu à cette digression.
Avant de la finir, il faut achever de voir les autres gentillesses des
présidents du parlement, qui ne purent être contents d'avoir égalé les
pairs avec les conseillers par le changement de la réception des pairs
aux hauts sièges, et par la plus qu'indécence de leur nouvelle manière
d'opinion aux procès par écrit.

Il faut revenir maintenant à expliquer ce nouveau dépit causé aux
présidents par les pairs, dont je viens de parler, et que j'ai remis ici
par les queues qu'il a laissées et qui durent encore. Du temps du
premier président Lamoignon, les princes du sang se lassèrent enfin de
sortir de séance aux bas sièges à la suite des présidents, et Lamoignon
avait trop de sens et d'esprit pour ne pas sentir que cette indécence,
pour en parler sobrement, ne pourrait se soutenir que tant qu'il
plairait aux princes du sang de la laisser durer. Il comprit en même
temps que les pairs, qui ne pouvaient se plaindre de ce qui leur était
commun avec les princes du sang, ne s'accommoderaient pas d'une marche
qui n'aurait plus ce bouclier, tellement que sans querelle\,; et sans
bruit M. le Prince, dont ce premier président était ami, convint avec
lui d'une autre façon de sortir de séance aux bas sièges, tant pour les
princes du sang que pour les pairs, où les premiers prirent un avantage
fort marqué sur les seconds, qui ne témoignèrent seulement pas le
sentir. Voici donc ce qui fut réglé pour les princes du sang entre M. le
Prince et le premier président, et qui s'est toujours pratiqué depuis.

La petite audience finie en bas, le premier président ôte son bonnet,
demeure assis, et regarde les princes du sang\,; aussitôt ils se
découvrent, se lèvent, et en même temps les pairs et les présidents en
font autant. Les princes du sang se tournent à droite et à gauche en
s'inclinant, traversent le parquet et s'en vont. Avant qu'ils soient
sortis du parquet, les présidents ont soin de se rasseoir\,; les pairs
en même temps se rassoient. Les uns et les autres demeurent quelques
moments de la sorte, puis toute la séance se lève en même temps\,; les
présidents s'inclinent aux pairs, les pairs à eux sans remuer et
découverts\,; puis le premier des pairs et le premier président se
mettent en marche en traversant le parquet. Le premier pair en coulant
par-devant les pairs debout devant leurs places, qui tous le suivent à
mesure un à un, tandis que les présidents, suivis des conseillers,
débouchent le parquet, les conseillers se retirant le long de leurs
bancs, et en sortent ainsi un à un par l'ouverture qui est entre la
chaire de l'interprète et le bureau du greffier. En débouchant, ils se
couvrent et sortent de la grand'chambre par le parquet des huissiers.
Les pairs débouchent la séance ou le parquet par l'ouverture qui est au
barreau joignant la lanterne de la cheminée, s'arrêtent quelques pas au
delà, l'un après l'autre, pour marcher deux à deux, se couvrent et
sortent de la grand'chambre par la grande porte qui donne dans la grande
salle. C'est ce qui s'observe encore aujourd'hui pour les princes du
sang, et que j'ai vu observer longtemps pour les pairs depuis aux
réceptions au parlement.

Cette ouverture du barreau, tout proche la lanterne de la cheminée, a
une porte de la hauteur du barreau, c'est-à-dire à hauteur d'appui quand
on est debout, et les avocats qui plaident derrière l'ouvrent et entrent
dans l'ouverture pour conclure. Fort peu avant que le premier président
Harlay se retirât, cette porte se trouva si bien fermée aux pairs
sortant de la séance qu'ils ne la purent ouvrir, en sorte qu'ils
montèrent par les marches tout joignantes des sièges hauts, et passèrent
par la lanterne\,; je m'y suis trouvé deux fois. Cette affectation fit
craindre la clôture de la porte de la lanterne même, ce qui aurait rendu
toute autre sortie impossible que celle des présidents et des
conseillers\,; tellement que, depuis cela, les pairs demeurent assis
lorsque la séance se lève après que les princes du sang sont partis,
demeurent découverts comme les présidents et les conseillers, et les
voient tous sortir du parquet jusqu'au dernier, sans se lever de leurs
places.

Les présidents en passant s'inclinent à eux, et eux aux présidents, mais
sans aucune contenance de se soulever\,; puis quand toute la robe,
jusqu'au dernier, est hors du parquet, les pairs se lèvent et en sortent
il n'importe plus par où. Je l'ai toujours vu faire par la lanterne de
la cheminée, car la porte du barreau est demeurée alors fermée. On sort
ainsi tumultuairement de la lanterne, et on se met après deux à deux en
ordre d'ancienneté. Un huissier du parlement les attend au débouché de
la séance, et, son bonnet à la main, marche devant eux, et leur fait
faire place jusque par delà la grande salle, à certaine distance de la
galerie, où il prend congé d'eux. C'est aussi en cet endroit que les
pairs se découvrent et se séparent pour aller trouver chacun son
carrosse. Les présidents trouvent deux huissiers au sortir du parquet,
qui marchent devant eux, et leur font faire place jusque prés de la
Sainte-Chapelle, frappant de leurs baguettes, en traversant la grande
salle, sur les boutiques. Quand il n'y aurait qu'un pair en séance, et
sans autre occasion que de ce qu'il l'aurait prise, il serait également
conduit par un huissier, et jusqu'aussi loin. Lorsqu'un pair arrive au
parlement pour y être reçu, il trouve un huissier à la descente de son
carrosse qui le conduit à la grand'chambre, marchant devant lui
découvert et faisant faire place. Cela était en usage, indépendamment de
réception, à l'égard de tous les pairs. Ce devoir a disparu sous
prétexte du grand nombre, depuis les quatorze érections de 1663, et que
les huissiers n'y pourraient suffire. Les princes du sang en trouvent
toujours deux à la descente de leur carrosse, et qui les y reconduisent
chaque fois qu'ils vont au parlement. Les présidents, qui y sont les
maîtres et qui ont ces huissiers dans leur main, s'en font précéder
seuls et sans être à la tête de la grand'chambre, allant par le palais.

Je ne sais d'où cela a commencé. Pour le frappement de baguettes, je n'y
vois d'origine que la foule, et d'avertir plus fortement de faire place,
chose qui a depuis tourné en distinction par des gens si attentifs à y
tourner les moindres choses, et d'en faire naître de toutes espèces,
comme on le va voir. Ils furent fort peinés du peu de succès de la
clôture de la porte du barreau joignant la lanterne de la cheminée, et
se plaignirent que les pairs demeurassent en séance lorsque les
magistrats en sortent, et que c'était pour voir passer les présidents
sans se lever pour eux. Je reviendrai après à cet article, mais ils ne
purent les en empêcher par eux-mêmes. Ils n'osèrent aussi en faire une
plainte au roi, parce qu'ils sentaient la réponse de la porte fermée si
nouvellement\,; ainsi les choses en sont demeurées là jusqu'à
aujourd'hui.

Les princes du sang trouvèrent leur distinction dans cette façon de
sortir seuls de la séance des bas sièges\,; et les présidents, pour n'en
être pas précédés, ont toujours eu grand soin de se rasseoir après les
avoir salués, pour montrer, par cette pause après cette sortie, que la
cour est toujours en séance, et que les princes du sang se sont retirés
avant qu'elle fût levée. Le premier président Harlay donna de son chef
une distinction nouvelle aux princes du sang, quelque temps après qu'il
fut en place, pour leur sortie des hauts sièges, où ils entrent encore
aujourd'hui, et sortaient alors, à la tête des pairs\,: ce fut de leur
ouvrir le petit degré du roi, qui, de son coin, descend à la place du
greffier aux grandes audiences, qui est celle que le chancelier occupe
aux lits justice. Depuis cette invention d'Harlay, lorsque la séance se
lève aux hauts sièges, les princes du sang, au lieu de se reployer comme
ils faisaient sur les pairs, et comme les pairs font encore, pour sortir
le long de leur banc par la lanterne de la cheminée, les princes du
sang, dis-je, s'avancent vers le coin du roi, après avoir salué les
pairs à leur droite, saluent les présidents vers ce coin, et descendent
le petit degré du roi, au bas duquel ils trouvent leurs deux huissiers
pour marcher devant eux.

De cette sortie séparée, Harlay a fait naître une indécence que je
m'abstiens de qualifier\,: c'est qu'à l'instant que le dernier des
princes du sang en séance a enfilé le degré qui n'est que de cinq
marches, comme ceux des deux lanternes, et par lequel personne ne doit
passer, un huissier escalade aux hauts sièges en montant sur les sièges
bas, et en enjambant le dossier vis-à-vis les plus anciens pairs, passe
tout de suite devant le premier président qui l'attend pour marcher
devant lui, et qui, resté debout avec toute la séance depuis la sortie
des princes du sang, ne se met en marche, rebroussant le long de son
banc, comme il a été dit ailleurs, que lorsqu'il a cet huissier devant
lui. Avant cette sortie des princes du sang par ce petit degré du roi,
cet huissier attendait avec un autre huissier le premier président au
débouché de la lanterne de la buvette, où le second huissier l'attend
encore, par où le premier président sort de la séance haute, suivi des
présidents et des conseillers qui sont sur ce banc. Les conseillers qui
sont du côté des pairs attendent que le dernier pair ait débouché la
lanterne de la cheminée pour aller joindre leurs confrères parmi la
grand'chambre, sans huissier. On est honteux de décrire ces misères et
ces petites inventions de distinctions et d'orgueil\,; mais on décrit
par là le caractère qui les fait imaginer et exécuter. On en va
expliquer d'autres incessamment, et encore plus ridicules.

Depuis que les princes du sang, et les pairs après eux, ont cessé de
suivre les présidents à la sortie de la séance des bas sièges, le
premier président cessa de faire venir la tournelle à la grand'chambre
aux affaires des ecclésiastiques et des nobles qui sont criminelles et
qui exigent l'assemblée des deux chambres, laquelle y venait auparavant.
La morgue de la dignité de la grand'chambre a cédé à la malice d'exclure
les pairs de cette séance de la tournelle, parce que, n'y ayant point
deux chemins séparés pour aller de l'une à l'autre, comme pour sortir
simplement de séance, il n'y peut rester que les pairs seuls qui ne
veulent pas suivre les présidents. En cela les princes du sang sont
enveloppés dans la même privation, et par même cause, de laquelle il
résulte que les princes du sang ni les pairs ne vont plus à la
tournelle, par la même cessation d'usage qui les a privés du conseil des
parties, où ils avaient droit de séance et d'opinion.

Le premier président de Novion, non content du bonnet, voulut pousser
plus loin ses entreprises et y donner aux conseillers une part
particulière, et ameuter mieux par là le parlement sur le bonnet. Il
imagina de faire demeurer un conseiller sur le banc des pairs, en sorte
que, lorsque leur nombre en occupe plus d'un, la dernière place de
chaque banc qu'ils remplissent, soit aux bas sièges, soit aux hauts, est
remplie par un conseiller, qui se trouve ainsi coupant la séance des
pairs et placé au milieu d'eux. Cette entreprise eut le même succès de
tant d'autres, et dure jusqu'à aujourd'hui. Il est vrai que le premier
président, jusqu'à cette heure aussi, a eu la modestie de ne pas
demander l'avis à ces conseillers qui coupent les pairs dans le rang de
la séance parmi eux. Il le passe et revient à lui en son rang, comme
s'il y était en séance parmi les conseillers. Ils appellent cela garder
le banc. Contre qui et pour qui, c'est ce qu'ils ne sauraient
expliquer\,; mais aux usurpations de fait on voit qu'ils y sont maîtres.

Je leur en vis tenter une autre en 1700, où il y eut plusieurs
réceptions de pairs au parlement coup sur coup. Je vis un conseiller
demeurer à la tête du troisième banc aux bas sièges, les princes du sang
et autres pairs en remplissant plus de deux. Je le fis remarquer à mes
voisins, qui le trouvèrent aussi mauvais que nouveau, mais qui se
contentèrent d'en gronder tout bas. Cette mollesse, qui a tourné toutes
ces usurpations en prétentions soutenues, me détermina sur-le-champ à en
faire un signe très marqué au premier président Harlay (quoique, depuis
l'affaire de M. de Luxembourg, je fusse demeuré hors de toute mesure
avec lui), résolu de faire un éclat sur-le-champ, et de sortir de séance
avec les pairs, s'il eût soutenu la gageure\,; mais il n'osa, et dans
l'instant fit signe des yeux et de la main à ce conseiller de s'ôter de
la, et à moi un d'excuse. Le conseiller obéit aussitôt\,; mais, si on
l'y avait laissé cette première fois, comme on le laissa à la dernière
place lorsqu'il l'usurpa la première fois, la chose en serait demeurée
comme l'autre. Ils n'ont pas hasardé celle-ci depuis.

Venons maintenant à deux entreprises qui en tout genre se peuvent dire
n'avoir point de nom, et qu'il est aussi nécessaire que honteux de
décrire, pour voir jusqu'à quel excès d'orgueil et de vétilles les
choses sont poussées par les présidents. Le récit en est aussi curieux
qu'il est en soi dégoûtant.

La grand'chambre est vaste et fait un carré plus long que large, et la
séance qui la coupe par le dossier des bancs de séance en équerre, comme
on le verra mieux sur le plan, fait un autre carré. De ce carré
particulier, et conséquemment de la totalité de la grand'chambre, la
droite et la gauche se règlent et se prennent de celles de la place que
le roi prend quand il y vient, qui est dans l'angle du fond, ce qui
s'appelle le coin du roi. Le banc des pairs, tant en haut qu'en bas, la
lanterne de la cheminée, la cheminée qui est hors le barreau et dans la
grand'chambre près de cette lanterne qui en a pris son nom, sont à la
droite du coin du roi\,; et le banc des présidents, tant en haut qu'en
bas, est à sa gauche, ainsi que la lanterne de la buvette.

Outre que par le fait et la simple inspection cela est ainsi, il y en a
deux autres preuves\,: l'une que le roi séant, la reine régente, s'il y
en a une, les fils de France, les princes du sang et les autres pairs
sont de suite, et sans distinction que la préséance, assis sur ce banc à
droite, et les pairs ecclésiastiques de l'autre qui est à gauche\,; or
les pairs ecclésiastiques ni les cardinaux, lorsqu'ils y venaient, ne
l'auraient pas emporté sur la reine régente et sur les fils de France\,;
ni même en cette séance en haut les pairs ecclésiastiques sur les
séculiers, parce que ces deux bancs sont affectés et demeurés suivant
l'ancienneté de la séance, et alors les six anciens pairs laïques
précédaient comme plus anciens les six ecclésiastiques. Il n'a donc
nulle difficulté pour reconnaître ce banc des pairs pour être à la
droite du roi, et le plus honorable.

Alors, comme on l'a dit, toute la magistrature est aux bas sièges, et
les présidents ont mieux aimé en ces occasions demeurer sur leur banc
ordinaire, qui est aussi à gauche quand la séance est à l'ordinaire en
bas, parce que le banc à droite y est aussi pour les pairs, que de
changer de place pour se mettre sur ce banc en bas à droite, que nul
magistrat ne pourrait leur disputer, et où les pairs, le roi présent, ne
peuvent venir parce qu'ils ne peuvent être alors qu'aux hauts sièges\,;
les présidents, dis-je, aiment mieux demeurer en leur place accoutumée
en bas que de montrer qu'ils ne se peuvent mettre sur celui de droite
que lorsque les pairs ne seoient point en bas, mais ce choix des
présidents ne change pas la droite et la gauche.

Une autre preuve encore, c'est qu'entre les avocats contraires de
parties inégales, celui de la première en dignité, demandeur ou
défendeur, prend de droit le barreau de la cheminée. Cela est sans
difficulté pour les princes du sang, les pairs, les ducs vérifiés, les
officiers de la couronne. C'est ce qui s'appelle le choix du barreau. Et
quand il y a dispute de rang reconnu au parlement, car celui de prince
étranger y est constamment ignoré, par exemple entre deux pairs en
contestation pour leur ancienneté, c'est un préalable nécessaire de
juger cette préférence, et c'est un préjugé favorable à la prétention
d'ancienneté de l'un sur l'autre que cette préférence de barreau adjugée
à l'un des deux. C'est à ce même barreau encore que les avocats généraux
plaident, et que le procureur général parle, et jamais à celui de la
buvette qui est de même joignant la lanterne de la buvette. Or il n'y a
que ces deux barreaux. Par toutes ces choses il est donc clair qu'en
haut et en bas les pairs seoient à droite et les présidents à gauche.
Cette gauche déplaît infiniment aux présidents, et voici ce qu'ils ont
imaginé pour la masquer tant en haut qu'en bas.

En haut le banc des pairs à droite et celui des présidents à gauche
joignent l'un et l'autre le coin du roi tout contre également. Ce coin
est juste dans l'angle de la muraille, et y est adossé tout contre,
comme y sont aussi adossés les deux bancs à droite et à gauche. Quand le
roi n'y est point, et c'est le temps dont on parle, ce coin est nu,
tapissé comme les bancs, sans autre marchepied que celui des deux bancs,
qui est de même hauteur et largeur le long des deux et devant le coin où
ils se joignent. Le coin est élevé de deux pieds plus que le siège des
bancs\,; il est plus profond, d'un peu de saillie devant et derrière à
cause de l'encoignure, mais sans déborder la largeur du siège des bancs,
et à s'y asseoir sur sa nudité il n'est guère plus large qu'il ne
faut\,; rien derrière que la tapisserie qui suit les deux pans de
muraille, et quoi que ce soit au-dessus. Ainsi le premier des princes du
sang ou des pairs du côté droit et le premier président du côté gauche
touchent également du coude ce coin du roi.

Cette égalité déplut au premier président de Novion. Il fit débourrer le
banc des pairs à huit pieds de long près le coin du roi, de manière que
qui s'y assoirait serait si bas que, outre l'incommodité de la simple
planche sous le mince tapis fleurdelisé comme le reste du banc, le haut
de sa tête n'atteindrait pas l'épaule, à taille égale, de celui qui
serait sur le bout du même banc qui n'a pas été débourré\,; d'où il
arrive que, tandis que le premier président touche du coude le coin du
roi, le premier des princes du sang en est à huit ou dix pieds. M. le
duc de Berry et M. le duc d'Orléans l'éprouvèrent eux-mêmes avec grand
scandale à la séance des renonciations, mais ils se contentèrent d'en
parler sans ménagement, et eurent la mollesse d'en demeurer là. Cette
même distance, les princes du sang, qui viennent toujours aux réceptions
des pairs et qui toujours demeurent après à la grande audience,
l'éprouvent toutes les fois qu'ils s'y trouvent.

On croirait peut-être que le premier président de Novion s'en tint là\,;
mais le moyen d'avoir la grand'chambre et des tapisseries à sa
disposition, et de n'en pas profiter de toutes les façons\,! Le banc des
pairs et celui des présidents tout semblables, et de même hauteur à
s'asseoir, et de même largeur déplut à Novion. Il voulut un petit trône,
et pour cela fit rembourrer d'un pied et demi par-dessus le rembourrage
ordinaire des bancs les six premières places les plus proches du coin du
roi. Avec cette invention, les présidents à mortier se trouvent avoir un
pied et demi d'élévation de séance au-dessus des princes du sang et des
pairs. Ce fut encore une autre indignation de M. le duc de Berry et de
M. le duc d'Orléans qui essuyèrent cette élévation au-dessus d'eux,
élévation que les princes du sang essuient avec l'intervalle toutes les
fois qu'ils se trouvent en séance aux grandes audiences. Il faut ajouter
que les conseillers qui sont tout de suite sur le banc des présidents ne
se mettent point sur l'élévation présidentale. C'est un trône
nouvellement imaginé qui ne convient qu'aux inventeurs, tellement que,
s'il n'y a qu'un président ou deux à la grande audience, le premier des
conseillers qui est sur le banc est à six ou sept places de distance de
lui, qui demeurent vides, et si ce conseiller n'est pas bien grand, il a
la commodité de s'appuyer sur cette élévation, comme on fait sur le bras
d'un haut fauteuil. Telle est la nouvelle industrie pour relever la
majesté de la présidence, paisiblement soufferte de grands et de petits,
de princes du sang et de conseillers. Il est vrai qu'il est besoin que
la stature des présidents réponde un peu à l'exhaussement de leur siège,
et que j'en ai vu quelquefois gambiller de petits qui avaient peine à se
tenir, et qui donnaient un peu à rire à la compagnie.

En bas c'est autre chose\,; les inventions veulent de la variété. Il y a
un peu d'air de campement dans celle-ci, qui se donne sous prétexte du
vent, mais qui ne laisse pas d'être dans toutes les saisons. Elle fait
souvenir de ces anciens tribunaux militaires qu'on tendait en pleine
campagne, où les empereurs recevaient les tributs des nations vaincues,
et d'où les chefs des armées haranguaient leurs troupes ou leur
partageaient les dépouilles. Il y a des tringles et des machines, qui se
tendent si subtilement sur le banc des présidents qu'en un clin d'œil il
se trouve sous un dais fleurdelisé, qui a un dossier et deux pentes pour
les côtés, qui le déborde un peu par devant, et qui est un peu sur eux
en berceau. Le banc n'a point été rehaussé de rembourrage comme celui
d'en haut. Cela viendra peut-être avec le temps, et alors ce banc
deviendra un véritable trône un peu allongé, comme lorsqu'ils étaient
plusieurs associés à l'empire.

Quoique ce dais ne disparaisse pas devant les princes du sang, à plus
forte raison devant les pairs, ils n'osèrent pourtant le produire devant
M. le duc de Berry et M. le duc d'Orléans à la séance en bas des
renonciations\,; mais j'ai vu une fois, toutes les chambres assemblées,
je ne me souviens plus pourquoi (et alors, comme la place manque en bas
où est la séance, les chambres se placent aux hauts sièges), moi étant
en place avec les princes du sang et les autres pairs, que ce dais était
tendu, un murmure aux hauts sièges derrière à qui ce dais ôtait la vue
de la séance, un message ou deux venir à l'oreille du premier président
Harlay, et aussitôt le dais se détendre et disparaître en un instant.

Ce serait abuser que d'en dire davantage. Il faut laisser ces choses aux
réflexions des lecteurs, qui seront sans doute plus fortes et plus
justes que ce qui s'en pourrait faire ici avec décence. Mais il faut
encore dire un mot de l'indécence des saluts.

Il est réciproque entre les fils de France, les princes du sang et les
pairs. Les fils de France et les princes du sang se découvrent et se
lèvent en pied aussitôt qu'un pair paraît à l'angle d'entrée de la
séance en bas, ou débouchant en haut la lanterne de la cheminée, comme
il en arrive toujours quelqu'un depuis qu'on est en séance. Les fils de
France et les princes du sang leur rendent la révérence qu'ils en
reçoivent en allant à leur place, attendant qu'il y soit arrivé, et ne
se rassoient et couvrent qu'en même temps que lui. Il serait superflu
d'ajouter que les pairs en usent de même pour les fils de France et pour
les princes du sang. Les fils de France demeurent assis, se découvrent
et s'inclinent un peu sans se soulever, pour un président qui arrive en
séance\,; les princes du sang en usent pour eux comme pour les pairs\,;
et les présidents réciproquement. Ils se découvrent et se lèvent pour un
fils de France, et ne se rassoient et ne se couvrent qu'en même temps
que lui. M. le duc d'Orléans en usa comme les fils de France toutes les
fois qu'il a été au parlement, et les présidents de même pour lui, quant
au salut, que pour les fils de France.

Le salut est aussi réciproque entre les pairs et les présidents. Dès
qu'un pair paraît à l'entrée de la séance en haut ou en bas, comme il
vient d'être expliqué, tous les présidents se découvrent, et quand il
arrive à sa place, s'inclinent à lui, mais sans se lever ni même se
soulever, et ne se couvrent que lorsqu'il s'est incliné à eux, qu'il
s'assait et qu'il se couvre. Les pairs en usent précisément de même pour
les présidents.

Cela fait un effet un peu étrange de voir en séance les fils de France,
les princes du sang et les pairs debout pour un pair qui entre, et toute
la robe qui ne fait que se découvrir sans bouger. C'en est un second de
voir aussi les princes du sang debout tout seuls pour un président qui
entre, tout le reste de la séance découvert, mais assis sans bouger.
Enfin c'en est un troisième de voir les fils de France, les princes du
sang, les pairs et les présidents debout pour un prince du sang qui
entre, et les conseillers demeurer assis, découverts, car ils ne se
lèvent pour qui que ce soit excepté les fils de France, pas même pour la
tournelle qui, aux réceptions des pairs, vient à la grand'chambre, ayant
ses présidents à sa tête, pour lesquels les princes du sang et les
présidents de la grand'chambre se lèvent seuls, et de même à la sortie
de la tournelle après la réception. Il semble que ce soit un reste de
ces légistes assis sur le marchepied du banc des pairs, des barons, des
prélats, etc., et qui ne se levaient peut-être pas de si bas qu'ils
étaient assis pour des nobles qui survenaient, comme si subalternes et
si disproportionnés qu'il ne s'agissait pas d'en être salué.

Les présidents ni les conseillers ne remuent en rien pour un conseiller
qui entre ou qui sort, aux hauts sièges et aux bas\,; c'est même
observation pour les saluts. Il faut seulement ajouter que le
chancelier, qui entre en séance avant le roi, et les pairs aussi, se
lève, lui, pour un pair qui entre, et les pairs réciproquement pour lui.
Il n'y peut avoir de remarques à faire sur les autres officiers de la
couronne, parce que ceux que le roi a mandés entrent en séance derrière
lui, et qu'il n'est point alors d'occasion de salut.

Venons maintenant à l'explication du plan de la grand'chambre, qui est à
la page suivante, en remarquant qu'elle a été fort rajustée en 1720,
mais sans aucun autre changement que celui de la cheminée, ôtée d'où
elle est marquée sur ce plan et portée près de la grand'porte, qui entre
sans milieu de la grand'chambre dans la grande salle du palais, par où
les princes du sang et les pairs sortent de séance, comme il a été dit.

EXPLICATION DU PLAN CI À CÔTÉ DE LA GRAND'CHAMBRE DU PARLEMENT DE PARIS.

A. Hauts sièges adossés aux murailles.

\begin{enumerate}
\def\labelenumi{\arabic{enumi}.}
\item
  Élévation dans l'angle. C'est la place du roi quand il vient au
  parlement, que personne ne remplit jamais en son absence. Il est
  couvert de la même tapisserie fleurdelisée qui couvre les murailles,
  qui est pareille à l'étoffe qui couvre aussi tous les bancs et petits
  bureaux de la séance. Cette place du roi s'appelle de sa situation, le
  coin du roi. Il est orné d'autres tapis et de carreaux, couvert d'un
  dais, et accommodé d'un marchepied de plusieurs marches, lorsqu'il y
  vient.
\item
  Espace pour les marches du marchepied du roi lorsqu'il vient au
  parlement. Elles sont couvertes du tapis du marchepied. Sur ces
  marches où on met des carreaux, c'est la séance du grand chambellan,
  qui y est comme couché. En son absence le premier gentilhomme de la
  chambre en année la prend. C'est une ancienne nouveauté en faveur de
  Louis, duc de Longueville, qui n'était point pair, et qui, dans le
  grand état où ceux de Longueville s'étaient élevés, se trouvait peiné
  de seoir en son rang d'officier de la couronne. Il obtint cette
  distinction, mais attachée à son office, par le crédit du premier duc
  de Guise, dont il avait épousé la fille. Leur fils unique ne vécut
  pas. Léonor, duc de Longueville après Louis, son cousin germain, fut
  celui qui mit le comble à leur grandeur par tout ce qu'il obtint de
  Charles IX. Ce Léonor est le grand-père du duc de Longueville, père du
  comte de Saint-Paul, tué au passage du Rhin, et du dernier des
  Longueville, mort prêtre, fou et enfermé dans l'abbaye de
  Saint-Georges, près de Rouen, en 1696. Ce même Léonor était père de la
  marquise de Belle-Ile-Retz, et de la comtesse de Thorigny-Matignon. Le
  sieur de Rothelin était son frère bâtard, dont tous les Rothelin sont
  sortis.
\item
  Degré de cinq marches par lequel le roi monte et descend de séance.
  Quelquefois les fils de France aussi avec lui, toujours en son
  absence. On a vu, ci-devant, comment le premier président Harlay a
  ouvert ce degré aux princes du sang. Depuis cette nouveauté Louis XIV
  n'a point été au parlement, et dans la minorité de Louis XV M. le duc
  d'Orléans régent, les y a laissés passer avec le roi. On a vu qu'ils
  entraient et sortaient de séance auparavant à la tête et par le même
  chemin des pairs. Ce degré est couvert de la queue du tapis du
  marchepied du roi. C'est la séance, mais sans carreaux, du prévôt de
  Paris, qui y est aussi couché avec son bâton de velours blanc à la
  main\,; mais il demeure découvert, n'a point de voix, et se range pour
  faire place au chancelier ou au garde des sceaux, qui monte par ce
  degré pour aller parler au roi, et le redescend pour revenir à sa
  place.
\item
  Séance du chancelier, ou, en son absence, du garde des sceaux. C'est
  la place du greffier aux grandes audiences, qui est au bas des marches
  du petit degré du roi. Le greffier, en l'absence du roi, est là sur un
  tabouret, son petit bureau devant lui dans l'angle, et tourné en
  angle. Le roi présent, le chancelier est tourné de même avec le même
  petit bureau devant lui. Au lieu du tabouret du greffier, il a un
  siège à bras, sans aucun dossier, couvert de la même queue du tapis du
  marchepied du roi, mais de façon qu'elle vient à fleur de terre devant
  son siège, et qu'il n'a point les pieds dessus. Cette espèce de siège
  unique pour lui, et dont le garde des sceaux use en son absence, et
  qui sert aussi aux \emph{Te Deum}, est moins une distinction qu'un
  secours donné à la vieillesse si ordinaire à ces officiers-là de la
  couronne, qui ne pourraient demeurer longtemps assis sans quelque
  appui.
\item
  Petit bureau du greffier devant le chancelier, qui n'est couvert alors
  que comme à l'ordinaire. Quoique le chancelier et son petit bureau
  soient en bas comme tous les magistrats, on l'a marqué ici de suite, à
  cause de ses allées vers le roi, et du tapis du marchepied du roi, qui
  couvre son siège.
\item
  Séance de la reine régente ou du régent s'il y en a, du sang royal et
  des pairs. Le roi présent ou absent, ils sont assis de suite sans
  intervalle ni autre distinction en rang d'aînesse et d'ancienneté.
  Après eux les officiers de la couronne au rang de leurs offices entre
  eux, excepté le chancelier et le grand chambellan dont on a marqué la
  séance. Les officiers de la couronne qui sont pairs siègent en leur
  rang d'ancienneté parmi les pairs. Si le grand chambellan est pair, il
  demeure en la séance de son office et opine seul après tout le côté
  droit, et avant tout le côté gauche. Le roi n'étant pas présent, les
  pairs ecclésiastiques siègent sur ce même banc, après eux tous les
  pairs, ensuite les conseillers d'honneur, puis quatre maîtres des
  requêtes et non plus, après eux le doyen du parlement et les
  conseillers, et parmi, les conseillers honoraires. Mais il n'y a
  jamais place pour ces magistrats.
\item
  Espace de trois ou quatre places joignant le coin du roi entièrement
  débourré, et bien plus bas que les bancs de séance qui sont à droite
  et à gauche d'égale hauteur, largeur et profondeur, avec un marchepied
  tout du long des deux côtés, d'égale hauteur et largeur. Ces bancs
  d'égale façon, couverts de la même étoffe bleue fleurdelisée jusqu'à
  terre sans traîner et les dossiers de même. Sur ce débourré, dont on a
  parlé ci-devant, personne n'y seoit. C'est du côté droit, ce qui reste
  vide par respect du roi quand il est au parlement, et fait l'espace
  qu'occupent, en s'élargissant également des deux côtés, les marches du
  marchepied du roi, où le débourré paraît alors en espace comme de
  l'autre côté qui est en l'absence du roi ce plus haut rembourré des
  présidents dont on a parlé plus haut.
\item
  Lieu de séance du premier de ce banc, soit du sang royal, soit pair
  s'il n'y a point de princes du sang, le roi présent ou absent, soit
  magistrat si le roi n'y est point (car en sa présence nul magistrat
  n'est aux hauts sièges), s'il n'y a ni prince du sang ni autre pair.
  Ce même lieu fut celui de la séance de M. le duc de Berry à la séance
  des renonciations aux hauts sièges, sans distinction aucune de tout le
  reste du banc.
\item
  Espace entre le marchepied des hauts sièges et le haut du dossier des
  bas sièges, où on pousse tout du long un banc sans dossier, mais
  couvert et fleurdelisé comme les autres, lorsque le banc adossé à la
  muraille ne suffit pas pour les pairs.
\item
  Marchepied d'une marche régnant le long des hauts sièges des deux
  côtés, partout égal en hauteur et largeur sans différence en nul
  endroit.
\item
  Espace égal partout en largeur entre les hauts sièges et les bas
  sièges des deux côtés, à la hauteur presque du dossier des bas sièges.
\item
  Banc des pairs ecclésiastiques, le roi présent. Les cardinaux s'y
  mettaient aussi. Ils n'y sont pas venus depuis la décision de la
  préséance sur eux de pairs ecclésiastiques que M. de
  Clermont-Tonnerre, évêque-comte de Noyon, fit prononcer par Louis XIV,
  allant tenir un lit de justice où les cardinaux de Bouillon et Bonzi
  prétendaient se trouver comme il a été dit ailleurs. Le cardinal
  Dubois, premier ministre tout puissant, entreprit de se trouver à un
  lit de justice de Louis XV et en fit grand bruit et menaces. M. de
  Tavannes, évêque-comte de Châlons, depuis archevêque de Rouen, qui se
  trouva seul à Paris des pairs ecclésiastiques, lui fit dire qu'il y
  irait résolûment, et que s'il se mettait en fait de se placer
  au-dessus de lui, ou d'y demeurer s'il arrivait avant lui, il le
  jetterait des hauts sièges en bas quoi qu'il en pût arriver, et qu'il
  y serait assisté et soutenu des pairs laïques avec qui la résolution
  était prise. Elle l'était en effet et avait passé par moi, et aurait
  été exécutée si le cardinal Dubois s'y fût commis. M. de Châlons
  arriva de bonne heure en séance. Le cardinal Dubois n'y parut point.
  Le roi absent, c'est où siègent aux grandes audiences les présidents
  et les conseillers clercs.
\item
  Élévation moderne de surrembourrage fort haute au-dessus des bancs de
  séance. Elle joint le coin du roi et a cinq ou six places et en aurait
  bien huit sans l'ampleur des habits des présidents qui seoient dessus.
  Le même espace était de ce côté gauche comme il est encore du côté
  droit avant cette invention et innovation et y est encore le roi
  présent.
\item
  Lieu où sied le premier président ou le président qui préside en sa
  place. Je leur ai vu mettre familièrement leur mortier, et leur bonnet
  quelquefois sur le coin du roi.
\item
  Endroit où le surrembourrage finit, et tout à coup tombe au niveau du
  rembourrage des bancs de séance sous la même tapisserie fleurdelisée
  qui couvre tous les bancs.
\item
  Lieu de séance du premier conseiller clerc, lors même qu'il n'y a
  qu'un président en place\,; alors le reste de l'élévation demeure vide
  parce qu'il n'y a que les présidents qui s'y mettent, et cela arrive
  très ordinairement. Lorsque tous y sont, ce qui est fort rare, comme à
  la séance aux hauts sièges des renonciations, les cinq premiers
  présidents s'assirent sur cette élévation, les autres au bas de
  l'élévation à la place des conseillers clercs, lesquels se mirent de
  suite auprès d'eux et sans intervalle.
\item
  Degré de cinq marches qui communique les hauts et les bas sièges au
  bout du banc des pairs près la lanterne de la cheminée.
\item
  Lanterne de la cheminée.
\item
  Banc adossé au mur dans la lanterne de la cheminée.
\item
  Échelle par où on monte dans la tribune de la lanterne de la cheminée.
\item
  Degré de cinq marches dans la porte qui donne de la lanterne de la
  cheminée dans la grand'chambre, par lequel les pairs entrent et
  sortent de séance aux hauts sièges, au bas duquel en sortant ils
  trouvent un huissier pour leur faire faire place et les conduire comme
  on l'a dit. Le sang royal à la tête des pairs, entre encore par là en
  séance aux hauts sièges, mais n'en sort plus par la, comme on l'a dit.
  Les conseillers laïques y entrent aussi par là, mais ils en sortent
  par ailleurs.
\item
  Lanterne de la buvette.
\item
  Banc adossé au mur dans la lanterne de la buvette.
\item
  Degré par où on monte dans la lanterne de la buvette.
\item
  Degré de cinq marches de la lanterne de la buvette par lequel les
  pairs ecclésiastiques, le roi présent seulement, et, en son absence,
  les présidents et les conseillers clercs entrent et sortent de séance
  aux hauts sièges, au bas duquel deux huissiers, avant l'innovation de
  l'escalade dont on a parlé, et maintenant un huissier, attendent les
  présidents pour marcher devant eux, leur faire faire place avec leurs
  baguettes frappantes sur les bois qu'ils trouvent, et les conduire
  comme on l'a dit.
\item
  Porte de la lanterne de la buvette qui donne dans la grand'chambre
  dans laquelle est le degré susdit. Mais cette partie de la
  grand'chambre où cette porte donne est une allée entre la clôture du
  parquet des bas sièges et la muraille, qui conduit sans séparation
  dans la partie pleine de la grand'chambre, au lieu que la porte de la
  lanterne de la cheminée, qui est le chemin des pairs donne
  immédiatement dans la pleine grand'chambre. Les conseillers laïques,
  qui, en l'absence du roi, peuvent avoir place du côté des pairs,
  attendent en place qu'ils soient tous entrés jusqu'au dernier dans la
  lanterne de la cheminée, puis longent le banc, passent devant le coin
  du roi, et longeant l'autre banc atteignent les magistrats par la
  lanterne de la buvette.
\item
  Porte de la lanterne de la buvette qui mène à la buvette.
\end{enumerate}

Avant de quitter les hauts sièges, il faut remarquer que le nombre des
pairs étant augmenté, les officiers de la couronne qui ne sont pas
pairs, et il n'y a plus guère que des maréchaux de France, mais aussi
bien plus nombreux qu'ils n'étaient, proposèrent aux pairs de se mettre
à gauche aux lits de justice, au-dessous des pairs ecclésiastiques dont
le banc, par leur petit nombre, est toujours très peu rempli. Être
au-dessous des pairs laïques comme ils étaient, ou au-dessous des pairs
ecclésiastiques comme ils le demandaient, parut égal aux pairs, qui y
consentirent, et M. le duc d'Orléans le trouva bon. Cela s'exécuta ainsi
au premier lit de justice de Louis XV, et s'est toujours continué
depuis. Le duc de Coislin, évêque de Metz, eut le choix des deux
côtés\,; il préféra le droit comme n'étant point pair par son siège,
mais par soi, et y a toujours siégé en son rang d'ancienneté dans
l'habit des pairs ecclésiastiques.

B. Bas sièges.

\begin{enumerate}
\def\labelenumi{\arabic{enumi}.}
\setcounter{enumi}{27}
\item
  Ils sont sans marchepied, à la différence des hauts sièges, qui est un
  monument que ces bas sièges le sont comme on l'a dit, devenus, de
  marchepied qu'ils étaient des hauts, pour seoir les légistes aux pieds
  des nobles seuls juges, à portée d'en être consultés tout bas quand il
  leur plaisait. Ces bas sièges, depuis qu'ils le sont devenus, ont un
  dossier, parce qu'ils ne sont pas comme les hauts sièges appuyés à la
  muraille. Ils ont aussi un bras à chaque bout du banc, parce que,
  comme les hauts sièges, ils ne joignent pas le coin du roi d'un côté,
  et les lanternes de l'autre. Excepté ce qui a été marqué de débourré
  et surrembourré près du coin du roi aux hauts sièges, de l'invention
  des présidents, tous les bancs de la grand'chambre sont égaux en
  hauteur et largeur, sans nulle différence des uns aux autres. Ceux de
  séance sont couverts, comme les murailles et les petits bureaux,
  d'étoffe bleue fleurdelisée sans nombre, en jaune. Ces petits bureaux
  sont portatifs, et sont, comme un prie-Dieu, sans marchepied à appuyer
  à l'étroit une personne. Il y en a cinq ou six épars devant les bancs
  aux bas sièges, pour la commodité des rapporteurs. Les bancs hors de
  séance et leurs dossiers sont nus de bois, et pour asseoir les gens du
  roi, les parties, les plaideurs et les avocats qui veulent entendre
  plaider.
\item
  Dossier des bas sièges égal à tous.
\item
  Sièges, ou endroits où on s'assied sur tous les bancs.
\item
  Hauteur des bancs.
\item
  Chaires et bureaux du greffier et de son commis, rangés lorsqu'on est
  aux bas sièges.
\item
  Rideau à hauteur d'appui qui, lorsqu'on est aux bas sièges, enferme et
  cache le degré du coin du roi, et les chaires et bureaux du greffier
  et de son commis qui seoient là, le roi absent, lorsqu'on est aux
  hauts sièges. Quand on y doit monter, on ôte ce rideau pendant la
  buvette, et on y place les chaires et bureaux du greffier et de son
  commis.
\end{enumerate}

34 Parquet.

\begin{enumerate}
\def\labelenumi{\arabic{enumi}.}
\setcounter{enumi}{34}
\item
  Banc des présidents. Ils l'occupent seuls lorsque la séance est aux
  bas sièges, n'y eût-il qu'un président, et si par un cas très rare il
  ne se trouvait aucun président, le conseiller le plus ancien qui
  présiderait demeurerait à sa place, et laisserait le banc des
  présidents vide. On voit très clairement que c'est une usurpation des
  présidents sur les conseillers, puisque les conseillers clercs sont
  aux hauts sièges, sur le même banc avec les présidents, parce que
  c'est aux hauts sièges le côté des clercs, qui n'ont aucune
  distinction sur les conseillers laïques. Aux lits de justice ce banc
  est encore celui des présidents\,; en absence du roi aux grandes
  audiences, lorsque la séance est aux hauts sièges, ce même banc est
  celui des gens du roi où nul autre ne se met.
\item
  Surdossier moderne et avancé sur le banc des présidents, en manière de
  dais postiche, comme en berceau sur leur tête, avec une pente de
  chaque côté du banc. L'étoffe en est fleurdelisée, pareille à la
  couverture des bancs et des murailles. Il ne se tend pas encore en
  été\,; on n'ose le donner encore en distinction\,; elle s'introduit en
  attendant, sous prétexte du vent et du froid, comme si ce banc seul y
  était exposé. On a vu ce qui ci-devant a été dit de cette machine, qui
  avec des tringles se tend et s'ôte en peu de moments. On l'ôte
  toujours pendant la buvette, lorsqu'on doit monter après aux hauts
  sièges pour la grande audience. On ne l'a osé hasarder en présence du
  roi.
\item
  Petit bureau derrière lequel sied le premier président, ou le
  président qui préside en sa place. Si le chancelier vient au parlement
  sans que le roi y doive venir, il prend cette place, préside, fait
  toutes les fonctions du premier président en sa présence, l'efface
  totalement\,; de même aux hauts sièges où il le déplace. En haut et en
  bas, le roi absent, le premier président est assis à la gauche du
  chancelier, et le joignant. Si le chancelier arrive au parlement, le
  roi y venant, il déplace de même le premier président et l'efface, et
  ne se met en sa place au bas du petit degré du roi, qu'après que le
  roi est arrivé et place au coin orné en trône qu'il occupe. Le
  chancelier en bas et en haut, le roi absent, entre et sort de séance
  par le même chemin du premier président. Si le chancelier est absent
  et privé des sceaux, le garde des sceaux fait au parlement tout ce
  qu'y fait le chancelier et en a la séance.
\item
  Banc du sang royal des pairs ecclésiastiques et laïques, et des
  conseillers clercs.
\item
  Bureau derrière lequel sied le premier du sang royal, ou le plus
  ancien pair, et quand il n'y a ni princes du sang ni autres pairs, le
  premier des magistrats non président à mortier. Ce même lieu fut celui
  de la séance de M. le duc de Berry aux renonciations, où la séance fut
  d'abord en bas, puis en haut. Ni en bas ni en haut il n'y eut ni
  distance ni distinction aucune de sa place à celle du dernier pair. Ce
  même lieu est encore où se met le premier huissier aux grandes
  audiences ordinaires, le roi absent mais hors de séance.
\item
  Dernière place au bout de ce banc, où par l'usurpation moderne demeure
  séant le plus ancien des conseillers clercs, lors même que ce banc ne
  suffit pas aux pairs.
\item
  Second banc souvent rempli de pairs à leurs réceptions et autres
  solennités.
\item
  Dernière place de ce banc derrière le bureau, où par l'usurpation
  moderne demeure séant le deuxième conseiller clerc, lors même que ce
  deuxième banc ne suffit pas au nombre des pairs.
\item
  Bureau. Il faut remarquer que tous ces bureaux, tels qu'on les a
  décrits ci-devant, sont tous égaux entre eux et sans aucune
  différence. Le premier président n'en a mis aucune au sien jusqu'à
  cette heure.
\item
  Bureau du milieu, devant lequel on ne passe point pour entrer ni
  sortir de séance. On passe donc entre le banc et ce bureau, autrement
  ce serait traverser le parquet. On a ci devant expliqué ce que c'est
  que traverser le parquet, et qui sont ceux qui le traversent.
\item
  Chaire nue du greffier au bout du second banc susdit où il sied
  lorsque la séance est aux bas sièges.
\item
  Bureau dudit greffier.
\item
  Chaire nue de l'interprète, elle tient au bout du troisième banc.
  Entre elle et celle du greffier est le passage pour entrer et sortir
  de séance. Toutes deux sont à bras. Le siège et le dossier sont un peu
  plus élevés que ceux des bancs auxquels elles tiennent, et ces
  dossiers un peu arrondis au milieu du haut. Les pays étrangers ont
  assez souvent consulté autrefois le parlement sur leurs questions, et
  y faisaient quelquefois juger leurs causes. Comme leurs langues
  étaient inconnues au parlement, on plaça cette chaire pour celui qui
  interprétait les pièces et les écritures produites en langues
  étrangères. Depuis, cette chaire est demeurée comme en monument de son
  usage passé, que le parlement ne veut pas laisser oublier. Cette
  chaire, non plus que celle du greffier, n'est point réputée de la
  séance.
\item
  Troisième banc sur lequel se mettent les pairs lorsque les deux
  premiers ne suffisent pas à leur nombre. Alors les plus anciens de
  ceux qui y passent se mettent les plus proches de la chaire de
  l'interprète qui est vide et les moins anciens les plus près du banc
  des présidents. À mesure que les pairs remplissent ces bancs, les
  conseillers en sortent et vont se mettre aux hauts sièges.
\item
  Bureau au bout de ce troisième banc tout prés du banc des présidents.
  La séance du doyen du parlement est derrière ce bureau. Depuis
  l'usurpation moderne, lui ou un autre conseiller laïque y demeure
  séant, lors même que ce troisième banc ne suffit pas au nombre des
  pairs, ce que j'ai vu arriver plus d'une fois par la présence de tout
  le sang royal, légitime et illégitime du feu roi, et du grand nombre
  de pairs ecclésiastiques et séculiers. Tout ancien pair que je suis,
  je me trouvai sur ce banc à la séance de l'ouverture du testament de
  Louis XIV. Il faut remarquer que les pairs y siéent entre eux à
  retours de ce que font les conseillers, dont les plus anciens se
  mettent les plus proches du doyen et ainsi de suite, en sorte que le
  moins ancien conseiller du banc se trouve joignant la chaire de
  l'interprète.
\item
  Espace dans le parquet devant ce troisième banc, où se met un banc
  sans dossier, mais couvert et fleurdelisé comme tous les autres, pour
  y seoir ce qui reste de pairs, lorsque l'on présume que les trois
  bancs ne suffiront pas à leur nombre. Sur ce banc ajouté aucun
  conseiller n'y seoit, encore qu'il y eût peu de pairs dessus, ou qu'il
  demeurât entièrement vide, comme je l'ai vu arriver quelquefois. Il
  faut remarquer que les pairs qui passent sur ce troisième banc ne s'y
  placent pas comme sur celui qui est derrière\,; les plus anciens s'y
  mettent les plus près du banc des présidents et ainsi de suite.
\item
  Lieu où, debout et sans chapeau ni épée, les pairs qui n'ont pas
  encore pris séance prêtent le serment de pair de France prononcé par
  le premier président de sa place assis et couvert, tous les princes du
  sang, autres pairs et magistrats assis et couverts en séance. Ce
  serment, quoique ancien, a été introduit. Les pairs entraient pour la
  première fois en séance sans information et sans serment, comme font
  encore les princes du sang. Le premier huissier, qui se tient près du
  pair qui prête serment, lui rend son chapeau et son épée sitôt que
  l'arrêt de réception est prononcé, qui n'est autre que dès qu'il a
  levé la main, et que le premier président lui a dit\,: \emph{Ainsi le
  jurez et le promettez}, il ajoute\,: \emph{Monsieur, montez à votre
  place\,;} et à l'instant il remet son épée à son côté\,; il entre en
  séance, et se va seoir en son rang. Ce prononcé\,: \emph{Montez à
  votre place}, est l'ancien, qui n'a pas été changé depuis que les
  réceptions ont été changées des hauts sièges où on monte, aux bas où
  il n'y a pas une seule marche à monter.
\item
  Banc des gens du roi lorsque la séance est aux bas sièges, ou que le
  roi est présent.
\item
  Bancs des parties et des spectateurs en absence du roi. Ceux-ci le
  précèdent, et d'autres redoublés derrière servent aussi de séance aux
  \emph{enquêtes et requêtes} aux assemblées de toutes les chambres et
  aux lits de justice, à ceux dont le roi se fait accompagner, comme
  gouverneurs ou lieutenants généraux de province, baillis d'épée,
  chevaliers du Saint-Esprit, mais qui n'ont point de voix et qui
  demeurent découverts.
\item
  Premier barreau de choix ou de supériorité, où plaident les avocats
  généraux lorsque la séance est aux bas sièges et où les avocats, qui
  ont ce barreau par la supériorité de leurs parties, plaident aussi,
  soit que la séance soit aux hauts sièges ou aux bas sièges.
\item
  Lieu où plaide l'avocat.
\item
  Passage dans lequel l'avocat s'avance pour conclure à l'entrée du
  parquet, et qui sert aux pairs à sortir de séance aux bas sièges
  lorsqu'ils la lèvent avec la cour.
\item
  Porte de ce passage à hauteur d'appui debout, où il y a un pas pour
  l'arrêter. C'est cette porte que les pairs ont trouvée fermée comme on
  l'a dit, et qui les fait demeurer en séance sans se lever quand la
  cour se lève et sort, comme il a été expliqué.
\item
  Passage sans porte par lequel la cour entre et sort de séance aux bas
  sièges, et par lequel les princes du sang et les pairs y entrent
  aussi, la cour séante à mesure qu'ils arrivent. Les princes du sang en
  sortent aussi par là avant que la cour lève la séance, comme on l'a
  dit, et vont à la cheminée de la grand'chambre pour l'ordinaire
  attendre la grande audience ou les pairs viennent aussi après.
\item
  Second barreau, et il n'y a que ces deux.
\item
  Lieu où plaide l'avocat, soit que la séance soit aux bas sièges ou aux
  hauts.
\item
  Passage dans lequel l'avocat s'avance pour conclure à l'entrée du
  parquet, qui n'a point d'autre usage.
\item
  Porte de ce passage à hauteur d'appui debout, qui a un pas pour
  l'arrêter.
\item
  Espace long et étroit entre le second barreau et la muraille, qui
  conduit de la buvette et de la lanterne de la buvette dans le grand
  espace de la grand'chambre derrière le premier barreau. C'est par cet
  espace que la cour va de la séance des bas sièges à la buvette et
  qu'elle sort de séance aux hauts sièges
\item
  Vaste espace de la grand'chambre entre la muraille mitoyenne de la
  grande salle et le premier barreau, et la muraille mitoyenne à la
  quatrième chambre des enquêtes et le parquet des huissiers.
\item
  Cheminée de la grand'chambre qui, comme on l'a dit, a été supprimée et
  portée contre le mur mitoyen de la grand'chambre et de la grande
  salle, lorsqu'on répara la grand'chambre en 1721.
\item
  Porte de la grand'chambre qui donne dans la quatrième chambre des
  enquêtes.
\item
  Porte de la grand'chambre à deux battants qui s'ouvrent pour les
  pairs, qui donne immédiatement dans la grande salle, plus grande de
  beaucoup que les autres.
\item
  Porte de la grand'chambre qui donne dans le parquet des huissiers, par
  où tout le monde entre d'ordinaire dans la grand'chambre et par où la
  cour ensemble en sort. Les pairs ensemble sortent par la grande porte
  dans la grande salle immédiatement, même seuls quand il ne s'y en
  trouve qu'un.
\item
  Fenêtres de la grand'chambre.
\item
  Chemin du sang royal, pour sortir de séance des hauts sièges, depuis
  que le premier président Harlay lui a ouvert le petit degré du roi\,;
  quelquefois aussi, lorsque le roi y vient, pour entrer en séance en
  même temps que lui. Lors des renonciations, M. le duc de Berry et M.
  le duc d'Orléans après la séance aux bas sièges, et pendant la buvette
  montèrent aux hauts sièges avec les princes du sang et tous les pairs,
  mais sans ordre, et y demeurèrent en séance et en rang, tous jusqu'à
  l'arrivée de la cour sortant de la buvette. On a vu ailleurs que ces
  princes ne se soulevèrent seulement pas, et qu'ils ne rendirent aux
  présidents le salut que par une inclination légère, étant restés
  découverts en les attendant. Les princes du sang en usèrent cette
  fois-là de même, et les pairs aussi comme ils font toujours. M. le duc
  de Berry et M. le duc d'Orléans se trouvèrent fort scandalisés de la
  longueur de la buvette et du long changement d'habit des présidents,
  dont ils auraient pu abréger leur toilette au moins ce jour-là.
\item
  Chemin du sang royal pour entrer et sortir de séance aux bas sièges.
\item
  Chemin des pairs pour entrer et sortir de séance aux hauts sièges. Il
  est le même des conseillers clercs, le roi absent, pour entrer, non
  pour sortir.
\item
  Chemin des présidents pour entrer et sortir de séance aux bas sièges,
  et aussi des conseillers clercs.
\item
  Chemin des présidents pour entrer et sortir de séance aux bas sièges.
\item
  Chemin ordinaire des pairs pour entrer en séance aux bas sièges, pour
  ceux qui sont sur le premier banc et sur la première moitié du second.
\item
  Chemin quelquefois usité par quelques pairs pour entrer en séance aux
  bas sièges, pour ceux qui sont sur le premier banc et la première
  moitié du second. C'est le même par lequel les pairs sortent de séance
  quand ils se lèvent avec la cour.
\item
  Chemin rarement usité par quelques pairs pour entrer en séance aux bas
  sièges, pour ceux qui sont sur le premier banc et la première moitié
  du second.
\item
  Chemin des pairs pour entrer en séance aux bas sièges, pour ceux qui
  sont sur la seconde moitié du deuxième banc.
\item
  Chemin des pairs pour entrer en séance aux bas sièges, pour ceux qui
  sont sur le troisième banc et sur le banc ajouté.
\item
  Chemin des conseillers laïques pour sortir de séance aux hauts sièges.
\end{enumerate}

81 et 82. Chemin ordinaire des pairs d'entrer en la grand'chambre, et
d'en sortir ensemble précédés d'un huissier. C'est aussi celui du sang
royal, mais presque toujours les pairs arrivent un à un chacun à son gré
jusque dans la grand'chambre par le parquet des huissiers, et les
princes du sang aussi.

\begin{enumerate}
\def\labelenumi{\arabic{enumi}.}
\setcounter{enumi}{82}
\item
  Chemin par lequel les pairs sortent ensemble de la grand'chambre
  quelquefois toujours précédés par un huissier.
\item
  Endroit par où le premier huissier, par une invention et usurpation
  moderne, escalade par-dessus le banc des sièges bas, et son dossier
  depuis quelque temps pour grimper aux hauts sièges lorsque la séance
  s'en lève, pour se mettre au-devant du premier président, ou du
  président qui préside en sa place, lorsqu'il se lève, et marcher
  devant lui.
\end{enumerate}

Il faut avertir que, lorsqu'on est au haut des sièges, le roi absent,
tout le monde indifféremment s'assied sur les bancs de séance aux bas
sièges plaideurs, auditeurs, en un mot qui veut et peut, excepté sur
celui des présidents qui, comme on l'a dit, est alors pour les gens du
roi. Le reste de la foule s'assied en bas à terre, pêle-mêle dans le
parquet, et qui peut sur les petits bureaux, qu'ils couchent. Cela se
fait à grand bruit et impétuosité dès que la grande audience en haut
ouvre.

Il faut, une fois pour toutes, remarquer que, lorsqu'on parle ici des
présidents, il ne s'agit que des présidents à mortier, qui sont seuls
présidents du parlement. Les présidents des chambres des enquêtes et des
requêtes ne sont que des conseillers avec commission pour présider en
telle chambre, si bien qu'en l'assemblée de toutes les chambres dans la
grand'chambre, ou partout ailleurs où le parlement est assemblé en
entier ou par députés de tout le corps, ils ne précèdent point les
conseillers de la grand'chambre, et en tout et partout ne sont réputés
que conseillers

Malgré cela, il y a une dispute dont les ministres se sont utilement
servis, et qu'on a grand soin d'entretenir sous main\,; c'est quand il
arrive, et cela n'est pas rare, que, dans une assemblée de toutes les
chambres, le gros du parlement est opposé à ce que la cour veut faire
passer, et que le premier président n'a pu venir à bout d'y amener la
compagnie, il prend plutôt le parti de se retirer que de hasarder d'être
tondu. Très ordinairement il est suivi de tous les présidents à mortier,
gens qui, ayant à perdre et à gagner, veulent plaire, qui désirent leur
survivance pour leurs enfants et d'autres grâces. Alors qui présidera\,?
Le doyen du parlement, en son absence le plus ancien conseiller de la
grand'chambre, de ceux qui demeurent en séance, prétend que c'est à lui,
le plus ancien président des enquêtes le lui dispute\,; le premier des
présidents de la première chambre des enquêtes allègue la primauté de sa
chambre et de sa présidence dans cette chambre. Dans ce conflit où aucun
n'a jusqu'à présent voulu céder, personne ne préside, et, faute de
président, la séance est forcée de se rompre et de lever. Ils sentent
bien tout ce qu'ils perdent à cette dispute, mais l'orgueil l'emporte
sur la raison et sur l'intérêt général de la compagnie.

\hypertarget{chapitre-xix.}{%
\chapter{CHAPITRE XIX.}\label{chapitre-xix.}}

~

{\textsc{Courte récapitulation.}} {\textsc{- État premier des
légistes.}} {\textsc{- Second état des légistes.}} {\textsc{- Troisième
état des légistes.}} {\textsc{- Quatrième état des légistes.}}
{\textsc{- Cinquième état des légistes.}} {\textsc{- Sixième état des
légistes.}} {\textsc{- Septième état des légistes devenus magistrats.}}
{\textsc{- Parlements et autres tribunaux.}} {\textsc{- Légistes devenus
magistrats ne changent point de nature.}} {\textsc{- Origine du nom de
cour des pairs arrogé à soi par le parlement de Paris.}} {\textsc{-
Origine des enregistrements.}} {\textsc{- Incroyables abus.}} {\textsc{-
Fausse mais utile équivoque du nom de parlement\,; sa protection\,; son
démêlement.}} {\textsc{- Anciens parlements de France.}} {\textsc{-
Parlements d'Angleterre.}} {\textsc{- Moderne chimère du parlement de se
prétendre le premier corps de l'État, réfutée.}} {\textsc{- Époque du
tiers état.}} {\textsc{- Parlement uniquement cour de justice pour la
rendre aux particuliers, incompétent des choses majeures et des
publiques.}} {\textsc{- Parlement ne parle au roi, et dans son plus
grand lustre, que découvert et à genoux comme tiers état.}} {\textsc{-
Inhérence de la partie de légiste jusque dans le chancelier.}}
{\textsc{- Jamais magistrat du parlement ni d'ailleurs, député aux états
généraux, ne l'a été que pour le tiers état, quand même il serait
d'extraction noble.}} {\textsc{- Exemples d'assemblées où la justice a
fait un corps à part, jamais en égalité avec l'Église ni la noblesse, et
jamais aux états généraux jusqu'aux derniers inclus de 1614.}}
{\textsc{- Absurdité de la représentation ou de l'abrégé des états
généraux dans le parlement.}} {\textsc{- Court parallèle du conseil avec
le parlement.}} {\textsc{- Conclusion de toute la longue digression.}}

~

Il se trouvera encore en leur ordre d'autres usurpations du parlement
aussi peu fondées, et plus fortes encore, s'il est possible, que celles
qui viennent d'être expliquées, qui demandent une récapitulation en très
peu de mots, depuis le premier état des légistes, jusqu'à celui où on
les voit arrivés.

Le peuple conquis, longtemps serf et dans la dernière servitude, ne fut
affranchi que longtemps après la conquête, et par parties. De ce qui fut
affranchi les uns demeurèrent colons dans la campagne et laboureurs,
soit pour eux-mêmes dans les rotures qu'ils avaient obtenues à certaines
conditions, ou pour autrui, comme fermiers\,; les autres continuèrent à
s'adonner à la profession mécanique, c'est-à-dire aux différents métiers
nécessaires à la vie dans les villes, et cela de gens de même espèce de
peuple affranchi. Des uns et des autres il s'en fit une autre portion de
gens plus aisés par leur travail, qui se mirent à quelque négoce, et
dont les seigneurs se servirent pour la direction commune de leurs
villes, d'où sont venus les échevins et autres sous divers noms. De
ceux-là il y en eut qui s'appliquèrent à l'étude des lois, des coutumes,
des ordonnances qui multiplièrent avec le partage des fiefs, leurs
hypothèques, etc., et les procès qui en naquirent, et ceux-là devinrent
le conseil des particuliers, dans leurs affaires domestiques\,; ils
furent connus sous le nom de légistes, qui gagnèrent leur vie à ce
métier, comme ils font encore aujourd'hui, qui étaient parties de ce
peuple serf mais affranchi, et qui, au lieu du labourage et des métiers,
choisirent celui de l'étude des procès. Tel est le premier état des
légistes.

Ces légistes furent placés par saint Louis sur le marchepied des nobles
et des ecclésiastiques, qui {[}étaient{]} nommément choisis par les rois
pour rendre la justice entre particuliers, dans les différentes tenues
d'assemblées pour cela, qui de parler ensemble s'appelèrent parlements,
quoique totalement différentes des assemblées majeures aussi appelées
parlements, qui avaient succédé aux champs de mars, puis de mai, où le
roi jugeait les causes majeures de pairs et des grands vassaux, et
faisait avec eux les grandes sanctions du royaume. Saint Louis,
scrupuleux sur l'équité, crut devoir soulager celle de ces nobles et de
ces ecclésiastiques, juges tantôt les uns tantôt les autres dans ces
parlements de la Pentecôte, de la Toussaint, etc., qui duraient peu de
jours, en les mettant à portée de s'éclaircir tout bas de leurs doutes
dans les jugements qu'ils avaient à rendre sur-le-champ, en consultant
tout bas ces légistes assis à leurs pieds qui ne leur disaient leur avis
qu'à l'oreille, et lors seulement qu'il leur était demandé, avis
d'ailleurs qui n'obligeait en rien celui qui avait consulté de le
suivre, s'il ne lui semblait bon de le faire. Tel est le second état des
légistes, qui dura fort longtemps.

La multiplication des affaires et de leurs formes, dont est née la
chicane, lèpre devenue si ruineuse et si universelle, multiplia et
allongea les tenues des parlements, en dégoûta les nobles et les
ecclésiastiques nommés pour chaque tenue, qui s'excusèrent la plupart,
occupés de guerres, d'affaires domestiques, de fonctions
ecclésiastiques\,; plus encore les pairs qui, de droit et sans être
nommés, étaient de tous ces parlements toutes fois qu'il leur plaisait
d'y assister, à la différence de tous autres, même des hauts barons, qui
n'y pouvaient entrer sans y être expressément et nommément mandés. Cette
espèce de désertion et la nécessité de vider les procès acquit aux
légistes la faculté de les juger avec ce peu de nobles et
d'ecclésiastiques qui se trouvaient à ces parlements du nombre de ceux
qui y étaient mandés et qui envoyaient leurs excuses, mais demeurant
toutefois assis sur le même marchepied, et c'est le troisième état des
légistes.

Bientôt après, ce peu d'ecclésiastiques et de nobles d'entre les mandés
pour composer ces parlements achevèrent de s'en dégoûter. Alors les
légistes, devenus d'abord juges avec eux, le demeurèrent sans eux par la
même nécessité de vider les causes. C'est le quatrième état des
légistes\,; mais toujours sur le marchepied, parce qu'il pouvait venir
de ces nobles et de ces ecclésiastiques mandés, dont souvent il s'en
trouvait à quelques séances.

La maladie de Charles VI et le choc continuel des factions d'Orléans et
de Bourgogne fit prendre le parti de ne changer plus les membres de ces
parlements qui demeurèrent à vie. Ce fut l'époque de la manumission des
légistes. Les nobles et les ecclésiastiques choisis pour ces parlements,
voyant qu'il fallait désormais assister à tous, ne purent s'y résoudre,
trop occupés de leurs guerres, de leurs fonctions, de leurs affaires.
Presque tous s'en retirèrent, de sorte que les légistes demeurèrent
seuls membres des parlements et seuls juges des procès. C'est leur
cinquième état, qui n'a fait que croître depuis à pas de géant.

Le parlement, devenu fixe à Paris et sédentaire toute l'année par la
multiplication sans nombre des procès, éleva de plus en plus les
logistes\,; ce qui fut leur sixième état.

Les malheurs de l'État et la nécessité d'argent tourna en offices
vénaux, puis héréditaires, leurs commissions devenues à vie, et forma le
septième état des légistes, qui alors, juges à titre d'office vénal et
héréditaire, devinrent magistrats, firent une compagnie réglée et
permanente, tels qu'ils sont demeurés depuis. De là sortit la formation
successive des autres parlements du royaume et de tant d'autres sortes
de tribunaux partout. C'est le septième état des légistes, qui forme
leur consistance jusqu'à aujourd'hui.

Ces gradations néanmoins ne changèrent pas la nature originelle et
purement populaire des légistes devenus magistrats, comme on le
démontrera bientôt, et ne l'a pas changée jusqu'à présent, quelques
efforts que dans la suite ils aient pu faire pour sortir de cette
essentielle bassesse, dont l'idée ne leur est venue que longtemps
depuis.

Devenu cour de justice, pour juger les causes des particuliers, le
parlement de Paris prit occasion de s'arroger le titre de cour des
pairs, de ce qu'étant la plus ordinaire à la portée des rois et de leur
accompagnement, les pairs y prenaient bien plus ordinairement séance, et
que, pour les choses que les rois voulaient rendre notoires par quelque
solennité publique, ils allaient avec les pairs les déclarer en
parlement. Cette même raison de rendre notoire ce qui émanait du roi,
comme édits, ordonnances, déclarations, érections de pairies, lettres
patentes, etc., les engagea de les envoyer registrer au parlement,
\emph{ut nota fierent}, et afin que les tribunaux y conformassent leurs
jugements. C'est ce qui fit envoyer les mêmes actes aux autres
parlements, et aux divers autres tribunaux qui pourraient avoir à rendre
des jugements en conformité.

À quelque distance déjà prodigieuse que ces divers degrés aient porté
les légistes de leur source et de leur état primitif, mais sans avoir
lors, ni jamais depuis, pu changer leur nature originelle, qui
d'eux-mêmes, dans l'élévation où on les voit ici, aurait osé imaginer de
se parangonner\footnote{De s'égaler.} aux pairs, de précéder les pairs
nés successibles de droit à la couronne, d'opiner devant une reine
régente en lit de justice, malgré la différence immense du lieu et de la
posture d'opiner, de parler aux pairs en public comme on ne parle même
plus aux valets d'autrui, de n'oublier rien pour les égaler en tout aux
légistes et pour oser se former un trône, l'un fort élevé, l'autre sous
une sorte de pavillon royal, et de là voir en places communes les pairs,
les princes du sang et les fils de France, et que les entreprises se
souffrent depuis tant d'années, et s'augmentent encore au gré de
l'orgueil et de l'industrie\,? Enfin, qui de ces légistes si parvenus au
point où on les voit arriver à cette cause, eût pu croire qu'il fût
tombé dans l'esprit de leurs successeurs de s'ériger en tuteurs des rois
mineurs, en modérateurs des rois majeurs, dont l'autorité a besoin de la
leur jusqu'à demeurer inutile et sans effet sans son concours, et
prétendre faire d'une simple cour de justice le premier corps de l'État,
ayant tout pouvoir par soi sur tous les grands actes concernant le
royaume\,? On a déjà vu la plupart de ces usurpations monstrueuses, dont
on a tellement abrégé tout ce qui pouvait l'être sans en affaiblir la
lumière que la récapitulation en serait presque aussi longue que l'a été
le récit, si on ne se contentait de ce peu de lignes. Venons, en
attendant des détails qui seront fournis par la régence de M. le duc
d'Orléans, à cette prétention si moderne d'être le premier corps de
l'État, et qui est telle qu'il n'est point de nom qu'on puisse lui
donner.

Le nom de parlement a été d'un grand usage pour éblouir. Les ignorants
qui font plus que jamais le très grand nombre dans tous les États\,; la
magistrature et ses suppôts, qui composent un peuple entier, dont
l'intérêt n'a cessé de donner cours aux idées les plus absurdes\,; les
gens faibles et bas qui ne veulent pas choquer des gens qui peuvent
avoir leurs biens entre leurs mains, quelquefois même leur vie, et qui
s'en servent avec la dernière hardiesse et liberté pour leurs
vengeances\,; tout ce qu'il y a de gens de condition magistrale, ou qui
en ont le but en sortant des bas emplois de finance et de plume qui
maintenant inonde tous les parlements\,; toute la bourgeoisie qui ne
peut avoir que le même but pour leurs familles\,; les marchands, ceux
qui se sont enrichis dans les métiers mécaniques pour relever leurs
enfants\,; tout cela fait un groupe qui ne s'éloigne guère de
l'universalité. Ajoutons à ce poids l'idée flatteuse qui en entraîne
tant d'autres, que le parlement est le rempart contre les entreprises
des ministres bursaux sur les biens des sujets, et il se trouvera que
presque tout ce qui est en France applaudira à toutes les plus folles
chimères de grandeur en faveur du parlement, par crainte, par besoin,
par basse politique, par intérêt ou par ignorance. Cette compagnie a
bien connu de si favorables dispositions, et bien su s'en prévaloir\,;
son nom de parlement, le même pour le son que celui de ces anciens
parlements de France où se faisaient les grandes sanctions de l'État, le
même encore que celui des parlements d'Angleterre, leur a été d'un
merveilleux usage pour se mettre dans l'idée publique à l'unisson de ces
assemblées, avec qui le parlement n'a rien de commun que le nom.

On a vu quelle est la totale différence de la nature des anciens
parlements de France et de ceux d'aujourd'hui, et quelle est la distance
et la disproportion des matières, des membres, du pouvoir de ces
anciennes assemblées, d'avec celles et ceux d'un tribunal qui n'est
uniquement qu'une cour de justice pour juger les causes entre
particuliers, et dont les membres légistes devenus juges et magistrats,
comme on l'a vu, sans avoir changé de nature, n'ont plus que des offices
vénaux à qui en veut, héréditaires, et qui font une portion de leur
patrimoine, tant par le sort principal\footnote{C'est-à-dire le prix de
  l'office.}, que par les gages, les taxations devacations,
d'épices\footnote{Voy., sur les épices, les notes à la fin du volume.},
et toutes les ordures d'un produit auquel tous, depuis le premier
président jusqu'au dernier du parlement, tendent journellement la main
et y reçoivent le salaire de chaque heure de travail ou de prétendu tel.

De tels membres sont plus distants, s'il se peut, des pairs et des hauts
barons qui composaient seuls les anciens parlements, que le morceau de
pré ou de terre, que l'hypothèque sur tel bien et les chicanes
mercenaires qui font la matière des jugements des parlements
d'aujourd'hui, des jugements des causes majeures des grands feudataires,
et les grandes sanctions du royaume, qui étaient la matière de la
décision de ces anciens parlements. Que si l'on compare à ceux
d'aujourd'hui ces parlements tenus en divers temps de l'année, il n'y a
qu'à comparer les nobles et les ecclésiastiques nommés par le roi pour
les composer, avec les légistes assis sur le marchepied de leurs bancs
pour les conseillers quand ils voulaient s'éclaircir tout bas de quelque
chose\,; et quant aux matières, si elles se rapprochent un peu plus, il
ne se trouvera pas que ces parlements tenus en divers temps de l'année
aient imaginé de pouvoir juger les causes majeures, ni de délibérer sur
rien de public.

Si on cherche plus de similitude avec les parlements d'Angleterre, ceux
dont il s'agit ici n'y trouveront pas mieux. Le parlement d'Angleterre
est l'assemblée de la nation, ou, suivant nos idées, la tenue des états
généraux, avec cette différence des nôtres, que ceux-là ont le pouvoir
tellement en propre pour faire ou changer les lois et pour tout ce qui
est droit et imposition, que le pouvoir des rois d'Angleterre est de
droit et de fait nul en ces deux genres sans le leur, et qu'il ne s'y
peut rien faire que par l'autorité du parlement. Elle est telle,
qu'encore que le droit de déclarer la guerre et de faire la paix y soit
une des prérogatives royales, on voit néanmoins que les rois veulent
avoir l'avis et le consentement de leurs parlements sur ces matières, et
qu'ils n'entreprennent rien de considérable au dehors ni au dedans sans
le consulter. Ce qui fait voir que subsides, levées de troupes,
fortifications, armements et mille autres choses publiques sont sous la
main du parlement autant ou plus que des rois.

En serait-ce là que nos parlements d'aujourd'hui en voudraient venir,
après avoir terrassé les grands du royaume, précédé les princes du sang,
opiné devant la reine régente, montré leurs présidents au sang royal,
eux sur une sorte de trône, et ces princes sur les bancs communs, cassé
les arrêts du conseil, et s'être faits les tuteurs des rois mineurs, les
modérateurs des rois majeurs, et les soutiens des droits des peuples
contre les édits, du bon ordre contre les lettres patentes, enfin, comme
ils se plaisent d'être nommés, le sénat auguste qui tient la balance
entre le roi et ses sujets\,? Dans de tels desseins, que d'éloignement
du parlement d'Angleterre où rien ne peut passer sans le concours des
deux chambres, ou la basse a plus de gentilshommes et de cadets de
seigneurs que d'autres députés, où la haute n'est composée que de pairs,
et qui, privativement à la chambre basse, juge tout ce qui se porte de
causes contentieuses devant le parlement, comme la basse, privativement
à la haute, se mêle des subsides, des impositions, des comptes et de
tout ce qui est commerce et finance, avec cette différence, toutefois,
qu'elle a besoin pour l'exécution de toutes ces choses du consentement
de la chambre haute, et que la chambre haute fait exécuter tous les
jugements qu'elle rend, sans aucun concours de la chambre basse. Où
trouver là une ombre, je ne dis pas de similitude, mais de ressemblance
la plus légère avec nos parlements\,?

Malgré une disparité si parfaite, si entière, si complète de la nature
et des membres de nos parlements d'aujourd'hui, d'avec la nature et les
membres de nos anciens parlements, et d'avec ceux d'Angleterre, jusqu'à
présent, et des matières de chicane et de questions de droit ou de fait
à juger entre des particuliers par des magistrats légistes d'origine
jusqu'à nos jours, et qui reçoivent eux-mêmes des plaideurs un écu par
heure de salaire à la sortie de chaque vacation, et les matières
publiques et d'État, comme les jugements des grands fiefs et des grands
feudataires, et les grandes sanctions du royaume réservées au roi, à
tout ce qu'il y a de plus grand et de plus auguste dans l'État avec lui,
et quant à l'Angleterre, ce qui vient d'en être expliqué et qui repousse
nos parlements à l'état des shérifs et des jurés, s'ils veulent toujours
une similitude anglaise, le parlement flatté de ce nom s'est plu à jouer
sur le mot et à tromper le monde par des équivoques que le monde a
reçues par les raisons d'ignorance, d'intérêt et de faiblesse qui en ont
été d'abord expliquées.

Ces fausses lueurs qui s'évanouissent si précipitamment au plus léger
rayon de lumière, appuyées du bruit que la cour a souvent fait faire au
parlement contre celle de Rome, par les raisons qui en ont été dites, et
des dernières régences déclarées au parlement pour les conjonctures et
les causes qui en ont été expliquées, ont enhardi le parlement aux
prétentions, et apprivoisé lui-même par les succès inespérables avec les
plus inconcevables absurdités, \emph{accinxit se}\footnote{Il se
  prépara, et en quelque sorte s'arma pour.} pour y accoutumer le monde.
C'est ce qui m'a obligé de faire céder la honte à la nécessité de
réfuter sérieusement cette prétention si moderne et si absurde du
parlement d'être le premier corps de l'État, par un écrit qui se
trouvera dans les Pièces\,; je dis la honte, parce qu'une telle
proposition ne peut en elle-même que mériter le silence et le mépris. La
pièce que je cite me dispensera de m'étendre ici autant qu'il aurait
fallu le faire sans ce renvoi, pour montrer jusqu'où se porte un orgueil
heureux, organisé, toujours subsistant et consultant, qui de degré en
degré, tous plus étonnants les uns que les autres, arrive enfin à un
comble dont le prodigieux ôte la parole et la lumière et se présente
comme probable à force d'accablement\footnote{Passage omis par les
  précédents éditeurs depuis \emph{C'est ce qui m'a obligé}.} .

Tout l'État n'est composé que de trois ordres, ainsi qu'on l'a montré au
commencement de cette longue mais nécessaire digression. Nul François
qui ne soit membre de l'un de ces trois ordres, par conséquent nul
François qui puisse être autre chose qu'ecclésiastique, noble ou du
tiers état. Chaque ordre a ses subdivisions\,; celui qui est devenu le
premier est composé du corps des pasteurs du premier et du second ordre,
des chapitres du clergé séculier, et du régulier qui se divise encore en
ordres et en communautés différentes. Il en est de même de l'ordre de la
noblesse et de celui du tiers état. Avec cette démonstration, comment se
peut-il entendre qu'une cour de justice qui, par son essence, n'est ni
du premier ni du second ordre, et qui n'est établie que pour juger les
causes des particuliers, puisse être le premier corps de l'État\,? Voilà
une exclusion dont l'évidence frappe.

On ne peut comprendre comment un corps du tiers état se met au-dessus de
ces trois ordres, si on a jamais su que la partie ne peut être plus
grande que son tout, et que le tiers état dont le parlement fait partie,
non seulement ne précède pas les deux autres, ordres, et que de cela
même il est connu sous le nom de tiers état, mais qu'il ne leur est pas
égal et leur est inférieur en quantité de choses très marquées. Ce
raisonnement seul devrait suffire, mais la chicane maîtresse des
cavillations, et féconde en refaites, veut être forcée dans ses
retranchements.

Je n'en vois ici que deux, l'un que le parlement ne soit pas du tiers
ordre\,; l'autre qu'il soit autre qu'une simple cour de justice. Ce
serait revenir sur ses pas par une ennuyeuse répétition, que s'étendre
ici sur la nature du parlement qui a été ci-dessus montrée simple cour
de justice, non compétente d'autre chose que de juger les procès entre
particuliers. On l'a fait voir par son origine, ses degrés, son aveu
même en plein parlement, par la bouche de son premier président La
Vacquerie, par l'usage constant et reconnu jusqu'aux prétentions
modernes, toujours durement réprimées par nos rois, et aux troubles et
aux désordres, protecteurs et appuis de ces mêmes prétentions tombées
d'effet avec les troubles et les désordres, quoique demeurées dans le
cœur et dans la tête des nouveaux prétendants. On renverra donc sur cet
article à ce qui en a été dit plus haut.

Celui que le parlement est du tiers état pourrait être renvoyé de même
aux preuves si claires et si certaines qui s'en trouvent dans cette
digression, si les efforts que les parlements ont essayé de faire à cet
égard en divers temps modernes n'obligeaient à quelque nouvel
éclaircissement.

Saint Louis, comme on l'a vu, est le premier qui pour
éclaircir\footnote{Le manuscrit porte ce mot au lieu d'\emph{éclairer}
  qu'on y a substitué.} les prélats et les nobles qui, dans les divers
parlements convoqués aux principales fêtes de l'année pour juger les
procès des particuliers avec les pairs, qui, de droit et sans y être
appelés, s'y trouvaient quand il leur plaisait, mit des légistes à leurs
pieds, assis sur le marchepied de leurs bancs. On a vu quels étaient ces
légistes et quelles étaient alors leurs fonctions sans voix. Il n'y
avait alors que deux corps ou ordres dans le royaume, et le peuple
partagé en serfs, en affranchis, et ces affranchis en colons de la
campagne, en bourgeois des villes, en gens de loi et de métiers, étaient
encore éloignés de faire le troisième corps ou ordre du royaume\,; ce ne
fut que sous Philippe le Bel, petit-fils de saint Louis, qui, après
force conquêtes en Flandre et en avoir pris le comte prisonnier, les
reperdit toutes à la bataille de Courtrai, en 1302, et eut besoin
d'argent qu'il chercha dans la bourse de ce peuple affranchi et enrichi,
et qui dès lors commença à pointer.

Les malheurs du règne de Philippe de Valois, qui, en vertu de la loi
salique, succéda aux trois rois fils de Philippe le Bel, morts sans
postérité masculine, et les guerres des Anglais, dont le roi, gendre de
Philippe le Bel, prétendit à la couronne, et mû par l'infidélité de
Robert d'Artois, après avoir acquiescé au jugement des pairs, rendu en
faveur de la loi salique, mirent Philippe de Valois dans la nécessité de
faire du peuple un troisième corps ou ordre du royaume pour les secours
pécuniaires qu'il y trouva\,: et ce n'est que depuis ces temps
infortunés que ce qui n'est ni ecclésiastique ni noble a été reconnu
sous le nom de tiers état, et associé aux deux autres ordres.

Ce nouvel ordre se trouva, comme les deux premiers, composé de divers
corps, et en plus grand nombre encore que les deux autres. Les corps de
justice, les légistes qui les composaient, et qui ne les composaient pas
comme les consultants et les suppôts de ces corps, tous alors
subalternes à ces parlements convoqués en divers temps de l'année pour
juger les causes des particuliers, les corps de ville, les divers corps
des marchands, des bourgeois des métiers, les colons de la campagne, et
leurs subdivisions infinies par bailliages et par provinces, composaient
ce tiers état que rien n'a changé depuis.

Les légistes, devenus par degrés et par la désertion des ecclésiastiques
et des nobles seuls juges, comme on l'a vu, et magistrats, ne composent
au parlement qu'une cour de justice pour juger, comme ces précédents
parlements généraux des divers temps de l'année, seulement les causes
des particuliers, non les causes majeures, si ce n'est par la présence
des pairs et la volonté du roi, ni les grandes sanctions de l'État,
ainsi qu'on l'a vu du premier président de La Vacquerie le dire
nettement en plein parlement au duc d'Orléans, depuis roi Louis XII, sur
sa prétention à la régence, contre M\textsuperscript{me} de Beaujeu,
qui, sans nul concours du parlement, en était et en demeura en
possession. Tel est le droit constant. Voici l'usage\,:

On a vu celui qui a toujours subsisté jusqu'à aujourd'hui que le premier
président et tous les magistrats du parlement ne parlent qu'à genoux et
découverts dans le parlement même, lorsque le roi y est présent, et que
si depuis un temps ils parlent debout, mais toujours découverts, ils
commencent tous à genoux, ne se lèvent qu'au commandement du roi, par la
bouche du chancelier, et concluent leurs discours à genoux, pour marquer
que cette bonté du roi de les faire parler debout ne déroge en rien à
l'essence du tiers état, dont ils sont, de parler à genoux en présence
du roi et découverts, à la différence des deux premiers ordres, qui
parlent assis et couverts.

On a vu aussi que le chancelier, second officier de la couronne et chef
de la justice, n'a pu, malgré cet éclat, déposer sa nature originelle de
légiste. Il est aux bas sièges, il ne parle au roi qu'à genoux\,: voilà
le légiste. Quand il parle de sa place il est assis et couvert\,: voilà
l'officier de la couronne. Il est le seul de ce caractère qui n'ait pas
du roi le traitement de cousin, et voilà le légiste\,; tandis que tous
les autres, et les maréchaux de France venus du plus bas lieu, comme on
a vu plusieurs, devenus nobles par leurs fonctions militaires, de
roturiers et du tiers état qu'ils étaient nés, ont comme leurs autres
confrères le traitement de cousin et néanmoins cèdent au chancelier, qui
a un rang fort distingué comme officier de la couronne. Il est donc
évident que rien ne peut dénaturer le légiste ni le tirer du tiers état,
puisque, si quelque chose le pouvait, ce serait sans doute le second
office de la couronne, chef suprême de la justice, et le supérieur né de
tous magistrats. On voit néanmoins en lui toute la distinction de son
office et toute sa nature de légiste parfaitement distinguées, et ce qui
lui reste de légiste ne l'être en rien du tiers état.

Enfin, et ceci tranche tout, c'est que depuis que les non
ecclésiastiques et non nobles ont fait un troisième ordre dans l'État,
connu sous le nom de tiers état dans l'assemblée des états généraux du
royaume formant et représentant toute la nation\,; jamais nul magistrat
n'y a été député que du tiers ordre. Il y a eu des premiers présidents
du parlement de Paris et nombre d'autres magistrats de ce parlement et
des autres parlements du royaume\,; il y en a eu quantité de tous les
autres tribunaux supérieurs, sans qu'il ait jamais été question qu'ils
passent être d'ailleurs que du tiers état, où constamment tous ont été
députés. La raison en est évidente, puisque n'étant ni ecclésiastiques
ni nobles, mais étant François, il faut nécessairement qu'ils soient
d'un des trois ordres qui seuls composent la nation, et que, n'étant pas
des deux premiers, il faut donc de nécessité qu'ils se trouvent du
troisième\,; et c'est ce qui s'est vu jusqu'aux derniers états généraux
qui aient été assemblés en 1614.

Mais il y a davantage, c'est qu'un noble et dont l'extraction n'est
point douteuse, mais qui se trouve revêtu d'une charge de judicature
quelle qu'elle soit au parlement ou ailleurs, est par cela même réputé
du tiers état, et ne peut être député aux états généraux qu'au tiers
état, tant cette qualité de légiste y est par nature inhérente et n'en
peut être arrachée par quelque raison que ce soit, et c'est ce qui s'est
vu en plusieurs députés des parlements aux états généraux. Après ces
preuves, comment pouvoir révoquer en doute que. Le parlement ne soit,
par sa nature et par l'usage jamais interrompu, et comme tous autres
magistrats, membre nécessaire et par essence du tiers état\,?

Il est vrai, car il ne faut aucune réticence, qu'il y a un exemple ou
deux où la justice a fait un corps à part dans les assemblées générales,
mais point jamais aux états généraux, et si peu, que ces assemblées où
elle a fait corps à part n'ont jamais été ni passées ni comptées ni
réputées être états généraux\,: secondement, c'est antérieurement et
postérieurement à ces assemblées, qui ne furent point états généraux, et
n'ont jamais passé pour tels, les officiers de justice, et ceux du
parlement de Paris et des autres parlements ont été députés du tiers
état sans aucune réclamation. C'est donc une exception singulière faite
à l'occasion de la perte de la bataille de Saint-Quentin, où il
s'agissait d'efforts extraordinaires, que la justice fut mise à part,
parce qu'elle avait fourni sa quote-part avant l'assemblée générale qui
ne fut convoquée que pour cela, et avec laquelle on n'eût pu la mêler
sans l'exposer à payer deux fois. Cette assemblée ne fut point d'états
généraux, et si\footnote{Pourtant.} encore la justice dans ce qu'elle
fut avec elle, céda sans difficulté à la noblesse\,: ainsi rien qui
fasse contre ce qui vient d'être expliqué.

Si le parlement prétendait participer et représenter même les états
généraux comme en contenant les trois ordres en abrégé, la réponse
serait facile. Il n'y a qu'à désosser cette composition, et on trouvera
qu'elle ne sera pas plus heureuse à imposer que l'équivoque du nom de
parlement. L'avantage des propositions fausses est le captieux et
l'implicite qu'elles présentent à la paresse ou à l'ignorance qui ne les
développent pas. L'artifice sait faire valoir le spécieux. Mais, si on
prend quelque soin d'approfondir, on voit bientôt le piège à découvert,
et on n'est plus qu'étonné de la hardiesse qui débite une absurdité avec
l'autorité d'une chose de notoriété publique.

On dira donc, si on veut, que les pairs ecclésiastiques et les
conseillers clercs, les pairs laïques et les conseillers d'honneur, et
les magistrats du parlement y représentent les trois ordres du royaume.
Il est vrai qu'ils sont de ces trois ordres, mais il ne s'ensuit pas ce
qu'on en prétend.

Les pairs, quelques efforts que le parlement puisse faire, ne sont point
du corps du parlement\,: autre chose est d'y avoir séance et voix
délibérative, autre chose est d'être de cette compagnie. Les pairs ont
la même voix et séance dans tous les parlements\,: dira-t-on qu'ils sont
de tous les parlements\,? le dira-t-on du chancelier qui préside à tous
quand il lui plaît\,? le dira-t-on des maîtres des requêtes qui y
entrent à ce titre\,? On a vu quel est celui qui a conservé aux seuls
pairs cette séance et voix, lorsque tous les autres nobles et
ecclésiastiques en ont été exclus. Cela a-t-il quelque trait à une
qualité particulière de membre du parlement\,? Jamais un grand fief de
la couronne ayant par nature la majesté d'apanage, et du plus grand
office de l'État et du plus ancien, ne ressembla à l'office vénal de
judicature qui s'acquiert et se vend par un légiste. Ainsi voilà les
deux premiers ordres que les pairs ne sauraient représenter dans le
parlement. On ne sera pas plus heureux à y montrer le premier ordre dans
les conseillers clercs. Les prélats des parlements assemblés en divers
temps de l'année pour juger les causes des particuliers n'en étaient
point par office, encore moins vénal, beaucoup moins comme docteurs ès
lois et légistes, puisque les légistes y étaient assis à leurs pieds
sans voix, et pour les conseiller à l'oreille quand il plaisait à ces
prélats de leur demander quelque éclaircissement. Il en était de même
des nobles, et les uns et les autres y étaient nommés et mandés par le
roi comme tels, tantôt les uns, tantôt les autres. Rien de plus
dissemblable aux conseillers clercs qui, comme légistes et non
autrement, mais aussi comme clercs pour protéger l'Église quand les
prélats se furent retirés de ces trop fréquentes et trop longues tenues,
ont eu des offices vénaux de conseillers affectés aux clercs\,: ce sont
donc des clercs, mais légistes, et qui sans être légistes ne pourraient
pas être conseillers. Ces légistes clercs ne peuvent donc représenter le
premier ordre de l'État au parlement pour leur argent, et pour leurs
examens et leurs degrés en lois.

La nobles se y est aussi peu représenté et par les conseillers
d'honneur. Jusqu'au tiers du règne de Louis XIV, ces places se donnaient
à des gens de qualité, même à des maréchaux de France. Mais ces
messieurs entraient au parlement comme autrefois les ecclésiastiques et
les nobles dans ces parlements tenus en divers temps de l'année, sans
degrés, sans examen, sans quoi que ce soit qui sentit le légiste, comme
font encore les pairs. C'était un honneur pour le parlement, et une
distinction pour ces seigneurs, qui, comme les pairs après eux, mais
personnellement et dans un seul parlement, avaient voix et séance, sans
pouvoir être dits être du parlement, puisqu'ils n'avaient point d'office
que la nomination du roi. Mais cet argument, tout faux qu'il est, est
maintenant tombé, puisqu'il y a tant d'années qu'aucun noble n'a obtenu
de ces places de conseillers d'honneur, qui sont devenues la récompense
de magistrats recommandables par leur mérite, leur ancienneté ou leur
faveur, tellement qu'elles ne sont plus remplies que par des légistes.
On voit donc l'absurdité de cette représentation des trois ordres du
royaume dans le parlement, et d'en faire membres, comme les légistes
qui, à titre de degrés aux lois et d'argent y sont pourvus d'office, les
pairs, les gouverneurs de province, les évêques diocésains, qui entrent
les premiers dans tous les parlements du royaume, et les autres dans
celui de leur province ou de leur ville épiscopale, comme le chancelier
de France qui préside à tous, enfin comme les maîtres des requêtes pour
ne rien oublier, qui tous les jours y peuvent aller juger quatre à la
fois.

À la suite de ce raisonnement qui paraît clair et sensible, on doit être
surpris de la pensée d'une simple cour de justice, qui, toute
majestueuse qu'elle soit devenue, n'est toutefois que cela, de prétendre
devenir le premier corps de l'État. Si {[}elle{]} l'était, et dans son
plus grand lustre, qui est lorsque le roi, accompagné de tout ce qu'il y
a de plus grand dans l'État, l'honore de sa présence, ce corps entier
qui ne parle que découvert et à genoux aux pieds des pairs et des
officiers de la couronne qui parlent assis et couverts, comment tous les
autres corps du royaume pourraient-ils paraître devant le roi\,? Il n'y
a plus que le prosternement et le visage contre terre qui pût être leur
posture, avec ce silence profond des Orientaux d'aujourd'hui. En vérité,
le premier corps de l'État, en même temps partie intégrante,
essentielle, membre de tous les temps jusqu'à aujourd'hui du tiers état,
sont deux extrémités par trop incompatibles.

Que le parlement se dise le premier de tous les corps qui tous ensemble
composent le tiers état, aucun de ceux des deux premiers ordres ne
prendra, je crois, le soin de le contester\,; ce sera alors à cette
compagnie à voir comment le grand conseil\footnote{Voy., sur le grand
  conseil, t. III, p.~98, note.}, qui lui dispute la préséance, trouvera
cette proposition et le conseil privé\footnote{Le conseil prive est le
  même que le conseil des parties dont il a été question, t. Ier,
  p.~446.} qui casse ses arrêts, dont les conseillers, qui sont connus
sous le nom de conseillers d'État, le disputent partout aux présidents à
mortier, et leur doyen au premier président, et dont les maîtres des
requêtes qui n'y sont jamais assis, viennent quand il leur plaît, à
titre unique de maîtres des requêtes, s'asseoir et juger à la
grand'chambre, et y précéder le doyen du parlement.

Enfin, un premier corps de l'État, n'être de nature et d'effet que des
gens du tiers état revêtus d'office de pure judicature pour leur argent
et comme légistes, pour juger uniquement les causes des particuliers, et
sans compétence par eux-mêmes pour les grandes sanctions de l'État et le
jugement des causes majeures, c'est un paradoxe que tout l'art et le
pouvoir ne saurait persuader.

Après une digression si étendue mais si nécessaire pour l'intelligence
de l'affaire qu'on va raconter et pour beau coup d'autres suites qui se
retrouveront dans le cours des années de la régence, il est temps de
revenir à ce qui y a donné lieu. On se souviendra donc ici de ce qui a
été expliqué avant la digression, de la situation suprême du duc du
Maine auprès du roi, de sa frayeur de ce qu'il pouvait perdre par sa
mort, qu'il voyait peu éloignée, de son projet de s'en mettre à couvert
par mettre aux mains d'une manière irréconciliable les ducs et le
parlement, qu'il craignait également l'un et l'autre\,; et plus
anciennement ce qu'on a vu de son caractère, de celui de la duchesse du
Maine, de leurs profonds artifices, de leur ambition, du comble aussi
effrayant que prodigieux où les menées de M. du Maine l'avaient porté,
et de tout ce qu'il avait à perdre. Voyons maintenant la trame qu'il sut
ourdir.

\hypertarget{chapitre-xx.}{%
\chapter{CHAPITRE XX.}\label{chapitre-xx.}}

1714

~

{\textsc{M. du Maine, devenu prince du sang, me dit un mot du bonnet,
que je laisse tomber.}} {\textsc{- M. du Maine, sans qu'on pût s'y
attendre, s'offre sur l'affaire du bonnet, dont il n'était pas question,
et, à force d'art et d'avances, jette les ducs dans le danger du refus
ou de l'acceptation.}} {\textsc{- Il répond du roi, du premier président
et du parlement.}} {\textsc{- On accepte, et pourquoi, mais malgré soi,
les offres du duc du Maine.}} {\textsc{- M. du Maine répond des princes
du sang et de M\textsuperscript{me} la Princesse.}} {\textsc{-
Merveilles du premier président aux ducs de Noailles et d'Aumont.}}
{\textsc{- Le roi parle le premier à d'Antin du bonnet.}} {\textsc{-
Échappatoire préparée.}} {\textsc{- M. du Maine exige un court mémoire
au roi.}} {\textsc{- Précautions extrêmes sur ce mémoire.}} {\textsc{-
M. le duc d'Orléans me donne sa parole positive, et
M\textsuperscript{me} la Duchesse aux ducs de La Rochefoucauld, Villeroy
et d'Antin, d'être en tout favorables aux ducs sur le bonnet, et la
tiennent exactement et parfaitement.}} {\textsc{- Précédentes avances
sur le bonnet à moi et à d'autres ducs froidement reçues, et de plus en
plus redoublées par le duc du Maine jusqu'à l'engagement forcé de
l'affaire.}} {\textsc{- Premier président à Marly, tout changé, y reçoit
la recommandation de M. le duc d'Orléans et le mémoire du roi, qui lui
parle favorablement.}} {\textsc{- État du premier président sur le
mémoire, contre parole et vérité, de propos délibéré. --Il fait
longtemps le malade.}} {\textsc{- Premier président visité des ducs de
Noailles et d'Antin, leur propose, en équivalent du bonnet, de suivre
les présidents entrant et sortant de séance.}} {\textsc{- Divers points
singulièrement discutés, sans que les deux ducs eussent compté de parler
de quoi que ce fût au premier président, lesquels rejettent cette suite
et tout équivalent du bonnet.}} {\textsc{- Inquiétude des présidents.
--Personnage de Maisons\,; son extraction.}} {\textsc{- Ruse de Novion
qui dévoue Maisons aux présidents.}} {\textsc{- Dîner engagé chez
d'Antin, à Paris, avec le premier président\,; convives.}} {\textsc{- Le
roi y envoie les seigneurs de son service\,; s'en passe pour la première
fois de sa vie\,; est servi par Souvré, maître de la garde-robe, et cela
se répète trois fois\,; les deux dernières sans repas, simples
conférences.}} {\textsc{- Tout sans succès. Premier président manque
malhonnêtement au dîner.}} {\textsc{- Maisons s'y trouve, sa conduite\,;
se relie plus que jamais au duc et à la duchesse du Maine, dont il était
mécontent.}}

~

Il y avait grand nombre d'années que MM. du parlement jouissaient
paisiblement de leurs usurpations et de leurs entreprises sur les pairs,
dont la faiblesse et l'incurie les laissait en pleine tranquillité, sans
que rien les eût réveillés à cet égard. Lorsque je fis mon compliment à
M. du Maine sur son nouvel être de prince du sang, comme on l'a vu en
son lieu, il me dit un mot du bonnet dans les protestations qu'il me fit
sur les ducs, et personnelles. Je pris cela pour un enthousiasme d'un
homme comblé au delà de toutes mesures, qui cherchait à rabattre
l'indignation des plus intéressés, et qui veut ramener à lui par des
offres vagues et fausses. Je glissai donc fort légèrement, et j'étouffai
une réponse vague dans l'entassement des compliments, en quoi je fus
favorisé de l'heure, qui était pendant le souper du roi, comme on l'a
vu. J'ai différé exprès à mettre ici cette circonstance pour la
rapprocher de l'affaire du bonnet. Je ne sais si, comme je le crus
alors, ce propos me fut jeté dans l'esprit que je viens de marquer, ou
si dès lors il avait conçu la noirceur profonde qu'on va expliquer,
lorsqu'il serait parvenu à se faire prince du sang, et que, suivant
cette idée, il m'en voulût jeter quelque propos dès qu'il le fut, pour
sonder comment cela prendrait. Si ce fut son projet, il ne fut
apparemment pas content de l'effet de son amorce, puisqu'il différa
longtemps après à la pousser, et que ce fut à d'autres qu'à moi qu'il la
présenta, sans m'en plus parler que dans les suites, dont aussi je ne
lui donnais pas occasion, car jamais on ne le rencontrait que dans les
cabinets du roi, rarement chez M\textsuperscript{me} la duchesse
d'Orléans où il allait à des heures rompues, et je n'allais jamais chez
lui que pour des compliments publics dont je ne pouvais me dispenser,
excepté cette affaire sur Blaye avec le maréchal de Montrevel, dont j'ai
parlé en son temps. Il faut encore se rafraîchir la mémoire du caractère
du premier président de Mesmes, et de son abandon de tout temps à M. du
Maine, qui lui avait valu une place, dont il était entièrement éloigné
sans l'intérêt que M. du Maine trouva pressant pour soi de vaincre tous
les obstacles pour l'y mettre. Enfin on doit être averti que cette
affaire du bonnet qui commença en novembre de cette année, ne rompit
qu'en mars de la suivante. Comme elle est de nature à n'en pouvoir
interrompre le récit, je l'ai mise la dernière de cette année\,; et
comme elle entre assez avant dans la suivante, je ne la commencerai
qu'après avoir achevé ce récit.

Un matin que le roi, à l'issue de son lever, donnait dans son cabinet
l'ordre pour sa journée, comme il le donnait tous les jours à ceux qui
étaient en fonction auprès de lui, en présence des courtisans qui
avaient l'entrée de son cabinet en ces heures-là, M. du Maine s'approcha
de d'Antin, et sans préliminaire lui parla de l'indécence du bonnet. Il
en dit autant deux jours après au duc d'Aumont, puis au duc d'Harcourt,
s'offrit à eux avec force compliments, et n'oublia rien pour les exciter
là-dessus. Chacun d'eux répondit vaguement et froidement. Aucun d'eux ne
se présenta pour être promoteur d'un embarquement, où le temps présent
ne permettait pas de s'engager avec prudence\,: ils furent surpris de
ces propos, mais ils les laissèrent tomber. Ce n'était pas pour cela que
M. du Maine les avait tenus. Voyant leur peu de succès, et que ses
offres de services n'étaient reçues que par des compliments généraux, il
prit à part quelques jours après, toujours au même lieu et à la même
heure, le duc de Noailles et d'Antin. Il leur dit qu'il ne comprenait
pas la froideur qu'il trouvait en ceux à qui il avait déjà parlé, sur
une affaire qui les avait si animés dans d'autres temps et avec tant de
raison\,; qu'il avait toujours été choqué d'une indécence si
extraordinaire\,; qu'il n'avait dit mot tant qu'elle lui avait servi de
distinction\,; mais qu'à présent qu'il en avait d'autres, cela lui
paraissait insupportable\,; qu'il était ami de quelques ducs, serviteur
en général de tous\,; qu'il honorait leur dignité, la première du
royaume\,; qu'il désirait leur amitié et de la mériter en les servant
sur un point aussi intéressant. Enfin il ajouta que son désir était si
sincère qu'il avait déjà pressenti le roi\,; que ses dispositions
étaient favorables\,; qu'il avait aussi parlé au premier président, qui,
dit-il, gouvernait le parlement\,; qu'il se faisait fort du premier
président, et du parlement par lui\,; et qu'il leur pouvait répondre que
le roi ne ferait aucune difficulté, dès que le parlement consentirait.
Il revint après à la froideur qu'il avait remarquée avec tant de
surprise\,; enfin il les pria de se voir quelques-uns ensemble, de se
communiquer la conversation qu'il avait avec eux, et de lui dire après
ce qu'ils désireraient de lui.

Les premiers propos avaient fort surpris ceux à qui il les avait tenus,
mais ce compliment redoublé et si marqué les étonna bien davantage. Il
leur parut trop pressant, et la chose trop suivie, pour pouvoir se
dispenser de se voir entre eux\,; et le jour même le duc d'Harcourt
boiteux, infirme et qui marchait difficilement, envoya prier
quelques-uns des principaux qui se trouvaient à Versailles de venir chez
lui un peu avant midi. Nous y trouvâmes les ducs de La Rochefoucauld, de
Villeroy, de Noailles, d'Aumont, Charost et moi. Harcourt exposa ce qui
vient d'être raconté, mais en plus grand détail, et la nécessité de
prendre un parti pour répondre à M. du Maine. M. de Noailles, en
l'absence de d'Antin qui n'avait pu venir, et qui, dès le cabinet du
roi, avait conté au duc d'Harcourt ce qui venait de se passer entre M.
du Maine, d'Antin et lui, en reprit des circonstances. Il fut après
question de raisonner Personne ne prit à l'hameçon, excepté Noailles et
Aumont, et fort légèrement encore. Tous connaissoient la duplicité de
celui qui le jetait, ennemi des rangs de l'État, de son ordre, de ses
règles, pour qui toutes avaient été violées et renversées, dont
l'intérêt était de maintenir toute confusion, qui regardait les ducs,
avec l'éloignement naturel à l'usurpateur de ce qui est le plus cher aux
hommes, et qui n'était pas tout à coup tombé amoureux d'eux. Tous
jugèrent que M. du Maine voulait engager cette affaire pour commettre
les ducs avec le parlement, se garantir, à la mort du roi qu'on voyait
diminuer, d'une union qui pouvait lui être funeste, et abaisser les ducs
de plus en plus par le mauvais succès de leur entreprise. On ne put
imaginer que cette vue dans cette proposition de M. du Maine, que rien
n'avait amenée, et qu'il poussait avec tant de suite et
d'empressement\,; et dans la vérité il n'y en pouvait avoir d'autre,
comme on l'éprouva enfin bien clairement. On convint donc aisément du
motif de ces offres si obligeantes et si pressantes, auxquelles on
devait s'attendre si peu\,; mais la conduite à tenir avec lui n'était
pas si facile à résoudre.

De ce moment nous vîmes deux précipices ouverts\,: le danger des suites,
plus que très apparentes, qu'on vient de toucher en deux mots, de donner
dans le panneau qui nous était tendu, et la cruauté d'y donner
sciemment\,; et le danger de refuser les empressements du duc du Maine.
C'était lui déclarer tacitement, mais clairement, qu'on pénétrait son
dessein, ou qu'on ne voulait lui rien devoir, parce qu'on était résolu à
l'attaquer\,; et l'un et l'autre exposait à toutes sortes
d'inconvénients et de périls en général et en particulier, dans le degré
d'empire où M. du Maine, un avec M\textsuperscript{me} de Maintenon,
était parvenu sur l'esprit du roi. On débattit l'un avec l'autre. Il
parut que le péril de donner lieu à M. du Maine de faire passer les ducs
pour ses ennemis auprès du roi était encore plus grand que l'autre,
qu'accepter ses offres n'était point un parti de choix mais de
nécessité, dans l'état où la chose se trouvait portée\,; qu'il ne
restait qu'à s'y conduire avec toute la prudence qu'on y pourrait
employer\,; que, puisqu'on ne pouvait s'en défendre, il fallait voir
sagement, puisque forcément, quel parti on en pourrait tirer. La réponse
fut donc faite dans cet esprit à M. du Maine le lendemain matin, au même
lieu où il avait fait sa proposition, et l'avait si fort serrée. Il
parut ravi et pressé de se mettre en besogne, avec les compliments les
plus flatteurs et les protestations les plus fortes. Il répondit des
princes du sang, dont l'âge et la situation, dit-il, ne leur
permettraient pas de balancer la volonté du roi. On lui objecta
M\textsuperscript{me} la Princesse et M\textsuperscript{me} la Duchesse.
Sur la première il se mit à rire, à hausser les épaules\,; et, après
quelques courts brocards sur son imbécillité et le peu de crédit qu'elle
avait dans sa famille, il en répondit, et assura qu'elle ne traverserait
pas une affaire qui devenait à lui la sienne. Sur M\textsuperscript{me}
la Duchesse, il répondit qu'il ne croyait pas qu'elle se souciât du
bonnet, moins encore qu'elle osât rien tenter contre le goût et le
vouloir du roi\,; qu'au reste on savait combien il était peu à portée
d'elle, et que c'était aux ducs à lui parler ainsi qu'à M. le duc
d'Orléans, duquel il n'osait se charger. Il exhorta ensuite d'Antin, qui
s'était approché d'eux parce qu'il était averti de ne perdre pas de
temps à en dire un mot au roi, et assura qu'il verrait incessamment le
premier président.

Ce magistrat répondit des merveilles au duc du Maine, sur la parole
duquel les ducs d'Aumont et d'Antin le virent, et qui le trouvèrent tout
sucre et tout miel. D'Antin n'eut pas la peine d'en parler au roi, le
roi lui parla le premier. Il lui dit que M. du Maine lui avait parlé de
l'affaire du bonnet\,; que, pourvu que la chose se passât de concert, il
ne demandait pas mieux que d'ôter ce scandale qu'il trouvait
insoutenable (ce fut son expression), et qu'il serait fort aise de faire
ce plaisir aux ducs. Là était la pierre d'achoppement, et dès lors j'eus
de plus en plus mauvaise opinion du succès. Je ne fus pas seul de mon
avis. M. d'Harcourt craignit, comme moi, l'échappatoire préparée dans ce
mot «\,de concert.\,» D'Antin lui-même ne savait trop qu'en penser. MM.
de Noailles et d'Aumont étaient, ou voulaient paraître convaincus de la
droiture et des bonnes intentions de M. du Maine et du premier
président. Mais l'embarquement n'avait pu s'éviter\,: il était fait\,;
il ne s'agissait plus que de voguer avec toute la prudence qui s'y
pouvait mettre.

M. du Maine, conducteur de la barque, voulut que les ducs présentassent
un court mémoire au roi, pour servir, disait-il, de base au jugement. Le
premier président le désira aussi. Il fallut donc en passer par là. J'en
craignis le piège, Harcourt le sentit aussi\,; nous en raisonnâmes sans
trouver moyen de le parer. Tout ce qu'il se put de précaution y fut
employé. D'Antin en fut chargé. Il le fit d'une page de papier à lettre,
sage, honnête, mesuré en choses et en termes pour le parlement et le
premier président. Il le montra à M. du Maine, qui le loua et
l'approuva. Il le lut au roi, qui l'assura qu'il le trouvait très bien
et {[}sans{]} quoi que ce soit à y reprendre. Il l'envoya au premier
président, avec un billet, par lequel il le priait de le corriger, s'il
y trouvait, contre son intention, quelque chose qui lui parût le
mériter, et de lui renvoyer après, pour qu'il le présentât au roi. Il
paraît donc que toutes sortes de précautions étaient prises, puisque,
après l'approbation de M. du Maine et celle du roi, il était encore
envoyé à l'examen du premier président, et soumis à sa correction. Deux
jours après, le premier président le renvoya à d'Antin, mais sans
lettre\,; et d'Antin le remit au roi, en lui rendant compte du renvoi
que lui en avait fait le premier président, qui en était apparemment
content, ajouta-t-il, puisqu'il le lui avait renvoyé sans note ni
correction\,; et le roi le prit de même ou en fit le semblant. Il loua
encore le mémoire et le procédé, et assura d'Antin qu'il remettrait le
mémoire au premier président, la première fois qu'il le verrait, et lui
recommanderait l'affaire. On verra bientôt la raison du renvoi du
mémoire à d'Antin sans correction, ni notes, ni billet, par le premier
président.

Cependant je m'étais chargé de parler à M. le duc d'Orléans sur le
bonnet, et les ducs de La Rochefoucauld et de Villeroy à
M\textsuperscript{me} la Duchesse, pour y fortifier d'Antin. Ni eux ni
moi ne trouvâmes aucune répugnance ni difficulté à vaincre. Nous eûmes
leur parole de consentir purement et simplement au bonnet, et l'un et
l'autre convinrent parfaitement que l'indécence en était insoutenable.
Tous deux aussi tinrent parole exactement et entièrement. Pour le comte
de Toulouse, il ne fut pas mention de lui dans une chose que M. du Maine
traitait ainsi de lui-même, outre qu'il n'avait pas approuvé l'élévation
que son frère leur avait procurée, et qu'il n'était pas homme à vouloir
s'opposer au bonnet.

Pour ne rien omettre, il faut dire que le duc du Maine, à l'instant
qu'il fut prince du sang, et lorsque je lui fis mon compliment le soir
même, m'avait témoigné qu'il voudrait pouvoir finir l'affaire du bonnet,
dont il me parlait pour la première fois, à son installation de prince
du sang au parlement, et que ce jour-là fût celui de la fin de cette
incroyable indécence, mais que le temps en était si court et si pressé
qu'il doutait que cela se pût exécuter en si peu de jours. Ce leurre ne
m'éblouit point, et me parut au contraire un verbiage très conforme au
naturel de celui qui me le tenait. Le jour qu'il fut au parlement comme
prince du sang, il en parla à d'Antin, et me prit après en particulier,
pendant la buvette, pour me renouveler les protestations de ses désirs
là-dessus, qu'il comptait bien montrer efficacement après le voyage de
Fontainebleau. Pendant ce voyage, le premier président y fit un tour, et
y vit M. du Maine, lequel conta aux ducs de Noailles et d'Antin que le
premier président lui avait parlé du déplaisir qu'il avait de ce que ces
deux ducs avaient rompu trop légèrement quelques conversations qu'ils
avaient eues avec lui comme ses amis particuliers, dès qu'il fut premier
président, sur le bonnet\,; qu'il l'avait même pressé d'y concourir,
puisque, devenu prince du sang, il avait changé d'intérêt\,; et qu'il
lui répondait de lui-même et du parlement là-dessus. Toutes ces avances
avaient été reçues avec la dernière froideur, et ne furent communiquées
à presque aucun des pairs. Ces deux-là lui dirent que la résolution
était prise depuis longtemps de demeurer en profond silence, d'éviter
les dégoûts qu'une autre conduite attirerait, dans l'impuissance ou on
se sentait d'obtenir la moindre justice\,; et d'Antin ajouta qu'il avait
assuré le roi qu'il ne l'importunerait jamais là-dessus.

Au retour de Fontainebleau, M. du Maine parla encore plus fortement au
duc de La Force à Sceaux. Il y allait souvent\,; il y apprit donc ce qui
s'était passé à Fontainebleau, la peine où M. du Maine disait être de
n'avoir pu remuer MM. de Saint-Simon, de Noailles et d'Antin. Il ajouta
qu'il comptait sur son amitié, et qu'il lui en demandait une marque\,:
c'était de rendre compte de sa conversation avec lui au plus grand
nombre de ducs qu'il pourrait, et de faire qu'ils ne perdissent pas de
gaieté de coeur une occasion si favorable, où le premier président
répondait du succès de son côté et du parlement, et lui duc du Maine du
côté du roi, auprès duquel il se chargeait de rompre utilement toutes
les glaces. Ce fut dans ce même temps qu'il parla dans le cabinet à
trois reprises aux ducs de Noailles, etc., comme je l'ai raconté, et que
nous nous assemblâmes chez M. d'Harcourt. Ainsi tout se fit à la fois,
parce que M. de La Force parla en même temps à plusieurs autres, qui
tous furent aussi d'avis d'accepter les offres de M. du Maine, que nous
venions de résoudre, comme on l'a vu, de ne pas refuser, parce que le
danger nous en parut encore plus grand que celui d'accepter.

C'était de Marly que le mémoire avait été envoyé au premier président,
et que, après son renvoi à d'Antin, il l'avait remis au roi, qui
l'avait, comme on l'a dit, déjà vu et approuvé pour le donner au premier
président. Il fut quelque temps à venir à Marly\,; et lorsqu'il y arriva
le matin, d'Antin se trouva au lit avec un gros rhume. Le premier
président descendit chez M. du Maine, avec qui il fut seul assez
longtemps\,; puis chez d'Antin, où il trouva les ducs de La
Rochefoucauld, Noailles et Aumont. Il leur parut tout différent de ce
qu'ils l'avaient vu chez lui\,; il était froncé, et avait l'air
embarrassé. Il dit qu'il n'avait encore parlé à personne, en attendant
les ordres du roi\,; mais, sans s'expliquer davantage, il lui échappa
que l'usage présent sur le bonnet était une chose ancienne dont le
parlement serait difficile à se départir. Il se montra pressé d'aller
chez le roi, et laissa ces messieurs fort étonnés d'un changement si
grand, si prompt, et si peu attendu. Je l'attendais au passage dans le
salon, avec M. le duc d'Orléans, qui, dès qu'il le vit, alla à lui, lui
dit qu'il savait l'affaire qui était sur le tapis, que non seulement il
ne s'y opposait pas, mais qu'il la trouvait juste et raisonnable, et
qu'il lui ferait plaisir d'y apporter toute facilité. Le premier
président paya ce prince de respects généraux, de l'ancienneté de
l'usage et de gravité, et dit qu'il allait recevoir les ordres du roi.
Il entra aussitôt après dans son cabinet\,; il y demeura peu, et sortit
fort allumé. Il trouva en sortant les ducs de Villeroy, Noailles,
Aumont, Charost et Harcourt ensemble, à qui il dit fort sèchement que le
roi lui avait remis un mémoire\,; qu'il lui avait permis de consulter le
parlement, et eu la bonté de l'assurer qu'il n'entendait pas rien exiger
d'eux. Passant tout de suite à la prétendue ancienneté de l'usage du
bonnet, il s'échauffa dans son discours, les quitta brusquement, et les
laissa encore plus étonnés que le matin chez d'Antin, où il ne retourna
pas. Il alla chez M. du Maine, d'où il monta en carrosse pour retourner
à Paris.

Le roi manda le lendemain matin à d'Antin par Bontems qu'il avait
balancé à donner le mémoire au premier président\,; mais que, n'y ayant
rien vu que de bien, et se souvenant qu'il l'avait prié de le donner, il
l'avait fait. D'Antin étant allé le lendemain chez le roi, il lui dit
qu'il avait dit au premier président de voir le mémoire avec qui il
jugerait à propos de sa compagnie\,; que ce que les ducs demandaient lui
paraissait raisonnable\,; que, pour ce qui le regardait, il le trouvait
bon\,; que les princes du sang y consentaient\,; que c'était à lui à
examiner ce qu'il y avait à faire là-dessus, sans en faire une dispute
ni un procès, et que cependant il était bien aise d'avoir appris que
cette affaire, où il ne voulait forcer personne, se passait de concert
et avec honnêteté entre tous. Le roi ajouta que le premier président
n'avait pas fait la moindre difficulté, avouant même que les ducs
n'avaient pas tort de se plaindre, et répondu qu'il prendrait son temps
pour en parler à sa compagnie, après quoi il viendrait lui en rendre
compte. La même chose nous revint par le duc du Maine. Cette facilité
dans le cabinet du roi parut si dissemblable à ce que le premier
président avait montré, avant d'y être entré et après en être sorti,
qu'il y en eut qui se persuadèrent qu'il avait envie de bien faire, mais
de se faire valoir, et montrer en même temps à sa compagnie qu'il
n'abandonnait pas ce qu'elle voulait croire de son intérêt, parce qu'il
s'était passé plusieurs choses qui l'avaient fort éloignée de lui. Pour
moi, qui avais toujours présent le danger que j'ai expliqué d'avance, et
devant les yeux le brouillard du mémoire exigé sans la moindre
nécessité, communiqué au premier président, et renvoyé sans réponse
d'approbation ni d'improbation, je ne pus m'endormir sur ce que je ne
voyais point, et M. d'Harcourt fut encore en cela de mon avis.

Jusqu'alors le secret entier avait été si exactement gardé, qu'il y a
lieu de s'étonner qu'il eût duré six semaines parmi tant de personnes,
sans qu'il en eût transpiré quoi que ce fût. À quatre jours de là, il
éclata par les plaintes que les magistrats faisaient à Paris, et qui
revinrent à Marly, du mémoire qui leur avait été communiqué. Le premier
président avait assemblé chez lui les présidents à mortier Novion,
Maisons, Aligre, Lamoignon et Portail, le doyen du parlement Le Nain, et
les conseillers Dreux, Le Ferron, Ferrand, laïques, Le Meusnier, Robert
et de Vienne, clercs. Ils voulurent trouver dans les premières lignes du
mémoire un souvenir malin des troubles de la minorité du roi\,; ils s'en
montrèrent extrêmement blessés, et ne trouvèrent rien de propre à les
calmer dans les expressions du premier président. Ce fut lui qui s'éleva
le premier sur le mémoire, qui excita les autres, et qui tâcha de rendre
le mécontentement contagieux dans le parlement.

D'Antin lui en écrivit sa surprise et ses plaintes par une lettre très
mesurée qu'il communiqua auparavant à quelques ducs. Il le somma sur
leur parole réciproque, donnée en présence du duc de Noailles\,: lui, de
lui envoyer le mémoire avant de le présenter au roi, ce qu'il avait
exécuté, le premier président, d'y remarquer et d'y corriger même ce
qu'il voudrait, et lui renvoyer ainsi, s'il y trouvait quelque chose qui
le méritât\,; parole qu'il n'avait pas tenue, puisqu'il le lui avait
renvoyé sans remarque ni correction, et s'en plaignait si amèrement
après. Il ajoutait que sa conduite n'était pas celle de gens qui eussent
dessein d'offenser, puisqu'il avait remis ce mémoire à leur censure
avant de s'en servir\,; et il finissait par expliquer l'endroit dont ils
se plaignaient d'une manière sans réplique, parce qu'en effet il y
fallait donner d'étranges contorsions pour y entendre ce que d'Antin
n'avait jamais pensé à y mettre. Il ne s'y agissait en effet que de
l'intérêt de la maison de Guise, et du duc de Guise qui, pour s'acquérir
le parlement pendant la Ligue, avait le premier souffert dans le serment
de pair à sa réception, l'addition de la qualité de conseiller. Or,
cette qualité y était supprimée depuis longtemps, et le souvenir du
temps de la Ligue avait des endroits qui faisaient honneur au parlement.
Cependant la pierre était jetée, elle fit tout son effet.

Presque en même temps, le premier président tomba malade ou le fit. Il
craignait un abcès dans la tête, qui est un mal qui ne se voit point. Un
voyage à la campagne lui parut nécessaire à sa santé\,; il en revint
avec la goutte, et fit durer tout cela deux mois. La raison ou le
prétexte était bon pour éloigner la réponse à rendre au roi, attiser le
feu, et bien prendre toutes ses mesures. On le soupçonna ainsi\,; et ce
soupçon lui attira une visite des ducs de Noailles et d'Antin ensemble,
qui lui dirent, en entrant, qu'ils ne venaient point pour lui parler
d'affaires, mais pour savoir des nouvelles de sa santé\,; mais lui leur
en voulut parler. Il entra d'abord dans une explication légère sur le
bruit que le mémoire excitait. Il ne fit qu'effleurer, par l'extrême
embarras d'avoir à répondre au silence qu'il avait gardé sur ce mémoire,
qu'il avait eu à examiner et à corriger à son gré avant qu'on en fît
usage, et qu'il avait renvoyé sans rien témoigner. Les autres ne
voulurent pas aigrir les choses plus qu'elles l'étaient\,; ainsi
personne ne chercha qu'à sauter pardessus.

De là, le premier président leur fit une proposition, qui les surprit
extrêmement. Rogue ou accort, selon le personnage qu'il avait à faire,
il exposa le plus amiablement du monde aux deux ducs qu'il n'était ni le
seul président, quoique le premier, ni le maître de sa compagnie,
quoiqu'il en fût le chef\,; que les autres présidents, communs avec lui
dans le même intérêt, ne le considéraient pas avec les mêmes yeux que
lui\,; qu'il trouvait en eux une opposition fort vive\,; que la
compagnie y prenait beaucoup de part\,; qu'il n'avait pas oublié que le
désir de l'union avait fait naître la pensée de finir les contestations
qui l'altéraient\,; que ce serait la remplir, et lever en même temps
tous les obstacles, si les ducs voulaient se relâcher de quelque chose
en faveur des prétentions des magistrats du parlement. À une proposition
si singulière de gens qui peu à peu avaient, comme on l'a vu ci-dessus,
tout emblé aux ducs, de force ou d'artifice, la réponse fut que ce qu'on
demandait était juste, ou ne l'était pas\,; qu'il s'agissait de
supprimer une incivilité très indécente, et une nouveauté sans fondement
aucun, telle que la séance d'un conseiller au bout de chaque banc des
pairs l'était avouée par eux-mêmes\,; qu'il n'était donc question, quant
à ce point, que de le remettre dans l'ordre ancien de tout temps\,; et
qu'à l'égard du bonnet, s'ils ne le voulaient pas donner, d'ôter au
moins une manière d'insulte, qui, tant qu'elle subsisterait, ne pouvait
cesser d'être une pierre de scandale\,; que ni l'un ni l'autre par sa
nature ne demandait de compensation\,; que, de plus, il ne restait rien
aux pairs dont ils se pussent dépouiller, après l'avoir été en tant de
manières.

Le premier président, toujours doux et honnête, n'oublia rien de poli ni
de respectueux, mais insistant toujours sur un équivalent dans un
esprit, à ce qu'il protesta souvent, d'accord et de paix, il leur fit
deux propositions\,: pour la première, il leur dit qu'il n'était pas
convenable à des personnes qui, comme eux, se plaignaient de l'indécence
et de la nouveauté de certains usages, d'en soutenir eux-mêmes de
pareils\,; que tel était celui des pairs de rester en séance quand la
cour levait celle des bas sièges, ce qui était indécent pour tout le
parlement. L'autre proposition fut de suivre les présidents tant en
entrant qu'en sortant de séance. Il ajouta qu'avec cela tout serait
bientôt agréablement fini. MM. de Noailles et d'Antin avaient une
réponse péremptoire à la première proposition, s'ils avaient bien voulu
se souvenir de l'usage qu'ils avaient vu tant de fois. Ils n'avaient
qu'à répondre que cette nouveauté cesserait aussitôt que la petite
porte, par où l'avocat qui a le barreau de la cheminée entre deux pas
dans le parquet pour conclure, ne serait plus fermée, pour forcer les
pairs à demeurer séants comme ils faisaient depuis cette nouveauté,
puisque, avant qu'elle fût pratiquée, la séance se levait en bas comme
en haut, les pairs et les magistrats se levant en même temps, le premier
des pairs marchant le long du banc et tous les autres à sa suite vers
cette petite porte, en même temps que le premier président, suivi des
magistrats, marchait vers l'ouverture qui est entre la chaire de
l'interprète et celle du greffier. Mais ces deux ducs, sans alléguer
cette raison, à laquelle le premier président n'avait point de réponse,
se contentèrent d'avouer la nouveauté et l'indécence de demeurer en
place quand la cour levait\,; et se contentèrent de donner un change, en
mettant sur le tapis d'ôter l'indécence du refus réciproque du salut
entre les pairs et les présidents lorsqu'ils entrent en séance,
condamnée par l'usage des princes du sang qui se lèvent également, et
entièrement, pour chaque pair et pour chaque président qui arrive à la
séance. Le premier président se tira de l'embarras de substituer
l'honnêteté réciproque à la malhonnêteté réciproque, par dire que cela
ne regardait que les présidents, au lieu de demeurer en séance quand la
cour levait était une indécence pour tout le parlement.

MM\hspace{0pt}. de Noailles et d'Antin n'étaient point allés chez le
premier président pour rien discuter avec lui. Ils n'avaient ni mission
ni encore moins pouvoir de rien\,; et ce n'était pas aussi le dessein du
premier président de convenir de quelque chose, mais d'entasser des
difficultés auxquelles on n'avait pas lieu de s'attendre après ce qui
s'était passé avec M. du Maine, et de lui-même à ces deux ducs. Ce point
de levée de séance en demeura donc là, pour venir au second qui était le
grand point d'ambition des présidents, pour en tirer après toutes les
suites et les conséquences que leur orgueil et leur art leur aurait
suggérées. Aussi ces deux ducs, qui ne l'ignoraient pas, par ce qui en
avait été jeté en d'autres occasions, ne mollirent pas sur cet article.
Le premier président allégua l'exemple du grand Condé, dont j'ai parlé
en son lieu. À cela les deux ducs répondirent que, inséparables des
princes du sang, ils les suivraient en quelque rang qu'ils voulussent
bien s'abaisser\,; qu'ainsi c'était non à eux, mais à ces princes, qu'il
devait s'adresser là dessus. Le premier président, se sentant si
adroitement rétorquer la force qu'il comptait tirer de son argument,
répondit, un peu ému, qu'il ne croyait pas que ces princes se
souciassent d'en faire difficulté, à moins que les pairs ne la leur
insinuassent\,; mais qu'indépendamment de cela, l'exemple de M. Le
Tellier, archevêque-duc de Reims, et de M. de Gordes, évêque-duc de
Langres, leur témaignait que cette suite des présidents n'était pas
nouvelle. MM. de Noailles et d'Antin rappelèrent au premier président ce
qui se trouve ici plus haut sur cette bévue de ces deux prélats\,; et
lui déclarèrent nettement que jamais les pairs ne renouvelleraient un
abus, unique en ces prélats, si court encore et fini sans plaintes,
après avoir eu sa source dans l'usage aboli aussitôt qu'introduit par
les princes du sang.

Ce fut par où finit cette longue visite. Elle se termina par les
civilités et les protestations qui l'avaient commencée. Le premier
président leur dit qu'il verrait incessamment MM. du parlement sur cette
affaire, et le roi ensuite dès que sa santé le lui permettrait, qu'il
trouvait se rétablissant tous les jours. En effet, il ne tarda guère
après à sortir et à rendre à la marquise de Créqui, à
M\textsuperscript{me} de Beringhen et à M\textsuperscript{me} de Vassé
ses assiduités accoutumées. Les deux premières étaient sœurs du duc
d'Aumont, et la dernière, fille de M\textsuperscript{me} de Beringhen et
logeant avec elle.

Les présidents étaient cependant fort en peine, parce qu'ils n'étaient
pas dans la confidence du duc du Maine, ni dans celle du premier
président. J'ai assez parlé ailleurs de Novion et de Maisons pour les
faire connaître. Ce dernier avait profité des dégoûts que le premier
président et le parlement se donnaient sans cesse. Quoique Novion fût de
même nom que les Gesvres, et que le premier président n'oubliât rien
pour faire l'homme de qualité, Maisons les effaçait là-dessus l'un et
l'autre. Ces Longueil sortaient récemment d'un huissier fieffé du
village de Longueil, en Normandie, où tout est plein de titres qui en
font foi. Le surintendant des finances, qui était aussi président à
mortier et grand-père de celui-ci, s'enta, par l'autorité de sa place,
sur la maison des anciens seigneurs de Longueil, de la terre desquels ce
village est le chef-lieu, qui était éteinte, qui avait eu des
gouverneurs de Normandie, et qui était très bonne et très ancienne. Elle
s'appelait Longueil, du nom de son fief, qui était une belle terre et
qui a été depuis dans la maison de Longueville, comme l'aïeul du
surintendant s'appelait aussi Longueil, mais du nom du village dont il
était. La faveur et la place du surintendant avait établi cette
fausseté, et le parlement s'applaudissait d'avoir, de père en fils, un
président de l'ancienne chevalerie. Il avait su en profiter\,; et, en
gagnant comme on l'a vu la cour et la ville, il avait conservé le bon
sens de ménager le parlement de plus en plus, dont les membres lui
savaient un gré infini du bon accueil qu'ils en recevaient, et de
trouver comme l'un d'eux avec eux un seigneur de cette naissance, et qui
vivait avec ce qu'il y avait de plus distingué à la ville et à la cour.
Le crédit qu'il s'était acquis dans le parlement lui faisait effacer
tous les autres présidents, et le premier président même, qui, en ayant
emporté la première place à la pointe du crédit du duc du Maine, se
trouva trop heureux de faire sa cour à Maisons, qui passait même pour le
gouverner, et pour ne s'en donner la peine que lorsqu'il lui convenait
de la prendre.

Novion craignit tout de lui\,; il n'ignorait pas son ambition, à
laquelle la cour le pouvait servir plus utilement que des gens de robe.
Il n'espéra donc rien de lui sur le bonnet qu'autant qu'il
l'intéresserait puissamment, et il eut assez d'esprit pour le faire d'un
seul coup, par les deux passions qui ont le plus de pouvoir sur la
plupart des hommes. Il l'alla trouver chez lui, où, accommodant son air
et son ton à ce qu'il voulait faire, il lui dit qu'il venait implorer sa
protection pour le parlement. La surprise d'un compliment si étrange ne
fit que mieux sentir ce que Novion lui voulait dire, d'autant plus qu'il
ne tarda pas à s'expliquer. Maisons trembla de perdre en un moment tout
ce qu'il avait pris tant de soin de s'acquérir dans sa compagnie. Il
voulait en être le dictateur, et considérait cette situation comme la
base de toute la fortune à laquelle il tendait par les amis qu'il
s'était faits à la cour, et dont sans cette maîtresse roue, l'amitié lui
deviendrait inutile. La légèreté de la cour ne lui parut pas comparable
en choix avec la solidité d'une compagnie toujours subsistante, que les
derniers exemples relevaient, avec l'espérance de ceux qui pouvaient
être prochains. Il connut assez le monde pour compter sur son adresse
auprès de ses amis de la cour, au moins sur la facilité de la
réconciliation après l'affaire finie, au lieu qu'en ne prenant pas parti
tout de bon il se perdait sans retour avec ses confrères, et par eux
avec le parlement, auquel ils persuadèrent qu'ils soutenaient moins leur
propre distinction que celle du parlement en leurs personnes. Ce fut
l'époque du changement de Maisons. Jusque-là il s'était extrêmement
mesuré. Il s'était contenté d'ambiguïtés, et de laisser voir une sorte
de suspension, pressant toutefois les ducs de ses amis, moi, entre
autres, de ne pas empêcher les princes du sang de les suivre, ce qui,
consenti par ces princes, levait toute difficulté à l'égard des ducs, et
tout obstacle du côté du parlement pour changer ce qu'ils désiraient.
Tel était le langage de Maisons.

Le récit que les ducs de Noailles et d'Antin firent aux autres ducs de
leur visite au premier président commença à les détromper de ses bonnes
intentions\,; car pour sa droiture, il y avait maintes années que
personne en France n'en était plus la dupe, ou plutôt on ne l'avait
jamais été. Ses amis avaient fort assuré les ducs qu'il ne faisait le
difficile que pour s'acquérir plus de confiance dans sa compagnie, et se
mettre en état de la ramener. Ses délais, ses difficultés entassées
répondaient peu à ses paroles si précises, si expresses, si nettes,
données par lui aux mêmes ducs, et à eux et à plusieurs autres par le
duc du Maine. On y avait donc compté, et nullement sur des équivalents
dont il n'avait jamais été la moindre question, et sur la plus légère
mention desquels on ne se serait jamais embarqué, parce qu'on l'aurait
pu éviter sur un si bon prétexte, sans montrer à M. du Maine un
dangereux refus personnel. Il ne s'agissait que du bonnet, et, par ce
qui s'était de là engrené, du conseiller sur le bout du banc des pairs
dont le premier président et M. du Maine avaient même parlé les premiers
comme d'une nouveauté également ridicule, inutile et insoutenable\,; les
autres usurpations dont ils avaient gardé le silence n'avaient pas été
mises sur le tapis par les ducs, trop accoutumés à perdre pour
entreprendre de regagner tant de larcins à la fois.

Cependant le voyage de Marly s'avançait. Le premier président était dans
les rues, et ne parlait point d'y aller. M. du Maine trouvait cette
conduite un peu étrange, en l'excusant cependant, et répondait toujours
de lui. On y voulut voir encore plus clair, et pour serrer la mesure, on
engagea un dîner à Paris, chez d'Antin, sous prétexte d'exposer sa belle
maison et ses magnifiques meubles à la censure et au bon goût en ce
genre du premier président, mais en effet pour avancer l'affaire. Il
promit de s'y rendre avec le président de Maisons et les duchesses
d'Elbœuf et de Lesdiguières, sœurs de beaucoup d'esprit, ses amies
intimes, dont la mère était Mesmes, héritière d'Avaux si connu par
l'éclat, le nombre et le succès de ses ambassades, frère aîné du
grand-père du premier président. Elles ne tenaient rien de la crasse
maternelle, pas même leur propre mère qui en était\,; elles étaient de
plus amies intimes aussi, et cousines germaines de d'Antin, enfants du
frère et de la sœur. II fut convenu que les ducs de La Rochefoucauld, La
Force, Guiche, Villeroy, Noailles et Aumont en seraient. Ce dernier
était en année de premier gentilhomme de la chambre\,; et, par un hasard
presque unique, ni M. de Bouillon, grand chambellan, ni pas un des
autres premiers gentilshommes de la chambre n'étaient à Marly, ni à
portée d'y venir par absence ou maladie\,: cela fit un cas qui n'était
jamais arrivé, et qui devint l'étonnement de toute la cour. Le roi,
infiniment attaché à tout l'extérieur possible, n'avait jamais vu les
fonctions de ses grands officiers auprès de sa personne tomber à de
moindres qu'eux\,; et ces cinq titulaires, avec leurs survivanciers,
s'étaient tellement entendus pour l'assiduité du service qu'il n'y avait
point de mémoire qu'il eût été suppléé plus de deux ou trois fois, et
encore par M. de La Rochefoucauld, grand maître de la garde-robe. Malgré
ce grand attachement du roi à la dignité de son service, il ordonna au
duc d'Aumont et au duc de La Rochefoucauld d'aller dîner à Paris chez
d'Antin, quoi qu'ils pussent lui représenter l'un et l'autre, et dit
qu'il le voulait ainsi, et que Souvré, maître de la garde-robe en année,
le servirait. J'écris les faits avec exactitude, je supprime les
réflexions. Souvré était allé avec congé passer quelques jours à Paris,
où le roi l'envoya chercher\,; et, pour n'y pas revenir, il y eut après
deux autres conférences à Paris, où le roi voulut encore que les mêmes
assistassent, et fut encore, ces deux divers jours qui font trois en
huit ou dix jours, servi uniquement par Souvré.

Les conviés, tous en liaison particulière avec le premier président, qui
avait toute sa vie fait son capital d'être du plus grand et du meilleur
monde, avaient été choisis par rapport à lui. Ils arrivèrent chez
d'Antin\,; ils y attendirent assez longtemps\,; enfin, Maisons vint,
chargé des excuses du premier président, qui s'était, dit-il, trouvé un
peu incommodé, et qui ne laissa pas le jour même de souper chez la
marquise de Créqui avec M\textsuperscript{me} de Vassé. Ce procédé
préparait mal la matière\,; on y entra pourtant avant et après dîner.
Tout roula sur l'origine ancienne ou nouvelle du bonnet, sur sa plus
qu'indécence, sur l'équivalent de la suite des présidents. Maisons, avec
tout son esprit, son monde, ses adresses, fut souvent réduit à
l'embarras, même au silence\,; mais l'opiniâtreté ne se démentit point,
et cette partie se sépara d'une manière fort infructueuse. Maisons en
eut honte\,; il pria d'Antin à l'oreille de passer chez lui sur le soir,
où tête à tête ils seraient plus libres. Je n'ai point pénétré le projet
de ce convi\footnote{Vieux mot synonyme d'\emph{invitation}.}\,; mais
d'Antin y fut, et rien n'avança entre eux deux plus qu'avec toute la
compagnie. Maisons de ce moment prit ouvertement couleur. Il n'avait pu
digérer que, après avoir fait toute sa vie une cour plus secrète que
publique au duc du Maine et avoir eu lieu de s'en promettre tout, il eût
fait Mesmes premier président, et Voysin chancelier, gens d'âge et de
santé à le laisser pourrir sur le grand banc. Il n'avait vu, depuis ces
extrêmes dégoûts, M. du Maine que le moins qu'il avait pu, et ce qu'il
n'avait seulement osé omettre pour ne pas s'en faire un ennemi. Tout à
coup il retourna à Sceaux, où le duc du Maine allait de deux jours l'un,
et d'où M\textsuperscript{me} du Maine ne sortait point. Il y redoubla
ses visites plusieurs fois la semaine, y fut souvent seul avec
M\textsuperscript{me} du Maine, et en tiers avec elle et son mari\,; et
à Versailles allait souvent chez lui et longtemps dans son cabinet tête
à tête. Toute rancune fut déposée, et pour les ducs avec qui il était en
liaison, et il ne feignit plus de se montrer absolument contraire avec
les paroles les plus douces et les plus dorées.

\hypertarget{chapitre-xxi.}{%
\chapter{CHAPITRE XXI.}\label{chapitre-xxi.}}

1714

~

{\textsc{Duc d'Aumont essaye de me tonneler sur la suite des
présidents.}} {\textsc{- Délais sans fin du premier président.}}
{\textsc{- Il est mandé à Marly, et pressé par le roi très favorablement
pour les ducs\,; sort furieux.}} {\textsc{- Impudence de ses plaintes et
des propos qu'il faisait semer.}} {\textsc{- Cause de son dépit.}}
{\textsc{- Maisons mène d'Aligre au duc et à la duchesse du Maine
demander grâce pour le parlement.}} {\textsc{- Efforts de Maisons à me
persuader, et à quelques autres, la suite des présidents.}} {\textsc{-
Le roi cru de moitié avec le duc du Maine.}} {\textsc{- Raisons de ne le
pas croire.}} {\textsc{- Opinion du roi du duc du Maine.}} {\textsc{-
Profondeurs du duc du Maine.}} {\textsc{- Embarras du premier
président.}} {\textsc{- Manèges qui font durer l'affaire.}} {\textsc{-
Noires impostures du premier président au roi contre les ducs, à qui le
roi les fait rendre aussitôt.}} {\textsc{- Éclat sans mesure contre le
premier président.}} {\textsc{- Premier président se plaint au roi du
duc de Tresmes dont il a peu de contentement.}} {\textsc{- Affront fait
au premier président de Novion, par le duc d'Aumont, dans la chambre du
roi, tout près de lui, dont il ne fut rien.}} {\textsc{- Double embarras
du duc du Maine avec le premier président, avec les ducs, engage les
ducs, et toujours malgré eux, à une conférence à Sceaux avec la duchesse
du Maine seule.}} {\textsc{- Personnage étrange du duc d'Aumont.}}
{\textsc{- Conférence à Sceaux entre la duchesse du Maine et les ducs de
La Force et d'Aumont.}} {\textsc{- Propositions énormes de la duchesse
du Maine.}} {\textsc{- Monstrueuses paroles de la duchesse du Maine, qui
terminent la conférence.}} {\textsc{- Exactitude du récit de la
conférence de Sceaux.}} {\textsc{- Le duc du Maine introduit
M\textsuperscript{me} la Princesse, dont il avait nommément répondu, et
finit l'affaire du bonnet, en le laissant comme il était.}} {\textsc{-
Évidence du jeu du duc du Maine.}} {\textsc{- Je visite le duc du Maine
et lui tiens les plus durs propos.}} {\textsc{- Réflexion sur le péril
de former des monstres de grandeur.}} {\textsc{- Réflexion sur le
bonnet.}} {\textsc{- Présidents ne représentent point le roi au
parlement. Les pairs y ont sur eux la droite, etc., tant aux hauts
sièges qu'aux bas sièges.}} {\textsc{- Comparaison du chancelier, qui se
découvre au conseil pour prendre l'avis des ducs, et du premier
président.}} {\textsc{- Étrange pension donnée au premier président.}}

~

Deux jours après, le duc d'Aumont m'envoya dire qu'il serait bien aise
de m'entretenir le lendemain matin chez le roi. Je soupçonnais déjà ce
que je ne pouvais me persuader, mais toutefois je ne voulus pas refuser
ce rendez-vous\,: je n'en fus pas dans la peine. Le lendemain matin,
comme je voulais aller chez le roi, je vis le duc d'Aumont entrer dans
ma chambre\,; j'étais sorti lorsqu'il avait envoyé chez moi, il n'eut
donc point de réponse, et il ne voulait point manquer une conversation
où il se promettait tout de son esprit et de son éloquence. Il avait en
effet beaucoup de l'un et de l'autre, mais il n'avait rien de plus. Il
entra d'abord en matière, exposa les difficultés qu'il voyait se
multiplier dans une affaire qui n'avait été entreprise que sur les
facilités qui s'y étaient d'abord présentées, livra le premier président
comme un homme sans parole, sans foi, à qui tout serait bon pour se
conserver son bonnet, remontra fortement l'aversion du roi à prononcer
dès qu'il s'agirait de le faire en juge, exagéra le dégoût d'être
éconduit d'une entreprise telle et si mûrement délibérée, conclut que,
tout valant mieux que d'y échouer, il fallait suivre les présidents.

J'écoutai tout en grand silence et beaucoup d'attention. Je lui
représentai que ce serait une belle récompense d'une civilité qui ne se
refuse pas à un honnête domestique d'autrui, lorsqu'on lui parle, de
l'artifice d'avoir changé l'ordre des réceptions des pairs, de la
violence de leur avoir fermé la petite porte du barreau de la lanterne
par où ils sortaient, en même temps que les présidents et les autres
magistrats par entre la chaire de l'interprète et celle du greffier\,;
que nous souffrions dans le bonnet une entreprise soutenue de l'intérêt
des princes du sang d'abord, fortifié depuis de celui des bâtards que
nous ne pouvions empêcher, mais en nous récriant toujours contre\,; au
lieu que d'accorder de suivre les présidents, ce serait la dernière
ignominie, se faire de simples conseillers, et mettre au-dessus de ce
que la plus haute noblesse peut espérer de plus grand, des gens du tiers
état, que nous voyons assis et couverts de nos hauts sièges, parler à
genoux et découverts dans les bas sièges, c'est-à-dire sur notre
marchepied comme légistes, dont ces bas sièges, devenus tels de
marchepieds qu'ils étaient, sont encore le monument, et leur séance
comme leur posture est le monument de leur état essentiel de légistes et
de tiers état\,; que pour comprendre l'usage que les présidents feraient
de ce consentement et de l'introduction de marcher à leur suite pour
entrer et sortir de séance, on n'avait qu'à se souvenir de celui qu'ils
avaient fait de leur usurpation d'opiner devant nous aux lits de
justice, malgré l'infinie disproportion d'y seoir et d'y parler, qui les
avait conduits de degré en degré à opiner avant les fils de France,
enfin devant la reine mère et régente\,; qu'il ne fallait point se
flatter que la position des princes du sang entre eux et nous, quand il
serait possible qu'ils les voulussent suivre, nous préservât de leurs
entreprises fondées sur ce nouvel usage que nous aurions accordé, parce
que l'état des princes du sang était invulnérable, et leur rang
aujourd'hui plus que jamais, duquel nous ne serions pas reçus à faire
bouclier, et qu'au lieu de l'union que nous devions nous proposer de la
levée des excès offensants, ce serait par nous-mêmes, et par notre
propre fait, ouvrir une large porte à toutes les plus folles
prétentions, et à la défensive de notre part la plus honteuse, quand,
contre toute apparence, après tant d'énormes exemples, ils ne
réussiraient à rien. Je supprime ici beaucoup d'autres raisons qui
seraient plus en leur place dans un morceau à part, mais qui n'existe
point parce qu'il n'y a pas eu lieu\,; et je conclus qu'il était de
notre plus pressant intérêt de rejeter un hameçon si grossier, et de
détourner les princes du sang par les plus vives remontrances de
consentir à suivre les présidents, s'il était possible qu'ils fussent
ébranlés à le faire.

Le duc d'Aumont insista sur les mêmes principes, ou plutôt motifs, qui
l'avaient amené\,; et, avec beaucoup de fleurs, se rabattit à me vouloir
persuader que nous n'avions rien à craindre, ayant les princes du sang
entre nous et les présidents à leur suite, et me conjurer d'y porter M.
le duc d'Orléans. Je répondis froidement que je serais méchant avocat
d'une cause que je tenais aussi mauvaise, et que ce prince de plus
s'était fort moqué avec moi d'une idée si ridicule à leur égard, et si
visiblement nuisible aux pairs. Pressé à l'excès par un homme fort
abondant, et que je vis déterminé à ne point sortir de ma chambre, je
lui dis que tout ce que ma déférence lui pouvait accorder était de
contribuer à une assemblée où cette matière des princes du sang fût de
nouveau mise en délibération, mais nombreuse et non autrement, où chacun
exposerait ses raisons et où la pluralité déciderait\,; et qu'au cas
qu'il y passât de faire ce que l'on pourrait pour persuader les princes
du sang de suivre les présidents, je verrais là-dessus M. le duc
d'Orléans, non pour lui dire des raisons où je n'en trouvais aucune,
mais pour lui exposer respectueusement les désirs qu'on avait cru devoir
former. De guerre lasse ou autrement, M. d'Aumont se contenta de ce
qu'il remportait\,; mais en s'en allant, il me pria de l'attendre chez
moi le lendemain matin à pareille heure pour raisonner du fruit de nos
communes réflexions. Cette seconde conversation fut plus courte\,; nous
demeurâmes tous deux dans nos mêmes sentiments.

On se lassait cependant des délais du premier président, ils n'étaient
plus fondés sur sa compagnie, puisqu'il avait tenu plusieurs assemblées
chez lui là-dessus\,; ni sur sa santé, puisqu'il était tous les matins à
la grand'chambre, et les après-dînées dans les rues. Il était même bien
peu respectueux pour le roi de différer si longtemps, et sans prétexte,
de lui rendre compte d'une affaire qu'il lui avait recommandée, et à
laquelle il lui avait dit qu'il ne trouvait point de difficulté. À la
fin, d'Antin en paria au roi, sur ce qu'il vit que ces lenteurs ne
tendaient qu'à soulever le parlement, comme on le va voir, et commettre
les ducs avec ses membres. Il se garda bien pourtant d'alléguer cette
raison au roi\,; il y en avait assez d'autres à dire. On avait sagement
résolu de mépriser tout, de ne relever rien, de ne faire pas la plus
légère plainte, mais d'aller droit au fait, sans se détourner ni à
droite ni à gauche, et sans l'embarrasser d'épines. Le roi fit donc dire
au premier président de lui venir parler\,: il fallut obéir. Le roi lui
dit qu'il était enfin temps de donner sa réponse\,; que ce que les ducs
demandaient lui semblait juste\,; qu'il serait bien aise que cela fût\,;
qu'il n'entendait pas commander, mais qu'il lui serait agréable que
cette affaire finît incessamment à leur satisfaction. Sur plusieurs
difficultés alléguées par le premier président, le roi lui dit qu'il ne
lui avait pas paru difficile d'abord\,; qu'il était surpris de ce
changement\,; qu'il y avait assez longtemps que l'affaire traînait\,;
que de façon ou d'autre il désirait qu'il ne tardât plus à donner la
réponse qu'il s'était chargé de lui rendre. Le premier président
s'excusa sur sa santé comme il put, et sortit tout enflammé du cabinet
du roi.

C'était encore à Marly. Il y était entré doux, poli, gracieux,
accueillant tout le monde, surtout les ducs qu'il rencontra\,; mais il
n'était plus le même, son audience l'avait entièrement changé. Les ducs
qui se trouvèrent sous sa main en furent surpris. Il se plaignit à eux
avec amertume qu'ils voulaient étrangler leur affaire, qu'il était inouï
qu'on eût cette précipitation\,; il allégua sa maladie. Il lui échappa
même que d'Antin avait bien recordé le roi, brossa à travers la
compagnie, et disparut. Il ne disait pas la cause principale de son
chagrin, qui fut sue avec le reste de la conversation que je viens de
rapporter une demi-heure après de d'Antin, à qui le roi le dit aussitôt
que le premier président l'eut quitté. Un petit nombre de membres du
parlement avaient tenu force propos sur les ducs\,: «\,que le roi
faisait trop de pairs\,; qu'il fallait les traiter comme de simples
conseillers, et n'en souffrir pas plus de douze en séance à la fois.\,»
Le roi le sut de point en point, et se trouva fort choqué de la licence
de ces messieurs\,; et le froid et le silence de d'Antin, à qui il en
avait parlé, l'aigrit encore davantage. Il sentit apparemment par là la
même différence de procédés qu'il y en avait dans les personnes, et que
ces discours portaient moins sur les ducs que sur son autorité. Il en
parla fortement au premier président, lui ordonna positivement d'en
marquer son mécontentement à sa compagnie et aux impertinents, et le
chargea fort expressément d'arrêter toute sorte de discours sur cette
affaire et sur les ducs. C'était saper par les fondements le projet du
premier président, qui voulait étouffer l'affaire par les procédés et
les éclats, et s'en tenir extérieurement à côté tant qu'il pourrait\,;
de là vint le dépit et la colère qu'il ne put cacher en sortant du
cabinet du roi.

Bientôt après Maisons donna une autre scène. Initié, comme il l'était de
nouveau, avec M. et M\textsuperscript{me} du Maine sur cette affaire, et
sans cesse en particulier avec eux, il ne devait pas être tourmenté de
leur part. Ce fut donc moins son inquiétude qu'un concert de comédie
pris avec eux, qui lui fit choisir le plus imbécile, non pas de ses
confrères mais du parlement entier, pour le leur mener. Il leur présenta
donc le président d'Aligre pour leur demander grâce pour le parlement,
car ce fut ainsi qu'ils se mirent à parler d'une affaire qui était toute
particulière aux présidents. Maisons n'allait pas là pour réussir. Aussi
furent-ils payés de toutes les civilités imaginables, dont sur la parole
de Maisons, mais qui ne disait pas la véritable bonne, d'Aligre et lui
se retirèrent contents. Toutefois il fallait finir. Le roi s'en était
expliqué. Les présidents trouvaient un si monstrueux avantage à lâcher
le bonnet pour être suivis, qu'ils ne voulurent rien oublier pour y
réussir.

On a vu en son lieu les liaisons que Maisons était venu à bout de me
faire prendre avec lui, et combien il les avait cultivées. Il avait
lestement glissé sur le refroidissement, et plus encore, qu'y avaient
mis de ma part les procédés de cette affaire du bonnet. Avec autant de
monde que le duc d'Aumont, plus d'esprit, et surtout de profondeur
encore et de manège, il se mit dans la tête qu'il n'était pas impossible
de me persuader, et que, venant à y réussir, j'entraînerais tous les
autres. Ma franchise, et la vivacité qu'on m'attribuait, lui faisaient
espérer qu'il découvrirait par moi notre dernier mot sur cette affaire.
Il s'attacha aussi à d'Antin, et il attaqua tous ceux qu'il crut pouvoir
gagner, faisant croire à chacun d'eux qu'il ne parlait qu'à lui, pour
donner plus de poids à ses paroles. J'eus donc à essuyer des visites
aussi longues que fréquentes, et des péroraisons où, à travers
l'impatience, j'admirais la souplesse et la fécondité qui par cent tours
divers tendait toujours au même but. L'esprit, le tour, les
\emph{sproposito} suppléaient d'ordinaire aux raisons, et sa patience
fut inaltérable aux coups de boutoir que mon impatience porta souvent
sur les présidents et leurs usurpations. L'utilité de l'union pour le
bien de l'État, dans les circonstances que l'âge du roi laissait
envisager de près, fut par lui tournée de toutes les manières, parce
qu'il me faisait l'honneur de me croire fort susceptible d'une si grande
raison\,; et il ne se rebuta point de la réponse, si présente et si
péremptoire, que c'était à eux à la mettre entre nous par la restitution
d'une usurpation de si nouvelle date, et de si injurieuse nature, non à
nous à l'acheter par un avilissement volontaire et inconcevable. Cette
persécution dura jusqu'à la bombe qui fit tout sauter, et qui en
attendant se chargeait.

Les plus profondes noirceurs laissent bien des embarras, quoique tissues
par tout l'art, l'esprit et l'expérience, et appuyées du plus puissant
crédit. L'affaire ne pouvait plus durer, j'en abrège mille choses qui ne
donneraient pas plus de connaissance que celle qui se peut tirer de ce
récit, de l'esprit qui enfanta ce projet, qui en ourdit la trame, et qui
la conduisit jusqu'au bout, et de celui dans lequel les ducs s'y
conduisirent, après avoir été forcés comme on l'a vu. Le respect dû à la
mémoire d'un grand roi dont je suis né sujet ne me permet pas de le
soupçonner d'avoir été de moitié là-dessus avec son bâtard favori.
Indépendamment de cette grande raison, c'est ici le lieu d'expliquer ce
qu'on sait par lui-même de ce qu'il pensait de M. du Maine, et l'équité
m'y engage aussi.

Il est souvent échappé au roi de dire dans son intérieur, et je l'ai su
de plusieurs de ceux qui en ont été témoins en diverses occasions, entre
autres de Maréchal, premier chirurgien du roi, et qui était l'honneur et
la vérité même, et à qui personne ne l'a disputé, que le roi disait que
M. du Maine avait à la vérité beaucoup d'esprit et de talents, mais
qu'il n'en savait rien faire\,; que toutes ses journées se passaient
entre son assiduité auprès de lui à ses heures, la chasse où il était
tout seul, et son cabinet de Versailles ou de Sceaux où il était aussi
tout seul, et où son temps était partagé entre la prière, la lecture et
les fonctions de ses charges où il travaillait beaucoup\,; que c'était
un idiot avec tout son esprit, qui ne savait jamais quoi que ce soit qui
se passât hors la sphère de ses charges, qui ne se souciait point de le
savoir, qui n'avait pas la moindre vue, et roulait du jour au jour, et
qui, étant fort plaisant, amusant et de bonne compagnie, était sauvage
au point de ne vouloir voir personne, et d'apprendre quelquefois les
choses qui occupaient la cour et qui étaient arrivées un mois et souvent
deux et trois auparavant, qui ne pensait jamais à soi, et qui était de
son propre aveu incapable de gouverner sa propre maison. Le roi s'en
était expliqué ainsi plusieurs fois avant la mort de M. {[}le Dauphin{]}
et de M\textsuperscript{me} la Dauphine\,; et il n'est pas impossible,
avec la ténacité prodigieuse qu'il avait dans les impressions qu'il
avait une fois prises, que les violences, que nous avons vu qu'il
souffrit depuis pour porter ses bâtards jusqu'à la couronne et les
affermir par son testament, ne lui aient été assez adroitement masquées
du bien de l'État et du péril des établissements, de la grandeur et de
la personne même de M. du Maine, pour qu'il ne soit jamais revenu de
cette impression sur lui. Elle fut le chef-d'œuvre de son ambition et de
sa politique et de la profondeur de sa connaissance du roi qui le
conduisit à tout. Ce fut aussi celui de l'art de M\textsuperscript{me}
de Maintenon qui lui aida de tous ses soins, et qui tenait souvent de
lui le même langage. Or, le roi disposé de la sorte, comme il est très
certain qu'il le fut toujours avant la mort de M. {[}le Dauphin{]} et de
M\textsuperscript{me} la Dauphine, et très douteux qu'il eût changé
depuis d'opinion, quelques raisons qu'il en ait pu avoir, sa conduite se
trouve éclaircie.

M. du Maine, qui veut ouvrir un précipice sous les ducs, qui les rende
pour son intérêt irréconciliables avec le parlement, a beau jeu
d'engager le roi avec un air de modestie et de contentement du nouvel
état de prince du sang où il l'a élevé et les siens, de le rendre
favorable sur le bonnet où il n'a plus d'intérêt que commun avec les
princes du sang, avec qui il partage tant d'autres distinctions.
L'intérêt des bâtards rendait le roi contraire au bonnet\,; et il y
devient favorable, lorsqu'il voit leur intérêt à regagner tant de gens
considérables, par l'abrogation d'une nouveauté sans fondement et très
injurieuse. M. du Maine, sûr du premier président, ne risquait rien à
mettre le roi ainsi dans cette affaire\,; il connaissoit bien sa
répugnance extrême pour toute décision. Il s'en met à l'abri en flattant
cette répugnance. Non seulement il lui donne le bonnet comme une affaire
de concert, mais il va au-devant de tout, jusqu'à faire que, dès la
première fois que le roi en parle au premier président, c'est en
l'assurant expressément qu'il n'entend rien commander, et qu'il lui
renouvelle d'autres fois encore la même assurance. Par là M. du Maine
s'assure que, quoi qu'il puisse arriver, le roi ne décidera rien, et
laissera les ducs dans la nasse, à qui, s'ils le pressaient, il aura sa
réponse toute prête\,: qu'il n'est entré dans cette affaire que parce
qu'elle lui a été présentée de concert\,; qu'il a promis dès le premier
jour au premier président de ne point commander\,; qu'il lui a dit, en
faveur des ducs, qu'il trouvait ce qu'ils demandent juste et
raisonnable, et qu'il aurait fort agréable qu'ils fussent contents\,;
que c'est tout ce qu'il pouvait faire\,; qu'après l'engagement pris de
ne point commander, et de leur su, et n'y être entré qu'à cette
condition, il ne peut aller plus loin. Ainsi M. du Maine jouait sa
comédie en sûreté, et s'était habilement mis à couvert d'avoir la main
forcée\,; mais elle ne pouvait finir que par un éclat, et c'était son
embarras. Il voulait s'en mettre à l'abri, le premier président ne
voulait pas l'essuyer tout seul, et c'est ce qui fit traîner l'affaire.

Le duc du Maine voulait engager le premier président en des procédés, et
se cacher derrière lui. Ce magistrat en sentait les conséquences\,; mais
asservi à M. du Maine qui le cajolait avec douceur, et à
M\textsuperscript{me} du Maine qui le traitait avec impétuosité, il se
trouvait étrangement en presse\,; et, outre les grands avantages dont
lui et les autres présidents se flattaient de l'échange du bonnet avec
leur suite, cette voie le tirait de tout embarras, et laissait à son
tour M. du Maine dans la nasse, qui n'aurait rien fait pour soi, et
n'aurait fait que l'avantage des présidents, avec une union passagère
des ducs avec le parlement, mais qui eût suffi pour ruiner tout ce qu'il
avait acquis de grandeur et de puissance, ce qu'il craignait
mortellement. Dans ce détroit néanmoins, il n'en fit aucun semblant. Il
sentit que montrer sa crainte de cet accord montrerait trop la corde\,;
il espéra que les ducs ne se laisseraient pas prendre à un hameçon si
grossier, et il ne s'y trompa pas. M. d'Aumont eut beau faire, il
n'ébranla aucun de ceux sans le concours desquels rien ne se pouvait
faire\,; au pis aller, M. du Maine était sur ses pieds, par le roi,
d'empêcher les princes du sang de consentir à suivre les présidents,
moyennant quoi il n'était pas possible de croire les ducs assez
destitués de sens pour vouloir se séparer de ces princes et se livrer à
une si honteuse prostitution. Le premier président, qui sentait qu'il
n'y avait pourtant que cette suite qui pût le tirer du détroit où M. du
Maine l'avait engagé, et qui, léger et présomptueux comme il était, n'en
vit l'affre que lorsqu'il y toucha, allongeait toujours, dans
l'espérance que le duc d'Aumont et Maisons, à force d'art, d'éloquence,
d'intrigue et de délais, réussiraient enfin à persuader les ducs d'en
sortir par là, après quoi il s'excuserait à M. du Maine sur les
présidents qui l'auraient forcé, parmi lesquels il n'avait que sa voix,
lesquels avaient mis le parlement de leur côté, et ce qu'il n'y avait
aucun lieu de pouvoir imaginer, les ducs aussi. Il prolongea donc tant
qu'il put, et au delà de toute mesure, de rendre réponse au roi.

Outré de rage de se voir trompé enfin dans l'espérance qu'il avait
conçue, piqué à l'excès d'avoir été arrêté par le roi sur les propos
qu'il avait fait semer sur cette affaire et sur les ducs, et d'être
privé de faire faire les éclats par un gros de gens de robe inconnus
dont il serait le moteur, et se donnerait cependant pour amiable
compositeur, brouillé pour brouillé comme il prévit bien qu'il allait
l'être avec les ducs par le refus du bonnet après ce qu'il avait si
nettement et si positivement promis plusieurs fois, et forcé enfin
d'aller rendre réponse au roi, il lui dit que les ducs faisaient entre
eux des assemblées continuelles sous prétexte de cette affaire, mais en
effet dans les vues d'un avenir qu'on ne devait prévoir qu'avec horreur,
et la plupart d'eux plus qu'aucun par les grâces dont Sa Majesté les
avait comblés\,; qu'ils étaient les plus grands ennemis de ses enfants
naturels\,; qu'ils prenaient toutes leurs mesures ensemble pour les
dépouiller dès que Sa Majesté ne serait plus, et en même temps pour se
rendre les seuls maîtres des affaires. Qu'il y avait plus\,: que,
flattés par les malheurs qui en si peu de temps ont détruit une partie
de la maison royale, ils comptaient bien que ce qui en restait ne
durerait guère, de faire après comme en Pologne et comme l'exemple de la
Suède les y invitait, rendre la couronne élective, et choisir l'un
d'entre eux pour la porter. Ce furent les principaux points qui furent
avancés au roi par le premier président, et qui furent accompagnés des
réflexions les mieux ajustées à de si horribles impostures. Elles ne
laissèrent pas de frapper le roi, qui les raconta un quart d'heure après
à d'Antin comme touché, effrayé, mais en suspens et cherchant
éclaircissement. Il ne fut pas difficile. D'Antin lui parla avec tant de
netteté sur des inventions si éloignées de toutes pensées, et si
évidemment sur l'impossibilité de les concevoir et d'en espérer sans la
plus parfaite folie, que le roi, peiné d'en avoir été ému, et piqué
contre la hardiesse d'une délation si atroce et en même temps si
absurde, permit à d'Antin d'en instruire les ducs pour qu'ils sussent à
quel homme ils avaient affaire. D'Antin ne laissa pas échapper
l'occasion d'un parallèle aisé entre les ducs et le parlement sur la
fidélité, l'obéissance et l'attachement au roi\,; et, sans la précaution
que l'habile duc du Maine avait su prendre de faire engager le roi au
parlement, en la personne du premier président, de ne point commander,
le bonnet eût été emporté de ce coup de haute lutte. L'exposé est seul
dans sa simple et pure vérité plus fort que tous les commentaires. On se
contentera de dire que l'instrument était digne de celui qui s'en
servait, et n'était pas inférieur aux plus exécrables usages, et avec un
front d'airain, et avoir tout promis et aux ducs et au roi même, sans
que les ducs eussent pensé à rien et rien demandé.

D'Antin, dans le reste de la journée, rendit compte à plusieurs ducs de
ce {[}dont{]} le roi lui avait permis de les informer. On peut juger
avec quel effet. En moins de deux jours tous les ducs se donnèrent
parole de ne jamais voir le premier président, et de ne garder avec lui
aucunes sortes de mesures en choses et en paroles, d'y entraîner leurs
familles, et d'en user comme avec un ennemi public et un imposteur
perfide et déshonoré\,: ce n'est pas trop dire. L'éclat fut porté aussi
loin qu'il le put être, et se soutint très longtemps dans tout le feu
que méritait une scélératesse, et gratuite, d'une nature aussi
complètement infâme. L'imposteur fut étourdi d'un unisson auquel il ne
s'était pas attendu des ducs. M. d'Aumont, et peut-être quelques autres
qui ne l'étaient que de nom, et dont il se servait parmi eux, n'osèrent
plus le voir\,; et cet homme qui avait toujours fait son capital de la
cour et du grand monde, se trouva en un instant délaissé de ce qui par
les ducs, leurs plus proches familles et leurs amis plus particuliers,
en faisait la partie la plus considérable. Aucun ne le salua, et hors
des insultes personnelles, indécentes à faire à un homme qui, par état,
ne porte point d'épée, il n'est affronts qu'il ne reçût tous les jours.
Outré d'un état si pénible et qui n'était pas prêt à finir, et appuyé du
duc du Maine, il saisit une occasion de se plaindre au roi. Le duc
deTresmes avait fait entrer peu à peu tout le monde au lever du roi, et
l'avait laissé dans l'antichambre. Il obtint que le roi dît au duc de
Tresmes qu'il ne devait pas faire servir sa charge à sa vengeance
particulière, mais sans aigreur, et d'ailleurs fut sourd à tout ce que
le premier président lui put dire, et ne se voulut mêler de rien.

Le roi avait oublié que, lorsque après l'opération de la fistule, il
commença à voir du monde dans son lit, le duc d'Aumont, père de celui
dont il s'agit ici, était en année, et les ducs très offensés des
entreprises du premier président de Novion. Il vint à Versailles à
l'heure qu'on devait bientôt voir le roi, et pria l'huissier de dire au
duc d'Aumont qu'il était-là\,; le duc d'Aumont le laissa jusque vers la
fin du fruit du dîner du roi dans l'antichambre, ayant fait entrer tout
ce qui pouvait entrer. À la fin il le fit appeler. Il ne put se mettre
en vue du roi, qui était au lit. Il attendit que le monde sortit, et
comme il commençait à s'écouler, il s'approcha du balustre. Le duc
d'Aumont, qui l'observait, l'y laissa entrer deux pas pour qu'il ne pût
s'en dédire, et le tira après fort rudement par sa robe, et lui dit
rudement aussi\,: «\,Où allez-vous\,? sortez\,; des gens comme vous
n'entrent pas dans le balustre si le roi ne les appelle pour leur
parler.\,» Novion, déjà outré de sa longue attente dans l'antichambre,
fut si confondu qu'il n'eut pas un mot à répondre. Il se retira plein de
honte et de rage, et comme il n'avait point de bâtard derrière lui, il
n'osa s'en plaindre, et demeura avec l'affront.

M. du Maine, ravi d'avoir mis ainsi les ducs hors de toute mesure avec
le premier président, ne laissait pas d'être en peine de la conclusion.
Les impostures n'avaient pas fait l'effet sur le roi qu'ils en avaient
tous deux espéré\,; et M. du Maine se voyait avec beaucoup d'angoisses
découvert à travers le premier président. Il n'en sentait pas moins du
désespoir où il voyait ce magistrat des suites de ses impostures, parce
qu'il ne voulait pas se brouiller avec un homme qui avait son secret et
qu'il avait mis à la tête du parlement. Il voulut donc montrer que rien
ne le rebutait pour chercher des expédients de sortir honnêtement les
ducs d'une affaire où il les avait embarqués par force, sur sa parole et
sur celle du premier président\,; et, en finissant, le tirer, s'il était
possible, de l'embarras étrange où il l'avait livré. Il se mit donc à
montrer aux ducs ses désespoirs, ses désirs, toujours son espérance,
glissant légèrement de faibles excuses du premier président. On ne lui
répondait que par des révérences sérieuses et silencieuses qui lui
donnaient fort à penser. Enfin il proposa aux mêmes ducs à qui il
s'était adressé sur le bonnet une conférence à Sceaux avec
M\textsuperscript{me} la duchesse du Maine seule, qui n'avait point
encore paru à découvert dans cette affaire, dans laquelle il espérait
qu'on pourrait trouver de bons expédients. Ce qu'on va voir qu'il s'y
traita montrera dans la dernière évidence le dernier degré de sa
puissance sur l'esprit du roi, et l'excès de ses inquiétudes sur tout ce
qu'il en avait obtenu. Les ducs s'en défendirent tant qu'ils purent et
jusqu'à l'opiniâtreté\,; mais, à force de recharges et d'empressements
les plus vifs et les plus redoublés, la même raison qui les avait
embarqués avec lui malgré eux dans l'affaire du bonnet les entraîna
encore à céder, quoiqu'ils vissent assez qu'il n'y avait rien à en
attendre qu'un prétexte de faire casser la corde sur eux. Ce fut donc à
qui n'irait point.

M. d'Aumont, qui tôt après ne se cacha plus guère d'avoir été un pigeon
privé, profita du refus de chacun pour se proposer. On se regarda\,; il
n'était pas encore assez à découvert pour lui faire un affront public\,;
et c'en eût été un de le refuser\,; ainsi, tout se faisant par force
dans l'embarquement et dans toute la suite de cette affaire, ce fut
force d'y consentir\,; mais comme on était aussi bien éloigné de se fier
en lui, on proposa tout de suite qu'il en fallait mettre un autre avec
lui. Le duc d'Aumont demanda pourquoi, et se mit à pérorer pour y aller
tout seul. S'il n'avait pas été plus que suspect déjà, cette offre si
aisée d'aller, cet empressement d'y aller seul auraient dû ouvrir les
yeux. L'embarras fut du compagnon. La commission de soi n'était rien
moins qu'agréable\,; l'union de M. d'Aumont la rendait encore plus
dégoûtante. Heureusement M. de La Force, dont j'aurai lieu de parler
ailleurs, se proposa, et il fut accepté avec joie. Il avait beaucoup
d'esprit\,; il était fort instruit\,; il était fort duc et pair, et très
incapable de gauchir. Il était depuis longtemps beaucoup de la société
de M\textsuperscript{me} la duchesse du Maine, enfin il était l'ancien
du duc d'Aumont\,; il avait fort la parole en main, et entre eux deux
c'était sur lui qu'elle devait naturellement rouler. Il n'avait pas été
des derniers à voir clair sur la conduite du duc d'Aumont, et il fut de
plus bien averti de s'en défier continuellement à Sceaux, et de l'y
regarder et se conduire comme avec le croupier de M\textsuperscript{me}
du Maine. Parmi tant de choses sinistres dans cette affaire, ce fut un
bonheur que tout fût bon au duc de La Force pourvu qu'il se mêlât de
quelque chose, et que ce goût lui eût donné envie de doubler le duc
d'Aumont.

Les voilà donc tous deux à Sceaux à jour marqué, qui suivit de fort près
le consentement arraché d'y aller. M\textsuperscript{me} la duchesse du
Maine les y reçut avec des politesses et des empressements non
pareils\,; et, un moment après leur arrivée, elle les mena dans son
cabinet, où elle fut en tiers avec eux. Là M\textsuperscript{me} du
Maine, après tous les jargons de préface, leur dit nettement que,
puisque c'était M. du Maine qui les avait engagés dans cette affaire,
qu'il s'était fait fort d'y réussir, qu'ils la regardaient comme si
principale surtout depuis qu'elle avait été embarquée et qu'elle
semblait avoir mal bâté, il était raisonnable que M. du Maine mît le
tout pour le tout pour les en bien sortir\,; mais qu'aussi était-il
juste qu'il fût assuré d'eux qu'il n'obligerait pas des ingrats, et
qu'ils entrassent avec lui en des engagements sur lesquels il pût
compter. À ce début, ces messieurs se regardèrent l'un l'autre, et
parurent fort surpris d'une proposition qu'ils entendirent pour la
première fois de leur vie\,; et si elle fut moins nouvelle au duc
d'Aumont, il joua bien d'abord.

M\textsuperscript{me} du Maine, qui s'en aperçut, et qui sans doute s'y
était bien attendue, les cajola l'un après l'autre, puis les ducs en
général, leur dit qu'ils ne devaient point s'étonner de ce qu'elle leur
proposait\,; qu'il était de leur intérêt d'emporter ce qui était
entamé\,; de celui de M. du Maine de s'assurer de tant de grands
seigneurs qui n'avaient pas vu sans peine ces diverses élévations\,;
qu'il en était bien informé il y avait longtemps\,; qu'il ne laissait
pas de désirer leur amitié, et qu'ils le voyaient bien par les démarches
qu'il avait faites sur cette affaire\,; mais qu'il entendait aussi que
le succès les lui concilierait de manière à éteindre en eux leurs
anciens déplaisirs à son égard, et à former un attachement (quelle
expression\,!) dont il se pût assurer\,; que c'était sur quoi elle les
priait de lui répondre. Là-dessus force compliments, force verbiages\,;
mais elle leur déclara qu'elle ne s'en contentait point. Eux répondirent
qu'ils ne savaient rien de plus à répondre que lui dire les sentiments
qu'ils lui exposaient, puisque, ne s'agissant de rien de précis, ils
n'avaient rien à refuser ni à accepter. M\textsuperscript{me} du Maine,
voyant que tous ses propos ne les faisaient point s'avancer, et que M.
de La Force comme l'ancien prenait toujours la parole sur M. d'Aumont
sans jamais la lui laisser, prit son parti de parler la première. Elle
leur dit donc que, après toutes les grâces dont le roi venait de combler
M. du Maine, et particulièrement celle de l'habilité à succéder à la
couronne, il n'avait plus rien à en désirer, mais qu'en même temps il
n'était pas assez peu considéré pour ne pas voir que cette disposition
et d'autres qui avaient précédé celle-là pouvaient, non pas être
contestées après le roi (elle ne disait pas ce qu'elle en pensait) qui
les avait bien solidement munies de tout ce qui les pouvait bien
assurer, mais donner occasion d'aboyer (quel terme\,!), de crier,
d'exciter les princes du sang, jeunes et sans expérience, quoique si
liés à eux par les alliances si proches et si redoublées, donner envie
aux pairs de se joindre à eux contre M. du Maine, enfin de les
tracasser\,; que M. du Maine voulait éviter cet inconvénient, jouir
paisiblement de tout ce qui lui avait été accordé, et que c'était à eux
à voir s'ils se voulaient engager à lui sur ce pied-là d'une manière non
équivoque.

Le duc d'Aumont saisit la parole. Le duc de La Force la lui prit à
l'instant, en l'interrompant sur ce qu'il enfilait plus que des
compliments. Après en avoir fait quelques-uns, La Force se mit à vanter
la solidité de tout ce que M. du Maine avait obtenu, la solennité des
formes qui y avaient été gardées, conclut que c'était là une terreur
panique sur des choses que personne n'avait aucun moyen d'attaquer. La
duchesse du Maine répondit que, s'ils n'avaient point de moyens, il n'en
fallait pas conserver la volonté\,; que cela ne se prouvait point par
des propos, mais par des choses\,; que c'était à eux à voir quelles
étaient ces choses dans lesquelles ils voudraient s'engager. Le duc de
La Force, de plus en plus surpris de tout ce qu'il entendait, et qui
voyait déjà où elle en voulait venir, se défendit sur ce qu'ils
n'imaginaient rien au delà de ce qu'il venait de lui dire\,; qu'il y
ajouterait de plus toutes les protestations qu'elle estimerait l'assurer
de leurs intentions\,; qu'elle avait vu que pas un d'eux n'avait opposé
quoi que ce fût à toutes les volontés du roi à l'égard du duc du
Maine\,; et revint encore à leur solidité. M\textsuperscript{me} du
Maine, forcée enfin d'articuler, leur déclara que si c'était sincèrement
qu'ils parlaient, tant pour eux que pour les autres ducs, il ne leur
coûterait rien de leur donner une assurance par écrit de soutenir après
le roi ce qu'il avait réglé de son vivant pour ses fils naturels et leur
postérité, tant pour leurs rangs et honneurs que pour la succession à la
couronne.

M. de La Force, qui dès le commencement de cette forte conversation
avait prévu cette proposition, la supplia de considérer ce qu'elle leur
proposait\,; de faire réflexion si des sujets, quels qu'ils fussent,
pouvaient sans crime s'arroger l'autorité et le droit de confirmer les
dispositions du roi vivant et régnant, enfin de jeter les yeux sur la
juste jalousie du roi de son autorité, et sur les folles calomnies que
le premier président avait osé leur imputer à ce même égard d'autorité,
et au roi même, lesquelles ils ne pouvaient ignorer, puisque le roi les
avait aussitôt après rendues au duc d'Antin avec permission d'en
informer les ducs, lequel lui en avait démontré la noirceur et la folie.
Le duc de La Force continuait en étendant sa réponse\,; mais la duchesse
du Maine, qui avait eu à peine la patience de l'écouter jusque-là,
l'interrompit avec un feu qu'elle ne put contenir. Elle lui dit qu'elle
s'en était toujours bien doutée\,; que les ducs ne cherchaient que des
échappatoires\,; mais que pour celle-là elle les tenait, et qu'elle leur
répondait que non seulement le roi ne serait point offensé de l'écrit
qu'elle leur demandait, mais qu'il leur en saurait même fort bon gré, et
que M. du Maine s'en faisait fort. Le duc d'Aumont profita prestement de
l'étourdissement où cette vive réponse jeta le duc de La Force, et de la
réflexion dans laquelle il tomba, quelque prévoyance qu'il en eût eue.
«\, Monsieur, lui dit Aumont, si nous ne trouvons plus de difficulté
comme madame l'assure, et que M. du Maine s'en fait fort, que
risquons-nous\,? et au contraire cette assurance de notre part n'est
qu'honorable.\,»

La Force retint l'indignation dont cette apostrophe le saisit, et avec
un sourire modeste lui répondit\,: «\,Mais qui nous assurera, monsieur,
que ce que le roi approuvera aujourd'hui, par considération pour M. le
duc du Maine, ne lui soit pas empoisonné demain contre nous sur son
autorité, à laquelle nous aurions attenté par la concurrence de la
nôtre\,; et contre M. le duc du Maine même qui, non content de toute
celle de la majesté royale, aurait en sus montré qu'il comptait ce
concours de notre part nécessaire, et qu'il y a eu recours\,? Qui nous
assurera que le premier président, dans la rage qu'il témoigne, que le
parlement, dans l'aliénation où il l'a mis de nous, n'aura pas encore
plus de jalousie que le roi de nous voir confirmer ce que cette
compagnie a solennellement enregistré\,; et que dans le temps que ces
messieurs n'épargnent rien pour nous réduire au simple état de membres
de leur corps, comme eux-mêmes et sans rien qui nous en distingue, ils
ne feront pas tous leurs efforts pour traiter d'attentat cette autorité
arrogée par-dessus, et en confirmation de la leur\,? Madame, se tournant
vers la duchesse du Maine, cela est trop délicat, ajouta-t-il\,; il
n'est aucun de nous qui en osât tenter le hasard.\,»
M\textsuperscript{me} du Maine rageait et le montrait bien à son visage.
Ce coup de partie embrassait tout, soit en effet pour s'assurer des ducs
une bonne et solide fois, comme elle le témaignait, soit pour les perdre
sans ressource auprès du roi, en quoi M. du Maine, qui répondait de Sa
Majesté à cet égard, et qui avait tant et si fort répondu du premier
président, en aurait usé avec la même perfidie, soit pour les perdre
avec les princes du sang, sans la moindre participation desquels cette
assurance par écrit était demandée et eût été accordée, soit avec le
parlement, soit avec le public, qui aurait vu les ducs disposer autant
qu'il était en eux de leur propre et seule autorité, par un écrit signé
d'eux, du droit de succéder à la couronne, sans nulle cause que leur
désir du bonnet et la volonté de la duchesse du Maine, que le duc du
Maine eût dédite, protesté qu'elle avait imaginé l'écrit de sa tête sans
son su, l'avait demandé sans la moindre participation de sa part,
répondu du roi par lui de son chef et sans lui en avoir jamais parlé, si
ce désaveu lui eût convenu dans la suite, comme on lui a vu faire depuis
en choses où il y allait de plus pour l'État et pour lui, comme on le
verra en son lieu. C'était donc là un coup tellement de partie que la
duchesse du Maine se contint, ne se rebuta point, et se mit à répliquer,
dupliquer et à faire les derniers efforts pour l'emporter à force
d'esprit et d'autorité sur M. de La Force, à qui seul elle avait
affaire, le pied ayant déjà si bien glissé au duc d'Aumont. Celui-ci se
voulut mêler une ou deux fois dans la dispute, mais il fut toujours
repoussé par l'autre, qui, lui mettant la main sur le bras, ne
s'interrompait point, et lui étouffa toujours la parole.

La duchesse du Maine se trouvant à bout, céda enfin à sa colère. Elle
dit à ces messieurs qu'elle voyait bien qu'eux ni leurs confrères ne se
pouvaient regagner\,; qu'ils mettaient en avant une vaine crainte du roi
duquel elle leur répondait, une vaine crainte d'ailleurs, une vaine
modestie sur eux-mêmes, surtout beaucoup d'esprit et de compliments à la
place de réalités nécessaires\,; qu'ils voulaient leur fait, et se
réserver entiers pour ce qui leur conviendrait dans l'avenir\,; que
c'était à M. du Maine et à elle à savoir s'en garantir\,; et qu'elle
voulait bien leur dire (et ceci est étrangement remarquable, d'autant
plus qu'elle n'a rien oublié, ni M. du Maine, pour le bien effectuer
depuis, comme on le verra en son lieu), qu'elle voulait bien leur dire,
pour qu'ils n'en pussent douter, que quand on avait une fois acquis
l'habilité de succéder à la couronne, il fallait, plutôt que se la
laisser arracher, mettre le feu au milieu et aux quatre coins du
royaume. Ce furent ses dernières paroles. En les achevant elle se leva
brusquement, sans toutefois qu'il lui fût échappé quoi que ce soit
contre ces deux ducs ni contre les ducs en général. On se quitta avec
beaucoup de compliments forcés d'une part, et de respects de l'autre qui
ne l'étaient pas moins, le duc de La Force ayant toujours l'œil sur le
duc d'Aumont, qui n'osa rien dire en particulier à la duchesse du Maine,
ni la suivre. Ils partirent aussitôt de Sceaux et vinrent rendre compte
de leur voyage.

Ce qui vient d'être raconté de la conversation de Sceaux est copié mot à
mot sur le rapport qui en fut fait par le duc de La Force, en présence
du duc d'Aumont, qui n'y trouva rien à ajouter, à diminuer ni à changer.
Il parut si important et en même temps si curieux qu'il fut écrit
sur-le-champ même, et c'est d'ou il a été pris. On n'en a omis que ce
que ce premier écrit omit, qui est un fatras de répliques et de
dupliques de part et d'autre, qui n'étaient que des répétitions
continuelles en d'autres termes des premiers, et pour ainsi dire des
propos matrices, qui furent écrits, et qu'on a exactement copiés. On en
usera ici comme on a fait sur les impostures du premier président au
roi, c'est-à-dire qu'on supprimera tout commentaire. Le simple narré est
non seulement au-dessus de tous ceux qu'on pourrait faire, mais il se
peut dire que la proposition de la duchesse du Maine, et la menace de sa
part de culbuter l'État, et sa déclaration de le faire plutôt que perdre
la succession à la couronne, surpassent non seulement toute attente,
mais toute imagination. Resterait à savoir le véritable projet de cet
engagement de conférence avec la duchesse du Maine. Était-ce un panneau
tendu au désir du bonnet, à l'embarras honteux de l'état actuel de cette
affaire, et à la sottise espérée des ducs que cet écrit d'assurance pour
les en accabler après par le roi, par les princes du sang, par le
parlement, par le public\,? et il semble que le personnage infâme de
délateur et d'imposteur que le premier président venait de faire auprès
du roi contre les ducs conduise à le penser. N'était-ce aussi que la
peur extrême du futur qui saisissait un moment d'espérance d'obtenir cet
écrit, avec dessein effectif de faire donner le bonnet, et de laisser le
premier président dans la nasse après s'être assuré des ducs, et
peut-être du roi à cet égard d'avance\,? Mais qui pourrait sonder les
profondeurs du gouffre noir et sans fond du sein du duc du Maine, qui se
substituait son épouse après avoir paru plus qu'il ne voulait dans la
conduite affreuse du premier président\,? Dieu les a jugés tous deux, il
n'appartient pas aux hommes de le faire.

Quel qu'en ait été le dessein, il manqua, grâce au duc de La Force qui,
se voyant trahi par son adjoint, conserva toute la présence de son
esprit et de son courage pour s'en tirer habilement et nettement, sans
donner prise le moins du monde. M. du Maine, comblé au moins d'avoir
commis les ducs avec le premier président par un si vif éclat, et le
parlement par lui, ne perdait point de vue son premier projet de faire
casser la corde sur les ducs sans qu'il partît y avoir part, et délivrer
en même temps le premier président de faire au roi une réponse nettement
négative. Cette réponse de plus ou de moins, après ce qu'il avait dit au
roi des ducs, ne lui aurait pas, à leur égard, gâté sa robe davantage.
Mais soit que le premier président crût en avoir assez fait, soit que M.
du Maine craignît de se manifester davantage par cette dernière
démarche, soit encore, supposé que le roi ne fût pas de la partie, qu'il
craignît que, piqué de la conduite du premier président, il ne se fâchât
jusqu'à décider le bonnet en faveur des ducs, le duc du Maine eut
recours à une nouvelle scène, à travers laquelle il ne parut l'auteur de
tout le jeu que plus manifestement\,: ce fut d'y amener
M\textsuperscript{me} la Princesse. Il ne pouvait néanmoins ignorer que,
dès le commencement de l'affaire, il avait répondu des princes du sang,
et d'elle nommément, si bien qu'il usa pour elle du mot de
happelourde\footnote{Ce mot, qui se disait au propre d'une pierre
  fausse, désignait, au figure, une personne de belle apparence, mais
  sans esprit.}, du terme d'imbécile qui n'était comptée pour rien, et
qui ne s'était jamais mêlée de rien dans sa famille ni dehors, qui
n'aurait osé penser à s'opposer à l'inclination du roi, et qui ne
branlerait jamais au moindre mot que lui son gendre lui dirait. Cela ne
fut pas dit par lui pour une fois aux ducs, mais à plusieurs, et
plusieurs fois répété, en répondant lui-même, et y mêlant des
plaisanteries du peu de cas qu'il y avait à en faire. Mais l'affaire
pressait, il fallait une issue, il choisit celle-là, ou il n'en trouva
point d'autre. Dans cet instant M\textsuperscript{me} la Princesse
devint un esprit, une femme de tête et d'autorité qui alla parler au roi
pour sa famille. Elle dit que M. le Prince lui avait toujours parlé du
bonnet comme de la plus chère distinction des princes du sang sur les
pairs\,; qu'elle avait trop de respect pour sa mémoire, pour ses
sentiments, pour ses volontés, pour l'intégrité du rang des princes du
sang, pour ne pas supplier le roi de toutes ses forces de n'y rien
innover. Là-dessus le roi dit à d'Antin qu'il était fâché de cette
fantaisie qui avait pris à M\textsuperscript{me} la Princesse, qu'il ne
pouvait la persuader ni passer par-dessus\,; et qu'il ne voulait plus
ouïr parler du bonnet. D'Antin, qui vit bien que c'était une chose
préparée, ne laissa pas de répondre de son mieux. Mais il parut
clairement que le roi était convenu avec M. du Maine d'en sortir\,;le
cette façon, et rien ne le put ébranler.

Rien de si transparent que ce personnage de M\textsuperscript{me} la
Princesse. Personne n'ignorait le peu de figure qu'elle avait fait dans
sa famille toute sa vie, ni les mépris et les duretés avec lesquels M.
le Prince l'avait sans cesse traitée jusqu'à sa mort, bien loin de lui
parler du bonnet, ni même de la moindre chose la plus domestique. Avec
des millions dont elle pouvait disposer, elle n'eut pas le moindre
crédit ni moyen d'éteindre le feu que le testament de M. le Prince fit
naître parmi ses enfants\,; et si on a vu en son lieu qu'elle fit
résoudre en un instant, par l'autorité du roi qu'elle alla trouver, le
double mariage de M. le Duc et de M. le prince de Conti, c'est qu'elle
fut guidée et poussée par l'intérêt de M\textsuperscript{lle} de Conti,
brusquement, et à l'insu de tous, et que ce qu'elle apprit au roi, par
la trahison de M\textsuperscript{lle} de Conti, du mariage, résolu entre
M. {[}le duc{]} et M\textsuperscript{me} la duchesse d'Orléans et
M\textsuperscript{me} la princesse de Conti, de M\textsuperscript{lle}
de Chartres et de M. le prince de Conti, sans que le roi en sût le
premier mot, le détermina sur-le-champ à montrer son autorité en le
rompant et faisant en même temps épouser M\textsuperscript{lle} de
Bourbon à M. le prince de Conti, et M\textsuperscript{lle} de Conti à M.
le Duc. Ici le roi, loin d'être piqué contre les ducs, l'était contre le
premier président, et le crédit de M\textsuperscript{me} la Princesse
n'avait jamais paru en aucune existence auprès du roi. M. du Maine
n'apprit rien aux ducs sur M\textsuperscript{me} sa belle-mère\,; mais
les ducs, toujours en soupçon, voulurent se faire assurer par lui
plusieurs fois, non d'elle, trop incapable pour en avoir rien à
craindre, sûrs surtout que nous étions de M\textsuperscript{me} la
Duchesse par nous-mêmes qui était très bien avec elle, mais que, par les
assurances qu'il nous donnait de M\textsuperscript{me} la Princesse,
jusqu'à nous répondre d'elle, plusieurs fois, comme on l'a vu, il se
trouvât hors d'état de nous la produire, comme il n'eut pas honte après
tout cela de faire pour s'en servi contre nous. M\textsuperscript{me} la
Princesse, de plus, n'avait ni grâce, ni prétexte, ni raison\,; on ira
même plus loin, elle n'avait pas droit ni caractère de s'opposer à ce
que M\textsuperscript{me} sa belle-fille consentait pour MM. ses
enfants, beaucoup moins à ce que M. le duc d'Orléans, eux si reculés,
lui fils du frère unique du roi et père du premier prince du sang,
consentait pour soi, pour lui et pour sa postérité. Il n'y eut donc
personne qui ne reconnût le duc du Maine à travers M\textsuperscript{me}
la Princesse, sans lequel le roi, disposé comme il le paraissait, et si
accoutumé à ne compter M\textsuperscript{me} la Princesse que par
l'extérieur de princesse du sang, lui eût bien demandé de quoi elle se
mêlait quand M. le duc d'Orléans et M\textsuperscript{me} la Duchesse
consentaient à une chose que lui-même trouvait juste et raisonnable\,;
ou plutôt, sans M. du Maine, le bonnet eût été accordé ou refusé qu'elle
ne l'aurait peut-être pas su de six mois après, de la façon dont elle
vivait. Personne donc, même des non intéressés, ne prit aux plaintes de
M. du Maine, qui disait à qui voulait l'entendre que
M\textsuperscript{me} la Princesse lui avait bien lavé la tête d'avoir
mis en avant l'affaire du bonnet. Elle finit donc de cette manière.
D'Antin dit aux ducs ce que le roi lui avait déclaré après avoir écouté
M\textsuperscript{me} la Princesse, qui lui alla parler huit ou dix
jours après la conférence de Secaux.

J'avais toujours été dans cette affaire, depuis la première conférence
que j'ai marqué que nous eûmes cinq ou six ensemble chez le maréchal
d'Harcourt pour délibérer sur l'embarquement, et M. du Maine m'avait
raccroché plusieurs fois a Marly, quoique je l'évitasse, pour m'en
parler avant l'éclat du premier président. Je ne dissimulerai pas que je
fus outré de nous voir le jouet de l'art et de la puissance de M. du
Maine, et de la scélératesse du premier président. Ce fut un samedi au
soir que d'Antin nous rendit à Versailles la réponse définitive du roi.
J'eus la nuit devant moi. Elle ne put me persuader de laisser M. du
Maine jouir paisiblement du plein et plus que plein succès de ses
souplesses\,; ce terme, je pense, n'est pas trop fort. Il m'avait
répondu de soi, de M\textsuperscript{me} la Princesse, des princes du
sang, du premier président, du parlement, comme aux autres ducs\,; il
m'avait fait les mêmes protestations de son désir et de sa bonne foi\,;
il m'avait même pressé dans les premiers temps de m'assurer du
consentement de M. le duc d'Orléans. Aucun péril ne me put persuader une
servitude assez basse pour lui laisser ignorer ce que je sentais. Je n'y
voulus embarquer personne avec moi, mais je ne pus souffrir qu'il le
portât plus loin. Je logeais dans l'aile neuve de plain-pied à la
tribune, lui dans la même aile en bas, tout auprès de la grande porte de
la chapelle. Le lendemain dimanche, je le fis guetter au sortir de la
chapelle. Jamais les fêtes et dimanches il n'y manquait grand'messe,
vêpres et le salut, et toutefois sa piété ne trompait personne. Il
allait souvent à complies, à la prière, au sermon toujours quand il y en
avait, et au salut les jeudis.

Dès que je fus averti, je descendis chez lui. Je le trouvai seul dans
son cabinet, qui me reçut l'air ouvert, de la manière du monde la plus
polie et la plus aisée. Je n'ouvris la bouche qu'après que je fus assis
dans mon fauteuil, et M. du Maine dans le sien. Alors, d'un air fort
sérieux, je lui dis ce que j'avais appris. M. du Maine blâma
M\textsuperscript{me} la Princesse, tomba sur elle, s'excusa,
s'affligea. Je l'interrompis pour lui nommer seulement et gravement le
premier président. M. du Maine voulut un peu l'excuser, et promptement
ajouta qu'il ne fallait point désespérer de l'affaire ni la regarder
comme finie\,; que pour lui il ne cesserait d'y travailler, et qu'il ne
serait jamais content qu'il n'en fût venu à bout. Sans m'émouvoir je
l'écoutai, puis lui dis toutes les impostures du premier président au
roi contre les ducs, que le roi avait rendues sur-le-champ à d'Antin,
avec permission de nous les dire, duquel je les savais\,; et delà je
traitai le premier président sans mesure, mais sans colère, avec un
simple air du plus profond mépris et de l'horreur de sa scélératesse. Ce
n'était pas que je comptasse lui rien apprendre, mais lui montrer que je
n'ignorais rien\,; et tout de suite le regardant fixement entre deux
yeux\,: «\,C'est vous, monsieur, continuai-je, qui nous avez engagés
malgré nous dans cette affaire\,; c'est vous qui nous avez répondu du
roi, du premier président, et par lui du parlement\,; c'est vous qui
nous avez répondu de M\textsuperscript{me} la Princesse\,; c'est vous
qui la faites intervenir maintenant, après avoir fait jouer au premier
président un si indigne personnage\,; enfin, c'est vous, monsieur, qui
nous avez manqué de parole, et qui nous rendez le jouet du parlement et
la risée du monde.\,» M. du Maine, toujours si vermeil et si désinvolte,
devint interdit et pâle comme un mort. Il voulut s'excuser en
balbutiant, et témoigner sa considération pour les ducs, et en
particulier pour moi. Je l'écoutais sans avoir ôté un moment les yeux de
dessus les siens. Enfin, fixant les yeux de plus en plus sur lui, je
l'interrompis et lui dis d'un ton élevé et fier, mais toujours
tranquille et sans colère\,: «\,Monsieur, vous pouvez tout, vous nous le
montrez bien et à toute la France\,; jouissez de votre pouvoir et de
tout ce que vous avez obtenu,\,» mais en haussant la tête et la voix et
le regardant jusqu'au fond de l'âme\,: «\,Il vient quelquefois des temps
où on se repent trop tard d'en avoir abusé, et d'avoir joué et trompé de
sang-froid tous les principaux seigneurs du royaume en rang et en
établissements, qui ne l'oublieront jamais\,;» et brusquement je me
lève, et tourne pour m'en aller sans lui laisser le moment de répondre.
Le duc du Maine, l'air éperdu d'étonnement et peut-être de dépit, me
suivit, balbutiant encore des excuses et des compliments. J'allai
toujours, sans me tourner, jusqu'à la porte. Là, je me tournai, et d'un
air d'indignation je lui dis\,: «\, Oh\,! monsieur, me conduire après ce
qui s'est passé, c'est ajouter la dérision à l'insulte.\,» Je passai à
l'instant la porte, et m'en allai sans regarder derrière moi.

La même après-dînée je racontai cette visite aux autres ducs de point en
point. Je ne sais si beaucoup l'eussent voulu faire, mais tous en
parurent très satisfaits. Nul ne le fut plus que moi. Je n'ai point su
ce que M. du Maine fit de cette conversation, dont il n'avait, je pense,
éprouvé encore de pareille. S'il en parla au roi, s'il s'en ouvrit à
M\textsuperscript{me} de Maintenon, s'il la tint secrète de sa part,
c'est ce que je n'ai point démêlé, et dont je me mis peu en peine. Si le
roi la sut, il a fait comme s'il ne la savait pas\,;
M\textsuperscript{me} de Maintenon de même. Jamais M\textsuperscript{me}
de Saint-Simon et moi ne nous en sommes aperçus. Personne de chez M. du
Maine, ni de Sceaux, n'en a jamais parlé. On peut juger que M. du Maine
et moi ne retournâmes pas l'un chez l'autre, et ne nous cherchions pas.
Nous nous rencontrions rarement\,; alors M. du Maine s'arrêtait et me
saluait bas, et de la façon la plus marquée (son pied-bot l'obligeait à
s'arrêter ainsi quand il voulait saluer quelqu'un par une véritable
révérence)\,; je lui répondis fidèlement par une demie, toujours
marchant\,; et nous vécûmes ainsi jusqu'à la mort du roi.

Quoique les réflexions gâtent souvent des Mémoires, il est difficile de
s'empêcher d'en faire ici sur le renversement de toutes lois, droits et
ordre pour des élévations sans mesure. Ceux qui les obtiennent regardent
comme ennemi tout ce qui n'approuve pas leur fortune, et comme des gens
à perdre tous ceux qui dans d'autres temps les y pourraient troubler.
Semblables aux tyrans qui ont asservi leur patrie, ils craignent tout,
ils se défient de tout, des hommes de sens et de courage dont l'état est
blessé de cette étrange élévation\,; ils se croient tout permis contre
eux, et la crainte de déchoir devient en eux une passion si supérieure à
tout autre sentiment, qu'il n'est crime dont ils puissent avoir horreur,
dès qu'il devient utile à la conservation de ce qu'ils ont usurpé.

On voit ici le plus noir dessein du duc du Maine amené à succès par les
plus noirs procédés, et en même temps les plus profondément pourpensés.
La fausseté, la trahison, la perfidie, les manquements de parole sans
cesse multipliés, la violence adroite pour attirer forcément dans ses
pièges, les divers personnages également soutenus, le dernier abus d'une
âme de boue, que comme telle il a mise sur le chandelier, à qui il fait
souffler comme il veut le froid et le chaud, qu'il rend traître jusque
sans le plus léger prétexte, et dont il se sert enfin pour faire vomir
au roi les impostures les plus absurdes, mais les plus infernales contre
tout ce que sa cour a de plus distingué et qui l'approche de plus près.
À force de se cacher derrière des gazes, et de multiplier les horreurs,
on sent qu'il est auteur et moteur de toutes les machines, et qu'il
n'oublie rien pour n'être point aperçu. Il se voue aux ténèbres, et les
ténèbres mêmes le rejettent. On les voit ensuite, lui et son infâme
instrument, tenter tout pour se tromper l'un l'autre\,: le premier
président pour obtenir des ducs de suivre les présidents, et laisser M.
du Maine dans la nasse\,; M. du Maine chercher à s'assurer des ducs en
leur donnant ce qu'ils voulaient, en laissant le premier président dans
le fond du bourbier que sa servitude, à ce maître perfide, lui avait
fait creuser à lui-même. Couverts enfin l'un et l'autre de tout ce qui
peut rendre les hommes plus méprisables et plus odieux, sans plus de
ressource de n'être pas vus tels et à plein découverts, on voit M. du
Maine se servir de son épouse, et abuser du respect dû à sa naissance de
fille du premier prince du sang, pour faire nettement et distinctement
les propositions les plus criminelles et en même temps les plus farcies
de toutes les sortes de poisons, et qui, dans la rage de ne les pouvoir
faire accepter, ose déclarer que, plutôt que se voir arracher ce qui
n'est pas dans le pouvoir des rois, ni dans la nature des choses de
donner, je veux dire la succession à la couronne, ils mettront le feu au
milieu et aux quatre coins du royaume. Est-ce une {[}personne{]} issue
de la couronne qui parle\,? Est-ce quelqu'un dont les frères et les
neveux y sont incontestablement appelés\,? Le plus mortel ennemi de nos
rois, de nos princes, de notre patrie, pourrait-il emprunter de la plus
furieuse rage des paroles qui en fussent plus le langage\,? et ce
langage est celui d'une princesse du sang qui a oublié ce qu'elle est,
et la reconnaissance de tous les biens, charges et grandeurs qu'a
obtenus le mari qu'elle a épousé, qui ont passé à ses enfants, qui tous
sont les premiers doubles adultérins que le soleil ait vus paraître, et
que les lois violées ont soufferts hors du néant et de la non
existence\,! menace enfin qui, selon toutes les lois et suivant encore
toute politique, en cela parfaitement d'accord avec les lois, mérite ce
qu'on n'oserait exprimer. Et à qui s'adresse-t-elle pour vomir cette
criminelle menace\,? à des gens du plus grand état, qu'elle regarde
comme ses ennemis, et que dans ce moment elle rend tels, et à qui elle
ne craint pas de le dire. On verra dans la suite qu'il n'a pas tenu à
elle, ni à son mari, caché alors derrière elle tant qu'il put, et
jusqu'à la dernière comédie, comme il s'y cachait ici, qu'ils n'aient
renversé l'État et livré la France en proie\ldots. Que n'aurait-on pas à
ajouter\,!

Mesmes, trop vil pour s'arrêter à lui, et qui, par ce qu'on vient d'en
voir, s'est montré par trop infâme pour ne pas déshonorer par le seul
attouchement qui en voudrait réfléchir ou produire, laissera sauter
par-dessus son infecte pourriture pour faire une courte réflexion sur le
bonnet.

On en a vu ci-dessus la nouveauté, l'art et la plus qu'indécence\,; elle
est telle que les présidents eux-mêmes sont forcés de l'avouer. Toute
leur défense est de se couvrir du nom et de la majesté du roi qu'ils
prétendent représenter tous ensemble en leur commune présidence, et
c'est par cette représentation qu'ils essayent de soutenir leurs
prétentions. La fausseté de cet allégué se découvre en ce que les
représentants du roi auraient la première place dans le lieu et la
fonction de leur représentation. Or il est de fait que ce sont les pairs
qui l'ont sur eux, tant aux hauts sièges qu'aux bas sièges, puisqu'ils
sont à la droite du coin du roi, au haut bout derrière lequel il n'y a
point de passage, et du côté de la cheminée, du côté du barreau de
préférence, du côté de la place et du plaidoyer des gens du roi. Si on a
nouvellement changé la cheminée, il demeure constant que c'est une
nouveauté\,; et le côté droit, à ce qui vient d'en être expliqué,
demeure en existence et en évidence. Il faut donc dire que les
présidents président au nom du roi, et non pas que des légistes pour
leur argent le représentent. Cette représentation est même si fausse à
leurs propres yeux qu'ils ne la pouvaient alléguer en présence du roi en
lit de justice. Ils ne pouvaient pas même s'appuyer sur la simple
présidence, puisque la présence du chancelier la leur ôte, et les efface
totalement. Néanmoins on les a vus usurper d'opiner en lit de justice,
non seulement devant les pairs et les princes du sang, mais devant les
fils de France, et devant la reine mère et régente\,; et les mouvements
qu'ils se donnèrent montrent bien que c'était pour leurs personnes
uniquement, et dans lesquels ils engagèrent le parlement d'entrer,
quoiqu'il n'y eût pas le moindre intérêt, lorsque cette affaire fut
enfin portée devant le roi en 1662, qui, très contradictoirement, jugea
contre eux pour les pairs ce qui a toujours subsisté depuis. Il est donc
évident, par cet exemple dont on se contente ici, que ce n'est ni par la
représentation du roi qu'ils n'ont point, ni par la présidence qu'ils
exercent en son nom, qu'ils osent soutenir l'énorme usurpation du
bonnet, et que, si le roi les obligeait d'articuler à quel titre, ils
demeureraient confondus.

Mais que pourraient-ils alléguer au roi là-dessus, en leur laissant même
soutenir cette représentation fausse et idéale, dès que le roi consent
pour ce qui le regarde, et qu'il dit au premier président que ce que les
ducs demandent lui paraît juste et raisonnable, et qu'il désire qu'ils
soient contents\,? c'était les mettre au pied du mur. Aussi le premier
président n'osa jamais faire une dernière réponse au roi\,; et ce fut
pour l'en délivrer que M. du Maine n'eut pas honte, après avoir tant de
fois répondu de M\textsuperscript{me} la Princesse, de l'amener enfin
sur la scène pour finir l'affaire comme on l'a vu.

Finissons par un mot fort court. Le chancelier va au parlement toutes
les fois que bon lui semble, y préside, et y efface totalement le
premier président et tous les autres présidents\,; il y déplace le
premier président en l'absence du roi\,; il est le supérieur du
parlement. Quand cette compagnie va chez lui le haranguer, et il n'est
point de chancelier à qui cela n'arrive, c'est par députés, parmi
lesquels sont le premier président et d'autres présidents à mortier. Le
premier président lui porte la parole et le traite toujours de
monseigneur\,; la députation est très légèrement conduite par le
chancelier qui prend la main sur le premier président et sur tous, et, à
l'ordinaire de la vie, ne donne la main chez lui à aucun magistrat, ni
la chancelière, qui a d'ailleurs un rang fort inférieur au sien, ne
donne aussi la main chez elle ni à la première présidente ni à aucune
femme de robe, et la donne néanmoins à toutes les autres, à la
différence du chancelier qui ne la donne qu'aux gens titrés. Voilà donc
une supériorité entière du chancelier sur le premier président et sur
tous les présidents qui, en corps, et le premier président en
particulier, lui écrivent \emph{monseigneur} et en reçoivent réponse
fort disproportionnée. Le conseil privé, ou des parties, qui casse les
arrêts du parlement, n'a qu'un seul président qui est le chancelier. En
prenant les avis il est couvert, et le demeure lorsque les conseillers
d'État se découvrent lorsqu'il les nomme pour opiner. Il n'ôte son
chapeau qu'en nommant le doyen du conseil, et le nomme M. le doyen, et
non par son nom comme il fait tous les autres conseillers d'État.
Lorsqu'il y a eu des pairs, même M. de Vitry, qui n'était que duc à
brevet et conseiller d'État d'épée, le chancelier s'est toujours
découvert pour eux, et l'exemple de MM. de Reims et de Noyon en est
récent. Que l'on compare maintenant le chancelier et le premier
président et leur très différent usage\,; qu'est-il possible que les
présidents y répondent qui se puisse souffrir\,? En voilà assez sur
cette étrange affaire qui gagna le mois de mars 1715. Sa nature a obligé
à un récit de suite et non interrompu\,; reprenons maintenant les
matières accoutumées, et revenons sur nos pas au 1er janvier 1715.
Toutefois il ne faut pas que l'empressement de finir une si désagréable
matière fasse omettre que M. du Maine avait payé d'avance le premier
président, presque immédiatement avant de l'entamer. Ce magistrat, qui
était un panier percé qui jetait à tout, et beaucoup en breloques, avait
toujours grand besoin d'argent, et se gouvernait fort par ce continuel
désir. Il avait quatre cent mille livres de brevet de retenue qu'il
avait payées à son prédécesseur\,; il n'eut pas honte d'en demander la
jouissance par une nouvelle pension de vingt mille livres, ni le duc du
Maine de la solliciter auprès du roi, qui n'était plus à portée de
refuser quoi que ce fût à ce très cher bâtard, et cher en toutes les
sortes.

\hypertarget{chapitre-xxii.}{%
\chapter{CHAPITRE XXII.}\label{chapitre-xxii.}}

1715

~

{\textsc{Année 1715.}} {\textsc{- Grillo vient faire au roi les
remercîments de la reine d'Espagne.}} {\textsc{- Trois cent mille livres
de brevet de retenue au duc de Bouillon sur son gouvernement
d'Auvergne.}} {\textsc{- Trois mille livres de pension à Arpajon\,; six
mille à Celi, intendant à Pau.}} {\textsc{- Électeur de Bavière à
Versailles.}} {\textsc{- Électeur de Cologne y prend congé du roi et
retourne dans ses États.}} {\textsc{- Mariage du prince héréditaire de
Hesse-Cassel avec la soeur du roi de Suède.}} {\textsc{- Mort de la
princesse d'Isenghien (Pot), sans enfants.}} {\textsc{- Mort\,;
caractère et famille du comte de Grignan\,; sa dépouille.}} {\textsc{-
Mort et caractère du maréchal de Chamilly\,; sa dépouille.}} {\textsc{-
Caractère, vie, conduite et mort de Fénelon, archevêque de Cambrai. --
Menées de Fleury, évêque de Fréjus, pour être précepteur de Louis XV.}}
{\textsc{- Origine de la haine implacable et de la persécution sans
bornes ni mesure de Fleury, évêque de Fréjus, depuis cardinal et maître
du royaume, contre le P. Quesnel et les jansénistes.}} {\textsc{- La
Parisière, évêque de Nîmes, Zopyre du P. Tellier.}} {\textsc{- Son
invention ultramontaine\,; sa misérable mort.}} {\textsc{- Mort et
caractère de l'abbé de Lyonne et d'Henriot, évêque de Boulogne.}}
{\textsc{- Gesvres, archevêque de Bourges, obtient la nomination au
cardinalat des deux rois de Pologne, Stanislas et l'électeur de Saxe.}}
{\textsc{- Languet fait évêque de Soissons, et quelques autres bénéfices
donnés.}} {\textsc{- Mort et caractère de la duchesse de Nevers.}}
{\textsc{- Infructueuse malice de M. le Prince.}}

~

Cette année commença par les remercîments que la reine d'Espagne fit au
roi des présents qu'elle en avait reçus par le duc de Saint-Aignan. Elle
lui dépêcha le marquis Grillo, noble génois, qu'elle affectionnait, et
qu'elle fit grand d'Espagne dès qu'elle s'y fut rendue maîtresse.

M. de Bouillon obtint cent mille écus de brevet de retenue sur son
gouvernement d'Auvergne\,; le marquis d'Arpajon mille écus de pension\,;
et Harlay, fils de l'ambassadeur plénipotentiaire à la paix de Ryswick,
deux mille. Il était intendant à Pau. Le roi ne se démentit jamais en la
moindre chose de sa préférence distinguée et marquée en tout de la robe
sur l'épée, et du bourgeois sur le noble.

L'électeur de Bavière tira dans le petit pare, ce qui était une faveur
où les fils de France avaient rarement atteint\,; joua après chez
M\textsuperscript{me} la Duchesse, soupa et joua chez d'Antin, ne vit
point le roi, et s'en retourna. On sut en même temps que le roi de
Suède, qui était toujours à Stralsund, avait accordé la princesse
Ulrique, sa sœur, au prince héréditaire de Hesse-Cassel, qui l'allait
épouser à Stockholm. C'est le même prince qui avait toujours servi dans
les armées des alliés contre la France, et qui fut battu en Italie par
Médavy presque en même temps de la levée du siège de Turin. L'électeur
de Cologne prit congé du roi dans son cabinet l'après-dînée, pour
retourner enfin dans ses États\,; il entra et sortit de chez le roi à
l'ordinaire par les derrières.

M\textsuperscript{me} d'Isenghien mourut en couche d'un enfant mort.
Elle était Pot, fille unique du dernier marquis de Rhodes, et je crois
la dernière de cette illustre et ancienne maison. Elle était brouillée
avec sa mère qui était Simiane, nièce du feu évêque-duc de Langres,
malgré laquelle elle s'était mariée. Sa mort fit la réconciliation.

Le comte de Grignan, seul lieutenant général et commandant de Provence
et chevalier de l'ordre, gendre de M\textsuperscript{me} de Sévigné qui
en parle tant dans ses lettres, mourut à quatre-vingt-trois ans dans une
hôtellerie, allant de Lambesc à Marseille. C'était un grand homme, fort
bien fait, laid, qui sentait fort ce qu'il était, fort honnête homme,
fort poli, fort noble, en tout fort obligeant, et universellement
estimé, aimé et respecté en Provence, où, à force de manger et de n'être
point aidé, il se ruina. Il ne lui restait que deux filles\,:
M\textsuperscript{me} de Vibraie, fille de la sœur de la duchesse de
Montausier, que les mauvais traitements de la dernière
M\textsuperscript{me} de Grignan-Sévigné forcèrent à un mariage fort
inégal, et qui fut toujours brouillée avec eux\,; et
M\textsuperscript{me} de Simiane, fille de la Sévigné, adorée de sa mère
comme elle l'était de la sienne. Elle avait épousé Simiane par amour
réciproque. Il avait peu servi, et il était premier gentilhomme de la
chambre de M. le duc d'Orléans, léger emploi alors, mais qui par
l'événement lui valut la lieutenance générale de Provence, dont le roi
n'avait pas disposé lorsqu'il mourut.

Le maréchal de Chamilly mourut à Paris le 7 janvier, après une longue
maladie, à soixante-dix-neuf ans. C'était un grand et gros homme, fort
bien fait, extrêmement distingué par sa valeur, par plusieurs actions,
et devenu célèbre par la défense de Grave. On en a parlé ailleurs à
diverses reprises. Il était fort homme d'honneur et de bien, et vivait
partout très honorablement\,; mais il avait si peu d'esprit qu'on en
était toujours surpris, et sa femme, qui en avait beaucoup, souvent
embarrassée. Il avait servi jeune en Portugal, et ce fut à lui que
furent écrites ces fameuses \emph{Lettres Portugaises}, par une
religieuse qu'il y avait connue et qui était devenue folle de lui. Il
n'eut point d'enfants. Son nom était Bouton, dont il y a eu des
chambellans des derniers ducs de Bourgogne, province d'où ils étaient.
Il ne laissa vacant que le gouvernement de Strasbourg, que le roi donna
au maréchal d'Huxelles, qui fut un beau morceau ajouté à son
gouvernement d'Alsace où néanmoins il ne retourna plus. La vérité est
que, pour plus de trente mille livres de rentes que valait Strasbourg,
il en rendit douze mille d'appointements du gouvernement de Brisach.

En ce même commencement de janvier, Fénelon, aujourd'hui conseiller
d'État d'épée, lieutenant général, gouverneur du Quesnoy et chevalier de
l'ordre après avoir été ambassadeur en Hollande, entra chez moi à
Versailles comme j'achevais de dîner. Il me dit fort affligé qu'il
venait d'apprendre par un courrier que l'archevêque de Cambrai, son
grand-oncle, était extrêmement mal\,; et qu'il me venait prier d'obtenir
de M. le duc d'Orléans de lui envoyer Chirac, son médecin, sur-le-champ,
et de lui prêter ma chaise de poste. Je sortis de table aussitôt.
J'envoyai chercher ma chaise, et allai chez M. le duc d'Orléans, qui
envoya chercher Chirac, et lui ordonna de partir et de demeurer à
Cambrai tant qu'il y serait nécessaire. Entre l'arrivée de Fénelon chez
moi et le départ de Chirac il n'y eut pas une heure, et il alla tout de
suite à Cambrai. Il trouva l'archevêque hors d'espérance et d'état à
tenter aucun remède. Il y demeura néanmoins vingt-quatre heures, au bout
desquelles il mourut. Ainsi, moi qu'il craignait tant auprès de M. le
duc d'Orléans pour les temps futurs, ce fut moi qui lui rendis le
dernier service. Ce personnage a été si connu et si célèbre que, après
ce qui s'en voit en plusieurs endroits ici, il serait inutile de s'y
beaucoup étendre, quoiqu'il ne soit pourtant pas possible de ne s'y
arrêter pas un peu.

On a vu ici sa naissance d'ancienne et bonne noblesse, décorée
d'ambassades, de divers emplois, d'un collier du Saint-Esprit sous Henri
III, et d'alliances\,; sa pauvreté, ses obscurs commencements, ses
tentatives diverses vers les jansénistes, les jésuites, les pères de
l'Oratoire, le séminaire de Saint-Sulpice, auquel enfin non sans peine
il s'accrocha, et qui le produisit aux ducs de Chevreuse et de
Beauvilliers\,; le rapide progrès qu'il fit dans leur estime, la place
de précepteur des enfants de France qu'elle lui valut, ce qu'il en sut
faire, les sources et les progrès de la catastrophe de ses opinions et
de sa fortune\,; les ouvrages qu'il composa, ceux qui y répondirent\,;
les adresses qu'il employa et qui ne purent le sauver, la disgrâce de
ses partisans, de ses amis, de ses protecteurs, à combien peu il tint
qu'elle n'entraînât la ruine des ducs de Chevreuse et de Beauvilliers,
et l'incomparable action de Noailles, archevêque de Paris, depuis
cardinal, qui le brouilla pour longtemps avec le duc son frère et sa
belle-sœur\,; les divers contours de son affaire qu'il porta enfin à
Rome, où le roi fit agir en son nom comme partie contre lui\,; sa
condamnation canoniquement acceptée par toutes les assemblées des
provinces ecclésiastiques du royaume de l'obéissance du roi\,; la
promptitude, la netteté, l'éclat de sa soumission et sa conduite
admirable dans sa propre assemblée provinciale avec Valbelle, évêque de
Saint-Omer, qui s'en déshonora\,; enfin le bonheur qu'il eut de se
conserver en entier, et pour toujours, le cœur et l'estime de Mgr le duc
de Bourgogne, des ducs de Chevreuse et de Beauvilliers, et de tous ses
amis, sans l'affaiblissement d'aucun, malgré la roideur et la profondeur
de sa chute, la persécution toujours active de M\textsuperscript{me} de
Maintenon, le précipice ouvert du côté du roi, et dix-sept années
d'exil\,; tous aussi vifs pour lui, aussi attentifs, aussi faisant leur
chose capitale de ce qui le regardait, aussi assujettis à sa direction,
aussi ardents à profiter de tout pour le remettre en première place que
les premiers moments de sa disgrâce, et tous avec la plus grande mesure
de respect pour le roi, mais sans s'en cacher, et moins qu'aucun d'eux
les ducs de Chevreuse et de Beauvilliers, toute leur famille et Mgr le
duc de Bourgogne même.

Ce prélat était un grand homme maigre, bien fait, pâle, avec un grand
nez, des yeux dont le feu et l'esprit sortaient comme un torrent, et une
physionomie telle que je n'en ai point vu qui y ressemblât, et qui ne se
pouvait oublier quand on ne l'aurait vue qu'une fois. Elle rassemblait
tout, et les contraires ne s'y combattaient pas. Elle avait de la
gravité et de la galanterie, du sérieux et de la gaieté\,; elle sentait
également le docteur, l'évêque et le grand seigneur\,; ce qui y
surnageait, ainsi que dans toute sa personne, c'était la finesse,
l'esprit, les grâces, la décence, et surtout la noblesse. Il fallait
effort pour cesser de le regarder. Tous ses portraits sont parlants,
sans toutefois avoir pu attraper la justesse de l'harmonie qui frappait
dans l'original, et la délicatesse de chaque caractère que ce visage
rassemblait. Ses manières y répondaient dans la même proportion, avec
une aisance qui en donnait aux autres, et cet air et ce bon goût qu'on
ne tient que de l'usage de la meilleure compagnie et du grand monde, qui
se trouvait répandu de soi-même dans toutes ses conversations\,; avec
cela une éloquence naturelle, douce, fleurie\,; une politesse
insinuante, mais noble et proportionnée\,; une élocution facile, nette,
agréable\,; un air de clarté et de netteté pour se faire entendre dans
les matières les plus embarrassées et les plus dures\,; avec cela un
homme qui ne voulait jamais avoir plus d'esprit que ceux à qui il
parlait, qui se mettait à la portée de chacun sans le faire jamais
sentir, qui les mettait à l'aise et qui semblait enchanter, de façon
qu'on ne pouvait le quitter, ni s'en défendre, ni ne pas chercher à le
retrouver. C'est ce talent si rare, et qu'il avait au dernier degré, qui
lui tint tous ses amis si entièrement attachés toute sa vie, malgré sa
chute, et qui, dans leur dispersion, les réunissait pour se parler de
lui, pour le regretter, pour le désirer, pour se tenir de plus en plus à
lui, comme les Juifs pour Jérusalem, et soupirer après son retour, et
l'espérer toujours, comme ce malheureux peuple attend encore et soupire
après le Messie. C'est aussi par cette autorité de prophète, qu'il
s'était acquise sur les siens, qu'il s'était accoutumé à une domination
qui, dans sa douceur, ne voulait point de résistance. Aussi n'aurait-il
pas longtemps souffert de compagnon s'il fût revenu à la cour et entré
dans le conseil, qui fut toujours son grand but\,; et une fois ancré et
hors des besoins des autres, il eût été bien dangereux non seulement de
lui résister, mais de n'être pas toujours pour lui dans la souplesse et
dans l'admiration.

Retiré dans son diocèse, il y vécut avec la piété et l'application d'un
pasteur, avec l'art et la magnificence d'un homme qui n'a renoncé à
rien, qui se ménage tout le monde et toutes choses. Jamais homme n'a eu
plus que lui la passion de plaire, et au valet autant qu'au maître\,;
jamais homme ne l'a portée plus loin, avec une application plus suivie,
plus constante, plus universelle\,; jamais homme n'y a plus entièrement
réussi. Cambrai est un lieu de grand abord et de grand passage\,; rien
d'égal à la politesse, au discernement, à l'agrément avec lesquels il
recevait tout le monde. Dans les premières années on l'évitait, il ne
courait après personne\,; peu à peu les charmes de ses manières lui
rapprochèrent un certain gros. À la faveur de cette petite multitude,
plusieurs de ceux que la crainte avait écartés, mais qui désiraient
aussi de jeter des semences pour d'autres temps, furent bien aises des
occasions de passer à Cambrai. De l'un à l'autre tous y coururent. À
mesure que Mgr le duc de Bourgogne parut figurer, la cour du prélat
grossit\,; et elle en devint une effective aussitôt que son disciple fut
devenu Dauphin. Le nombre des gens qu'il y avait accueillis, la quantité
de ceux qu'il avait logés chez lui passant par Cambrai, les soins qu'il
avait pris des malades et des blessés qu'en diverses occasions on avait
portés dans sa ville, lui avaient acquis le coeur des troupes. Assidu
aux hôpitaux et chez les moindres officiers, attentif aux principaux, en
ayant chez lui en nombre et plusieurs mois de suite jusqu'à leur parfait
rétablissement, vigilant en vrai pasteur au salut de leurs âmes, avec
cette connaissance du monde qui les savait gagner et qui en engageait
beaucoup à s'adresser à lui-même, et il ne se refusait pas au moindre
des hôpitaux qui voulaient aller à lui, et qu'il suivait comme s'il
n'eût point eu d'autres soins à prendre, il n'était pas moins actif au
soulagement corporel. Les bouillons, les nourritures, les consolations
des dégoûts, souvent encore les remèdes sortaient en abondance de chez
lui\,; et dans ce grand nombre un ordre et un soin que chaque chose fût
du meilleur en sa sorte qui ne se peut comprendre. Il présidait aux
consultations les plus importantes\,; aussi est-il incroyable jusqu'à
quel point il devint l'idole des gens de guerre, et combien son nom
retentit jusqu'au milieu de la cour.

Ses aumônes, ses visites épiscopales réitérées plusieurs fois l'année,
et qui lui firent connaître par lui-même à fond toutes les parties de
son diocèse, la sagesse et la douceur de son gouvernement, ses
prédications fréquentes dans la ville et dans les villages, la facilité
de son accès, son humanité avec les petits, sa politesse avec les
autres, ses grâces naturelles qui rehaussaient le prix de tout ce qu'il
disait et faisait, le firent adorer de son peuple\,; et les prêtres dont
il se déclarait le père et le frère, et qu'il traitait tous ainsi, le
portaient tous dans leurs cœurs. Parmi tant d'art et d'ardeur de plaire,
et si générale, rien de bas, de commun, d'affecté, de déplacé, toujours
en convenance à l'égard de chacun\,; chez lui abord facile, expédition
prompte et désintéressée\,; un même esprit, inspiré par le sien, en tous
ceux qui travaillaient sous lui dans ce grand diocèse\,; jamais de
scandale ni rien de violent contre personne\,; tout en lui et chez lui
dans la plus grande décence. Ses matinées se passaient en affaires du
diocèse. Comme il avait le génie élevé et pénétrant, qu'il y résidait
toujours, qu'il ne se passait pas de jour qu'il ne réglât ce qui se
présentait, c'était chaque jour une occupation courte et légère. Il
recevait après qui le voulait voir, puis allait dire la messe, et il y
était prompt\,; c'était toujours dans sa chapelle, hors les jours qu'il
officiait, ou que quelque raison particulière l'engageait à l'aller dire
ailleurs. Revenu chez lui, il dînait avec la compagnie toujours
nombreuse, mangeait peu et peu solidement, mais demeurait longtemps à
table pour les autres, et les charmait par l'aisance, la variété, le
naturel, la gaieté de sa conversation, sans jamais descendre à rien qui
ne fût digne et d'un évêque et d'un grand seigneur\,; sortant de table
il demeurait peu avec la compagnie. Il l'avait accoutumée à vivre chez
lui sans contrainte, et à n'en pas prendre pour elle. Il entrait dans
son cabinet et y travaillait quelques heures, qu'il prolongeait s'il
faisait mauvais temps et qu'il n'eût rien à faire hors de chez lui.

Au sortir de son cabinet il allait faire des visites ou se promener à
pied hors la ville. Il aimait fort cet exercice et l'allongeait
volontiers\,; et, s'il n'y avait personne de ceux qu'il logeait, ou
quelque personne distinguée, il prenait quelque grand vicaire et quelque
autre ecclésiastique, et s'entretenait avec eux du diocèse, de matières
de piété ou de savoir\,; souvent il y mêlait des parenthèses agréables.
Les soirs, il les passait avec ce qui logeait chez lui, soupait avec les
principaux de ces passages d'armée quand il en arrivait, et alors sa
table était servie comme le matin. Il mangeait encore moins qu'à dîner,
et se couchait toujours avant minuit. Quoique sa table fût magnifique et
délicate, et que tout chez lui répondît à l'état d'un grand seigneur, il
n'y avait rien néanmoins qui ne sentît l'odeur de l'épiscopat et de la
règle la plus exacte, parmi la plus honnête et la plus douce liberté.
Lui-même était un exemple toujours présent, mais auquel on ne pouvait
atteindre\,; partout un vrai prélat, partout aussi un grand seigneur,
partout encore l'auteur de \emph{Télémaque}. Jamais un mot sur la cour,
sur les affaires, quoi que ce soit qui pût être repris, ni qui sentît le
moins du monde bassesse, regrets, flatterie\,; jamais rien qui pût
seulement laisser soupçonner ni ce qu'il avait été, ni ce qu'il pouvait
encore être. Parmi tant de grandes parties un grand ordre dans ses
affaires domestiques, et une grande règle dans son diocèse\,; mais sans
petitesse, sans pédanterie, sans avoir jamais importuné personne d'aucun
état sur la doctrine.

Les jansénistes étaient en paix profonde dans le diocèse de Cambrai, et
il y en avait grand nombre\,; ils s'y taisaient, et l'archevêque aussi à
leur égard. Il aurait été à désirer pour lui qu'il eût laissé ceux de
dehors dans le même repos\,; mais il tenait trop intimement aux
jésuites, et il espérait trop d'eux, pour ne leur pas donner ce qui ne
troublait pas le sien. Il était aussi trop attentif à son petit troupeau
choisi, dont il était le cœur, l'âme, la vie et l'oracle, pour ne lui
pas donner de temps en temps la pature de quelques ouvrages qui
couraient entre leurs mains avec la dernière avidité, et dont les éloges
retentissaient. Il fut rudement réfuté par les jansénistes\,; et il est
vrai de plus que le silence en matière de doctrines aurait convenu à
l'auteur si solennellement condamné du livre des \emph{Maximes des
saints}\,; mais l'ambition n'était rien moins que morte\,; les coups
qu'il recevait des réponses des jansénistes lui devenaient de nouveaux
mérites auprès de ses amis, et de nouvelles raisons aux jésuites de tout
faire et de tout entreprendre pour lui procurer le rang et les places
d'autorité dans l'Église et dans l'État. À mesure que les temps orageux
s'élaignaient, que ceux de son Dauphin s'approchaient, cette ambition se
réveillait fortement, quoique cachée sous une mesure qui, certainement,
lui devait coûter. Le célèbre Bossuet, évêque de Meaux, n'était plus, ni
Godet, évêque de Chartres. La constitution avait perdu le cardinal de
Noailles\,; le P. Tellier était devenu tout puissant. Ce confesseur du
roi était totalement à lui ainsi que l'élixir du gouvernement des
jésuites\,; et la société entière faisait profession de lui être
attachée depuis la mort du P. Bourdaloue, du P. Gaillard et de quelques
autres principaux qui lui étaient opposés, qui en retenaient d'autres,
et que la politique des supérieurs laissait agir, pour ne pas choquer le
roi ni M\textsuperscript{me} de Maintenon contre tout le corps\,; mais
ces temps étaient passés, et tout ce formidable corps lui était enfin
réuni. Le roi, en deux ou trois occasions depuis peu, n'avait pu
s'empêcher de le louer. Il avait ouvert ses greniers aux troupes dans un
temps de cherté et où les munitionnaires étaient à bout, et il s'était
bien gardé d'en rien recevoir, quoiqu'il eût pu en tirer de grosses
sommes en le vendant à l'ordinaire. On peut juger que ce service ne
demeura pas enfoui, et ce fut aussi ce qui fit hasarder pour la première
fois de nommer son nom au roi. Le duc de Chevreuse avait enfin osé
l'aller voir, et le recevoir une autre fois à Chaulnes\,; et on peut
juger que ce ne fut pas sans s'être assuré que le roi le trouvait bon.

Fénelon, rendu enfin aux plus flatteuses et aux plus hautes espérances,
laissa germer cette semence d'elle-même\,; mais elle ne put venir à
maturité. La mort si peu attendue du Dauphin l'accabla, et celle du duc
de Chevreuse qui ne tarda guère après aigrit cette profonde plaie\,; la
mort du duc de Beauvilliers la rendit incurable, et l'atterra. Ils
n'étaient qu'un coeur et qu'une âme, et, quoiqu'ils ne se fussent jamais
vus depuis l'exil, Fénelon le dirigeait de Cambrai jusque dans les plus
petits détails. Malgré sa profonde douleur de la mort du Dauphin, il
n'avait pas laissé d'embrasser une planche dans ce naufrage. L'ambition
surnageait à tout, se prenait à tout. Son esprit avait toujours plu à M.
le duc d'Orléans. M. de Chevreuse avait cultivé et entretenu entre eux
l'estime et l'amitié, et j'y avais aussi contribué par attachement pour
le duc de Beauvilliers qui pouvait tout sur moi. Après tant de pertes et
d'épreuves les plus dures, ce prélat était encore homme d'espérances\,;
il ne les avait pas mal placées. On a vu les mesures que les ducs de
Chevreuse et de Beauvilliers m'avaient engagé de prendre pour lui auprès
de ce prince, et qu'elles avaient réussi de façon que les premières
places lui étaient destinées, et que je lui en avais fait passer
l'assurance par ces deux ducs dont la piété s'intéressait si vivement en
lui, et qui étaient persuadés que rien ne pouvait être si utile à
l'Église, ni si important à l'État, que de le placer au timon du
gouvernement\,; mais il était arrêté qu'il n'aurait que des espérances.
On a vu que rien ne le pouvait rassurer sur moi, et que les ducs de
Chevreuse et de Beauvilliers me l'avouaient. Je ne sais si cette frayeur
s'augmenta par leur perte, et s'il crut que, ne les ayant plus pour me
tenir, je ne serais plus le même pour lui, avec qui je n'avais jamais eu
aucun commerce, trop jeune avant son exil, et sans nulle occasion
depuis. Quoi qu'il en soit, sa faible complexion ne put résister à tant
de soins et de traverses. La mort du duc de Beauvilliers lui donna le
dernier coup. Il se soutint quelque temps par effort de courage, mais
ses forces étaient à bout. Les eaux, ainsi qu'à Tantale, s'étaient trop
persévéramment retirées du bord de ses lèvres toutes les fois qu'il
croyait y toucher pour y éteindre l'ardeur de sa soif.

Il fit un court voyage de visite épiscopale, il versa dans un endroit
dangereux, personne ne fut blessé, mais il vit tout le péril, et eut
dans sa faible machine toute la commotion de cet accident. Il arriva
incommodé à Cambrai, la fièvre survint, et les accidents tellement coup
sur coup qu'il n'y eut plus de remède\,; mais sa tête fut toujours libre
et saine. Il mourut à Cambrai le 7 janvier de cette année, au milieu des
regrets intérieurs, et à la porte du comble de ses désirs. Il savait
l'état tombant du roi, il savait ce qui le regardait après lui. Il était
déjà consulté du dedans et recourtisé du dehors, parce que le goût du
soleil levant avait déjà percé. Il était porté par le zèle
infatigablement actif de son petit troupeau, devenu la portion d'élite
du grand parti de la constitution par la haine des anciens ennemis de
l'archevêque de Cambrai, qui ne l'étaient pas moins de la doctrine des
jésuites qu'il s'agissait, de tolérée à grande peine qu'elle avait été
depuis son père Molina, de rendre triomphante, maîtresse et unique. Que
de puissants motifs de regretter la vie\,; et que la mort est amère dans
des circonstances si parfaites et si à souhait de tous côtés\,!
Toutefois il n'y parut pas. Soit amour de la réputation, qui fut
toujours un objet auquel il donna toute préférence, soit grandeur d'âme
qui méprise enfin ce qu'elle ne peut atteindre, soit dégoût du monde si
continuellement trompeur pour lui, et de sa figure qui passe et qui
allait lui échapper, soit piété ranimée par un long usage, et ranimée
peut-être par ces tristes mais puissantes considérations, il parut
insensible à tout ce qu'il quittait, et uniquement occupé de ce qu'il
allait trouver, avec une tranquillité, une paix, qui n'excluait que le
trouble, et qui embrassait la pénitence, le détachement, le soin unique
des choses spirituelles et de son diocèse, enfin avec une confiance qui
ne faisait que surnager à l'humilité et à la crainte.

Dans cet état il écrivit au roi une lettre, sur le spirituel de son
diocèse, qui ne disait pas un mot sur lui-même, qui n'avait rien que de
touchant et qui ne convint au lit de la mort à un grand évêque. La
sienne, à moins de soixante-cinq ans, munie des sacrements de l'Église,
au milieu des siens et de son clergé, put passer pour une grande leçon à
ceux qui survivaient, et pour laisser de grandes espérances de celui qui
était appelé. La consternation dans tous les Pays-Bas fut extrême. Il y
avait apprivoisé jusqu'aux armées ennemies, qui avaient autant et même
plus de soin de conserver ses biens que les nôtres. Leurs généraux et la
cour de Bruxelles se piquaient de le combler d'honnêtetés et des plus
grandes marques de considération, et les protestants pour le moins
autant que les catholiques. Les regrets furent donc sincères et
universels dans toute l'étendue des Pays-Bas. Ses amis, surtout son
petit troupeau, tombèrent dans l'abîme de l'affliction la plus amère. À
tout prendre, c'était un bel esprit et un grand homme. L'humanité rougit
pour lui de M\textsuperscript{me} Guyon, dans l'admiration de laquelle,
vraie ou feinte, il a toujours vécu, sans que ses mœurs aient jamais été
le moins du monde soupçonnées, et est mort après en avoir été le martyr,
sans qu'il ait été jamais possible de l'en séparer. Malgré la fausseté
notoire de toutes ses prophéties, elle fut toujours le centre où tout
aboutit dans ce petit troupeau, et l'oracle suivant lequel Fénelon vécut
et conduisit les autres.

Si je me suis un peu étendu sur ce personnage, la singularité de ses
talents, de sa vie, de ses diverses fortunes, la figure et le bruit
qu'il a faits dans le monde, m'ont entraîné, persuadé aussi que je ne
devais pas moins au feu duc de Beauvilliers pour un ami et un maître qui
lui fut si cher, et pour montrer que ce n'était pas merveille qu'il en
fût aussi enchanté, lui qui avec sa candeur n'y vit jamais que la piété
la plus sublime, et qui n'y soupçonna pas même l'ambition. Tout était si
exactement compassé chez M. de Cambrai qu'il mourut sans devoir un sou
et sans nul argent.

Un prélat plus heureux pour le monde, mais qui n'a voulu rendre que soi
heureux, jeta en ce temps-ci le premier fondement d'un règne qui a
étonné l'Europe, et qui en même temps est devenu le plus grand et le
plus solide malheur de la France. Je parle du trop fameux Fleury, qui a
rendu à Dieu depuis plus de deux ans les comptes de sa longue vie et de
sa toute puissante et funeste administration, dont il n'est pas temps de
parler. On a vu ses plus qu'obscurs commencements, ses progrès par cause
plus que louche, avec quels efforts et combien tard il devint évêque de
Fréjus, et la prédiction du roi au cardinal de Noailles, qui lui arracha
cet évêché malgré lui. Il y languissait loin de la cour et du grand
monde, où il n'osait venir que rarement. On a vu aussi comment il
tâchait de s'en dédommager en Provence et en Languedoc\,; l'étrange
conduite qu'il eut, pour un évêque français, lorsque M. de Savoie vint à
Fréjus pour l'expédition de Toulon\,; la juste colère du roi, et l'art
et la hardiesse que Torcy employa pour lui parer les plus grandes
marques d'indignation\,; mais l'ambition ne se rebute d'aucun obstacle.
Il avait toute sa vie été courtisan du maréchal de Villeroy. Il voyait
M\textsuperscript{me} de Dangeau et M\textsuperscript{me} de Lévi dans
l'intimité de M\textsuperscript{me} de Maintenon et dans toutes les
parties intérieures du roi. Il avait toujours cultivé Dangeau et sa
femme, où la bonne compagnie de la cour était souvent, et qui étaient
amis intimes du maréchal de Villeroy. Il s'initia auprès de
M\textsuperscript{me} de Lévi, et la subjugua par ses manières son
liant, son langage. À la faveur suprême où il vit le maréchal de
Villeroy auprès du roi, ramené, puis porté par M\textsuperscript{me} de
Maintenon sans cesse, il ne douta pas qu'il ne fût dans les dispositions
du roi, surtout depuis qu'il le vit successeur des places du duc de
Beauvilliers dans le conseil. Il avait toujours courtisé M. du Maine\,;
et de tout cela, il conclut que, marchant par ces deux dames, il
pourrait se faire nommer précepteur. Toutes deux étaient parfaitement à
lui\,; M\textsuperscript{me} de Dangeau pouvait beaucoup sur le maréchal
de Villeroy. Celui-ci et M. du Maine étaient dans les mesures les plus
intimes, dont M\textsuperscript{me} de Maintenon était le lien. Les
jésuites le connaissoient trop pour s'y fier\,; et c'est ce qui
détermina sa fortune.

M\textsuperscript{me} de Maintenon les haïssait, et on en a vu ailleurs
les raisons. Le maréchal de Villeroy ne les aimait pas intérieurement
plus qu'elle. M. du Maine en savait trop pour vouloir un précepteur de
leur main, conduit, instruit et soutenu par eux. Les deux dames
rompirent la glace auprès de M\textsuperscript{me} de Maintenon, elles
furent bien reçues. M\textsuperscript{me} de Dangeau parla au maréchal
de Villeroy, qui devint aisément favorable à un homme qu'il avait
protégé toute sa vie jusqu'à l'avoir quelquefois logé chez lui. Il s'en
ouvrit à M. du Maine, qui, n'ayant rien contre Fleury, et, voyant le
goût de M\textsuperscript{me} de Maintenon, se rendit aisément à le
porter. Ces mesures prises, Fleury comprit qu'il fallait ôter tout
prétexte aux refus en quittant un évêché situé à l'extrémité du royaume.
Sur ces espérances, il demanda à s'en défaire sous prétexte de sa santé.
Le P. Tellier, tout habile et prévoyant qu'il fût, n'en sentit pas le
piège. La démarche lui parut indifférente, c'était un évêché à remplir
d'une de ses créatures, il ne songea qu'à en être quitte à bon marché,
en ne donnant à Fleury qu'une légère abbaye. Celle de Tournus vaqua
bientôt après\,; elle lui fut offerte, et Fleury l'accepta sans
marchander. En attendant, pressé de pouvoir veiller de près au grand
objet qui lui faisait quitter Fréjus, il fit un mandement d'adieu à ses
diocésains, dont le tour ne fut pas fort approuvé\,; le démon en sut
profiter.

Fleury, dont la science, les mœurs ni la religion n'avait jamais fait le
capital de sa vie, avait toujours évité les questions de doctrine. Peu
aimé des jésuites et lié avec la meilleure compagnie, il ne s'était pas
contraint de blâmer l'inquisition et la tyrannie qui s'exerçait sur le
jansénisme, et avait toujours laissé son diocèse en paix. L'idée d'être
précepteur le fit changer de conduite\,; il voulut ranger les écueils,
et aller au-devant de tout en matière si délicate et si sûrement
exclusive, tellement que les derniers six mois de son épiscopat à Fréjus
ne furent employés qu'à la recherche de la doctrine, des livres, des
confesseurs, et à tourmenter le peu de religieuses de son diocèse. Comme
il voulait du bruit, il en fit plus que de mal\,; mais ce bruit, qui
entrait si bien dans ses vues, et que ses amis surent faire valoir à la
cour, retentit jusque dans les Pays-Bas et dans la retraite du fameux P.
Quesnel. Il venait d'achever son septième mémoire pour servir à l'examen
de la constitution, qui n'a été imprimé qu'en 1716\footnote{Voy., dans
  les Pièces, l'extrait du P. Quesnel sur ce prélat. (Note de
  Saint-Simon.)}, et il travaillait à la préface lorsque, irrité du
nouveau personnage de persécuteur que Fleury venait de prendre, il reçut
le mandement de ses adieux à ses diocésains. Il ne put résister au désir
de châtier le nouveau zèle de Fleury par le ridicule de cette pièce,
qu'il sut enchâsser dans sa préface avec l'ironie la plus amère, la plus
méprisante, et qui en effet mit en pièces ce beau mandement. \emph{Inde
iræ}. Fleury, avec son air doux, riant, modeste, était l'homme le plus
superbe en dedans et le plus implacable que j'aie jamais connu. Il ne le
pardonna pas au P. Quesnel\,; et c'est la cause unique qui a produit en
Fleury cette fureur jusqu'à lui inouïe, et qui s'est portée sans cesse
aux derniers excès de cruauté et de tyrannie contre les jansénistes et
les anticonstitutionnaires, et les infernales mesures pour les perpétuer
après sa mort, aux dépens de l'Église et de l'État.

À propos de la constitution, un trait du P. Tellier et de ses créatures,
arrivé en ce même temps-ci, ne sera pas déplacé en ce lieu, et mérite
d'y tenir place. La Parisière, homme de la condition la plus obscure, et
dont le savoir ne consistait qu'en manèges et en intrigues, avait
succédé au savant et célèbre Fléchier en l'évêché de Nîmes. C'étaient là
les gens d'élite du P. Tellier. Instruit par lui, il fit sourdement le
zélé contre la constitution, refusa même de l'accepter\,; et par cette
démarche s'initia aux états de Languedoc, parmi les évêques. Il y lit si
bien son personnage qu'étant député pour le clergé par les états, il
reçut défense de venir à la cour, et les états ordre de nommer un autre
évêque. Cette éclatante disgrâce acheva de lui ouvrir tous les cœurs
opposés à la constitution. Il sut donc le nombre des évêques, des curés,
des supérieurs séculiers et réguliers, les prêtres, les moines, les
personnes principales séculières qui ne voulaient point de la
constitution, leur force en capacité, en zèle, en amis, en soutiens, en
un mot tout le secret de gens opprimés qui se concertent. Ce nouveau
Zopyre mit en mémoires toutes ses connaissances et les envoya au P.
Tellier. Quand il se crut en état de n'avoir plus rien à apprendre, il
monta tout à coup en chaire dans sa cathédrale, fit un sermon foudroyant
contre les réfractaires aux ordres du roi et du pape, reçut là même la
constitution de la manière la plus précise et la plus absolue\,; et peu
de jours après montra un ordre du roi pour lui rendre la députation des
états, dont il apporta les cahiers à Versailles avec un front d'airain.
Ce fut lui qui dans la suite se licencia de donner l'exemple de
consulter les évêques et les universités d'Espagne, de Portugal et
d'Italie, sur la constitution, qui n'avaient garde de n'y pas adhérer,
dans la frayeur de l'inquisition, et dans l'opinion ultramontaine de
l'infaillibilité du pape. Ce malheureux, abhorré partout et dans son
diocèse, y mourut banqueroutier, et en homme sans foi ni loi, quelques
années après.

L'abbé de Lyonne, fils du célèbre ministre d'État, mourut aussi en ce
mois de janvier. Ses mœurs, son jeu, sa conduite, l'avaient éloigné de
l'épiscopat et de la compagnie des honnêtes gens. Il était extrêmement
riche en bénéfices qui lui donnaient de grandes collations\footnote{Droit
  de conférer des bénéfices ecclésiastiques.}. L'abus qu'il en faisait
engagea sa famille à lui donner quelqu'un qui y veillât avec autorité.
Il fallut avoir recours à celle du roi, par conséquent aux jésuites,
puisqu'il s'agissait de biens et de collations ecclésiastiques. Ils
découvrirent un certain Henriot de la plus basse lie du peuple, décrié
pour ses mœurs et pour ses friponneries. Ce fut leur homme\,; ils le
firent tuteur de l'abbé de Lyonne, chez lequel il s'enrichit par la
vente de toutes ses collations. Ce nonobstant, Henriot, valet à tout
faire, parut un si grand sujet au P. Tellier, et si à sa main, qu'il le
chargea dans Paris de plusieurs commissions extraordinaires dans des
couvents de filles, appuyé par Pontchartrain, qui se délectait de mal
faire, et qui faisait bassement sa cour au P. Tellier. Tous deux firent
l'impossible auprès du roi pour le faire évêque, sans que jamais le roi,
qui était instruit sur ce compagnon, les voulût écouter. Les chefs de la
constitution se firent un capital de le faire évêque dans la régence, et
réussirent enfin à le faire évêque, ou pour mieux dire, loup de
Boulogne, à la mort de M. de Langle. Rien en tout ne pouvait être plus
parfaitement dissemblable. Henriot, connu et par conséquent parfaitement
méprisé et détesté, y vécut et y mourut en loup. Ce fut un des premiers
évêques que le cardinal Fleury voulut sacrer. Il en fit la cérémonie à
Fontainebleau dans la paroisse, au scandale universel. Pour revenir à
l'abbé de Lyonne, il passa toute sa vie dans la dernière obscurité. Il
logeait à Paris dans son beau prieuré de Saint-Martin des Champs, où
tous les matins, les vingt dernières années de sa vie, il buvait, depuis
cinq heures du matin jusqu'à midi, vingt et quelquefois vingt-deux
pintes d'eau de la Seine, sans se pouvoir passer à moins, outre ce qu'il
en avalait encore à son dîner. Il n'était pas fort vieux, et ne laissait
pas d'avoir de l'esprit et des lettres.

On a vu en son lieu, en parlant du vieux duc de Gesvres, et de tout ce
qu'il fit auprès du roi contre son fils revenant de Rome, pour
l'empêcher d'être archevêque de Bourges, quel était ce prélat, et
combien il était en faveur auprès d'Innocent XI, dont il était camérier
d'honneur, et en espérance de la pourpre romaine, lorsque l'éclat arrivé
entre le roi et le pape, pour la franchise du quartier des ambassadeurs,
fit en 1688 rappeler tous les François de Rome\,; et que l'archevêché de
Bourges lui fut donné en récompense des espérances qu'il perdait, contre
l'usage constamment observé jusqu'alors de ne donner les archevêchés
qu'à des évêques. Cet abbé, devenu ainsi archevêque de plein saut, ne
perdit jamais de vue le chapeau qu'il avait tant espéré. Il avait
conservé à Rome des amis et un commerce secret. Il avait réussi à
s'acquérir l'amitié de Croissy, et de Torcy, secrétaire d'État des
affaires étrangères. Il avait accoutumé le roi à trouver bon qu'il fît
de son mieux pour devenir cardinal. La nomination du roi Jacques qu'il
avait eue d'abord n'ayant pas réussi, il trouva moyen de se faire donner
celle de Pologne par le roi Stanislas, dans le court intervalle de son
règne\,; et il fut encore assez habile pour obtenir la même grâce de
l'électeur de Saxe, après qu'il fut remonté sur ce trône. Ce chapeau
faisait toute l'occupation et la vie de l'archevêque de Bourges. On
verra qu'il attendit encore des années qui lui parurent bien longues, et
pendant lesquelles il travailla sans cesse à son objet, auquel à la fin
il arriva.

Le roi, contre sa coutume de ne donner les bénéfices que les jours qu'il
avait communié le matin, le samedi saint, la veille de la Pentecôte, de
l'Assomption, de la Toussaint et de Noël, en donna à la mi-janvier de
cette année, mais seulement au fils plus que disgracié de corps, de
mœurs et d'esprit, de son ministre des finances, et à trois favoris de
la constitution. L'abbé Desmarets, qui avait déjà une grosse abbaye et
d'autres bénéfices, eut l'abbaye de Saint-Antoine aux Bois\,; et l'abbé
de Montbazon la riche abbaye du Gard, près de Metz, de plus de cinquante
mille livres de rente. Le cardinal de Rohan s'était enfin trop
entièrement vendu au P. Tellier, et ce père avait encore trop besoin de
lui, pour ne se le pas assurer de plus en plus. Languet, de la plus
nouvelle et petite robe du parlement de Dijon, qui était aumônier de
M\textsuperscript{me} la duchesse de Bourgogne, et que je voyais sans
cesse dans les antichambres des dames du palais, eut l'évêché de
Soissons, où il fit bientôt après parler de son zèle pour la
constitution. Le frère d'Argenson, si nécessaire dans Paris, et à
l'oreille du roi, aux jésuites, passa du triste évêché de Dol à
l'archevêché d'Embrun, vacant par la mort de Brûlart-Genlis, le plus
ancien des archevêques\,; et Dol fut donné au fils de Sourches qui
pourrissait aumônier du roi en grand mépris.

La duchesse de Nevers mourut en ce temps-ci. On a assez fait connaître
quelle elle était, et le duc de Nevers, son mari, pour n'avoir ici
besoin que d'une addition légère. Peu de femmes l'avaient surpassée en
beauté. La sienne était de toutes les sortes, avec une singularité qui
charmait. On ne se pouvait lasser de lui entendre raconter les aventures
de ses voyages d'Italie. M. le Prince avait été extrêmement amoureux
d'elle. Il voulut lui donner une fête sous un autre prétexte, et c'était
l'homme du monde qui s'y entendait le mieux. Mais comme il n'était pas
moins malin qu'amoureux, il imagina d'engager M. de Nevers de faire les
vers de la pièce qui devait être le principal divertissement de la fête,
et dont toute la galanterie était pour M\textsuperscript{me} de Nevers.
Il le cajola si bien, que M. de Nevers lui promit de faire ces vers, et
il y réussit au delà des espérances de M. le Prince. Il prépara donc la
fête, dans le double plaisir de plaire à sa dame et de se moquer du
mari. Celui-ci tout jaloux, tout Italien, tout plein d'esprit qu'il fût,
n'avait pas conçu le plus léger soupçon de cette fête, quoiqu'il
n'ignorât pas l'amour de M. le Prince. Quatre ou cinq jours avant celui
de la fête, il découvrit de quoi il s'agissait, il n'en dit mot, et
partit le lendemain pour Rome avec sa femme, où il demeura longtemps, et
à son tour se moqua bien de M. le Prince. M\textsuperscript{me} de
Nevers à plus de soixante ans était encore parfaitement belle,
lorsqu'elle mourut d'une maladie fort courte. Depuis qu'elle était
veuve, elle était devenue fort avare, et ne quittait plus la duchesse du
Maine.

\hypertarget{note-i.-morceau-inuxe9dit-de-saint-simon-relatif-uxe0-lacaduxe9mie-franuxe7aise.}{%
\chapter{NOTE I. MORCEAU INÉDIT DE SAINT-SIMON RELATIF À L'ACADÉMIE
FRANÇAISE.}\label{note-i.-morceau-inuxe9dit-de-saint-simon-relatif-uxe0-lacaduxe9mie-franuxe7aise.}}

La plupart des notes de Saint-Simon sur le \emph{Journal de Dangeau} se
retrouvent dans ses Mémoires. En voici une qui fait exception. En
parlant des faveurs dont Villars fut comblé à son retour, en 1714 (Voy.
p.~51 et suiv. de ce volume), Saint-Simon ne dit rien de son élection à
l'Académie française (17 mai 1714). Mais dans ses notes sur le Journal
de Dangeau qui mentionne cette nouvelle, on trouve le passage
suivant\footnote{Je dois ce morceau inédit de Saint-Simon à l'obligeance
  de M. Amédée Lefèvre-Pontalis, auteur d'un excellent \emph{Discours
  sur la Vie et les Mémoires de Saint-Simon}, qui a été couronné par
  l'Académie française.}\,:

«\,L'Académie françoise se perdit peu à peu par sa vanité et par sa
complaisance. Elle seroit demeurée en lustre si elle s'en étoit tenue à
son institution\,; la complaisance commença à la gâter\,: des personnes
puissantes par leur élévation ou par leur crédit protégèrent des sujets
qui ne pouvoient lui être utiles, conséquemment ne pouvoient lui faire
honneur. Ces protections s'étendirent après jusque sur leurs domestiques
par orgueil, et ces domestiques qui n'avoient souvent pas d'autre mérite
littéraire furent admis. De là cela se tourna en espèce de droit que
l'usage autorisa, et qui remplit étrangement l'Académie. Pour essayer de
se relever au moins par la qualité de ses membres, elle élut des gens
considérables, mais qui ne l'étoient que par leur naissance ou leurs
emplois, sans lesquels les lettres ne les auroient jamais admis dans une
société littéraire, et ces personnes eurent la petitesse de s'imaginer
que la qualité d'académiciens les rendoit académiques. De l'un à l'autre
cette mode s'introduisit, et l'Académie s'en applaudit par la vanité de
faire subir à ces hommes distingués une égalité littéraire en places, en
sièges, en voix, en emplois de directeur et de chancelier par tour ou
par élection\,; et tel qui eût été à peine assis chez un autre, se
croyoit quelque chose de grand par ce mélange avec lui au dedans de
l'Académie, et ne sentoit pas que cette distinction intérieure et
momentanée ne différoit guère de celle des rois de théâtre et des héros
d'opéra.

«\,Que pour honorer l'Académie, la distinction des personnes ne fût pas
un obstacle à les admettre, quand d'ailleurs ils avoient de quoi payer
de leurs personnes par leur savoir et par leur bon goût et s'en tenir
là, c'étoit chose raisonnable\,; on avoit commencé de la sorte\,; cela
honoroit qui que ce fût\,; l'égalité littéraire contribuoit à
l'émulation et à l'union des divers membres dans un lieu où l'esprit et
les lettres seules étoient considérées, et ou tout autre éclat ne devoit
pas être compté. Tant que l'Académie n'a été ouverte qu'à des prélats et
à des magistrats en petit nombre, distingués en effet par les lettres,
et à des gens de qualité, même de dignité, s'il s'en trouvoit de tels,
elle leur a donné et en a reçu un éclat réciproque\,; mais depuis que,
de l'un à l'autre, par mode et par succession de temps, les grandes
places et celles de domestiques sans autre titre s'y sont réunies, la
mésalliance est tombée dans le ridicule, et les lettres dans le néant,
par le très petit nombre de gens de lettres qui y ont eu place et qui se
sont découragés par les confrères qui leur ont été donnés, parfaitement
inutiles aux lettres et bons seulement à y cabaler des élections. On
admirera la fatuité de plusieurs gens considérables qui s'y laissèrent
entraîner, et celle de l'Académie à les élire.\,»

\hypertarget{note-ii.-lettre-de-richelieu-mourant-uxe0-mazarin.}{%
\chapter{NOTE II. LETTRE DE RICHELIEU MOURANT À
MAZARIN.}\label{note-ii.-lettre-de-richelieu-mourant-uxe0-mazarin.}}

Le cardinal Mazarin et sa famille sont traités avec une grande sévérité
dans plusieurs passages de Saint-Simon, notamment p.~105 et suivantes de
ce volume. Je n'ai pas l'intention de faire l'apologie du cardinal ni de
ses nièces. Je me bornerai pour sa famille à renvoyer le lecteur au
curieux ouvrage de M. Am. Renée sur les Nièces de Mazarin\footnote{Pages
  88 et suivantes de la première édition.}. Quant au cardinal, il ne
faut pas croire qu'il dut uniquement son élévation à l'heureux caprice
d'une reine. On oublie trop que les services rendus à l'État l'avaient
signalé depuis longtemps et que Richelieu, sur son lit de mort, l'avait
désigné pour son successeur. Voici la lettre inédite, par laquelle le
cardinal mourant lègue à Mazarin le soin de continuer son œuvre.

«\,Monsieur,

«\,La providence de Dieu, qui prescrit des limites à la vie de tous les
hommes, m'ayant fait sentir en cette dernière maladie que mes jours
étoient comptés\,; qu'il a tiré de moi tous les services que je pouvois
rendre au monde, je ne le quitte qu'avec regret de n'avoir pas achevé
les grandes choses que j'avois entreprises pour la gloire de mon roi et
de ma patrie. Mais, parce qu'il nous faut soumettre aux lois qu'il nous
impose, je bénis cette sagesse infinie et je reçois l'arrêt de ma mort
avec autant de constance que j'ai de joie de voir le soin qu'elle prend
de m'en consoler. Comme le zèle que j'ai toujours eu pour l'avantage de
la France a fait mes plus solides contentements, j'ai un extrême
déplaisir de la laisser sans l'avoir affermie par une paix générale.
Mais, puisque les grands services que vous avez déjà rendus à l'État me
font assez connoître que vous serez capable d'exécuter ce que j'avois
commencé, je vous remets mon ouvrage entre les mains, sous l'aveu de
notre bon maître, pour le conduire à sa perfection, et je suis ravi
qu'il recouvre en votre personne plus qu'il ne sauroit perdre en la
mienne. Ne pouvant, sans faire tort à votre vertu, vous recommander
autre chose, je vous supplierai d'employer les prières de l'Église pour
celui qui meurt,

«\,Monsieur,

«\,Votre très humble serviteur,

ARMAND, cardinal-duc de richelieu.\,»

Cette lettre, qui fait le plus grand honneur aux deux cardinaux, est
conservée dans le dépôt des manuscrits de la Bibliothèque impériale.

\hypertarget{note-iii.-terres-distribuuxe9es-aux-leudes-francs-apruxe8s-la-conquuxeate.}{%
\chapter{NOTE III. TERRES DISTRIBUÉES AUX LEUDES FRANCS APRÈS LA
CONQUÊTE.}\label{note-iii.-terres-distribuuxe9es-aux-leudes-francs-apruxe8s-la-conquuxeate.}}

Saint-Simon dit que les terres distribuées aux leudes ou compagnons des
rois après la conquête s'appelèrent \emph{fiefs}. L'assertion n'est pas
entièrement exacte. Ces terres portèrent primitivement le nom de
\emph{bénéfices} (\emph{beneficia}), ou terres accordées en récompense
des services. Dans l'origine, elles ne donnaient pas à ceux qui les
obtenaient les droits de souveraineté, c'est-à-dire le droit de battre
monnaie, de lever des impôts, de rendre la justice et de faire la
guerre. Les rois pouvaient même enlever ces terres aux leudes qui ne
remplissaient pas avec exactitude les obligations qui leur étaient
imposées. Ce fut seulement par le traité d'Andelot (587) et surtout par
les usurpations si fréquentes dans ces temps d'anarchie que les leudes
rendirent les bénéfices inamovibles et héréditaires. Quant aux fiefs et
au régime féodal, il faut arriver au IXe siècle pour en trouver
l'organisation solidement établie et conférant les droits régaliens qui
ont été énumérés plus haut. La plupart des historiens antérieurs à notre
siècle ont confondu les \emph{bénéfices} et les \emph{fiefs}, comme le
fait Saint-Simon dans ce passage. On doit surtout à M. Guizot d'avoir
relevé cette erreur dans ses \emph{Essais sur l'histoire de France} et
dans son Cours de l'histoire de la civilisation en France, il a
nettement marqué la distinction entre les bénéfices et les fiefs, tout
en montrant que la distribution des bénéfices et les usurpations des
leudes ont conduit peu à peu au régime féodal.

\hypertarget{note-iv.-assembluxe9es-des-francs-dites-champs-de-mars-et-champs-de-mai.}{%
\chapter{NOTE IV. ASSEMBLÉES DES FRANCS, DITES CHAMPS DE MARS ET CHAMPS
DE
MAI.}\label{note-iv.-assembluxe9es-des-francs-dites-champs-de-mars-et-champs-de-mai.}}

Les assemblées des Francs n'étaient pas toujours divisées \emph{en deux
chambres}, pour employer les termes mêmes de Saint-Simon. Il serait
difficile de retrouver cette division dans les \emph{champs de mars} des
Mérovingiens\,; mais, sous les Carlovingiens, les usages rappelés par
Saint-Simon furent habituellement observés, comme le prouve un document
du IXe siècle, conservé dans une lettre écrite en 882 par Hincmar,
archevêque de Reims\footnote{Cette lettre forme un véritable traité sous
  le titre de \emph{De ordine palatii}. Elle reproduit un document plus
  ancien composé par Adalhard, abbé de Corbie, parent et conseiller de
  Charlemagne.}. Voici la traduction qu'en a donnée M. Guizot\,:

«\,C'était l'usage du temps de Charlemagne de tenir chaque année deux
assemblées\,: dans l'une et dans l'autre, on soumettait à l'examen ou à
la délibération des grands les articles de loi nommés \emph{capitula},
que le roi lui-même avait rédigés par l'inspiration de Dieu, ou dont la
nécessité lui avait été manifestée dans l'intervalle des réunions. Après
avoir reçu ces communications, ils en délibéraient un, deux ou trois
jours, ou plus, selon l'importance des affaires. Des messagers du
palais, allant et venant, recevaient leurs questions et rapportaient
leurs réponses, et aucun étranger n'approchait du lieu de leur réunion,
jusqu'à ce que le résultat de leurs délibérations pût être mis sous les
yeux du grand prince, qui, alors, avec la sagesse qu'il avait reçue de
Dieu, adoptait une résolution à laquelle tous obéissaient. Les choses se
passaient ainsi pour un, deux capitulaires, ou un plus grand nombre,
jusqu'à ce que, avec l'aide de Dieu, toutes les nécessités du temps
eussent été réglées.

«\,Pendant que ces affaires se traitaient de la sorte hors de la
présence du roi, le prince lui-même, au milieu de la multitude venue à
l'assemblée générale, était occupé à recevoir les présents, saluant les
hommes les plus considérables, s'entretenant avec ceux qu'il voyait
rarement, témoignant aux plus âges un intérêt affectueux, s'égayant avec
les plus jeunes, et faisant ces choses et autres semblables pour les
ecclésiastiques comme pour les séculiers. Cependant, si ceux qui
délibéraient sur les matières soumises à leur examen en manifestaient le
désir, le roi se rendait auprès d'eux, y restait aussi longtemps qu'ils
le voulaient, et là ils lui rapportaient avec une entière familiarité ce
qu'ils pensaient de toutes choses, et quelles étaient les discussions
amicales qui s'étaient élevées entre eux. Je ne dois pas oublier de dire
que, \emph{si le temps était beau, tout cela se passait en plein
air}\footnote{J'ai souligné les passages qui confirment le récit de
  Saint-Simon.}**, sinon dans plusieurs bâtiments distincts, \emph{où
ceux qui avaient à délibérer sur les propositions du roi étaient séparés
de la multitude des personnes venues à l'assemblée\,;} et alors les
hommes les moins considérables ne pouvaient entrer.

«\,Les lieux destinés à la réunion des seigneurs \emph{étaient divisés
en deux parties, de telle sorte que les évêques, les abbés et les clercs
élevés en dignité pussent se réunir sans aucun mélange de laïques}. De
même les comtes et les autres principaux de l'État se séparaient, dès le
matin, du reste de la multitude, jusqu'à ce que, le roi présent ou
absent, ils fussent tous réunis\,: et alors les seigneurs ci-dessus
désignés, les clercs de leur côté, les laïques du leur, se rendaient
dans la salle qui leur était assignée, et où on leur avait fait
honorablement préparer des sièges. \emph{Lorsque les seigneurs laïques
et ecclésiastiques étaient ainsi séparés de la multitude, il demeurait
en leur pouvoir de siéger ensemble ou séparément}, selon la nature des
questions qu'ils avaient à traiter, ecclésiastiques, séculières ou
mixtes. De même, s'ils voulaient faire venir quelqu'un, soit pour
demander des aliments, soit pour faire quelque question, et le renvoyer
après en avoir reçu ce dont ils avaient besoin, ils en étaient les
maîtres. Ainsi se passait l'examen des affaires que le roi proposait à
leurs délibérations.\,»

\hypertarget{note-v.-lits-de-justice-68-origine-du-nom.-cuxe9ruxe9monial-des-lits-de-justice.-toutes-les-suxe9ances-royales-en-parlement-nuxe9taient-pas-lits-de-justice.-suxe9ance-royale-pour-la-condamnation-du-prince-de-conduxe9-en-1654.}{%
\chapter{NOTE V. LITS DE JUSTICE {[}68{]} --- ORIGINE DU NOM. ---
CÉRÉMONIAL DES LITS DE JUSTICE. --- TOUTES LES SÉANCES ROYALES EN
PARLEMENT N'ÉTAIENT PAS LITS DE JUSTICE. --- SÉANCE ROYALE POUR LA
CONDAMNATION DU PRINCE DE CONDÉ EN
1654.}\label{note-v.-lits-de-justice-68-origine-du-nom.-cuxe9ruxe9monial-des-lits-de-justice.-toutes-les-suxe9ances-royales-en-parlement-nuxe9taient-pas-lits-de-justice.-suxe9ance-royale-pour-la-condamnation-du-prince-de-conduxe9-en-1654.}}

Les lits de justice, dont il est souvent question dans l'histoire de
l'ancienne monarchie, étaient des séances solennelles du parlement, où
le roi siégeait en personne entouré des princes du sang et des grands
officiers de la couronne. Les ducs et pairs y étaient convoqués et
devaient y prendre séance en leur rang, d'après l'ordre de leur
réception. Ces cérémonies tiraient leur nom de ce que le roi siégeait
sur une espèce de lit formé de coussins. Il en est déjà question dans
une ordonnance de Philippe de Valois en date du 11 mars 1344 (1345).
L'article 14 dit que, dans ces cérémonies, «\,nul ne doit venir siéger
auprès du lit du roi, les chambellans exceptés {[}69{]} .\,» C'est donc
à tort que certains historiens ont regardé comme le premier lit de
justice celui que tint Charles V en 1369 pour juger le prince de Galles,
duc de Guyenne, qui était accusé de félonie.

Le cérémonial des lits de justice était rigoureusement déterminé. Dans
le cas où le roi se rendait au parlement pour tenir un lit de justice,
un maître des cérémonies avertissait l'assemblée dès que le roi était
arrivé à la Sainte-Chapelle. Aussitôt quatre présidents à mortier avec
six conseillers laïques et deux conseillers clercs allaient le recevoir
et le saluer au nom du parlement. Ils le conduisaient ensuite à la
grand'chambre, les présidents marchant aux côtés du roi, les conseillers
derrière lui et le premier huissier entre les deux massiers du roi. Le
roi s'avançait précédé des gardes dont les trompettes sonnaient et les
tambours battaient jusque dans la grand'chambre. Le lit de justice du
roi surmonté d'un dais était placé dans un des angles de la
grand'chambre. Les grands officiers avaient leur place marquée\,: le
grand chambellan aux pieds du roi\,; à droite, sur un tabouret, le grand
écuyer portant suspendue au cou l'épée de parade du roi\,; à gauche, se
tenaient debout les quatre capitaines des gardes et le capitaine des
Cent-Suisses. Le chancelier siégeait au-dessous du roi dans le même
angle\,; il avait une chaire à bras que recouvrait le tapis de velours
violet semé de fleurs de lis d'or, qui servait de drap de pied au roi.
Le grand maître des cérémonies et un maître ordinaire prenaient place
sur des tabourets devant la chaire du chancelier. Le prévôt de Paris, un
bâton blanc à la main, se tenait sur un petit degré par lequel on
descendait dans le parquet. Dans le même parquet, deux huissiers du roi,
leurs masses d'armes à la main, et six hérauts d'armes étaient placés en
avant du lit de justice.

Les hauts sièges à la droite du roi étaient occupés par les princes du
sang et les pairs laïques\,; à gauche, par les pairs ecclésiastiques et
les maréchaux venus avec le roi. Le banc ordinaire des présidents à
mortier était rempli par le premier président et les présidents à
mortier revêtus de robes rouges et de leurs épitoges d'hermine. Sur les
autres bancs siégeaient les conseillers d'honneur, les quatre maîtres
des requêtes qui avaient séance au parlement, enfin les conseillers de
la grand'chambre, des chambres des enquêtes et des requêtes, tous en
robe rouge. Il y avait des bancs réservés pour les conseillers d'État et
les maîtres des requêtes qui accompagnaient le chancelier et qui étaient
revêtus de robes de satin noir, ainsi que pour les quatre secrétaires
d'État, les chevaliers des ordres du roi, les gouverneurs et lieutenants
généraux des provinces, les baillis d'épée, etc.

Lorsque le roi était assis et couvert et que toute l'assemblée avait
pris place, le roi ôtant et remettant immédiatement son chapeau, donnait
la parole au chancelier pour exposer l'objet de la séance. Le chancelier
montait alors vers le roi, s'agenouillait devant lui, et, après avoir
pris ses ordres, retournait à sa place, où assis et couvert il
prononçait une harangue d'apparat. Son discours fini, le premier
président et les présidents se levaient, mettaient un genou en terre
devant le roi, et, après qu'ils s'étaient relevés, le premier président,
debout et découvert, ainsi que tous les présidents, prononçait un
discours en réponse à celui du chancelier. Il parlait au nom du
parlement, tandis que le chancelier avait parlé au nom du roi.

Après ces harangues, le chancelier remontait vers le lit de justice du
roi, et, un genou en terre, prenait de nouveau ses ordres\,; de retour à
sa place, il disait que la volonté du roi était que l'on donnât lecture
de ses édits. Sur son ordre, un greffier faisait cette lecture. Le
chancelier appelait ensuite les gens du roi pour qu'ils donnassent leurs
conclusions. Un des avocats généraux prononçait alors un réquisitoire,
dont la conclusion était toujours que la cour devait ordonner
l'enregistrement des édits. Il arriva cependant que plusieurs avocats
généraux, parmi lesquels on remarque Omer Talon et Jérôme Bignon,
profitèrent de ces circonstances solennelles pour adresser au souverain
de sages remontrances.

Après le discours de l'avocat général, le chancelier recueillait les
voix, mais seulement pour la forme. Il montait pour la troisième fois au
lit de justice du roi et lui demandait son avis\,; il s'adressait
ensuite aux princes, pairs laïques et ecclésiastiques, maréchaux de
France, présidents du parlement, conseillers d'État, maîtres des
requêtes, conseillers au parlement, qui tous opinaient à voix basse et
pour la forme. Ce simulacre de vote terminé, le chancelier allait pour
la quatrième fois demander les ordres du roi, et, de retour à sa place,
il prononçait la formule d'enregistrement ainsi conçue\,: \emph{Le roi,
séant en son lit de justice, a ordonné et ordonne que les présents édits
seront enregistrés, publiés et adressés à tous les parlements et juges
du royaume}. La formule, dictée au greffier par le chancelier, au nom du
roi, se terminait ainsi\,: \emph{Fait en parlement, le roi y séant en
son lit de justice}. Le roi sortait ensuite du parlement entouré de la
même pompe et du même cortège qu'à son entrée.

Les lits de justice étaient considérés ordinairement comme des coups
d'État. Le parlement se réunissait quelquefois le lendemain du lit de
justice pour protester contre un enregistrement forcé, et de là
naissaient des conflits et des troubles. Telle fut, en 1648, l'occasion
de la Fronde.

La présence du roi au parlement ne suffisait pas pour qu'il y eût lit de
Justice. Le Journal d'Olivier d'Ormesson en fournit la preuve\,: à la
date du 2 décembre 1665, il mentionne la présence du roi au parlement,
sans que cette séance royale fût un véritable lit de justice. «\,Le roi,
dit-il, entra sans tambours, trompettes ni aucun bruit, à la différence
des lits de justice.\,» Il signale, à l'occasion du même événement, une
autre différence qui concerne le chancelier\,: «\,M. le chancelier,
dit-il, y vint, et l'on députa deux conseillers de la grand'chambre à
l'ordinaire pour le recevoir, sans qu'il eût des masses devant lui,
comme aux lits de justice.\,» André d'Ormesson retrace dans ses Mémoires
inédits une de ces séances royales qui n'étaient pas lits de justice. Il
s'agissait du procès criminel intenté au prince de Condé. L'auteur, qui
était conseiller d'État, entre dans tous les détails de la cérémonie,
dont il fut témoin oculaire\,:

«\,Cette journée (19 janvier 1654), je me trouvai chez M. le chancelier
{[}70{]} , sur les huit heures, en ayant été averti la veille par M.
Sainctot, maître des cérémonies. M. le chancelier me fit mettre au fond,
à côté de lui, pour donner place aux autres dans son carrosse. Étant
auprès de lui, il me dit que le duc d'Anjou {[}71{]} ne s'y trouveroit
point, n'étant pas en âge de juger, et que le roi n'en étoit capable que
par la loi du royaume qui le déclaroit majeur à treize ans\,; que les
capitaines des gardes ne seroient point auprès du roi, n'ayant point de
voix ni de séance au parlement\,; que le prévôt de Paris n'y seroit
point non plus, et que le duc de Joyeuse n'y entreroit que comme duc de
Joyeuse et ne seroit point aux pieds du roi comme grand chambellan\,;
que les gens du roi demeureroient présents pendant le procès, encore
qu'ils aient accoutumé de se retirer, après avoir donné leurs
conclusions par écrit\,; que les princes parents descendroient de leurs
places et demanderoient d'être excusés d'assister au procès, et que le
roi leur prononceroit qu'il trouvoit bon qu'ils y demeurassent.

«\,Étant arrivés en la Sainte-Chapelle et de là allant prendre nos
places, MM. Chevalier et Champron, conseillers au parlement, vinrent
au-devant de M. le chancelier. Il se mit au-dessus du premier président
et n'en bougea pendant la séance. Le roi, ayant pris sa place, étoit
accompagné, du côté des pairs laïques, à la main droite, des ducs de
Guise, de Joyeuse son frère, d'Épernon, d'Elbœuf, de Sully, de Candale,
et de quatre maréchaux de France, conseillers de la cour, qui prirent la
séance entre eux, non du jour qu'ils étoient maréchaux de France, mais
du jour qu'ils avoient été reçus conseillers de la cour au parlement,
comme M. le chancelier le leur avoit prononcé sur la difficulté qu'ils
lui en firent. Ainsi M. le maréchal de La Mothe-Houdancourt, le maréchal
de Grammont, le maréchal de L'Hôpital et le maréchal de Villeroy prirent
leurs places après les ducs et pairs. Du côté des pairs ecclésiastiques,
à main gauche, étoient assis M. d'Aumale, archevêque de Reims, duc et
pair de France, l'évêque de Beauvais (Chouart-Busenval), comte et pair,
l'évêque de Châlons (Viallard), comte et pair, l'évêque de Noyon
(Baradas), comte et pair. Au siège bas, au-dessous des ducs, le comte de
Brienne (Loménie) {[}72{]} , Bullion, sieur de Bonnelles, Le Fèvre
d'Ormesson {[}73{]} , Haligre et Morangis-Barrillon, conseillers d'État
reçus au parlement {[}74{]} . Tous les présidents de la cour étoient
présents, excepté le président de Maisons (Longueil), relégué à Conches
en Normandie, pour avoir suivi le parti des princes avec son frère
conseiller à la cour. Les présidents présents étoient MM. de Bellièvre,
premier président, de Nesmond, de Novion (Potier), de Mesmes (d'Irval),
Le Coigneux, Le Bailleul et Molé-Champlâtreux. Les maîtres des requêtes
présents étoient MM. Mangot, Laffemas, Le Lièvre et d'Orgeval-Lhuillier.

~

{\textsc{«\,La compagnie assise, M. du Bignon {[}75{]} , avocat général,
assisté de M. Fouquet, procureur général, et de M. Talon, aussi avocat
général, proposa au roi le sujet de cette assemblée, et parla contre la
désobéissance de M. le Prince {[}76{]} , et il sembloit à son discours
qu'il excitoit le roi à lui pardonner et à oublier toutes ses actions
passées, et à la fin donna ses conclusions à M. Doujat, rapporteur, par
écrit. M. le chancelier dit aux gens du roi qu'ils demeurassent dans
leurs places\,; dont la compagnie murmura, n'étant point de l'ordre qui
s'observe en telles occasions, et M. le chancelier, au retour, comme
j'étois encore près de lui, me dit qu'il ne le feroit plus. M. le
chancelier demanda l'avis à M. Chevalier, doyen du parlement, un des
rapporteurs, puis à M. Doujat, qui dit qu'il y avoit trois preuves
contre M. le Prince\,: la première, la notoriété de fait, la seconde les
lettres missives et les commissions signéesLouis De Bourbon, et puis les
témoins qui avoient déposé contre lui des actes d'hostilité. On avoit
lu, auparavant, les dépositions de cinq ou six témoins, quatre ou cinq
lettres du prince et ses commissions. Après que M. Doujat eut parlé,
toute la compagnie n'opina que du bonnet et fut d'avis des conclusions
qui étoient, \emph{que ledit prince seroit ajourné de comparoir en
personne, se mettre dans la Conciergerie et se représenter dans un
mois\,; qu'il seroit ajourné dans la ville de Péronne, à cri public, au
son de la trompette, et cependant que ses biens seroient saisis\,;
décret de prise de corps contre le président Viole, Lenet, Marchin
(Marsin), Persan et encore six ou sept autres seigneurs et capitaines\,;
leurs biens saisis}, etc.\,»}}

~

\hypertarget{note-vi.-louis-xiv-au-parlement-en-1655.}{%
\chapter{NOTE VI. LOUIS XIV AU PARLEMENT, EN
1655.}\label{note-vi.-louis-xiv-au-parlement-en-1655.}}

Saint-Simon rapporte que Louis XIV alla «\,en habit gris tenir son lit
de justice avec une houssine à la main, dont il menaça le parlement, en
lui parlant en termes répondant à ce geste.\,» Cette scène dramatique
s'est gravée profondément dans les esprits et est devenue un des lieux
communs de l'histoire traditionnelle. On y a ajouté un de ces mots à
effet qui ne sortent plus de la mémoire des peuples. Louis XIV, d'après
la tradition, aurait répondu au premier président qui lui parlait de
l'intérêt de l'État\,: «\,L'État, c'est moi\,» On place cette scène le
20 mars 1655.

Quel est sur ce point le récit des historiens contemporains\,? Les
principaux auteurs de Mémoires qui écrivaient à cette époque et
s'occupaient de l'intérieur de la France sont M\textsuperscript{lle} de
Montpensier, M\textsuperscript{me} de Motteville, Montglat et
Gourville\,; il faut y ajouter Gui Patin, dont les lettres forment une
véritable gazette de l'époque. M\textsuperscript{lle} de Montpensier,
qui vivait alors loin de Paris, ne traite que des intrigues de sa petite
cour. M\textsuperscript{me} de Motteville ne parle pas de ces scènes,
auxquelles Anne d'Autriche resta étrangère. Quant à Gourville, qui était
alors attaché à Nicolas Fouquet, à la fois surintendant et procureur
général, il donne de curieux renseignements sur le prix auquel les
conseillers du parlement vendaient leurs votes\footnote{«\,M. Fouquet me
  parlant un jour de la peine qu'il y avoit à faire vérifier des édits
  au parlement, je lui dis que, dans toutes les chambres, il y avoit des
  conseillers qui entraînaient la plupart des autres\,; que je croyais
  qu'on pouvoit leur faire parler par des gens de leur connoissance,
  leur donner à chacun cinq cents écus de gratification et leur en faire
  espérer autant dans la suite aux étrennes. J'en fis une liste
  particulière, et je fus chargé d'en voir une partie que je
  connaissais. On en fit de même pour d'autres\ldots. Quelque temps
  après il se présenta une occasion au parlement, où M. Fouquet jugea
  bien que ce qu'il avoit fait avoit utilement réussi.\,» \emph{Mémoires
  de Gourville}, à l'année 1655.}. Gui Patin, dont on connaît l'humeur
chagrine, se borne à dire dans une lettre du 26 mars 1665\,: «\,Le
lendemain matin le roi a été au palais, où il a fait vérifier quantité
d'édits de divers offices et autres. M. Bignon y a harangué devant le
roi très pathétiquement, et y a dit merveilles, et nonobstant tout a
passé\,; \emph{interea patitur justus\,; nec est qui recogitet
corde}.\,» Dans une lettre du 21 avril, il ajoute\,: «\,Le parlement
s'étoit assemblé de nouveau pour examiner les édits que le roi fit
vérifier en sa présence la dernière fois qu'il fut au palais, qui fut à
la fin du carême\,: cela a irrité le conseil, et défenses là-dessus leur
ont été envoyées de ne pas s'assembler davantage. Et de peur que le roi
ne fût pas obéi, il a pris lui-même la peine d'aller au palais bien
accompagné, \emph{où de sa propre bouche, sans autre cérémonie, il leur
a défendu de s'assembler davantage contre les édits qu'il fit l'autre
jour publier}.\,»

Le marquis de Montglat, dont les Mémoires se distinguent par leur
exactitude, est celui des auteurs contemporains qui insiste le plus sur
le costume insolite du roi. Voici le passage\,: «\,Le roi fut tenir son
lit de justice au parlement le 20 mars (1655), pour faire vérifier des
édits. Et, parce que l'autorité royale n'étoit pas encore bien rétablie,
les chambres s'assemblèrent pour revoir les édits, disant que la
présence du roi avoit ôté la liberté des suffrages, et qu'il étoit
nécessaire en son absence de les examiner pour voir s'ils étoient
justes. La mémoire des choses passées faisoit appréhender ces
assemblées, après les événements funestes qu'elles avoient causés. Cette
considération obligea le roi de partir du château de Vincennes le 10
d'avril, et de venir le matin au parlement en justaucorps rouge et
chapeau gris, accompagné de toute sa cour en même équipage\,: \emph{ce
qui étoit inusité jusqu'à ce jour}. Quand il fut dans son lit de
justice, il défendit au parlement de s'assembler\,; et après avoir dit
quatre mots, il se leva et sortit, sans ouïr aucune harangue.\,»

Montglat, qui, en sa qualité de maître de la garde-robe, était
parfaitement instruit du cérémonial, n'est frappé que du costume
insolente du roi et de sa cour. Quant à la houssine, que d'autres ont
remplacée par un fouet, il n'en dit pas un mot. Enfin un journal de
cette époque, dont l'auteur est resté inconnu, complète le récit de
Montglat et donne la scène entière avec toute l'étendue nécessaire pour
rectifier les assertions erronées\footnote{Ce journal, manuscrit, est
  conservé à la Bibliothèque Impériale sous le n° 1238 \emph{d} (bis)\,;
  S. F.}\,:

«\,Le parlement, dit l'auteur anonyme\footnote{Fol. 326, sqq.}, s'étant
assemblé le vendredi 9 avril (1655) pour entendre la lecture des édits
plus au long et plus attentivement qu'il n'avoit fait en présence de Sa
Majesté, il n'en put apprendre la conséquence et les incommodités que
tout le monde en recevroit sans horreur et sans confusion, tant ils
étoient à l'oppression de tous les particuliers que d'impossible
exécution\footnote{On voit, par ce passage, que l'auteur n'était pas
  favorable aux édits bursaux, et il n'aurait pas manqué de faire
  ressortir les circonstances qui auraient caractérisé la violence du
  gouvernement.} . M. le chancelier (Séguier) s'en défendit le mieux
qu'il lui étoit possible, «\,pour n'en avoir eu, disoit-il, aucune
communication.\,» M. le garde des sceaux\footnote{Mathieu Molé était
  garde des sceaux depuis 1651.} assuroit ne l'avoir vu qu'en le
scellant le matin du même jour qu'il avoit été porté au parlement, et
tout le conseil protestoit ingénument de n'y avoir participé en aucune
façon, si bien que, pour assoupir cette grande rumeur qui alloit se
répandre par toute la ville et ensuite dans toutes les provinces, si le
parlement eût continué ses assemblées, le roi fut conseillé d'y
retourner le mardi suivant, 13 du mois d'avril\footnote{Il y a une
  légère différence de date avec Montglat. Le journal anonyme donne
  l'indication précise des jours\,; ce qui ferait pencher la balancé en
  sa faveur.}, afin de les dissoudre et d'en empêcher le cours une fois
pour toutes. Sa Majesté y fut reçue en la manière accoutumée, mais sans
que la compagnie sût aucune chose de sa résolution. En entrant, elle ne
fit paroître que trop clairement sur son visage l'aigreur qu'elle avoit
dans le cœur. «\,Chacun sait, leur dit-elle d'un ton moins doux et moins
gracieux qu'à l'ordinaire, combien vos assemblées ont excité de trouble
dans \emph{mon État}\footnote{Serait-ce cette parole qui aurait donné
  lieu à la phrase célèbre\,: \emph{L'État, c'est moi\,?} Beaucoup de
  prétendus mots historiques n'ont pas une origine plus sérieuse.}**, et
combien de dangereux effets elles y ont produits. J'ai appris que vous
prétendiez encore les continuer, sous prétexte de délibérer sur les
édits qui naguère ont été lus et publiés en ma présence. Je suis venu
ici tout exprès pour en défendre (en montrant du doigt MM\hspace{0pt}.
des enquêtes) la continuation, ainsi que je fais absolument\,; et à
vous, monsieur le premier président\footnote{Le premier président était
  alors Pomponne de Bellièvre.} (en le montrant aussi du doigt), de les
souffrir ni de les accorder, quelques instances qu'en puissent faire les
enquêtes.\,» Après quoi, Sa Majesté s'étant levée promptement, sans
qu'aucun de la compagnie eût dit une seule parole, elle s'en retourna au
Louvre et de là au bois de Vincennes, dont elle étoit partie le matin,
et où M. le cardinal l'attendoit.\,»

Voilà le récit le plus complet et le plus circonstancié de cette scène
qui a été si singulièrement travestie par l'imagination de quelques
historiens. Louis XIV, qui avait alors dix-sept ans, était allé
s'établir au château de Vincennes pour se livrer plus facilement au
plaisir de la chasse\,; ce qui explique le costume insolite dont parle
Montglat. Malgré la défense formelle du roi, le parlement ne se tint pas
pour battu\,; le premier président entra en conférence avec le cardinal
Mazarin, et les enquêtes demandèrent l'assemblée des
chambres\footnote{Ces détails se trouvent dans le journal anonyme que
  j'ai cité plus haut.}. Le 29 avril, le premier président, avec les
députés du parlement, alla supplier le roi de la leur accorder. «\,Mais,
dit l'auteur du journal, le roi continuant dans la fermeté que son
conseil avoit jugée nécessaire à l'entier rétablissement de son
autorité, lui dit seulement\,: \emph{qu'il ne lui restoit aucune aigreur
contre aucun de la compagnie\,; qu'il ne vouloit point toucher à ses
privilèges\,; mais que le bien de ses affaires présentes ne pouvant
consentir à leurs assemblées, Sa Majesté leur en dépendoit d'abondant la
continuation}.\,»

\hypertarget{note-vii.-uxe9pices.}{%
\chapter{NOTE VII. ÉPICES.}\label{note-vii.-uxe9pices.}}

Les épices, dont il est question dans ce volume de Saint-Simon, étaient
primitivement des présents en nature que l'on offrait aux juges après le
gain d'un procès. Cet usage était très ancien. Saint Louis défendit aux
juges de recevoir en épices plus de la valeur de dix sous par semaine.
Philippe le Bel leur interdit d'en accepter au delà de ce qu'ils
pourraient consommer journellement dans leur maison. Peu à peu l'usage
s'introduisit de remplacer les épices par de l'argent\,; mais le nom
resta le même. On voit en 1369 un sire de Tournon donner vingt francs
d'or à ses deux rapporteurs, et ce après l'avoir obtenu en présentant
requête au parlement. Les juges finirent par considérer les épices comme
une redevance qui leur était due, et un arrêt de 1402 prononça dans ce
sens. On obligea même les plaideurs à les remettre d'avance, et depuis
cette époque on appela épices la somme que les juges des divers
tribunaux recevaient des parties dont ils avaient examiné le procès.

\end{document}
